\label{secc}

We now give some review to some ideas in classical category theory. 

First we give some definitions on cardinal theory:

\defn{
    Suppose $\kappa$ is an infinte cardinal. $\kappa$ is called \term{regular}, if for any collection of sets $\{A_i\}_{i\in I}$,
    such that $\abs I<\kappa$ and $\abs{A_i}<\kappa$ for all $I$, we have $\abs{\coprod_{i\in I}A_i}<\kappa$.
    $\kappa$ is called \term{strongly inaccessible}, if $\kappa$ is regular, uncountable, and for any cardinal $\tau<\kappa$
    we have $2^\tau<\kappa$.
}

By definition, if $\kappa$ is strongly inaccessible, all sets with cardinality less than $\kappa$ satisfies the ZF axioms.
Thus the existence of strongly inaccessible cardinals cannot be proven in the ZF axioms.

We note more that for any cardinal $\kappa$, there exists a cardinal $\kappa'>\kappa$ such that $\kappa$ is regular.

\defn{
    Suppose $\kappa$ and $\tau$ are regular cardinals. We write $\tau\ll\kappa$, if for any $\tau_0<\tau$ and $\kappa_0<\kappa$
    we have $\kappa_0^{\tau_0}<\kappa$.
}

We note that for any regular cardinal $\tau$, there exists a regular cardinal $\kappa>\tau$ such that $\kappa\gg\tau$.
Also note that a regular cardinal $\kappa$ satisfies that $\kappa\ll\kappa$ if and only if $\kappa=\omega$
or $\kappa$ is strongly inaccessible.

We now deduce with the set-theoretic technicalities. In the theory of 1-categories, mostly we may assume that 
the hom-set between two objects to be indeed a set but allowing the object to form a proper class.
(A category with all hom's between two objects to be (small) sets is called \term{locally small}.) However,
we will see that in our theory of $\infty$-categories, objects and morphisms are treated equally,
meaning that we are facing set-theoretic problems. To fix this problem, we shall use the following terminology:
for any cardinal $\tau$, we assume that there exists a strongly inaccessible cardinal
$\kappa$ that is greater than $\tau$. We will mostly fix a strongly inaccessible cardinal $\kappa$, and we will mainly deal with
mathematical objects that are $\kappa$-small. When we need to deal with mathematical objects that are too large to be
$\kappa$-small, we shall take another strongly inaccessible cardinal $\kappa'$ that is greater than $\kappa$, and to work
with mathematical objects that are $\kappa'$-small, in which case it is okay to talk about objects that are not $\kappa$-small.
This terminology will work throughout this chapter.

Next, we discuss the theory of enriched categories.

\defn{
    Suppose $(\catC,\ox,\1_{\catC})$ is a monoidal category (where $\1_{\catC}$ is the tensor unit). A \term{$\catC$-enriched category},
    $\catD$, consists of the following data:
    \begin{enumerate}[i)]
        \item A class of objects;
        \item A object $\Hom_{\catD}(X,Y)\in\catC$ for every pair of object $X,Y\in\catD$;
        \item An identity $\1_{\catC}\to\Hom_{\catD}(X,X)$ for every object $X\in\catD$;
        \item A composition map $\Hom_{\catD}(Y,Z)\ox\Hom_{\catD}(X,Y)\to\Hom_{catD}(X,Z)$ for every triple of objects $X,Y,Z\in\catD$;
        \item Such that the composition is associative and unital, i.e. the following diagrams are commutative:
        $$\xymatrix{(\Hom_{\catD}(Z,W)\ox\Hom_{\catD}(Y,Z))\ox\Hom_{\catD}(X,Y)\ar[dd]\ar[r]&\Hom_{\catD}(Y,W)\ox\Hom_{\catD}(X,Y)\ar[d]\\
        &\Hom_{\catD}(X,W)\\\Hom_{\catD}(Z,W)\ox(\Hom_{\catD}(Y,Z)\ox\Hom_{\catD}(X,Y))\ar[r]&\Hom_{\catD}(Z,W)\ox\Hom_{\catD}(X,Z),\ar[u]}$$
        $$\xymatrix{\1_{\catC}\ox\Hom_{\catD}(X,Y)\ar[rr]\ar[rd]&&\Hom_{\catD}(Y,Y)\ox\Hom_{\catD}(X,Y)\ar[ld]\\&\Hom_{\catD}(X,Y),&}$$
        $$\xymatrix{\Hom_{\catD}(X,Y)\ox\1_{\catC}\ar[rr]\ar[rd]&&\Hom_{\catD}(X,Y)\ox\Hom_{\catD}(X,X)\ar[ld]\\&\Hom_{\catD}(X,Y).&}$$
    \end{enumerate}
    For objects $X,Y$ in $\catD$, a \term{morphism} $f$ from $X$ to $Y$ is a map $\1_\catC\to\Hom_{\catD}(X,Y)$ in $\catC$.
    When $\catC=\cat{SSet}$, a $\catC$-enriched category is also called a \term{simplicial category}.
}

For example, for any closed left $\catC$-module $\catD$, $\catD$ can be regarded as a $\catC$-enriched category, by making
$\Hom_{\catD}(X,Y)=\sHom(X,Y)$.

\defn{
    Suppose $\catC$ is a monoidal category, and $\catD,\catD'$ are $\catC$-enriched categories. A \term{$\catC$-enriched functor}
    $F$ from $\catD$ to $\catD'$ consists of the following data:
    \begin{enumerate}[i)]
        \item A map from the objects of $\catD$ to the objects of $\catD'$;
        \item A morphism $\Hom_{\catD}(X,Y)\to\Hom_{\catD'}(FX,FY)$ for any pair of objects $X,Y\in\catD$;
        \item Such that the following diagrams are commutative:
        $$\xymatrix{\Hom_{\catD}(Y,Z)\ox\Hom_{\catD}(X,Y)\ar[r]\ar[d]&\Hom_{\catD}(X,Z)\ar[d]\\
        \Hom_{\catD'}(FY,FZ)\ox\Hom_{\catD'}(FX,FY)\ar[r]&\Hom_{\catD'}(FX,FZ),}$$
        $$\xymatrix{&\1\ar[ld]\ar[rd]\\\Hom_{\catD}(X,X)\ar[rr]&&\Hom_{\catD'}(FX,FX).}$$
    \end{enumerate}
    When $\catC=\cat{SSet}$, a $\catC$-enriched functor is also called a \term{simplicial functor}.
}

\defn{
    Suppose $\catC$ is a monoidal category, $F,F':\catD\to \catD'$ are two $\catC$-enriched functors between $\catC$-enriched categories.
    A \term{$\catC$-enriched natural transformation} $\tau$ from $F$ to $F'$ consists of a map $\1\to\Hom_{\catD'}(FX,F'X)$ for any object
    $X\in\catD$, such that the following diagram commutes:
    $$\xymatrix{\Hom_{\catD'}(FX,FY)\ar[r]&\1\ox\Hom_{\catD'}(FX,FY)\ar[r]&\Hom_{\catD'}(FY,F'Y)\ox\Hom_{\catD'}(FX,FY)\ar[d]\\
    \Hom_{\catD}(X,Y)\ar[d]\ar[u]&&\Hom_{\catD'}(FX,F'Y)\\
    \Hom_{\catD'}(F'X,F'Y)\ar[r]&\Hom_{\catD'}(F'X,F'Y)\ox\1\ar[r]&\Hom_{\catD'}(F'X,F'Y)\ox\Hom_{\catD'}(FX,F'X)\ar[u]}$$
    When $\catC=\cat{SSet}$, a $\catC$-enriched natural transformation is also called a \term{simplicial natural transformation}.
}

Note that these definitions restricts to ordinary category theory when $\catC=(\cat{Set},\times)$.

By the above definitions, we may form the category $\cat{Cat}_{\catC}$ of all (small) $\catC$-enriched categories,
and the 2-category $\catt{Cat}_{\catC}$ of all (small) $\catC$-enriched categories. Also, for any two $\catC$-enriched categories
$\catD,\catD'$, we have the category $\Fun(\catD,\catD')$ of all $\catC$-enriched functors.

We also make the following definition:

\defn{
    Suppose $\catC,\catD$ are monoidal categories. A \term{lax monoidal functor} between $\catC$ and $\catD$ is a triple $(F,\mu,\beta)$, 
    where $F:\catC\to\catD$ is a functor, $\mu_{XY}:FX\ox FY\to F(X\ox Y)$ is a natural transformation, 
    $\beta:\1_{\catD}\to F(\1_{\catC})$ is a map, such that the following diagrams are commutative for any objects $X,Y,Z$:
    $$\xymatrix @C=60pt{
    (FX\ox FY)\ox FZ\ar[d]_{\mu_{XY}\ox\1_{FZ}}\ar[r]^{a_{(FX)(FY)(FZ)}}&FX\ox(FY\ox FZ)\ar[d]^{\1_{FX}\ox\mu_{YZ}}\\
    F(X\ox Y)\ox FZ\ar[d]_{\mu_{(X\ox Y)Z}}&FX\ox F(Y\ox Z)\ar[d]^{\mu_{X(Y\ox Z)}}\\
    F((X\ox Y)\ox Z)\ar[r]^{F(a_{XYZ})}&F(X\ox(Y\ox Z))
    }$$ 
    $$\xymatrix @C=40pt{
    F\1\ox FX\ar[d]_{\mu_{\1X}}&\1\ox FX\ar[d]^{l_{FX}}\ar[l]^{\beta\ox\1_{FX}}\\
    F(\1\ox X)\ar[r]^{F(l_X)}&FX
    }\xymatrix @C=40pt{
    FX\ox F\1\ar[d]_{\mu_{X\1}}&FX\ox \1\ar[d]^{r_{FX}}\ar[l]^{\1_{FX}\ox\beta}\\
    F(X\ox \1)\ar[r]^{F(r_X)}&FX
    }$$ 
}

By definition, a monoidal functor is a lax monoidal functor. For example, if $\catC$ is a monoidal category,
then $\Hom_{\catC}(\1,-):\catC\to\cat{Set}$ is a lax monoidal functor.

\rmk{
    Now, suppose we have a lax monoidal functor $F:\catC\to\catC'$ between monoidal categories. Then for any $\catC$-enriched category
    $\catD$, we may form a $\catC'$-enriched category $F\catD$, with:
    \begin{enumerate}[i)]
        \item Objects being the objects of $\catD$;
        \item For any $X,Y\in\catD$, $\Hom_{F\catD}(X,Y)=F\Hom_{\catD}(X,Y)\in\catC'$;
        \item The unit is defined to be $$\1_{\catC'}\to F\1_{\catC}\to F\Hom_{\catD}(X,X)=\Hom_{F\catD}(X,X);$$
        \item The composition is defined to be $$\begin{aligned}\Hom_{F\catD}(Y,Z)\ox\Hom_{F\catD}(X,Y)
        &=(F\Hom_{\catD}(Y,Z))\ox(F\Hom_{\catD}(X,Y))\\&\to F(\Hom_{\catD}(Y,Z)\ox\Hom_{\catD}(X,Y))\\
        &\to F\Hom_{\catD}(X,Z)=\Hom_{F\catD}(X,Z).\end{aligned}$$
    \end{enumerate}
    This construction extends to a functor $F:\cat{Cat}_{\catC}\to\cat{Cat}_{\catC'}$ and a $2$-functor
    $F:\catt{Cat}_{\catC}\to\catt{Cat}_{\catC'}$, and also a functor $F:\Fun(\catD,\catD')\to\Fun(F\catD,F\catD')$ 
    for any two $\catC$-enriched categories $\catD,\catD'$.

    In particular, if the functor $F$ is chosen to be $U=\Hom_{\catC}(\1,-):\catC\to\cat{Set}$, then for any $\catC$-enriched category
    $\catD$, the ordinary category $U\catD$ is called the \term{underlying category} of $\catD$. In this case, 
    we notice that for objects $X,Y$ in $\catD$, the set of all morphisms from $X$ to $Y$ is simply $\Hom_{U\catD}(X,Y)$. 
    Therefore it is reasonable to call a morphism $f:X\to Y$ in $\catD$ an \term{isomorphism},
    if it is an isomorphism in $U\catD$.
}

We next make the following definition.

\defn{
    Suppose $\catD$ is a $\catC$-enriched category, where $\catC$ is a closed monoidal category. $\catD$ is called \term{tensored over $\catC$},
    if for all $D\in\catD$ and $C\in\catC$, the functor $$\catD\to\catC,E\mapsto\sHom_r(C,\Hom_{\catD}(D,E))$$ is representable. 
    We denote the representation of the functor by $C\ox D$. $\catD$ is called \term{cotensored over $\catC$},
    if for all $D\in\catD$ and $C\in\catC$, the functor $$\catD^\op\to\catC,E\mapsto\sHom_l(C,\Hom_{\catD}(E,D))$$ is representable. 
    We denote the representation of the functor by $D^C$.
}

We note that for any $\catC$-enriched category $\catD$ that is tensored over $\catC$, $\catD$ is naturally a left $\catC$-module.
If moreover $\catD$ is also tensored over $\catC$, the module structure is closed. 

We next give the following proposition, whose prove may be found in [nLab]:

\prop{[$\dagger$]
    For any two $\catC$-enriched category $\catD,\catD'$ where $\catC$ is closed symmetric monoidal, the functor category
    $\Fun(\catD,\catD')$ has a natural $\catC$-enriched structure, which we will denote by $(\catD')^\catD$ (or again $\Fun(\catD,\catD')$),
    whose underlying category is the ordinary category $\Fun(\catD,\catD')$, such that we have the following reciprocity formula
    between $\catC$-enriched categories:
    $$\Fun(\catD,\Fun(\catD',\catD''))\cong\Fun(\catD\ox\catD',\catD''),$$
    where $\catD\ox\catD'$ is the $\catC$-enriched category with objects being pairs $(D,D')$ where $D\in\catD,D'\in\catD'$,
    and $\Hom_{\catD\ox\catD'}((X,X'),(Y,Y'))=\Hom_{\catD}(X,Y)\ox\Hom_{\catD'}(X',Y')$ for $X,Y\in\catD,X',Y'\in\catD'$.
}

The next concept is the slice categories.

\defn{
    Suppose $\catC$ is a category, $p:\catI\to\catC$ is a diagram in $\catC$. The \term{category of objects of $\catC$ over $p$}, $\catC_{/p}$
    is the following category:
    \begin{enumerate}[i)]
        \item The objects in $\catC_{/p}$ are pairs $(X,\tau)$, where $X$ is an object in $\catC$ and $\tau:\const_X\to p$
        is a natural transformation;
        \item A morphism from $(X,\tau)$ to $(Y,\sigma)$ is a map $f:X\to Y$ in $\catC$ such that $\sigma\circ\const_f=\tau$.
    \end{enumerate}
    Dually, the \term{category of objects of $\catC$ under $p$}, $\catC_{p/}$ is the following category:
    \begin{enumerate}[i)]
        \item The objects in $\catC_{/p}$ are pairs $(X,\tau)$, where $X$ is an object in $\catC$ and $\tau:p\to\const_X$
        is a natural transformation;
        \item A morphism from $(X,\tau)$ to $(Y,\sigma)$ is a map $f:X\to Y$ in $\catC$ such that $\const_f\circ\tau=\sigma$.
    \end{enumerate}
}

We also use the following notation:

\defn{
    Suppose $\catC$ is a category and $Z\in\catC$. For any $X,Y\in\catC$, we define $\Hom_Z(X,Y):=\Hom_{\catC_{/Z}}(X,Y)$.
}

The slice category enjoy the following property:

\prop{
    Suppose $\catC$ is a category, $p:\catI\to\catC$ is a diagram in $\catC$. The initial object of $\catC_{p/}$ is exactly
    colimit of $p$ in $\catC$. Dually the terminal object of $\catC_{/p}$ is exactly the limit of $p$ in $\catC$.
}

The slice category can also be defined by a universal property. We first make the following property:

\defn{
    Suppose $\catC,\catC'$ are two categories. The \term{join} of the two categories, $\catC\star\catC'$, is defined as follows:
    \begin{enumerate}[i)]
        \item The objects of $\catC\star\catC'$ is the disjoint union of the objects $\catC$ and $\catC'$;
        \item $$\Hom_{\catC\star\catC'}(C,C')=\begin{cases}\Hom_{\catC}(C,C'),&(C,C'\in\catC)\\\{*\},&(C\in\catC,C'\in\catC')\\
        \varnothing,&(C\in\catC',C'\in\catC)\\\Hom_{\catC'}(C,C').&(C,C'\in\catC')\end{cases}$$
    \end{enumerate}
}

We note that there exists natural inclusions $\catC\to\catC\star\catC'$ and $\catC'\to\catC\star\catC'$.

Now we may show that the slice category enjoys the following universal property:

\prop{
    Suppose $\catC$ is a category, $p:\catI\to\catC$ is a diagram in $\catC$. Then for any category $\catJ$, there exists a
    natural isomorphism $$\Fun(\catJ,\catC_{/p})\cong\Fun(\catJ\star\catI,\catC)\times_{\Fun(\catI,\catC)}\{p\}.$$
    There is a dual statement for undercategories.
}

Now we can generalize this idea to joins, overcategories, undercategories, limits and colimits to $\catC$-enriched categories,
where $\catC$ is a closed symmetric monoidal category whose tensor unit is the terminal object. Details are left to the reader.

We next introduce the language of Kan extensions:

\defn{
    Suppose $\catC$ is a category, $i:\catI\to\catJ$ is a functor between categories, which induces a functor
    $i^*:\catC^\catJ\to\catC^\catI.$ The \term{left Kan extension} of a functor $F:\catI\to\catC$, is a functor 
    $i_!F:\catJ\to\catC$, together with a natural transformation $F\to i^*i_!F$, such that for any functor $G:\catJ\to\catC$,
    the natural transformation $$\Hom_{\catC^\catJ}(i_!F,G)\to\Hom_{\catC^\catI}(i^*i_!F,i^*G)\to\Hom_{\catC^\catI}(F,i^*G)$$
    is a natural isomorphism. Dually the \term{right Kan extension} of a functor $F:\catI\to\catC$, is a functor 
    $i_*F:\catJ\to\catC$, together with a natural transformation $i^*i_*F\to F$, such that for any functor $G:\catJ\to\catC$,
    the natural transformation $$\Hom_{\catC^\catJ}(G,i_*F)\to\Hom_{\catC^\catI}(i^*G,i^*i_*F)\to\Hom_{\catC^\catI}(i^*G,F)$$
    is a natural isomorphism. A functor $i_!:\catC^\catI\to\catC^\catJ$ is called a \term{left Kan extension functor}
    if it is a left adjoint to $i^*$. Dually a functor $i_*:\catC^\catI\to\catC^\catJ$ is called a \term{right Kan extension functor}
    if it is a right adjoint to $i^*$.
}

Kan extensions are generalizations of limits and colimits, as one can see from the following proposition:

\prop{
    Suppose $\catC$ is a category, $i:\catI\to\catJ$ is a functor between categories. If $\catC$ has all colimits,
    then the left Kan extension functor exists, and for any functor $F:\catI\to\catC$ and object $Y\in\catJ$,
    $$i_!F=\colim_{X\in\catI,iX\to Y}F(X).$$ Dually, if $\catC$ has all limits,
    then the right Kan extension functor exists, and for any functor $F:\catI\to\catC$ and object $Y\in\catJ$,
    $$i_*F=\lim_{X\in\catI,Y\to iX}F(X).$$
}

The following definition and proposition also hold in enriched category theory. We will discuss the theory of homotopy Kan extensions
in the next section, and its $\infty$-categorical generalization in Section \ref{secj}.

We conclude this section with a brief discussion of presentable categories and accessible categories:

\defn{
    Suppose $\kappa$ is a regular cardinal. A category $\catC$ is called \term{$\kappa$-filtered},
    if for any category $\catD$ that is $\kappa$-small (that is, the arrows in $\catD$ hava cardinality less than $\kappa$),
    and any diagram $p:\catD\to\catC$, the category $\catC_{p/}$ is nonempty. A diagram is called \term{$\kappa$-filtered}
    if it is indexed by a $\kappa$-filtered category. A colimit is called \term{$\kappa$-filtered} if 
    it is the colimit of a $\kappa$-filtered diagram. We say a structure is \term{filtered} if it is $\omega$-filtered.
}

\defn{
    Suppose $\kappa$ is a regular cardinal. We call a category \term{$\kappa$-accessible} if the category is locally small,
    every object of the category is small, the category admits small $\kappa$-filtered colimits, and there exists a small set of objects
    that generates the category under small $\kappa$-filtered colimits. We call a category \term{accessible} if it is $\kappa$-accessible
    for some regular cardinal $\kappa$.
}

\defn{
    We call a category \term{presentable} if it is accessible and has all small colimits.
}

\lem{
    A category is presentable if the category is locally small, every object of the category is small,
    the category admits small colimits, and there exists a small set of objects that generates the category under small colimits.
}

\defn{
    Suppose $\catC$ is a category, $S$ is a class of morphisms. $S$ is called \term{weakly saturated}, if $S$ is closed under
    pushouts, retracts and transfinite compositions.
}

By the small object argument, we deduce the following:

\prop{
    Suppose $\catC$ is a presentable category. Then for any set of morphisms $S$ in $\catC$, the smallest weakly saturated class
    of morphisms in $\catC$ containing $S$ is $\Cof(S)$.
}

For further use, we state the following proposition:

\prop{[$\dagger$]
    Suppose $\catC$ is a presentable category, $\catC'$ is a full subcategory of $\catC$ that is closed under small $\kappa$-filtered colimits
    for some regular cardinal $\kappa$. The following conditions are equivalent:
    \begin{enumerate}[i)]
        \item There exists a small set of objects in $\catC'$ that generates $\catC'$ under small colimits;
        \item For any sufficiently large cardinals $\tau$ with $\tau\gg\kappa$, any $\tau$-filtered partially ordered set $A$,
        and any diagram $\{X_\alpha\}_{\alpha\in A}$ of $\kappa$-small objects in $\catC$ indexed by $A$, if $X_A\in\catC'$,
        where $X_B$ is the colimit of $\{X_\alpha\}_{\alpha\in B}$ for all $B\subseteq A$, then for every $\tau$-small subset $C\subseteq A$,
        there exists a $\tau$-small $\kappa$-filtered subset $B\subseteq A$ containing $C$ such that $X_B\in\catC_0$.
    \end{enumerate}
}

\cor{
    Suppose $F:\catC\to\catD$ is a functor between presebntable categories which preserves $\kappa$-filtered colimits and 
    $\catD'$ is a $\kappa$-filtered full subcategory of $\catD$. Then $F^{-1}(\catD)$ is a $\kappa$-filtered full subcategory of $\catC$.
}

We will discuss more about presentable categories and accessible categories in Section \ref{seck} and Section \ref{secl}.
\footnote{We note that even though we are discussing on presentable and accessible $\infty$-categories in Section \ref{seck}
and Section \ref{secl}, all arguments are also available for ordinary categories, and not circularity will result from
using the results given in the two stated sections now.}