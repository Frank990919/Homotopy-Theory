\lem{
    Every topological space is small relative to the inclusions.
}

\defn{
    Define an inclusion to be a \term{closed $T_1$ inclusion} if it is closed and every point not in the image is closed.
}

\prop{
    Every compact space is finite relative to closed $T_1$ inclusions.
}

\defn{
    Define the set $I=\{S^{n-1}\to D^n\mid n\ge 0\}$ ($S^{-1}=\varnothing,D^0=*$), and the set $J=\{D^n\to D^n\times[0,1]\mid n\ge 0\}$. 
    Define $\Cof=\Cof(I)$, $\Fib=\RLP(J)$, and $\W$ to be all weak homotopy equivalences. Define a map to be a \term{Serre fibration} if and only it is in $\Fib$.
    Define a map to be a \term{relative cell complex} if and only it is in $\Cell(I)$.
}

\lem{
    Every relative cell complex is a closed $T_1$ inclusion.
}

\cor{
    Every map in $\Cof$ or $\Cof(J)$ is a closed $T_1$ inclusion.
}

\lem{
    Suppose $X$ is a relative cell complex. Then every compact set of $X$ intersects the interiors of only finitely many cells.
}

\lem{
    Closed $T_1$ inclusions that are also weak equivalence are closed under transfinite composition.
}

\prop{
    $\Cof(J)\subseteq\W\cap\Cof$.
}

\prop{
    $\RLP(I)\subseteq\W\cap\Fib$.
}

\lem{
    A map $p:X\to Y$ is in $\RLP(I)$ if and only for any map $i:A\to B\in I$, $\sHom_{\squ}(i,p):\sHom(B,X)\to\sHom(A,X)\times_{\sHom(A,Y)}\sHom(B,Y)$
    is surjective. In particular, if for any map $i\in I$, $\sHom_{\squ}(i,p)$ is a trivial fibration, then $p\in\RLP(I)$.
}

\lem{
    If $p$ is a fibration and $i\in I$, then $\sHom_{\squ}(i,p)$ is a fibration.
}

\cor{
    Every topological space is fibrant, and for any $n\ge 0$ and $Y$ a space the map $\sHom(D^n,Y)\to\sHom(S^{n-1},Y)$ is a fibration.
}

\lem{
    If $p$ is a weak equivalence, then for any $n\ge 0$, $\sHom(D^n,p)$ is a weak equivalence.
}

\lem{
    If $p$ is a weak equivalence, then for any $n\ge -1$, $\sHom(S^n,p)$ is a weak equivalence.
}

\prop{
    The pullback of any weak equivalence along a fibration is a weak equivalence.
}

\thm{
    $\W\cap\Fib\subseteq\RLP(I)$.
}

\thm{
    $\cat{Top}$ is a finitelyly generated model category with $I$ being its generating cofibrations, $J$ being its generating trivial cofibrations 
    and weak equivalences being weak homotopy equivalences. As a corollary, $\cat{Top}_*$ is a finitely generated model category with $I_+$ 
    being its generating cofibrations, $J_+$ being its generating trivial cofibrations and weak equivalences being weak homotopy equivalences.
}

\defn{
    Suppose $X$ is a space. A subset $U$ of $X$ is called \term{compactly open} if for any map $f:K\to X$ with $K$ compact Hausdorff, $f^{-1}(U)$ is open.
    $X$ is called a \term{$k$-space} if all its compactly open sets are open. Define $\cat{K}$ to be the full subcategory of $\cat{Top}$
    containing all $k$-spaces. Define the \term{$k$-space topology} on $X$, denoted $kX$, to be defining $U$ open in $kX$ if and only if
    it is compactly open in $X$.
}

\prop{
    $k$ is a functor from $\cat{Top}$ to $\cat{K}$, and it has a left adjoint and right inverse $i$, the inclusion.
}

\prop{
    $\cat{K}$ has all small colimits and limits, where colimits are directly taken in $\cat{Top}$ and limits are taken by applying $k$ 
    to the limit in $\cat{Top}$.
}

\prop{
    For any $X,Y\in\cat{K}$, define $C(X,Y)$ to be the space with underlying set $\Hom_{\cat{K}}(X,Y)$, with the topological subbasis
    $$\begin{array}{l}\{\{g:X\to Y\mid g(f(K))\subseteq U\}\mid f:K\to X,\\K\text{ compact, Hausdorff, }U\text{ an open set of }Y\}.\end{array}$$ 
    Define $\sHom(X,Y)=kC(X,Y)$, then we have a natural isomorphism
    $$\Hom_{\cat{K}}(k(X\times Y),Z)\cong\Hom_{\cat{K}}(X,\sHom(Y,Z))$$ for any $X,Y,Z\in\cat{K}$.
    \footnote{The usage of $\Hom,\sHom,\Map,-^-$ are as follows. Suppose we want to say that the Hom-object of $A$ and $B$ is $C$. 
    If $C$ is a set, or $C$ is a simplicial set, or we are talking about Hom-object in enriched categories, we use $\Hom$. If in addition
    $C$ is a simplicial set, we may add $\bullet$ to distinguish it with the case that $C$ is a set. If $C$ is a homotopy type, we use
    $\Map$. If $B,C$ are of the same type, and a) $A$ is of different type, or b) $B,C$ are sets, simplicial sets or homotopy types,
    we use the exponential $-^-$. In other cases, we use $\sHom$.}
}

\thm{
    $\cat{K}$ is a finitely generated model category with $I$ being its generating cofibrations, $J$ being its generating trivial cofibrations 
    and weak equivalences being weak homotopy equivalences. Moreover we have $\Cof_{\cat{K}}=\Cof_{\cat{Top}}\cap\cat{K},
    \Fib_{\cat{K}}=\Fib_{\cat{Top}}\cap\cat{K}$. As a corollary, $\cat{K}_*$ is a finitelyly generated model category with $I_+$ being
    its generating cofibrations, $J_+$ being its generating trivial cofibrations and weak equivalences being weak homotopy equivalences.
    Moreover we have $\Cof_{\cat{K}_*}=\Cof_{\cat{Top}_*}\cap\cat{K}_*,\Fib_{\cat{K}_*}=\Fib_{\cat{Top}_*}\cap\cat{K}_*$. Furthermore 
    the inclusions $\cat{K}\to\cat{Top},\cat{K}_*\to\cat{Top}_*$ are Quillen equivalences.
}