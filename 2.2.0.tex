In this section, we will discuss the Grothendieck construction in $\infty$-category theory.
We will see how this is useful to link simplicial categories and topological categories.
We will start with the Grothendieck construction in classical category theory to get some get some first sense,
and we will generalize it to the case of $\infty$-categories.

The Grothendieck construction in classical category theory is as follows. Suppose $F:\catC\to\catD$ is a functor of categories.
Then for every object $D\in\catD$, the fiber $\catC_{D}$ is a category. We sometimes wish that the action ``taking the fiber''
yields a functor $\catD\to\cat{Cat}$ (in the covariant fashion), or $\catD^\op\to\cat{Cat}$ (in the contravariant fashion);
or at least, in a homotopy-coherent sense. We will show that this can be done in the case that the functor $F$ is some ``fibration''.
The translation between certain ``fibrations'' over $\catD$, and homotopy-coherent functors $\catD\to\cat{Cat}$,
is called the \term{Grothendieck construction}.

\eg{
    Consider the category of all quasi-coherent sheaves, $\cat{QCoh}$. It has objects $(X,\mathscr{F})$, where $X$ is a scheme
    and $\mathscr{F}$ is a quasi-coherent sheaf on $X$; and a morphism from $(X,\mathscr{F})$ to $(Y,\mathscr{G})$
    consists of a map $f:Y\to X$ of schemes, together with a map $f^\sharp:\mathscr{F}\to f_*\mathscr{G}$ of $\mathscr{O}_X$-modules.
    Now take the functor $\cat{QCoh}\to\cat{Sch}^\op$ that takes a pair $(X,\mathscr{F})$ to its underlying scheme $X$.
    Then for every scheme $X$, the fiber of this functor at $X$ is the category of all quais-coherent sheaves on $X$,
    $\cat{QCoh}(X)$. Therefore the action ``taking the fiber'' actually yields a homotopy-coherent functor
    $\cat{QCoh}(-):\cat{Sch}^\op\to\cat{Cat}$.
}

To give a partial solution to the question, let us restrict to the case where the fibers of $F$ are groupoids, so we would
like to form a functor $\catD\to\cat{Gpd}$. The converse is easy: for any functor $\chi:\catD\to\cat{Gpd}$, we may form a functor
$F:\catC_\chi\to\catD$, where the category $\catC_\chi$ has:
\begin{itemize}
    \item Objects being $(D,\eta)$, where $D$ is an object in $\catD$ and $\eta$ is an object in $\chi(D)$;
    \item A morphism $(D,\eta)\to(D',\eta')$ consists of a morphism $f:D\to D'$ in $\catD$, and a morphism $\alpha:\chi(f)(\eta)\to\eta'$
    in $\chi(D')$;
\end{itemize}
and the functor $F$ takes every object $(D,\eta)$ to $D$, ignoring the $\eta$.
(In fact, this construction is also available in the case that $\chi$ is a homotopy-coherent functor,
or is a (homotopy-coherent) functor $\catD\to\cat{Cat}$.) 

For the converse, we make the following observation: for every map $f:D\to D'$ in $\catD$, we must construct a transport functor
$f_!:\catC_D\to\catC_{D'}$. In other words, we require the functor $F$ satisfies the following properties:
\begin{itemize}
    \item Transport of objects: for any object $C\in\catC$ such that $F(C)=D$, there exists a morphism $g:C\to C'$
    in $\catC$ such that $F(g)=f$;
    \item Transport of morphisms: for any diagram $$\xymatrix{C\ar[r]^{g}\ar[d]_{h}&C'\\\overline{C}\ar[r]^{\overline{g}}&\overline{C}',}$$
    such that $F(g)=F(\overline g)=f,F(h)=\1_D$, there exists UNIQUELY a map $h':C'\to\overline{C'}$ such that $h'g=\overline{g}h$
    and $F(h')=\1_{D'}$.
\end{itemize}
We now make this idea into a definition:

\defn{
    Suppose $F:\catC\to\catD$ is a functor. $F$ is called a \term{left fibration} if the following conditions are satisfied:
    \begin{itemize}
        \item The lifting problem $$\xymatrix{\Ho\l^0[1]\ar[r]\ar[d]&\catC\ar[d]\\\Ho\d[1]\ar[r]\ar@{-->}[ru]&\catD}$$ has a solution;
        \item The lifting problem $$\xymatrix{\Ho\l^0[2]\ar[r]\ar[d]&\catC\ar[d]\\\Ho\d[2]\ar[r]\ar@{-->}[ru]&\catD}$$ has a unique solution.
    \end{itemize}
}

The reader should verify that this definition is equivalent to the transport axioms given above, and that this definition
yields that every fiber of $F$ is a groupoid. Moreover, by the discussion given above, we have the following theorem:

\thm{
    A left fibration between small categories $\catC\to\catD$ gives rise to a 2-functor $\catD\to\cat{Gpd}$ by taking the fibers. 
    Conversely, every 2-functor $\catD\to\cat{Gpd}$ corresponds to a left fibration $\catC\to\catD$ in this manner.
}

We finally make the following definition.

\defn{
    We call a functor $F:\catC\to\catD$ a \term{right fibration} if $F^\op$ is a left fibration.
}

Under this definition, right fibrations $\catC\to\catD$ correspond to 2-functors $\catD^\op\to\cat{Gpd}$.

We will generalize the idea to left and right fibrations between simplicial sets, and discuss the corresponding
Grothendieck construction (on $\infty$-groupoids, which is equivalent to spaces) in the first three subsections.
In the last four subsections, we will be more general and discuss the Grothendieck construction on $\infty$-categories.

\rmk{
    In Lurie's book, he mainly considered the contravariant-fashioned structures, such as right fibrations and cartesian fibrations.
    This is also what the lecturer had done in class. However, the lecturer decided to write the notes mainly using
    the covariant-fashioned structures, such as left fibrations and cocartesian fibrations, in order to make it more intuitive
    and readable. The reader should feel free to transfer between the covariant-fashioned structures and 
    the contravariant-fashioned structures (after all, it is just taking the opposite category).
}