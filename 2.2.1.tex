\label{secg}

We first give the following definitions:

\defn{
    A map $p:X\to S$ is called:
    \begin{enumerate}[i)]
        \item An \term{inner fibration} if it is in $\RLP\{\l^i[n]\to\d[n]\mid 0<i<n\}$;
        \item A \term{left fibration} if it is in $\RLP\{\l^i[n]\to\d[n]\mid 0\le i<n\}$;
        \item A \term{right fibration} if it is in $\RLP\{\l^i[n]\to\d[n]\mid 0<i\le n\}$.
    \end{enumerate}
}

An inner fibration is what we generally want a ``good functor'' between $\infty$-categories look like, since for every functor $F:\catC\to\catD$
between ordinary categories, $\N(F)$ is an inner fibration. On the other hand, left and right fibrations come from the previous section,
and the reader may verify that a functor $F:\catC\to\catD$ between ordinary categories is a left (or right) fibration if and only if
$\N(F)$ is.

We also shall give the following definition:

\defn{
    A map $i:A\to B$ is called:
    \begin{enumerate}[i)]
        \item An \term{inner anodyne extension} if it is in $\Cof\{\l^i[n]\to\d[n]\mid 0<i<n\}$;
        \item A \term{left anodyne extension} if it is in $\Cof\{\l^i[n]\to\d[n]\mid 0\le i<n\}$;
        \item A \term{right anodyne extension} if it is in $\Cof\{\l^i[n]\to\d[n]\mid 0<i\le n\}$.
    \end{enumerate}
}

We will first study the stability properties of such maps under certain constructions.

\lem{
    Suppose $f:A\to A',g:B\to B'$ are cofibrations of simplicial sets. Suppose that $f$ is right anodyne or $g$ is left anodyne.
    Then the map $$(A\star B')\amalg_{A\star B}(A'\star B)\to A'\star B'$$ is inner anodyne.
}

\lem{
    Suppose $f:A\to A',g:B\to B'$ are cofibrations of simplicial sets. Suppose that $f$ is anodyne.
    Then the pushout product $$(A\star B')\amalg_{A\star B}(A'\star B)\to A'\star B'$$ is left anodyne.
    (The dual statement is left for the reader.)
}

We now deduce the following proposition.

\prop{
    Suppose $i:A\to B$ is a cofibration of simplicial sets, $p:B\to X$ is a map of simplicial sets, and $q:X\to S$
    is an inner fibration. Define $r=qp,r_0=ri,p_0=pi$. Then:
    \begin{enumerate}[i)]
        \item The induced map $$X_{p/}\to X_{p_0/}\times_{S_{r_0/}}S_{r/}$$ is a left fibration;
        \item If $q$ is a right fibration then the induced map $$X_{p/}\to X_{p_0/}\times_{S_{r_0/}}S_{r/}$$ is a Kan fibration;
        \item If $i$ is right anodyne then the induced map $$X_{p/}\to X_{p_0/}\times_{S_{r_0/}}S_{r/}$$ is a trivial fibration;
        \item If $i$ is anodyne and $q$ is a left fibration then the induced map $$X_{p/}\to X_{p_0/}\times_{S_{r_0/}}S_{r/}$$ is a trivial fibration.
    \end{enumerate} 
}

We may now deduce the following proposition mentioned in \ref{seci}:

\cor{[Part 1 of Proposition \ref{tagl}]
    Suppose $\catC$ is an $\infty$-category, $p:K\to\catC$ is a diagram in $\catC$. Then the projection $\catC_{p/}\to\catC$ 
    is a left fibration. In particular, $\catC_{p/}$ is an $\infty$-category.
}

We furthermore have the following propositions:

\prop{
    We define $$\begin{aligned}I&=\{\l^i[n]\to\d[n]\mid 0\le i<n\},\\
    I'&=\{(\p\d[n]\times\d[1])\amalg_{\p\d[n]\times\{0\}}(\d[n]\times\{0\})\to\d[n]\times\d[1]\mid n\ge 0\},\\
    I''&=\{(A\times\d[1])\amalg_{A\times\{0\}}(B\times\{0\})\to B\times\d[1]\mid n\ge 0,A\to B\text{ is a cofibration}\}.\end{aligned}$$
    Then $\Cof(I)=\Cof(I')=\Cof(I'')$.
}

\cor{
    The pushout product of any cofibration and left anodyne extension is a left anodyne extension.
}

\cor{
    For any cofibration $i$ and left fibration $p$, $p_{\squ}^i$ is a left fibration. If moreover $i$ is left anodyne,
    then $p$ is a trivial fibration.
}

\cor{
    A map $p:X\to S$ is a left fibration if and only if the induced map $$X^{\d[1]}\to X^{\{0\}}\times_{S^{\{0\}}}S^{\d[1]}$$
    is a trivial fibration.
}

\prop{
    We define $$\begin{aligned}I&=\{\l^i[n]\to\d[n]\mid 0<i<n\},\\
    I'&=\{(\p\d[n]\times\d[2])\amalg_{\p\d[n]\times\l^1[2]}(\d[n]\times\l^1[2])\to\d[n]\times\d[2]\mid n\ge 0\},\\
    I''&=\{(A\times\d[2])\amalg_{A\times\l^1[2]}(B\times\l^1[2])\to B\times\d[2]\mid n\ge 0,A\to B\text{ is a cofibration}\}.\end{aligned}$$
    Then $\Cof(I)=\Cof(I')=\Cof(I'')$.
}

\cor{
    The pushout product of any cofibration and inner anodyne extension is an inner anodyne extension.
}

\cor{
    For any cofibration $i$ and innder fibration $p$, $p_{\squ}^i$ is an inner fibration. If moreover $i$ is inner anodyne,
    then $p$ is a trivial fibration.
}

\cor{
    A map $p:X\to S$ is an inner fibration if and only if the induced map $$X^{\d[2]}\to X^{\l^1[2]}\times_{S^{\l^1[2]}}S^{\d[2]}$$
    is a trivial fibration.
}

We next establish some more properties enjoyed by left fibrations:

\prop{
    Suppose $p:\catC\to\catD$ is a left fibration, $f$ is a morphism in $\catC$ such that $p(f)$ is an equivalence.
    Then $f$ is an equivalence.
}

\prop{
    Suppose $p:\catC\to\catD$ is a left fibration, $y\in\catC$, $\overline{f}:\overline{x}\to p(y)$ is an equivalence in $\catD$.
    Then there exists a morphism $f$ in $\catC$ such that $p(f)=\overline f$.
}

Using these propositions, one can now deduce Proposition \ref{tagm} by noticing that 
$\l^0[n]\cong(\{0\}\star\d[n-2])\amalg_{\{0\}\star\p\d[n-2]}(\d[1]\star\p\d[n-2])$.