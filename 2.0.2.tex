\label{secb}

Although the study of model categories given in Chapter 1 are rather useful, in this chapter we still need to use some more properties
of model categories that we have not discussed yet. Most of these properties are related with enriched categories and presentable categories,
that have been discussed in the previous section. We will discuss these properties in this section.

We start with a lifting criterion:

\prop{
    Suppose $\catC$ is a model category, $i:A\to B$ is a cofibration between cofibrant objects, $X$ is a fibrant object,
    and $f:A\to X$ is a map. If there exists a map $\bar{g}\in[B,X]$ such that $[f]=\bar g[i]$ in $\Ho\catC$, then
    there exists a map $g\in\Hom_\catC(B,X)$ such that $f=gi$.
}

Next we discuss about the stability of weak equivalences under pushouts and pullbacks:

\defn{
    Suppose $\catC$ is a model category. $\catC$ is called \term{left proper}, if weak equivalences is stable under pushouts by cofibrations.
    Dually $\catC$ is called \term{right proper}, if weak equivalences is stable under pullback by fibrations.
}

The following lemma and corollary can be used to deduce whether a category is left proper or not:

\lem{
    In a model category, weak equivalences between cofibrant objects is stable under pushouts by cofibrations.
}

\cor{
    A model category with all objects cofibrant is left proper.
}

The following proposition can be deduced from the previous chapter:

\prop{
    $\cat{SSet}$, with the standard model structure, is right proper.
}

The study of left proper model categories is also related with homotopy pushouts:

\prop{
    Suppose $$\xymatrix{A\ar[r]^i\ar[d]_j&B\ar[d]\\C\ar[r]&B\amalg_AC}$$ is a pushout square in a model category $\catC$.
    If:
    \begin{enumerate}[i)]
        \item Either $A,B$ are cofibrant and $j$ is a cofibration;
        \item Or $\catC$ is left proper and $i$ is a cofibration;
    \end{enumerate}
    Then the square is a homotopy pushout square.
}

The next concept is the study of combinatorial model categories:

\defn{
    A model category is called \term{combinatorial}, if it is presentable and cofibrantly generated.
}

Combinatorial model categories enjoy certain accessibility properties:

\prop{
    Suppose $\catC$ is a combinatorial model category. The full subcategories $\Fib,\W,\W\cap\Fib$ of $\catC^{[1]}$ are accessible.
}

To the converse, given certain accessibilities, one can construct examples of combinatorial model categories:

\lem{[$\dagger$]
    Suppose $\catC$ is a presentable category, $\Cof$ and $\W$ are two classes of morphisms in $\catC$. Suppose more that:
    \begin{enumerate}[i)]
        \item $\Cof$ is a weakly saturated class of morphisms in $\catC$ and is \term{of small generation};
        that is, there exists a set of morphisms $I$ such that $\Cof=\Cof(I)$;
        \item $\Cof\cap\W$ is a weakly saturated class of morphisms;
        \item $\W$ is an accessible full subcategory of $\catC^{[1]}$;
        \item $\W$ has the 2-out-of-3 property.
    \end{enumerate}
    Then $\Cof\cap\W$ is of small generation.
}

\prop{\label{tagq}
    Suppose $\catC$ is a presentable category, $\Cof$ and $\W$ are two classes of morphisms in $\catC$. Suppose more that:
    \begin{enumerate}[i)]
        \item $\Cof$ is a weakly saturated class of morphisms in $\catC$ and is \term{of small generation};
        that is, there exists a set of morphisms $I$ such that $\Cof=\Cof(I)$;
        \item $\Cof\cap\W$ is a weakly saturated class of morphisms;
        \item $\W$ is an accessible full subcategory of $\catC^{[1]}$;
        \item $\W$ has the 2-out-of-3 property;
        \item $\RLP(\Cof)\subseteq\W$.
    \end{enumerate}
    Then $\catC$ admits a combinatorial model structure, with cofibrations being $\Cof$ and weak equivalences being $\W$.
}

\cor{
    Suppose $\catC$ is a presentable model category such that $\Cof$, as a weakly saturated class of morphisms, is of small generation.
    Then the following are equivalent:
    \begin{itemize}
        \item $\catC$ is combinatorial;
        \item $\W$ is an accessible full subcategory of $\catC^{[1]}$.
    \end{itemize}
}

We now present another version of the above proposition that we will frequently use in the following section:

\defn{
    Suppose $\catC$ is a presentable category. A class of morphisms $\W$ in $\catC$ is called \term{perfect},
    if it contains every isomorphism, satisfies the 2-out-of-3 property, and is an $\omega$-accessible full subcatgeory of $\catC^{[1]}$.
}

For example, the class of all isomorphisms in $\catC$ is perfect, whenever $\catC$ is presentable. Also, the class of weak homotopy equivalences
in $\cat{SSet}$ is perfect.

\prop{
    Suppose $F:\catC\to\catD$ is a functor between presentable categories which preserves filtered colimits and 
    $\W$ is a perfect class of morphisms in $\catD$. Then $F^{-1}(\W)$ is a perfect class of morphisms in $\catC$.
}

\prop{\label{tagp}
    Suppose $\catC$ is a presentable category, $\W$ is a class of morphisms in $\catC$, $I$ is a set of morphisms in $\catC$. Suppose more that:
    \begin{enumerate}[i)]
        \item $\W$ is a perfect class of morphisms;
        \item $\W$ is stble under pushouts by pushouts of elements in $I$;
        \item $\RLP(I)\subseteq\W$.
    \end{enumerate}
    Then $\catC$ admits a left proper combinatorial model structure, with cofibrations being $\Cof(I)$ and weak equivalences being $\W$.
}

The next concept is the theory of diagram categories and homotopy limits and colimits. We first make the following definitions:

\defn{
    Suppose $\catI$ is a small category and $\catC$ is a model category. We define a map $\alpha\in\catC^\catI$ to be:
    \begin{enumerate}[i)]
        \item An \term{injective cofibration} if it is a degreewise cofibration;
        \item A \term{projective fibration} if it is a degreewise fibration;
        \item A \term{weak equivalence} if it is a degreewise weak equivalence;
        \item A \term{projective cofibration} if it has LLP with respect to every map that is both a projective fibration and a weak equivalence;
        \item An \term{injective fibration} if it has RLP with respect to every map that is both a injective cofibration and a weak equivalence.
    \end{enumerate}
}

We now state the following proposition:

\prop{
    Suppose $\catC$ is a combinatorial model category, $\catI$ is a small category. Then the projective cofibrations, projective fibrations
    and weak equivalences form a combinatorial model structure on $\catC^\catI$, which will be called the \term{projective model structure},
    denoted $(\catC^\catI,\text{proj})$; the injective cofibrations, injective fibrations
    and weak equivalences form a combinatorial model structure on $\catC^\catI$, which will be called the \term{injective model structure},
    denoted $(\catC^\catI,\text{inj})$.
}

The projective model structure is easy: One can even write down the generating cofibrations and the generating trivial cofibrations.
For any $I\in\catI$ and $C\in\catC$ we define the functor $F_C^I:\catI\to\catC$ by $$F_C^I(I')=\coprod_{\alpha\in\Hom_\catI(I,I')}C.$$
Then, if $I_0$ is the set of generating cofibrations in $\catC$, $\{F_C^I\to F_{C'}^I\mid C\to C'\in I_0,I\in\catI\}$ is the 
set of generating cofibrations in $\catC^\catI$, and similar for the generating trivial cofibrations. The injective model structure 
is more complicated and requires Proposition \ref{tagq}. The following lemma is also needed:

\lem{
    Suppose $\catC$ is a presentable category and $\catI$ be a small category. Suppose $S_0$ is a weakly saturated class of morphisms in
    $\catC$ that is of small generation. Let $S$ be the class of morphisms in $\catC^\catI$ that degreewise belongs to $S_0$.
    Then $S$ is a weakly saturated class of morphisms that is of small generation.
}

We note that if $\catC$ is left or right proper, then the projective model structure and the injective model struture on $\catC^\catI$ 
are left or right proper, respetively. Also note that every projective cofibration is an injective cofibration, and every
injective fibration is a projective fibration.

The projective model structure and the injective model struture is fonctorial in $\catC$. Namely, if $(F,G):\catC\to\catD$ is a 
Quillem adjunction between combinatorial model categories, then $$(F^\catI,G^\catI):(\catC^\catI,\text{proj})\to(\catD^\catI,\text{proj})$$
and $$(F^\catI,G^\catI):(\catC^\catI,\text{inj})\to(\catD^\catI,\text{inj})$$ are Quillen adjunctions, that are Quillen equivalences 
if $(F,G)$ is one.

%We now focus on enriched homotopy theory. The 

\tbc

% To be added: Enriched homotopy theory