\defn{
    Suppose that $\catC,\catD,\catE$ are model categories. An adjunction of 2 variables $(\ox,\sHom_l,\sHom_r,\vp_l,\vp_r):\catC\times\catD\to\catE$ 
    is called a \term{Quillen adjunction of 2 variables} if for any $f:A\to B\in\Cof_\catC,g:C\to D\in\Cof_\catD$, the \term{pushout product}
    $$f\squ g:A\ox D\amalg_{A\ox C}B\ox C\to B\ox D$$ is a cofibration, and is trivial if $f$ or $g$ is. If so $\ox$ is called a \term{Quillen bifunctor}.
}

\lem{
    Suppose $(\ox,\sHom_l,\sHom_r):\catC\times\catD\to\catE$ is an adjunction of 2 variables between model categories $\catC,\catD,\catE$. 
    Then the following statements are equivalent:
    \begin{enumerate}[i)]
    \item $\ox$ is a Quillen bifunctor;
    \item For any $f:A\to B\in\Cof_\catC,p:X\to Y\in\Fib_\catE$, the map $$\sHom_{\squ,l}(f,p):
    \sHom_l(A,X)\times_{\sHom_l(B,X)}\sHom_l(B,Y)\to\sHom_l(A,Y)$$ is a fibration, and is trivial if $f$ or $p$ is;
    \item For any $g:C\to D\in\Cof_\catD,p:X\to Y\in\Fib_\catE$, the map $$\sHom_{\squ,r}(g,p):
    \sHom_r(C,X)\times_{\sHom_r(D,X)}\sHom_r(D,Y)\to\sHom_r(C,Y)$$ is a fibration, and is trivial if $g$ or $p$ is.
    \end{enumerate}
}

\lem{
    Suppose $(\ox,\sHom_l,\sHom_r):\catC\times\catD\to\catE$ is a Quillen adjunction of 2 variables between model categories $\catC,\catD,\catE$. 
    If $C$ is cofibrant in $\catC$, then $(C\ox-,\sHom_l(C,-)):\catD\to\catE$ is a Quillen adjunction. If $D$ is cofibrant in $\catD$, 
    then $(-\ox D,\sHom_r(D,-)):\catC\to\catE$ is a Quillen adjunction. If $E$ is fibrant in $\catE$, then $(\sHom_l(-,E),\sHom_r(-,E)):\catC\to\catD^\op$
    is a Quillen adjunction.
}

\lem{
    Suppose $(\ox,\sHom_l,\sHom_r):\catC\times\catD\to\catE$ is an adjunction of 2 variables between categories $\catC,\catD,\catE$. 
    Suppose $I,I'$ are collections of maps in $\catC,\catD$, respectively. Then $\Cof(I)\squ\Cof(I')\subseteq\Cof(I\squ I')$.
}

\cor{
    Suppose $(\ox,\sHom_l,\sHom_r):\catC\times\catD\to\catE$ is an adjunction of 2 variables between model categories $\catC,\catD,\catE$. 
    Suppose more that $\catC,\catD$ are cofibrantly generated model categories, with generating cofibrations $I,I'$, generating trivial cofibrations 
    $J,J'$, respectively. Then $\ox$ is a Quillen bifunctor, if and only if $I\squ I'\subseteq\Cof_\catE,I\squ J'\subseteq\W_\catE\cap\Cof_\catE,
    J\squ I'\subseteq\W_\catE\cap\Cof_\catE$.
}

\rmk{
    If $\catC$ is a closed monoidal category, then $(\squ,\sHom_{\squ,l},\sHom_{\squ,r})$ forms a closed monoidal structure on $\catC^{[1]}$.
}

\defn{
    A \term{monoidal model category} is a model category together with a closed monoidal structure, such that:
    \begin{enumerate}[i)]
    \item The monoidal structure bifunctor $\ox$ is a Quillen bifunctor;
    \item For any $X$ cofibrant, the natural transformation $QS\ox X\to S\ox X$ is a weak equivalence;
    \item For any $X$ cofibrant, the natural transformation $X\ox QS\to X\ox S$ is a weak equivalence.
    \end{enumerate}
    A \term{symmetric monoidal model category} is a monoidal model category that is also a symmetric monoidal category.
}

\lem{
    Suppose $\catC$ is a model category that is also a closed monoidal category. Then for any $X$ cofibrant, the natural transformation 
    $QS\ox X\to X$ is a weak equivalence, if and only if for any fibrant $X$, the natural transformation $X\to\sHom_l(QS,X)$ is a weak equivalence; 
    For any $X$ cofibrant, the natural transformation $X\ox QS\to X$ is a weak equivalence, if and only if for any fibrant $X$, 
    the natural transformation $X\to\sHom_r(QS,X)$ is a weak equivalence.
}

\prop{\label{tagd}
    Suppose $\catC$ is a monoidal model category with the unit being the terminal object $*$. Suppose more that $*$ is cofibrant. 
    Then $\catC_*$ is a monoidal model category, with $X\wedge Y$ defined to be the pushout
    $$\xymatrix @C=80pt{
    X\amalg Y\ar[r]^{(\1_X\ox w)\amalg(v\ox\1_Y)}\ar[d]&X\ox Y\ar[d]\\
    {*}\ar[r]&X\wedge Y
    }$$
    for any $(X,v),(Y,w)\in\catC_*$, with the unit of $\wedge$ being $*_+$; $\sHom_{l,*}(X,Y)$ defined to be the pullback 
    $$\xymatrix @C=80pt{
    \sHom_{l,*}(X,Y)\ar[r]\ar[d]& \sHom_l(X,Y)\ar[d]^{\sHom_l(v,Y)}\\
    {*}\ar[r]^{\sHom_l(X,w)}& \sHom_l(*,Y)
    }$$ 
    for any $(X,v),(Y,w)\in\catC_*$, with the basepoint of $\sHom_{l,*}(X,Y)$ being induced by the maps $*\to *$ and 
    $$\vp_l(X\ox*\cong X\to*\xrightarrow{w}Y):*\to\sHom_l(X,Y);$$ $\sHom_{r,*}(X,Y)$ defined similarly. If $\catC$ is a 
    symmetric monoidal model category, so is $\catC_*$.
}

\prop{
    $\cat{SSet}$ is a symmetric monoidal model category, with $\ox=\times,\sHom_l=\sHom_r=-^-$. $\cat{SSet}_*$ is a symmetric monoidal model category.
}

\prop{
    $\cat{K}$ is a symmetric monoidal model category, with $\ox=\times,\sHom_l=\sHom_r=\sHom$. $\cat{K}_*$ is a symmetric monoidal model category.
}

\rmk{
    $\cat{Top}$ is not a monoidal model category.
}

\prop{\label{tagc}
    If $R$ is a commutative ring, then $\cat{Ch}_R$ with the projective model structure is a symmetric monoidal model category, where $X\ox Y$ 
    is defined to be $$\begin{aligned}(X\ox Y)_n&=\bigoplus_{k\in\mathbb{Z}}X_k\ox_RY_{n-k},\\
    d_n(x\ox y)&=(dx)\ox y+(-1)^{\abs{x}}x\ox(dy);\end{aligned}$$ $\sHom(X,Y):=\sHom_l(X,Y)=\sHom_r(X,Y)$ is defined to be
    $$\begin{aligned}\sHom(X,Y)_n&=\prod_{k\in\mathbb{Z}}\Hom(X_k,Y_{n+k}),\\(d_n(f))(x)&=d(f(x))+(-1)^{n+1}f(dx);\end{aligned}$$ 
    The associativity isomorphism $T$ is defined to be $T(x\ox y)=(-1)^{\abs{x}\abs{y}}y\ox x$.
}

\prop{
    If $R$ is a ring, then the functor $\ox:\cat{Ch}_{R^\op}\times\cat{Ch}_R\to\cat{Ch}_{\mathbb{Z}}$ defined in Proposition \ref{tagc} 
    is a Quillen bifunctor.
}

\rmk{
    $\cat{Ch}_R$ with the injective model structure is not a monoidal model category. For example, in $\cat{Ch}_\mathbb{Z}$,
    $$(S^0(\mathbb{Z})\to S^0(\mathbb{Q}))\squ(0\to S^0(\mathbb{Z}/2\mathbb{Z}))=(S^0(\mathbb{Z}/2\mathbb{Z})\to 0),$$ which is certainly not injective.
}

\defn{
    Suppose $\catC,\catD$ are monoidal model categories. A \term{monoidal Quillen adjunction} from $\catC$ to $\catD$ is a Quillen adjunction $(F,G)$ 
    such that $F$ is monoidal, and the map $FQS\to QS$ is a weak equivalence. If so $F$ is called a \term{monoidal Quillen functor}. 
    A \term{symmetric monoidal Quillen functor} is a monoidal Quillen functor that is a symmetric monoidal functor.
}

\lem{
    The composition of two monoidal Quillen functors is a monoidal Quillen functor.
}

\lem{
    Monoidal model categories, monoidal Quillen functors and monoidal natural transformations form a 2-category, which we denote $\catt{MonModel}$.
    Symmetric monoidal model categories, symmetric monoidal Quillen functors and monoidal natural transformations form a 2-category, 
    which we denote $\catt{SymMonModel}$.
}

\eg{
    $-_+:\cat{SSet}\to\cat{SSet}_*,-_+:\cat{K}\to\cat{K}_*$ are monoidal Quillen adjunctions.
}

\eg{
    If $R\to S$ is a ring homomorphism, then the induced Quillen adjunction $\cat{Ch}_R\to\cat{Ch}_S$ is a monoidal Quillen adjunction.
}

\prop{
    $\abs{-}:\cat{SSet}\to\cat{K},\abs{-}:\cat{SSet}_*\to\cat{K}_*$ are monoidal Quillen adjunctions.
}

\defn{
    Suppose $\catC$ is a monoidal model category. A \term{$\catC$-model category} is a model category $\catD$ together with a 
    closed right $\catC$-module structure, such that $\ox:\catD\times\catC\to\catD$ is a Quillen bifunctor, and for any $X$ cofibrant, 
    the natural transformation $X\ox QS\to X\ox S$ is a weak equivalence. A $\cat{SSet}$-model category is called a \term{simplicial model category}.
}

\defn{
    Suppose $\catC$ is a monoidal model category. A \term{$\catC$-Quillen functor} between $\catC$-model categories is a Quillen functor 
    that is also a closed $\catC$-module functor.
}

\eg{
    $\cat{SSet},\cat{SSet}_*,\cat{K},\cat{K}_*$ are simplicial model categories.
}

\eg{
    For any ring $R$, $\cat{Ch}_R$ is a $\cat{Ch}_\mathbb{Z}$-model category.
}

\lem{
    If $\catC$ is a monoidal model category, all $\catC$-model categories, $\catC$-Quillen functors and $\catC$-module natural transformations 
    form a 2-category, which we denote $\catt{Model}_\catC$. Any monoidal Quillen functor between monoidal model categories $\catC$ and $\catD$ 
    induces a forgetful 2-functor from $\catt{Model}_\catD$ to $\catt{Model}_\catC$.
}

\lem{
    If $\catC$ is a pointed monoidal model category, every $\catC$-model category is pointed.
}

\prop{
    Suppose $\catC$ is a monoidal model category with the unit being the terminal object $*$. Suppose more that $*$ is cofibrant. 
    If $\catD$ is a $\catC$-model category, then $\catD_*$ is a $\catC_*$-model category, where the $\catC_*$-module structure 
    is given similar to Proposition \ref{tagd}.
}

\lem{
    If $\catC$ is a symmetric monoidal model category, the duality 2-functor on $\catt{Model}$ and the duality 2-functor on $\catt{Mod}_\catC$ 
    induces a duality 2-functor on $\catt{Model}_\catC$.
}

\defn{
    Suppose $\catC$ is a monoidal model category. A \term{monoidal $\catC$-model category} is a monoidal model category $\catD$ together with 
    a monoidal Quillen functor $\catC\to\catD$. A \term{monoidal $\catC$-Quillen functor} between monoidal $\catC$-model categories 
    is a Quillen functor that is also a closed $\catC$-algebra functor. We have similar definitions
     for a \term{symmetric monoidal $\catC$-model category} and a \term{symmetric monoidal $\catC$-Quillen functor}.
}

\lem{
    If $\catC$ is a monoidal model category, all monoidal $\catC$-model categories, monoidal $\catC$-Quillen functors and $\catC$-algebra 
    natural transformations form a 2-category, which we denote $\catt{MonModel}_\catC$. Any monoidal Quillen functor between 
    monoidal model categories $\catC$ and $\catD$ induces a forgetful 2-functor from $\catt{MonModel}_\catD$ to $\catt{MonModel}_\catC$. 
    We have a similar atatement for the symmetric case.
}

\eg{
    $\cat{K}$ is a symmetric monoidal $\cat{SSet}$-model category.
}

\prop{
    Suppose $(\ox,\sHom_l,\sHom_r,\vp_l,\vp_r):\catC\times\catD\to\catE$ is a Quillen adjunction of 2 variables between model categories 
    $\catC,\catD,\catE$. Then the derived functors defines an adjunction of 2 variables $$(\ox^L,R\sHom_l,R\sHom_r,\Ho\vp_l,\Ho\vp_r):
    \Ho\catC\times\Ho\catD\to\Ho\catE,$$ where $\Ho\vp_l$ is given by
    $$
    \begin{aligned}
    {[C\ox^LD,E]}&=[QC\ox QD,E]\\
    &\cong[QC\ox QD,RE]\\
    &\cong[QD,\sHom_l(QC,RE)]\\
    &\cong[D,\sHom_l(QC,RE)]=[D,(R\sHom_l)(C,E)];
    \end{aligned} 
    $$
    and $\Ho\vp_r$ is given similarly.
}

\thm{\label{tagj}
    The pseudo-2-functor $\Ho:\catt{Model}\to\catt{Cat}_{ad}$ lifts to a pseudo-2-functor $$\Ho:\catt{MonModel}\to\catt{CloMon},$$ 
    where a monoidal model category $$(\catC,\ox,\sHom_l,\sHom_r,\vp_l,\vp_r,a,l,r,S)$$ is mapped to 
    $$(\Ho\catC,\ox^L,R\sHom_l,R\sHom_r,\Ho\vp_l,\Ho\vp_r,a,l,r,QS),$$ where $a$ is given by
    $$
    \begin{aligned}
    (X\ox^LY)\ox^LZ&=Q(QX\ox QY)\ox QZ\\
    &\cong(QX\ox QY)\ox QZ\\
    &\cong QX\ox(QY\ox QZ)\\
    &\cong QX\ox Q(QY\ox QZ)=X\ox^L(Y\ox^LZ),
    \end{aligned} 
    $$
    $l$ is given by $$S\ox^LX=QS\ox QX\cong S\ox QX\cong QX\cong X,$$ $r$ is given similarly; a monoidal Quillen adjunction 
    $$(F,G,\vp,\mu,\alpha)$$ is mapped to $$(LF,RG,\Ho\vp,\mu,\alpha),$$ where $\mu$ is given by 
    $$
    \begin{aligned}
    (LF)X\ox^L(LF)Y&=QFQX\ox QFQY\\
    &\cong FQX\ox FQY\\
    &\cong F(QX\ox QY)\\
    &\cong FQ(QX\ox QY)=(LF)(X\ox^LY),
    \end{aligned} 
    $$
    $\alpha$ is given by $$(LF)S=FQS\cong FS\cong S;$$ a monoidal natural transformation $\tau$ is mapped to $L\tau$.
}

\thm{
    The pseudo-2-functor $\Ho:\catt{MonModel}\to\catt{CloMon}$ lifts to a pseudo-2-functor $$\Ho:\catt{SymMonModel}\to\catt{CloSymMon},$$ 
    where the natural isomorphism $T$ is given by $$X\ox^LY=QX\ox QY\cong QY\ox QX=Y\ox^LX.$$
}

\thm{
    Suppose $\catC$ is a monoidal model category, then the pseudo-2-functor $\Ho:\catt{Model}\to\catt{Cat}_{ad}$ lifts to a pseudo-2-functor 
    $$\Ho:\catt{Model}_\catC\to\catt{CloMod}_{\Ho\catC},$$ where the natural isomorphisms $a,r,\mu$ are defined similarly to that of 
    Theorem \ref{tagj}; lifts to a pseudo-2-functor $$\Ho:\catt{MonModel}_\catC\to\catt{CloAlg}_{\Ho\catC},$$ where $(\catD,i)$ is mapped to 
    $(\Ho\catD,Li)$, $(F,\rho)$ is mapped to $(LF,L\rho)$; and lifts to a pseudo-2-functor $$\Ho:\catt{SymMonModel}_\catC\to\catt{CloSymAlg}_{\Ho\catC}$$ 
    if $\catC$ is symmetric. In the latter case $$\Ho\circ\9-^\op\0=\9-^\op\0\circ\Ho:\catt{SymMonModel}_\catC\to\catt{CloSymAlg}_{\Ho\catC}.$$
}

\prop{
    $L\abs{-}:\Ho\cat{SSet}\to\Ho\cat{K}$ is a equivalence between closed symmetric monoidal categories. $L\abs{-}_*:\Ho\cat{SSet}_*\to\Ho\cat{K}_*$ 
    is a equivalence between closed symmetric monoidal categories.
}