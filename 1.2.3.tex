\lem{
    If $R$ is a ring, all objects in the category $\cat{Ch}_R$ of chain complexes of $R$-modules are small. All bounded complexes of
    finitely presented $R$-modules are finite.
}

\defn{
    Suppose $R$ is a ring, define the functor $S^n:\cat{Mod}_R\to\cat{Ch}_R$ with $S^n(M)_n=M$ and $S^n(M)_m=0$ if $n\ne m$; 
    define the functor $D^n:\cat{Mod}_R\to\cat{Ch}_R$ with $D^n(M)_n=D^n(M)_{n-1}=M$ and $D^n(M)_m=0$ if $n\ne m,m-1$, and $d_n=\1_M$. 
    Denote $S^n(R)$ by $S^n$ and $D^n(R)$ by $D^n$.
}

\lem{
    $S^n$ is left adjoint to $Z_n:X\to\ker d_n$, and $D^n$ is left adjoint to $-_n$.
}

\defn{
    Suppose $R$ is a ring. Define the set $I=\{S^{n-1}\to D^n|n\in\mathbb{Z}\}$, and the set $J=\{0\to D^n|n\in\mathbb{Z}\}$. 
    Define $\Cof=\Cof(I)$, $\Fib=\RLP(J)$, and $\W$ to be all quasi-isomorphisms.
}

\lem{
    A map $p\in\cat{Ch}_R$ is a fibration if and only if it is surjective.
}

\prop{
    A map $p\in\cat{Ch}_R$ is in $\RLP(I)$ if and only if it is a surjective quasi-isomorphism. In particular, $\RLP(I)=\W\cap\Fib$.
}

\lem{
    Any cofibrant chain complex is degreewise projective. Any bounded below degreewise projective chain complex is cofibrant.
}

\rmk{
    Not every degreewise projective chain complex is cofibrant. For example take $R=\Bbbk[x]/(x^2)$ where $\Bbbk$ is a field, and take $A$
    to be the complex with $A_n=R$, and $d_n$ to be the multiplication by $x$ for any $n$, then $A$ is degreewise projective but not cofibrant.
}

\lem{
    If $C$ is a cofibrant chain complex and $K$ is acyclic, then any map from $C$ to $K$ is nullhomotopic.
}

\prop{
    A map $i\in\cat{Ch}_R$ is a cofibration if and only if it is injective with cofibrant cokernel.
}

\prop{
    A map $i\in\cat{Ch}_R$ is in $\Cof(J)$ if and only if it is injective with projective cokernel. In particular, $\Cof(J)\subseteq\W\cap\Cof$.
}

\thm{
    $\cat{Ch}_R$ is a cofibrantly generated model category with $I$ being its generating cofibrations, $J$ being its generating trivial cofibrations
    and weak equivalences being quasi-isomorphisms. This model structure is called the \term{projective model structure}.
}

\cor{
    Any object in $\cat{Ch}_R$ is projectively fibrant. An object in $\cat{Ch}_R$ is projective if and only if it is acyclic and cofibrant. 
    If two maps in $\cat{Ch}_R$ are chain homotopic then they are right homotopic.
}

\defn{
    Define $\Cof'$ to be all injections in $\cat{Ch}_R$. Define $\Fib'$ to be $\RLP(\W\cap\Cof')$, and call elements in $\Fib'$ \term{injective fibrations}.
    For any $X\in\cat{Ch}_R$ define $|X|=\abs{\bigcup_nX_n}$. Define $\gamma=\sup\{|R|,|\mathbb{Z}|\}$. Define $I'$ to be the set containing a map
    from each isomorphism class of injections whose codomain has cardinality no larger than $\gamma$. Define $J'=I'\cap\W$.
}

\prop{
    $\Cof(I')=\Cof'$. Moreover a map $p\in\cat{Ch}_R$ is in $\RLP(I')$ if and only if it is surjective with injective kernel.
}

\cor{
    Any injective object in $\cat{Ch}_R$ is injectively fibrant and acyclic. Moreover $\RLP(I')\subseteq\W\cap\Fib'$.
}

\lem{
    Any injectively fibrant chain complex is degreewise injective. Any bounded above degreewise injective chain complex is injectively fibrant.
}

\rmk{
    Not every degreewise projective chain complex is injectively fibrant. For example take $R=\Bbbk[x]/(x^2)$ where $\Bbbk$ is a field, and take $A$
    to be the complex with $A_n=R$, and $d_n$ to be the multiplication by $x$ for any $n$, then $A$ is degreewise injective but not injectively fibrant.
}

\lem{
    If $C$ is an acyclic chain complex and $K$ is injectively fibrant, then any map from $C$ to $K$ is nullhomotopic.
}

\prop{
    A map $p\in\cat{Ch}_R$ is in $\Fib'$ if and only if it is surjective with injectively fibrant kernel.
}

\lem{
    Suppose $i:A\to B$ is a map in $\cat{Ch}_R$ such that $i\in\Cof'\cap\W$. If $C$ is a subcomplex of $B$ with $|C|\le\gamma$, 
    then there exists a subcomplex $D$ of $B$ containing $C$ such that $i:D\cap A\to D$ is in $\W$.
}

\prop{
    $\Cof(J')=\Cof'\cap\W$. $\RLP(J')=\Fib'$.
}

\thm{
    $\cat{Ch}_R$ is a cofibrantly generated model category with $I'$ being its generating cofibrations, $J'$ being its generating trivial cofibrations
    and weak equivalences being quasi-isomorphisms. This model structure is called the \term{injective model structure}.
}

\cor{
    Any object in $\cat{Ch}_R$ is injectively cofibrant. An object in $\cat{Ch}_R$ is injective if and only if it is acyclic and injectively fibrant.
}

\prop{
    The identity functor from the projective model structure to the injective model structure is a Quillen equivalence.
}

\prop{
    For any ring homomorphism $R\to S$, the adjunction $(\mathrm{Ind},\mathrm{Res}):\cat{Ch}_R\to\cat{Ch}_S$ is a Quillen adjunction between
    the projective model structures, it is a Quillen equivalence if and only if the homomorphism is an isomorphism.
}

\prop{
    For any ring homomorphism $R\to S$, the adjunction $(\mathrm{Ind},\mathrm{Res}):\cat{Ch}_R\to\cat{Ch}_S$ is a Quillen adjunction between 
    the injective model structures if and only if $S$ is a flat $R$-module, it is a Quillen equivalence if and only if the homomorphism is an isomorphism.
}

\prop{
    $[S^nM,S^0N]\cong\operatorname{Ext}_R^n(M,N)$ for any $R$-module $M,N$.
}