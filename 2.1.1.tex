We use some simple examples to give the motivation of $\infty$-categories.

\eg{\label{tagk}
    Suppose $X$ is a topological space. The \term{fundamental groupoid} of $X$, denoted $\Pi_{\le 1}(X)$, is the category with:
    \begin{itemize}
        \item objects being all points in $X$; and
        \item $\Hom_{\Pi_{\le 1}(X)}(a,b)$ being the homotopy classes of all paths from $a$ to $b$.
    \end{itemize}
    This is some definition that we shall already have seen in algebraic topology. This gives rise to a functor $\Pi_{\le 1}:\cat{Top}\to\cat{Gpd}$.
    Moreover, for any map $f:X\to Y$ of topological spaces, $\Pi_{\le 1}(f)$ is an equivalence of categories, if and only if 
    it is fully faithful and essentially surjective, where the first assertion indicates that $\pi_1(f)$ is a bijection, and the second assertion
    indicates that $\pi_0(f)$ is a bijection. 

    However, a topological space, contains far more information than its 0th and 1st homotopy groups. There also not only exist
    homotopies between paths, there also exist homotopies between homotopies, and homotopies between the homotopies between homotopies, etc.
    How shall we encode such higher structures? We may think of a ``category'', with not only morphisms between objects, but also morphisms
    between morphisms, and morphisms between the morphisms between morphisms, etc. Specifically, we would like to define $\Pi(X)$ as follows:
    \begin{itemize}
        \item Objects being all points $a,b,\cdots$ in $X$; 
        \item $\Hom_{\Pi(X)}(a,b)$ being all paths $\gamma,\delta,\cdots$ from $a$ to $b$;
        \item $\Hom_{\Pi(X)}(\gamma,\delta)$ being all homotopies $H,K,\cdots$ from $\gamma$ to $\delta$;
        \item $\cdots$
    \end{itemize}
    And we would like to obtain the property that for any map $f:X\to Y$ of topological spaces, $\Pi(f)$ is an ``equivalence'' of
    ``categories'', if and only if $f$ induces isomorphisms on all homotopy groups; that is, $f$ is a weak homotopy equivalence.
    The theory of $\infty$-categories are introduced to make this thought precise and to study their properties.
}

\eg{
    We recall homotopies between maps in model categories: Given a model category $\catC$ and cofibrant-fibrant objects $A,B$,
    maps $f,g:A\to B$, a homotopy from $f$ to $g$ is equivalent to a map $h:A\ox\d[1]\to B$, such that $h|_{A\ox0}=f,h|_{A\ox1}=g$.
    However, cosimplicial frames not only allow us to construct the cylinder object of $A$, but also higher structures.
    For example, a map $A\ox\d[2]\to B$ may be viewed as a homotopy between the three homotopies given by its restriction on
    $A\ox\p\d[2]$, a map $A\ox\d[3]\to B$ may be viewed as a homotopy between the four homotopies between homotopies
    given by its restriction on $A\ox\p\d[3]$. The theory of $\infty$-categories gives us a tool on how to 
    formalize this kind of higher structures, and to study their properties.
}

Notice, that in these two examples, higher morphisms are ``invertible'' (homotopies are invertible, at least up to higher homotopies).
This also coincides with most usages of $\infty$-categories, as in most applications we may see. Therefore, we only talk about
$\infty$-categories, whose ``$k$-morphisms'' are invertible whenever $k\ge 2$. \footnote{In higher category theory, we use 
the term $(\infty,n)$-category to indicate an $\infty$-category whose $k$-morphisms are invertible whenever $k\ge n+1$. In other words,
what we are focusing about is $(\infty,1)$-categories.}

So how shall we define them? The most na\"ive approach is to define them as what we did in Section \ref{seca}. However, notice one thing: 
in both examples, associativity and unit axioms fails to be strict, but only hold up to homotopy, i.e. higher morphisms. To fix this issue,
one must add certain extra diagrams to give the correct notion of higher categories, rather than the ones given in Section \ref{seca}.
(The reason we introduce them in the first chapter is to make the language we use concise. However, pseudo-2-functors, DO give an approach to
functors between $\infty$-categories, as they keep the homotopy-coherent ideas. From now on, we will call that kind of 2-category
introduced in Section \ref{seca} \term{strict 2-categories}.) On the other hand, even the explicit definition of a 3-category 
is EXTREMELY complicated, and things are just getting worse when we pass to 4-categories and beyond.

Fortunately, we have an alternative approach to the theory of $\infty$-categories.

From our na\"ive thought, if we take the ``Hom-thing'' between two object in an $\infty$-category, it will result in an $(\infty,0)$-category.
In other words, $\infty$-categories are categories enriched over $(\infty,0)$-categories. Consequencely, we only have to make a model for
$(\infty,0)$-categories.

Notice that in Example \ref{tagk}, all morphisms, including 1-morphisms, are invertible. This means that $\Pi(X)$ ``should be
an $(\infty,0)$-category''. On the other hand, there is a general axiom, known as the \term{homotopy hypothesis},
accepted by higher categorists, which is roughly stated as follows:

\begin{itemize}
    \item Every $(\infty,0)$-category has the form $\Pi(X)$ for some topological space $X$. Moreover two $(\infty,0)$-categories 
    are equivalent, if and only if their corresponding topological spaces are weakly equivalent.
\end{itemize}

This axiom ensures that we have a one-to-one correspondence between $(\infty,0)$-categories and equivalences classes of topological spaces,
where two spaces are equivalent if and only if they are weakly equivalent. That is, we use objects 
in $\Ho\cat{Top}\simeq\Ho\cat{K}\simeq\Ho\cat{SSet}$ to describe $(\infty,0)$-categories. Because of such important role they play in
higher category theory, we give them a special name: \term{homotopy types}. We write $\cat{H}$ for the category of homotopy types.

Now we have a schematic view of $(\infty,1)$-categories: they are just $\cat{H}$-enriched categories. But this view is really, really difficult
for us to discover properties enjoyed by $\infty$-categories. We need to give models that are easy to compute. We shall now give three 
different models for us to work with.

\rmk{
    Although we are giving different models, they are actually representing the same thing: $(\infty,1)$-categories. Therefore it is important 
    to show that these models are equivalent, at least up to equivalence between categories. We will devolope how equivalences between categories
    are defined in these different models, and how one model can be translated into another, and show that under equivalences between categories,
    all these models are equivalent.
}

The first model is topological categories. 

\defn{
    A \term{topological category} is a category enriched over $\cat K$, the category of all $k$-spaces.
    \footnote{The reason we use $k$-spaces here, is that $\cat K$ is a closed monoidal category, and the geometric realization plays well 
    with $k$-spaces. From now on, unless otherwise stated, when we say ``topological spaces'', what we actually mean is ``$k$-spaces.''}
}

By the definition of a topological category, if we apply the localization functor $\cat K\to\cat H$, a topological category becomes an
$\cat{H}$-enriched category. We will now specify this localization functor with the symbol $\h$. Therefore, topological categories are indeed
models for $\infty$-categories. 

\defn{
    The \term{homotopy category} of a topological category $\catC$ is $\h\catC$. Two topological categories are said to be 
    \term{(weakly) equivalent} if their homotopy categories are equivalent (viewed as $\cat H$-enriched categories).
}

Note that an equivalence between topological categories, at least in higher category theory, does not mean it is an equivalence
between $\cat K$-enriched categories. This is because essentially what we are dealing with are actually $\cat{H}$-enriched categories.

The second model is simplicial categories.

\defn{
    A \term{simplicial category} is a category enriched over $\cat{SSet}$, the category of all simplicial sets.
    \footnote{Note that a simplicial category is not equivalent to a simplicial object in the category of the categories.}
}

By the definition of a simplicial category, if we apply the localization functor $\cat{SSet}\to\cat H$, a simplicial category becomes an
$\cat{H}$-enriched category. We will, again, specify this localization functor with the symbol $\h$. Therefore, simplicial categories
are also indeed models for $\infty$-categories. 

\defn{
    The \term{homotopy category} of a simplicial category $\catC$ is $\h\catC$. Two simplicial categories are said to be 
    \term{(weakly) equivalent} if their homotopy categories are equivalent (viewed as $\cat H$-enriched categories).
}

We now consider the relations between these two relations. We already know that there exists an adjunction $(\abs{-},\Sing):\cat{SSet}\to\cat K$,
that is a closed symmetric monoidal Quillen equivalence. Consequencely, we get an adjunction 
$(\abs{-},\Sing):\cat{Cat}_{\cat{SSet}}\to\cat{Cat}_{\cat K}$. By Section \ref{secb}, this adjunction is a Quillen equivalence between the
canonical model structures on both sides. Thus both the left and right adjoints preserve equivalences, and both the unit morphisms 
and counit morphisms are equivalences. Thus the two models can be regarded as equivalent models for $(\infty,1)$-categories.

However, these two models have disadvantages. As we have said before, in most $(\infty,1)$-categories (that we actually work with), compositions
of morphisms is associative only up to homotopy. But in a topological category, composition of morphisms are associative up to nose. 
Therefore we must figure out how to transform a homotopy coherent diagram into a strictly commutative diagram. Although this is always
possible (See Section {\color{red}\texttt{[ERROR! SECTION NOT FOUND]}}), it is convenient for us to get a more flexible model 
of $(\infty,1)$-categories.

We note that in some cases, simplicial sets can regarded as categories. We may regard 0-simplices of the simplicial set as objects of the category,
and 1-simplices of the simplicial set as morphisms of the category. For example, the simplicial set $\d[n]$ can be regarded as the category, 
that comes from the partially ordered set with $n+1$ objects $[n]$. This gives rise to the following definition:

\defn{
    We consider the functor $\d\to\cat{Cat},[n]\mapsto[n]$. This gives a cosimplicial object in the category of all categories. By
    Proposition \ref{tagb}, this gives rise to an adjunction $\cat{SSet}\to\cat{Cat}$. We call the right adjoint of this adjunction
    the \term{nerve functor}, denoted $\N$.
}

Now, for any category $\catC$, all $n$-simplices in $\N\catC$ are simply all composable sequences of maps 
$$X_0\xrightarrow{f_0}X_1\xrightarrow{f_1}\cdots\xrightarrow{f_{n-1}}X_n.$$

The nerve functor enjoys the following property:

\prop{
    Suppose $K$ is a simplicial set. The following conditions are equivalent:
    \begin{enumerate}[i)]
        \item There exists a category $\catC$ and an isomorphism of simplicial sets $K\cong \N(\catC)$;
        \item Any map $\l^i[n]\to K$, where $0<i<n$, extends \textnormal{uniquely} to a map $\d[i]\to K$.
    \end{enumerate}
}

Let us take a pause here to see what does this proposition means. What is a map $\d[2]\to \N(\catC)$? it means taking three morphisms
$f:X\to Y,g:Y\to Z$ and $h:X\to Z$ in $\catC$, such that $h=gf$. Similarly, a map $\l^1[2]\to \N(\catC)$ means taking two morphisms 
$f:X\to Y,g:Y\to Z$. Now the meaning for the statement ``any map $\l^1[2]\to \N(\catC)$, extends uniquely to a map $\d[2]\to \N(\catC)$'' means: 
it simply means that every two morphisms that are tip-to-tail are uniquely composable. Similarly, the statement ``any map $\l^i[3]\to \N(\catC)$,
where $i=1,2$, extends uniquely to a map $\d[3]\to \N(\catC)$'' means that composition is strictly associative. Note that we do not
include the cases $i=0$ and $i=n$. This is because ``any map $\l^0[2]\to \N(\catC)$, extends uniquely to a map $\d[2]\to \N(\catC)$''
will indicate that every map has a left inverse, which is only possible whenever $\catC$ is a groupoid.

We have seen that certain simplicial sets can be used to model ordinary categories. We may want to show that certain simplicial sets can 
be used to model $(\infty,1)$-categories. 

Now, suppose $K$ is a simplicial set, that is considered to be an $(\infty,1)$-category. 
What kind of properties would we like $K$ to have? We would like to consider 0-simplices as objects, 1-simplices as morphisms, as usual.
We would also like every two morphisms that are tip-to-tail are composable. This means that any map $\l^1[2]\to K$, extends to a map $\d[2]\to K$.
However, if we still require the composition to be unique, we are on our way back to ordinary categories. In fact, it is unreasonable,
and unnatural, to require composition of two morphisms to be unique. For example, there exists multiple ways to join two paths 
to obtain a longer path in any topological space $X$, meaning that there exists multiple choices of compositions in $\Pi(X)$. Nevertheless,
these choices are homotopic. The condition that different compositions are homotopic, can be recovered by the statement
``any map $\l^i[3]\to K$, where $i=1,2$, extends to a map $\d[3]\to K$''. Similarly, higher homotopy relations can be recovered by the statement
``any map $\l^i[n]\to K$, where $0<i<n$, extends to a map $\d[i]\to K$''. Again, we do not include the cases $i=0$ and $i=n$,
since they are only satisfied when every morphism in $K$ is ``invertible''. 

We now make our discussion precise:

\defn{
    A \term{quasi-category}, is a simplicial set $K$, such that $$(K\to *)\in\RLP(\{\l^i[n]\to\d[n]\mid 0<i<n\}).$$
}

Since this is the mainly used model for $(\infty,1)$-categories, from now on, when we say an $\infty$-category, we refer to a quasi-category,
unless stated otherwise.

\eg{
    The nerve of an ordinary category is an $\infty$-category.
}

\eg{
    Kan complexes are $\infty$-categories. Moreover, they enjoy right lifting property with respect to $\l^i[n]\to\d[n]$ even if $i=0$ or $i=n$,
    meaning that every morphism in a Kan complex is ``invertible''. For this reason, we simply define \term{$\infty$-groupoids} or
    \term{$(\infty,0)$-categories} to be Kan complexes, and say that two $\infty$-groupoids are \term{equivalent} if they are weak equivalent.
    This give a nice realization of the homotopy hypothesis, and the construction $\Pi$ taking a topological space to an $\infty$-groupoid
    can be simply realized by $\Sing$.
}

We now discuss the relations between $\infty$-categories and simplicial categories. In order to show that these two models are equivalent
(in an appropriate sense), we must from a pair of functors $\cat{SSet}\to\cat{Cat}_{\cat{SSet}}$ and $\cat{Cat}_{\cat{SSet}}\to\cat{SSet}$.
As shown in Proposition \ref{tagb}, a cosimplicial simplicial category will give rise to an adjunction $\cat{SSet}\to\cat{Cat}_{\cat{SSet}}$.
In the case of ordinary categories, we gave the cosimplicial category $\d\to\cat{Cat},[n]\mapsto[n]$. This makes the composition of morphisms
in a simplicial set strictly associative, which is what in the case of an ordinary category we want. Consequencely, we would like
to ``thicken the morphism space'' to give rise to the homotopy-coherent idea of composing the morphisms. This gives rise to
the following definition:

\defn{
    We define the functor $\C:\cat{TOSet}\to\cat{Cat}_{\cat{SSet}}$ as follows. For any totally ordered set $J$, define $\C[J]$ to be the
    simplicial category with:
    \begin{itemize}
        \item Objects being all objects of $J$;
        \item For any $i,j\in J$, $\Hom_{\C[J]}(i,j)=\emptyset$ if $i>j$, and $\Hom_{\C[J]}(i,j)=\N(P_{i,j})$ if $i\le j$, where
        $P_{i,j}$ is the partially ordered set $\{I\subseteq J\mid i,j\in I;\forall k\in I,i\le k\le j\}$;
        \item For $i\le j\le k$, the composition law $\Hom_{\C[J]}(i,j)\times\Hom_{\C[J]}(j,k)\to\Hom_{\C[J]}(i,k)$ is induced by the map
        of partially ordered sets $P_{i,j}\times P_{j,k}\to P_{i,k},(I,I')\mapsto I\cup I'$;
    \end{itemize}
    and for any maps $f:J\to J'$ between totally ordered sets, define $\C[f]$ to be the functor with:
    \begin{itemize}
        \item $\C[f](i)=f(i)$ for any object $i\in J$;
        \item For any $i,j\in J$, the map $\Hom_{\C[J]}(i,j)\to \Hom_{\C[J']}(f(i),f(j))$ is induced by the map 
        $P_{i,j}\to P_{f(i),f(j)},I\mapsto f(I)$.
    \end{itemize}
    This functor induces a cosimplicial object in the category of simplicial categories, which further induces an adjunction
    $\cat{SSet}\to\cat{Cat}_{\cat{SSet}}$. We denote this adjunction by $(\C,\N)$, and call the right adjoint the \term{simplicial nerve}
    functor.
}

This is really a complicated definition, so we shall pause again to figure out this definition. Take $\C[\d[3]]$ as example. We have
$$\Hom_{\C[\Delta[3]]}=\9\vcenter{\xymatrix{03\ar[r]\ar[d]\ar[rd]&013\ar[d]\\023\ar[r]&0123}}\0.$$
(Notice that it also contains two nondegenerate $2$-simplices that are not shown.) We notice that each vertex of this 
Hom-space represents a way from $0$ to $3$. In particular, there is a way from $0$ to $3$ that does not pass through any other vertices,
which is the $03$ vertex. We denote the morphism $0\to 3$ represented by the vertex $03$ as $p_{03}$. Now we notice that
$p_{03}\ne p_{13}p_{01}$. However, the $1$-simplex from $03$ to $013$ indicates that these two morphisms are homotopic to each other.
Moreover, the fact that the morphism space is weakly contractible, means that all possible compositions are canonically homotopic.
This corresponds to the point that $\C[n]$ is the ``thicken version'' of $[n]$ that we have dropped the strict associative condition
and replace it by the idea that compositions are commutative only up to coherent homotopy. Take another example, $\C[\p\d[3]]$.
We have $$\Hom_{\C[\p\d[3]]}=\9\vcenter{\xymatrix{03\ar[r]\ar[d]&013\ar[d]\\023\ar[r]&0123}}\0.$$ We may see that the path of maps
$03\to 013\to 0123\leftarrow 023\leftarrow 03$ is not homotopic to the constant path, which corresponds well with the fact that
$\p\d[3]$ is not weakly contractible.

As an exercise, the reader can show that if $0\le i\le j\le n$, then the map $\Hom_{\C[\p\d[n]]}(i,j)\to\Hom_{\C[\d[n]]}(i,j)$
is identity unless $i=0,j=n$, in which case it is a cofibration; then the map $\Hom_{\C[\l[n]]}(i,j)\to\Hom_{\C[\d[n]]}(i,j)$
is identity unless $i=0,j=n$, in which case it is an anodyne extension.

\defn{
    The \term{topological nerve} of a topological category $\catC$ is defined to be $\N\Sing\catC$.
}

\prop{
    Suppose $\catC$ is a fibrant simplicial category (in the canonical model structure of the category of simplicial categories)
    (which is equivalent to all of its Hom-spaces are Kan complexes). Then $\N\catC$ is an $\infty$-category.
}

\cor{
    The topological nerve of any topological category is an $\infty$-category.
}

We now give the definition to equivalences between $\infty$-categories.

\defn{
    Suppose $S$ is a simplicial set. The \term{homotopy category} of $S$ is $\h S:=\h\C[S]$. A map $f:S\to T$ between simplicial sets
    is called a \term{categorical equiavlence} if $\h f$ is an equivalence of $\cat{H}$-enriched categories.
}

We now state the following theorem, which will be proved in Section {\color{red}\texttt{[ERROR! SECTION NOT FOUND]}}.

\thm{
    There exists a model structure on $\cat{SSet}$, called the \term{Joyal model structure}, where weak equivalences are 
    all the categorical equivalences and fibrant objects are the quasi-categories, such that the adjunction
    $$\xymatrix{(\cat{SSet},\text{Joyal})\ar@(ru,lu)[r]^{\C}\ar@{}[r]|{\bot}&(\cat{Cat}_{\cat{SSet}},\text{std})\ar@(ld,rd)[l]^{\N}}$$
    is a Quillen equivalence.
}

This means that the theory of quasi-categories and the theory of simplicial categories are equivalent models of $\infty$-categories.

Finally we give the following definition:

\defn{
    Suppose $\catC$ is a simplicial category, or a topological category, or a simplicial set. We define $\Ho\catC:=\pi_0\h\catC$.
}

To get a better view of what we have done in this section, it would be better if you have this graph in mind. Here all the arrows
above the categories are left adjoints, where all the arrows below the categories are right adjoints, equivalences
means Quillen equivalences.
$$\xymatrix @C=40pt{
    (\cat{SSet},\text{Joyal})\ar@(ru,lu)[r]^{\C}\ar@{}[r]|{\simeq}
    &(\cat{Cat}_{\cat{SSet}},\text{std})\ar@(ld,rd)[l]^{\N}\ar@(ru,lu)[r]^{\abs-}\ar@{}[r]|{\simeq}
    &(\cat{Cat}_{\cat{K}},\text{std})\ar@(ld,rd)[l]^{\Sing}\ar[r]^{\h}\ar@(ru,lu)[rr]^{\Ho}
    &\cat{Cat}_{\cat H}\ar[r]^{\pi_0}
    &\cat{Cat}\ar@(ld,rd)[ll]
}$$