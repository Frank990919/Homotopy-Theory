We use some simple examples to give the motivation of $\infty$-categories.

\eg{\label{tagk}
    Suppose $X$ is a topological space. The \term{fundamental groupoid} of $X$, denoted $\Pi_{\le 1}(X)$, is the category with:
    \begin{itemize}
        \item objects being all points in $X$; and
        \item $\Hom_{\Pi_{\le 1}(X)}(a,b)$ being the homotopy classes of all paths from $a$ to $b$.
    \end{itemize}
    This is some definition that we shall already have seen in algebraic topology. This gives rise to a functor $\Pi_{\le 1}:\cat{Top}\to\cat{Gpd}$.
    Moreover, for any map $f:X\to Y$ of topological spaces, $\Pi_{\le 1}(f)$ is an equivalence of categories, if and only if 
    it is fully faithful and essentially surjective, where the first assertion indicates that $\pi_1(f)$ is a bijection, and the second assertion
    indicates that $\pi_0(f)$ is a bijection. 

    However, a topological space, contains far more information than its 0th and 1st homotopy groups. There also not only exist
    homotopies between paths, there also exist homotopies between homotopies, and homotopies between the homotopies between homotopies, etc.
    How shall we encode such higher structures? We may think of a ``category'', with not only morphisms between objects, but also morphisms
    between morphisms, and morphisms between the morphisms between morphisms, etc. Specifically, we would like to define $\Pi(X)$ as follows:
    \begin{itemize}
        \item Objects being all points $a,b,\cdots$ in $X$; 
        \item $\Hom_{\Pi(X)}(a,b)$ being all paths $\gamma,\delta,\cdots$ from $a$ to $b$;
        \item $\Hom_{\Pi(X)}(\gamma,\delta)$ being all homotopies $H,K,\cdots$ from $\gamma$ to $\delta$;
        \item $\cdots$
    \end{itemize}
    And we would like to obtain the property that for any map $f:X\to Y$ of topological spaces, $\Pi(f)$ is an ``equivalence'' of
    ``categories'', if and only if $f$ induces isomorphisms on all homotopy groups; that is, $f$ is a weak homotopy equivalence.
    The theory of $\infty$-categories are introduced to make this thought precise and to study their properties.
}

\eg{
    We recall homotopies between maps in model categories: Given a model category $\catC$ and cofibrant-fibrant objects $A,B$,
    maps $f,g:A\to B$, a homotopy from $f$ to $g$ is equivalent to a map $h:A\ox\d[1]\to B$, such that $h|_{A\ox0}=f,h|_{A\ox1}=g$.
    However, cosimplicial frames not only allow us to construct the cylinder object of $A$, but also higher structures.
    For example, a map $A\ox\d[2]\to B$ may be viewed as a homotopy between the three homotopies given by its restriction on
    $A\ox\p\d[2]$, a map $A\ox\d[3]\to B$ may be viewed as a homotopy between the four homotopies between homotopies
    given by its restriction on $A\ox\p\d[3]$. The theory of $\infty$-categories gives us a tool on how to 
    formalize this kind of higher structures, and to study their properties.
}

Notice, that in these two examples, higher morphisms are ``invertible'' (homotopies are invertible, at least up to higher homotopies).
This also coincides with most usages of $\infty$-categories, as in most applications we may see. Therefore, we only talk about
$\infty$-categories, whose ``$k$-morphisms'' are invertible whenever $k\ge 2$. \footnote{In higher category theory, we use 
the term $(\infty,n)$-category to indicate an $\infty$-category whose $k$-morphisms are invertioble whenever $k\ge n+1$. In other words,
what we are focusing about is $(\infty,1)$-categories.}

So how shall we define them? The most na\"ive approch is to define them as what we did in Section 1.1.3. However, notice one thing: 
in both examples, associativity and unit axioms fails to be strict, but only hold up to homotopy, i.e. higher morphisms. To fix this issue,
one must add certain extra diagrams to give the correct notion of higher categories, rather than the ones given in Section 1.1.3.
(The reason we introduce them in the first chapter is to make the languages concise. However, pseudo-2-functors, DO give an approch to
functors between $\infty$-categories, as they keep the homotopy-coherent ideas. From now on, we will call that kind of 2-category
introduced in Section 1.1.3 \term{strict 2-categories}.) On the other hand, even the explicit definition of a 3-category 
is EXTREMELY complicated, and things are just getting worse when we pass to 4-categories and beyond.

Fortunately, we have alternative approaches to the theory of $\infty$-categories. We shall now explain them.

\subsubsection*{i. Via Homotopy Types}

From our na\"ive thought, if we take the ``Hom-thing'' between two object in an $\infty$-category, it will result in an $(\infty,0)$-category.
In other words, $\infty$-categories are categories enriched over $(\infty,0)$-categories. Consequencely, we only have to make a model for
$(\infty,0)$-categories.

Notice that in Example \ref{tagk}, all morphisms, including 1-morphisms, are invertible. This means that $\Pi(X)$ ``should be
an $(\infty,0)$-category''. On the other hand, there is a general axiom accepted by higher categorists, which is roughly stated as follows:

\begin{itemize}
    \item Every $(\infty,0)$-category has the form $\Pi(X)$ for some topological space $X$. Moreover two $(\infty,0)$-categories 
    are equivalent, if and only if their corresponding topological spaces are weakly equivalent.
\end{itemize}