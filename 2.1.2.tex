We now describe how most aspects of ordinary categories extends to infinity categories. 

\rmk{
    As we have stated above, since the three models are equivalent models for $(\infty,1)$-categories, what wen have defined for 
    $\infty$-categories (that are invariant under categorical equivalences) should extends naturally to topological categories
    and simplicial categories. The readers may feel free to use the results we have stated for $\infty$-categories on topological categories
    and simplicial categories, if they prefer.
}

\defn{
    Suppose $\catC$ is an $\infty$-category. An \term{object} in $\catC$ is a vertex in $\catC$. A \term{morphism} in $\catC$ is a $1$-simplex
    in $\catC$. The \term{domain} and \term{codomain} of a morphism $\phi:\d[1]\to\catC$ are $\phi(0)$ and $\phi(1)$, respectively.
    If a morphism $\phi$ has domain $x$ and codomain $y$, we write $\phi:x\to y$. The \term{identity morphism} of an object $x$ is
    $\1_x:=s_0(x):x\to y$.
}

\defn{
    Suppose $S$ is a simplicial set. We define the \term{opposite} of $S$ to be the simplicial set $S^\op$, with $S^\op_n=S_n$ but
    $$(d_i:S^\op_n\to S^\op_{n-1})=(d_{n-i}:S_n\to S_{n-1}),(s_i:S^\op_n\to S^\op_{n+1})=(s_{n-i}:S_n\to S_{n+1}).$$
    If $\catC$ is an $\infty$-category, we call $\catC^\op$ the \term{opposite category} of $\catC$.
}

We next consider equivalences between morphisms.

\defn{
    Suppose $\catC$ is an $\infty$-category. Two morphisms $f,g:x\to y$ in $\catC$ are said to be \term{homotopic}, 
    written as $f\simeq g$, if one of the following equivalence relations hold:
    \begin{enumerate}[i)]
        \item There exists a $2$-simplex $\vcenter{\xymatrix @R=5pt{x\ar[rd]^f\ar[dd]_{\1_x}&\\&y\\x\ar[ru]_g&}}$ in $\catC$;
        \item There exists a $2$-simplex $\vcenter{\xymatrix @R=5pt{x\ar[rd]^g\ar[dd]_{\1_x}&\\&y\\x\ar[ru]_f&}}$ in $\catC$;
        \item There exists a $2$-simplex $\vcenter{\xymatrix @R=5pt{&y\ar[dd]^{\1_y}\\x\ar[ru]^f\ar[rd]_g&\\&y}}$ in $\catC$;
        \item There exists a $2$-simplex $\vcenter{\xymatrix @R=5pt{&y\ar[dd]^{\1_y}\\x\ar[ru]^g\ar[rd]_f&\\&y}}$ in $\catC$.
    \end{enumerate}
}

\prop{
    The homotopic relation is an equivalence relation.
}

\defn{
    Suppose $\catC$ is an $\infty$-category. We define a category $\pi(\catC)$ as follows:
    \begin{enumerate}[i)]
    \item Its objects are vertices of $\catC$;
    \item For any $x,y\in\catC_0$, define
    $\Hom_{\pi(\catC)}(x,y)=\{f:x\to y\in\catC_1\}/\simeq$;
    \item For any $x\in\catC_0$, define $\1_x=[s_0x]$;
    \item For any morphisms $[f]:x\to y,[g]:y\to z$ in $\pi(\catC)$, take a lift
    of the following diagram:
    $$\xymatrix @C=40pt{
    \l^1[2]\ar[rr]^{f\text{ on }\d\{0,1\}}_{g\text{ on }\d\{1,2\}}\ar[d]&&\catC\\
    \d[2]\ar@{.>}[rru]_{h}&&
    }$$
    and define $[g]\circ[f]=[d_1h]$.
    \end{enumerate}
}

\prop{
    The category $\pi(\catC)$ is isomorphic to the category $\Ho(\catC)$, meaning that two morphisms $f\simeq g$ if and only if thet are equal in 
    $\Ho\catC$.
}

Next we discuss the mapping spaces between objects.

\defn{
    Suppose $S$ is a simplicial set, $x,y$ are vertices in $S$. We define the \term{mapping space} from $x$ to $y$ to be 
    $\Map_S(x,y):=\Hom_{\h S}(x,y)\in\cat H$.
}

So how shall we compute this mapping space? By definition, $\Map_S(x,y)$ is the homotopy type of $\Hom_{\C[S]}(x,y)$.
The advantage of this definition is that it has a natural composition law, and it works for all simplicial sets, not just $\infty$-categories.
However, this construction is quite complicated, and it is usually not a Kan complex, which makes it difficult to verify its properties.

We shall now introduce another simplicial set that represents the homotopy type $\Map_S(x,y)$, at least when $S$ is an $\infty$-category.

\defn{
    Suppose $S$ is a simplicial set, $x,y$ are vertices in $S$. We define $$\Hom^R_S(x,y)=\{\sigma\in S_{n+1}
    \mid\sigma|_{\d\{0,\cdots,n\}}=x,\sigma(n+1)=y\},$$ and $$\Hom^L_S(x,y)=\{\sigma\in S_{n+1}\mid\sigma(0)=x,\sigma|_{\d\{1,\cdots,n+1\}}=y\}.$$
    Moreover we define $$\Hom_S(x,y)=\{\sigma:\d[n]\times\d[1]\to S\mid \sigma|_{\d[n]\times\{0\}}=x,\sigma|_{\d[n]\times\{1\}}=y\}.$$
}

The following proposition will be proved in Section \texttt{\color{ff0000}[ERROR! SECTION NOT FOUND]}
and \texttt{\color{ff0000}[ERROR! SECTION NOT FOUND]}.

\prop{
    If $S$ is an $\infty$-category, then for any objects $x,y$, the simplicial sets $\Hom^R_S(x,y),\Hom^L_S(x,y),\Hom_S(x,y)$ are Kan complexes. 
    There exists a canonical isomorphism $$\Hom^R_S(x,y)\xrightarrow{\simeq}\Hom_{\C[S]}(x,y)$$ in the category $\cat H$. The natural inclusions
    $$\Hom^R_S(x,y)\hookrightarrow\Hom_S(x,y)\hookleftarrow\Hom^L_S(x,y)$$ are weak homotopy equivalences.
}

We next discuss when can we say a morphism is an equivalence.

\defn{
    Suppose $\catC$ is a topological category, of a simplicial category, or an $\infty$-category. A morphism $f$ in $\catC$ is called an
    \term{equivalence}, or \term{invertible}, if it is an isomorphism in $\h\catC$.
}

The following criterion is useful while we want to determine whether a morphism is an equivalence or not.

\prop{
    Suppose $\catC$ is a topological category, $f:X\to Y$ is a morphism in $\catC$. The following conditions are equivalent:
    \begin{enumerate}[i)]
        \item $f$ is an equivalence;
        \item $f$ has a homotopy inverse; that is, there exists a morphism $g:Y\to X$ such that $fg\simeq \1_Y$ and $gf\simeq\1_X$;
        \item For any $Z\in\catC$, composition with $f$ induces a homotopy equivalence $\Hom_\catC(Z,X)\to\Hom_\catC(Z,Y)$;
        \item For any $Z\in\catC$, composition with $f$ induces a weak homotopy equivalence $\Hom_\catC(Z,X)\to\Hom_\catC(Z,Y)$;
        \item For any $Z\in\catC$, composition with $f$ induces a homotopy equivalence $\Hom_\catC(Y,Z)\to\Hom_\catC(X,Z)$;
        \item For any $Z\in\catC$, composition with $f$ induces a weak homotopy equivalence $\Hom_\catC(Y,Z)\to\Hom_\catC(X,Z)$.
    \end{enumerate}
}

The following proposition will be proved in Section \texttt{\color{ff0000}[ERROR! SECTION NOT FOUND]}.

\prop{
    Suppose $\catC$ is an $\infty$-category, $\phi$ is a morphism in $\catC$. $\phi$ is an equivalence, if and only if for every
    map $f:\l^0[n]\to\catC$ such that $f|_{\d\{0,1\}}=\phi$, $f$ extends to a map $\d[n]\to\catC$.
}

Using this proposition we obtain the following result:

\prop{
    Suppose $\catC$ is a simplicial set. The following conditions are equivalent:
    \begin{enumerate}
        \item $\catC$ is an $\infty$-groupoid;
        \item $\catC$ is an $\infty$-category and $\Ho\catC$ is a groupoid;
        \item $\catC$ is an $\infty$-category and $\catC$ has extension property with respect to all inclusions $\l^0[n]\to\d[n]$;
        \item $\catC$ is an $\infty$-category and $\catC$ has extension property with respect to all inclusions $\l^n[n]\to\d[n]$.
    \end{enumerate}
}

Next we talk about the ``underlying $\infty$-groupoid of an $\infty$-category''.

\prop{
    Suppose that $\catC$ is an $\infty$-category. Let $\catC'\subseteq\catC$ be the lergest simplicial subset of $\catC$
    such that all edges of $\catC'$ are invertible in $\catC$. Then $\catC$ is a Kan complex. Moreover for any Kan complex $K$,
    the induced map $\Hom_{\cat{SSet}}(K,\catC')\to\Hom_{\cat{SSet}}(K,\catC)$ is an isomorphism. We call $\catC'$ the \term{largest
    Kan complex} contained in $\catC$. If $f:\catC\to\catD$ is a categorical equivalence between $\infty$-categories, then the induced map
    $f':\catC'\to\catD'$ between the largest Kan complexes is a homotopy equivalence.
    \footnote{The lecturer cannot figure out a proof to the last assertion while not using the Joyal model structure of simplicial sets,
    although the book said that it can be proved directly.}
}

We next consider diagrams on $\infty$-categories. This gives rises to the difference between homotopy-commutative diagrams
and homotopy-coherent diagrams, as it has been promised that we will discuss it in this section. 

The issue is as follows. Suppose $\catC$ is an $\infty$-category (or a topological category, simplicial category). To a first approximation,
it is very similar to work on $\catC$ and $\h\catC$, since they have the same objects and morphisms. However, in $\catC$, we should ask
whether or not two morphisms are equivalent or homotopic, but not equal. In this case, the homotopy between the morphisms need to
be taken in consider, since they may as well affect the higher-dimensional homotopies. However, a commutative diagram in $\h\catC$,
does not contain the information on the homotopies between the morphisms. Consequencely, commutative diagrams on $\h\catC$, which corresponds
to the notion of \term{homotopy-commutative diagrams} on $\catC$, is quite unnatural and need to be refined by the notion of
\term{homotopy-coherent diagrams} on $\catC$.

Let us give two examples to show their differences.

\eg{
    Consider a monoid object $G$ in $\cat H$, and consider the $\cat H$-enriched category $\cat BG$. 
    Then a $\cat BG$-valued diagram in a topological category is a homotopy-commutative diagram. On the other hand, the mutiplication on $G$
    is homotopy commutative, i.e. the maps $(xy)z,x(yz):G\times G\times G\to G$ are homotopic, which means that $G$ is an $A_3$-algebra.
    However, in practice we would like to require that $G$ is not just an $A_3$-algebra, but also satisfies higher 
    homotopy-coherent properties. For example, we would like the following diagram of homotopies between morphisms $G\times G\times G\times G\to G$
    to be commutative up to a homotopy:
    $$\xymatrix @C=-5pt @R=30pt{&(x(yz))w\ar@{-}[rr]\ar@{-}[ld]&&x((yz)w)\ar@{-}[rd]&\\
    ((xy)z)w\ar@{-}[rrd]&&&&x(y(zw))\ar@{-}[lld]\\&&(xy)(zw);&&}$$
    and also the following diagram of homotopies between homotopies between maps $G\times G\times G\times G\times G\to G$
    to be commutative up to a homotopy:
    $$\xymatrix @!0 @C=44pt{&&&&&(((ab)c)d)e\ar@{-}[rrrr]\ar@{-}[llld]\ar@{-}'[dd]'[ddddd][dddddd]&&&&((ab)(cd))e\ar@{-}[lldd]\ar@{-}[ddddd]\\
    &&((a(bc))d)e\ar@{-}[ld]\ar@{-}'[d]'[dddd][dddddd]&&&&&&&\\
    &(a((bc)d))e\ar@{-}[rrrrrr]\ar@{-}[dddl]&&&&&&(a(b(cd)))e\ar@{-}[dddl]&&\\
    &&&&&&&&&\\
    &&&&&&&&&\\
    a(((bc)d)e)\ar@{-}[rrrrrr]\ar@{-}[dddd]&&&&&&a((b(cd))e)\ar@{-}[ddd]&&&(ab)((cd)e)\ar@{-}[lld]\ar@{-}[lllddd]\\
    &&&&&((ab)c)(de)\ar@{-}[llld]\ar@{-}'[r][rr]&&(ab)(c(de))\ar@{-}'[ld][lllddd]&&\\
    &&(a(bc))(de)\ar@{-}[lldd]&&&&&&&\\
    &&&&&&a(b((cd)e))\ar@{-}[lld]&&&\\
    a((bc)(de))\ar@{-}[rrrr]&&&&a(b(c(de)))&&&&&}$$
    This requirement is equivalent to assuming that $G$ is an $A_\infty$-algebra. But there exists $A_3$ algebras that are not $A_\infty$-algebras.
    For such $G$, $\cat BG$-valued diagrams are ``bad'' since they cannot recover all the homotopy relations that we want. Also, 
    because of the lack of higher homotopies, the $\cat H$-enriched category $\cat BG$ should not be viewed as an $\infty$-category,
    which corresponds to the fact that not all $\cat{H}$-enriched categories are $(\infty,1)$-categories, as it has been mentioned
    in the previous section.
}

\eg{
    Now suppose that we have an $[n]$-valued diagram in the simplicial category $\cat{SSet}$, that we wish to be homotopy-coherent.
    We denote this diagram by $F$. For simplicity, we suppose that $F(i)=*$ for every $0\le i<n$.
    (The reader should also think of what will happen if $F(i)$ are arbitrary simplicial sets, as we will come back to this 
    in several sections.) We define $e_i$ to be the image of $F(i)$ in $F(n)$. Then, for every $0\le i\le j<n$,
    since the maps $i\to j\to n$ and $i\to n$ are equal, we would like to assume that there exists a $1$-simplex $e_{ij}$
    from $e_i$ to $e_j$ in $F(n)$. Also, for every $0\le i\le j\le k<n$, since the diagram 
    $$\xymatrix{&j\ar[r]\ar[rrd]&k\ar[rd]\\i\ar[ru]\ar[rru]\ar[rrr]&&&n}$$ is commutative, we would like to assume that 
    there exists a $2$-simplex $e_{ijk}$ in $F(n)$ filling the homotopies $e_{ij},e_{ik},e_{jk}$; and for every $0\le i\le j\le k\le l<n$,
    we would like to assume that there exists a $3$-simplex $e_{ijkl}$ in $F(n)$ filling the homotopies $e_{ijk},e_{ijl},e_{ikl},e_{jkl}$;
    and homotopies of higher degrees as well. Such a diagram contains much more information than a homotopy-commutative diagram,
    where the latter only contains the information of $e_{ij}$'s, and losing the informations of higher homotopies.
}

So how shall we define a homotopy-coherent diagram? We could define this by working degree by degree 
in topological categories or simplicial categories. However, this is a great amount of data and is really difficult to actually work with.
But in $\infty$-categories, things are really easy to formulate: Suppose $\catI$ is an ordinary category and $\catC$ is an $\infty$-category,
we may simply define an $\catI$-valued (homotopy-coherent) diagram in $\catC$ to be a map $\N(\catI)\to\catC$ of simplicial sets. (This is,
in fact, one of the reasons that quasi-categories are one of the most widely-used model for $(\infty,1)$-categories.)

The notion of homotopy-coherent diagrams in higher categories is a special case of functors between higher categories. Again, this is 
very easy to define in the settings of $\infty$-categories:

\defn{
    Suppose $\catC,\catD$ are $\infty$-categories. A \term{functor} from $\catC$ to $\catD$ is simply a map $F:\catC\to\catD$ of simplicial sets.
    We define $\Fun(K,\catD):=\catD^K$, for every simplicial set $K$. (In this notation, $K$ will often, but not always, be a simplicial set.)
}

The following proposition will be proved in Section \texttt{\color{ff0000}[ERROR! SECTION NOT FOUND]}.

\prop{
    Suppose $K,K'$ are simplicial sets, $\catC,\catD$ are $\infty$-categories. Then:
    \begin{enumerate}[i)]
        \item $\Fun(K,\catC)$ is an $\infty$-category;
        \item If $F:\catC\to\catD$ is a categorical equivalence, then the induced map $\Fun(K,\catC)\to\Fun(K,\catD)$ is a categorical equivalence;
        \item If $f:K\to K'$ is a categorical equivalence, then the induced map $\Fun(K',\catC)\to\Fun(K,\catC)$ is a categorical equivalence.
    \end{enumerate}
}

\defn{
    Suppose $\catC,\catD$ are $\infty$-categories. We define the \term{$\infty$-category of functors} from $\catC$ to $\catD$ to be 
    $\Fun(\catC,\catD)$. We define \term{natural transformations} between functors from $\catC$ to $\catD$ to be morphisms in $\Fun(\catC,\catD)$,
    and call a natural transformation a \term{natural equivalence} if it is invertible in $\Fun(\catC,\catD)$.
}

Next we discuss slice categories. It is shown in Section \ref{secc} that in the case of ordinary categories, if $p:\catC\to\catD$
is a diagram in $\catD$, then for any $\catC'$, we have isomorphism
$$\Fun(\catC',\catD_{/p})\cong\Fun(\catC'\star\catC,\catD)\times_{\Fun(\catC,\catD)}\{p\},$$ that is natural in $\catC'$. 
This gives rise to the following definitions:

\defn{
    Suppose $S,S'$ are simplicial sets. We define the \term{join} of $S$ and $S'$, denoted $S\star S'$, to be the simplicial set
    with $$\Hom_{\cat{SSet}}(\d[n],S\star S')=\left\{(f,\sigma,\sigma')\mathrel{\Bigg|}
    \begin{aligned}&f:[n]\to[1],\\&\sigma:\d(f^{-1}(0))\to S,\\&\sigma':\d(f^{-1}(1))\to S'\end{aligned}\right\}.$$
}

If the reader has trouble understanding the definition, it is suggested to take $S=\d[2],S'=\d[1]$ as an example to get a better understanding.

We also give the following definition:

\defn{
    Suppose $K$ is a simplicial set. We define the \term{left cone} of $K$ to be $K^\tril:=*\star K$, and the \term{right cone} of $K$ to be
    $K^\trir:=K\star *$. The distinguished vertex in either cone will be called the \term{cone point}, and will be written as
    $-\infty$ in the left cone or $\infty$ in the right cone.
}

With this definition, one can verify that $(\cat{SSet},\star)$ is a closed monoidal category, and the following proposition holds:

\prop{
    If $S,S'$ are $\infty$-categories, then $S\star S'$ is an $\infty$-category.
}

Moreover, we have the following proposition:

\prop{
    For any simplicial category $\catC,\catC'$ we have a canonical isomorphism $\N(\catC\star\catC')\cong\N(\catC)\star\N(\catC')$.
}

On the other hand, $\star$ does not commute with the functor $\C$. Nevertheless, we have the following proposition 
which will be proved in Section \texttt{\color{ff0000}[ERROR! SECTION NOT FOUND]}:

\prop{
    For any simplicial sets $S,S'$ there exists a canonical map $\C[S\star S']\to\C[S]\star\C[S']$, which is an equivalence
    between simplicial categories.
}

Now we can define what a slice category is:

\prop{
    Suppose $p:K\to S$ is a map of simplicial sets. The functor $$\cat{SSet}\to\cat{SSet},
    X\mapsto\Fun(X\star K,S)\times_{\Fun(K,S)}\{p\}$$ is representable. We define $S_{/p}$ to be the representation of this functor,
    and call it the \term{$\infty$-category of objects of $S$ over $p$} whenever $S$ is an $\infty$-category.
    Dually, we have the definition of the \term{$\infty$-category of objects of $S$ under $p$}, which will be denoted $S_{p/}$.
}

To be more specifically, $$(S_{/p})_n=\{\sigma:\d[n]\star K\to S\mid \sigma|_{K}=p\}.$$ Similarly we may directly characterize
$S_{p/}$.

This definition is indeed a generalization of the slice category of ordinary categories:

\prop{
    If $p:\catC\to\catD$ is a diagram in an ordinary category $\catD$, then $\N(\catD)_{/\N(p)}\cong\N(\catD_{/p})$. The case for undercategories
    is similar.
}

The following proposition will be proved in Section \texttt{\color{ff0000}[ERROR! SECTION NOT FOUND]}
and Section \texttt{\color{ff0000}[ERROR! SECTION NOT FOUND]}.

\prop{
    Suppose $\catC$ is an $\infty$-category, $p:K\to\catC$ is a \term{diagram} in $\catC$ (which simply means that 
    $p$ is a map of simplicial sets). Then $\catC_{/p}$ is an $\infty$-category. Moreover, if $q:\catC\to\catD$ 
    is a categorical equivalence between $\infty$-categories, then the induced map $\catC_{/p}\to\catD_{/qp}$ is also a categorical equivalence.
}

We next talk about fully faithful and essentially surjective functors, together with subcategories.

\defn{
    Suppose $f:\catC\to\catD$ is a map of simplicial sets (or other models for $\infty$-categories). $f$ is called \term{essentially surjective}
    if $\h f$ is esseentially surjective. $f$ is called \term{fully faithful} if $\h f$ is fully faithful.
}

By definition, $f$ is a categorical equivalence if and only if it is fully faithful and essentially surjective.

\defn{
    Suppose $\catD$ is a subcategory of $\Ho\catC$, where $\catC$ is an $\infty$-category. We define $\catC\times_{\N\Ho\catC}\N\catD$
    to be the \term{subcategory of $\catC$ spanned by $\catD$}. We say that a simplicial subset $K\subseteq\catC$ is a \term{subcategory}
    of $\catC$ if it is isomorphic to $\catC\times_{\N\Ho\catC}\N\catD$ for some subcategory $\catD$ of $\Ho\catC$. If $\catD$
    is a full subcategory of $\Ho\catC$, we also call $\catC\times_{\N\Ho\catC}\N\catD$ the \term{full subcategory of $\catC$ spanned by $S$},
    where $S$ is the set of objects in $\catD$.
}

By definition, if $\catC'$ is a full subcategory of $\catC$, then the inclusion $\catC'\hookrightarrow\catC$ is fully faithful.

