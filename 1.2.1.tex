\defn{
    Suppose $\catC$ is a category with all small colimits and $\lambda$ is a ordinal. A \term{$\lambda$-sequence} $X$ is a colimit-preserving functor
    $\lambda\to\catC$. We denote $X(\alpha)$ as $X_\alpha$. A \term{transfinite composition} of a $\lambda$-sequence is the induced map
    $X_0\to\colim_{\beta<\lambda}X_\beta$. If $I$ is a collection of maps of $\catC$ and for any $\beta+1<\lambda$ the map $X_\beta\to X_{\beta+1}$
    is in $I$, we call the $\lambda$-sequence to be in $I$ and transfinite composition a transfinite composition of maps of $I$.
}

\defn{
    Suppose $\gamma$ is a cardinal. A \term{$\gamma$-filtered ordinal} is a limit ordinal $\alpha$ such that for any $A\subseteq\alpha,|A|\le\gamma$
    we have $\sup A<\alpha$.
}

\defn{
    Suppose $\catC$ is a category with all small colimits, $I$ is a collection of maps of $\catC$, $A$ is an object in $\catC$ and $\kappa$ is a cardinal.
    We say $A$ is \term{$\kappa$-small relative to $I$} if for any $\kappa$-filtered ordinal $\lambda$ and all $\lambda$-sequence $X$ in $I$,
    the map of sets $$\colim_{\beta<\lambda}\Hom_\catC(A,X_\beta)\to\Hom_\catC(A,\colim_{\beta<\lambda}X_\beta)$$ is an isomorphism.
    We say $A$ is \term{small relative to $I$} if there exists some $\kappa$ such that $A$ is $\kappa$-small relative to $I$.
    We say $A$ is \term{small} if A is small relative to $\catC$. We say $A$ is \term{finite relative to $I$} if there exists some finite cardinal
    $\kappa$ such that $A$ is $\kappa$-small relative to $I$. We say $A$ is \term{finite} if A is finite relative to $\catC$.
}

\eg{
    In the category $\cat{Set}$ every object $A$ is $|A|$-small.
}

\eg{
    If $R$ is a ring, in the category $\cat{Mod}_R$ of left $R$-modules every object is small, every finitely presented $R$-module is finite.
}

\rmk{
    In the category $\cat{Top}$ not all objects are small; For example, the indiscrete space with 2 points.
}

\defn{
    Suppose $\catC$ is a category, $I$ is a collection of maps of $\catC$. 
    Denote $\RLP(I)$ to be the class of all maps that has RLP with respect to every may in $I$, and $\LLP(I)$
    to be the class of maps that has LLP with respect to every may in $I$. Denote 
    $\Cof(I)=\LLP(\RLP(I))$ and $\Fib(I)=\RLP(\LLP(I))$. We call the maps in $\Cof(I)$ to be
    \term{$I$-cofibrations} and $\Fib(I)$ to be \term{$I$-fibrations}.
}

\lem{
    Suppose $\catC$ is a category, $I$ is a collection of maps of $\catC$. Then $I\subseteq\Cof(I),I\subseteq\Fib(I)$. Moreover $\RLP(\Cof(I))=\RLP(I)$ 
    and $\LLP(\Fib(I))=\LLP(I)$.
}

\lem{
    Suppose $(F,G)$ is an adjunction between $\catC$ and $\catD$, $I$ is a collection of maps of $\catC$, $J$ is a collection of maps of $\catD$.
    Then we have $G(\RLP(FI))\subseteq\RLP(I),F(\Cof(I))\subseteq\Cof(FI),F(\LLP(GJ))\subseteq\LLP(J),G(\Fib(J))\subseteq\Fib(GJ)$.
}

\defn{
    Suppose $\catC$ is a category with all small colimits and $I$ is a collection of maps of $\catC$. A \term{relative $I$-cell complex}
    is a transfinite composition of pushouts of elements of $I$. Denote $\Cell(I)$ to be the class of all relative $I$-cell complexes. An object $A$ 
    is called an \term{$I$-cell complex} if the map $\phi\to A$ is a relative $I$-cell complex.
}

\lem{
    Suppose $\catC$ is a category with all small colimits and $I$ is a collection of maps of $\catC$. Then $\Cell(I)\subseteq\Cof(I)$.
}

\lem{
    Suppose $\catC$ is a category with all small colimits and $I$ is a collection of maps of $\catC$. Then transfinite composition of pushouts
    of elements of $I$ or isomorphisms is in $\Cell(I)$.
}

\lem{
    Suppose $\catC$ is a category with all small colimits and $I$ is a collection of maps of $\catC$. Then $\Cell(I)$ is closed
     under transfinite composition.
}

\lem{
    Suppose $\catC$ is a category with all small colimits and $I$ is a collection of maps of $\catC$. Then any pushout of coproducts of maps of $I$
    is in $\Cell(I)$.
}

\thm{[The Small Object Argument]
    Suppose $\catC$ is a category with all small colimits and $I$ is a set of maps of $\catC$. Suppose the domains of maps of $I$ are small
    relative to $\Cell(I)$. Then there exists a functorial factorization $(\gamma,\delta)$ on $\catC$ such that for any $f$ a morphism in $\catC$,
    $\gamma(f)\in\Cell(I)$ and $\delta(f)\in\RLP(I)$.
}

\cor{
    Suppose $\catC$ is a category with all small colimits and $I$ is a set of maps of $\catC$. Suppose the domains of maps of $I$ are small 
    relative to $\Cell(I)$. Then for any $f:A\to B\in\Cof(I)$ there exists $g:A\to C\in\Cell(I)$ such that $f$ is a retract of $g$, and the retract fixes $A$.
}

\prop{
    Suppose $\catC$ is a category with all small colimits and $I$ is a set of maps of $\catC$. Suppose the domains of maps of $I$ are small 
    relative to $\Cell(I)$. Then any object that is small relative to $\Cell(I)$ is small relative to $\Cof(I)$.
}

\defn{
    Suppose $\catC$ is a model category. $\catC$ is called \term{cofibrantly generated} if there exist sets of maps $I$ and $J$ of $\catC$,
    such that:
    \begin{enumerate}[i)]
    \item The domains of maps of $I$ are small relative to $\Cell(I)$;
    \item The domains of maps of $J$ are small relative to $\Cell(J)$;
    \item $\Fib=\RLP(J)$;
    \item $\W\cap\Fib=\RLP(I)$.
    \end{enumerate} 
    The sets $I$ and $J$ are called the \term{generating cofibrations} and the \term{generating trivial cofibrations},
    respectively. A cofibrantly generated model category is called \term{finitely generated} if the domains and codomains of maps of $I$ and $J$
    are finite relative to $\Cell(I)$.
}

\prop{
    Suppose $\catC$ is a cofibrantly generated model category with generating cofibrations $I$ and generating trivial cofibrations $J$. Then:
    \begin{enumerate}[i)]
    \item $\Cof=\Cof(I)$, $\W\cap\Cof=\Cof(J)$;
    \item Any cofibration is a retract of a relative $I$-cell complex; any trivial cofibration is a retract of a relative $J$-cell complex;
    \item The domains of maps of $I$ are small relative to $\Cof$; the domains of maps of $J$ are small relative to $\W\cap\Cof$.
    \end{enumerate}
    If $\catC$ is finitely generated then the domains and codomains of maps of $I$ and $J$ are finite relative to $\Cof$.
}

\thm{
    Suppose $\catC$ is a category with all small limits and colimits, $\W$ is a class of maps of $\catC$, $I$ and $J$ are sets of maps in $\catC$. 
    Then $\catC$ is a cofibrantly generated model category with weak equivalence $\W$, generating cofibrations $I$ and generating
    trivial cofibrations $J$ if and only if the following are satisfied:
    \begin{enumerate}[i)]
    \item $\W$ has the 2-out-of-3 property and is closed under retracts;
    \item The domains of maps of $I$ are small relative to $\Cell(I)$;
    \item The domains of maps of $J$ are small relative to $\Cell(J)$;
    \item $\Cell(J)\subseteq\W\cap\Cof(I)$;
    \item $\RLP(I)\subseteq\W\cap\RLP(J)$;
    \item Either $\W\cap \Cof(I)\subseteq \Cof(J)$ or $\W\cap\RLP(J)\subseteq\RLP(I)$.
    \end{enumerate}
}

\prop{
    Suppose $(F,G):\catC\to\catD$ is an adjunction, where $\catC$ and $\catD$ are model categories. Suppose more that $\catC$ is a
    cofibrantly generated model category with generating cofibrations $I$ and generating trivial cofibrations $J$. Then $(F,G)$
    is an Quillen adjunction if and only if $FI\subseteq\Cof_\catD$ and $FJ\subseteq\W_\catD\cap\Cof_\catD$. 
}

\prop{
    If $\catC$ is a cofibrantly generated model category then so is $\catC_*$. If $\catC$ is a finitely generated model category then so is $\catC_*$.
}