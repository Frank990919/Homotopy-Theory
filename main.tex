% !TEX program = xelatex

\documentclass[12pt]{report}

\usepackage{dsfont}
\usepackage{amsfonts}
\usepackage{amsmath}
\usepackage{amsthm}
\usepackage{amssymb}
\usepackage{geometry}
\usepackage[mathcal]{euscript}
\usepackage[all]{xy}
\usepackage{xcolor}
\usepackage{enumerate}
\usepackage{mathrsfs}

\definecolor{lightblue}{HTML}{2088d0}
\definecolor{pruple}{HTML}{8030dd}
\geometry{inner=1.91cm,outer=1.91cm,top=1.91cm,bottom=1.91cm,paperwidth=210mm,paperheight=297mm}
\usepackage[colorlinks,linkcolor=purple,anchorcolor=blue,citecolor=green]{hyperref}

\theoremstyle{plain}
\newtheorem{theorem}{Theorem}[subsection]
\newtheorem{corollary}[theorem]{Corollary}
\newtheorem{lemma}[theorem]{Lemma}
\newtheorem{claim}[theorem]{Claim}
\newtheorem{proposition}[theorem]{Proposition}
\theoremstyle{definition}
\newtheorem{definition}[theorem]{Definition}
\newtheorem{example}[theorem]{Example}
\newtheorem{remark}[theorem]{Remark}
\newtheorem*{notation}{Notation}

\def\refbib#1{\hyperref[#1]{\color{lightblue}\texttt{[#1]}}}
\def\term#1{\textbf{#1}}
\def\mathbb#1{\mathds{#1}}

\long\def\defn#1{\begin{definition}#1\end{definition}}
\long\def\prop#1{\begin{proposition}#1\end{proposition}}
\long\def\thm#1{\begin{theorem}#1\end{theorem}}
\long\def\rmk#1{\begin{remark}#1\end{remark}}
\long\def\eg#1{\begin{example}#1\end{example}}
\long\def\lem#1{\begin{lemma}#1\end{lemma}}
\long\def\cor#1{\begin{corollary}#1\end{corollary}}

%\def\refthm#1{Theorem \ref{#1}}
%\def\refprop#1{Proposition \ref{#1}}

\hyphenation{mon-oid-al}

\def\0{\right)}
\def\9{\left(}
\def\1{\mathbb1}

\def\d{\Delta}
\def\l{\Lambda}
\def\p{\partial}

\def\ox{\otimes}
\def\ov{\wedge}
\def\vp{\varphi}
\def\ve{\varepsilon}
\def\squ{\mathop{\operatorname{\square}}}

\def\Cof{\mathrm{Cof}}
\def\Fib{\mathrm{Fib}}
\def\W{\mathrm{W}}

\def\op{\mathrm{op}}
\def\h{\mathrm{h}}
\def\N{\mathrm{N}}
\def\C{\mathfrak{C}}
\def\pr{\operatorname{pr}}
\def\inn{\operatorname{in}}
\def\Cyl{\operatorname{Cyl}}
\def\Path{\operatorname{Path}}
\def\colim{\mathop{\operatorname{colim}}}
\def\Ho{\operatorname{Ho}}
\def\diag{\operatorname{diag}}
\def\Sing{\operatorname{Sing}}
\def\Adj{\operatorname{Adj}}
\def\Cell{\operatorname{Cell}}
\def\RLP{\operatorname{RLP}}
\def\LLP{\operatorname{LLP}}
\def\Map{\operatorname{Map}}
%\def\Mapsq{\Map_\square}
\def\Hom{\operatorname{Hom}}
\def\sHom{\operatorname{\mathscr{H}\mkern-3mu\mathit{om}}}

\def\catB{\mathcal{B}}
\def\catC{\mathcal{C}}
\def\catD{\mathcal{D}}
\def\catE{\mathcal{E}}
\def\catF{\mathcal{F}}
\def\catS{\mathcal{S}}
\def\catT{\mathcal{T}}
\def\cattC{\mathbb{C}}
\def\cattD{\mathbb{D}}

\def\siml{\stackrel{l}{\sim}}
\def\simr{\stackrel{r}{\sim}}

\def\cat#1{\underline{\smash{\mathsf{#1}}}}
\def\catt#1{\underline{\underline{\smash{\mathsf{#1}}}}}

\def\abs#1{\left|#1\right|}

\begin{document}

\title{\Huge \textbf{Homotopy Theory}}
\author{\Large Frank Kong, 2017012181}
\maketitle

\chapter*{Preface}

\section*{Introduction}

This is the lecture note to the course Homotopy Theory. 
The course started in the autumn semester in 2018, and 49 lectures has been given so far.
The statements given are unproved, the proofs can be found in references below.

\section*{References}

\begin{itemize}
\item The main page of the course on Banana Space: 

\url{https://www.bananaspace.net/page/51800441}.
\item The main page of the course on GitHub: 

\url{https://github.com/Frank990919/Homotopy-Theory}.
\item Hovey: \textit{Model Categories}.
\item Lurie: \textit{Higher Algebra}.
\item Lurie: \textit{Higher Topos Theory}.
\item The nLab: \url{https://ncatlab.org/}.
\item Lecture notes for the lecture \textit{Homotopic Algebra and Homological Algebra}, by Bu Chenjing:

Banana Space version: \url{https://www.bananaspace.net/page/25842635};

GitHub version: \url{https://github.com/abccsss/Homotopical-Algebra}.
\end{itemize}

\section*{Acknowledgements}

The lecturer would like to thank Bu Chenjing for giving the lecturer support on the lecture, and discussions, ideas and thoughts 
that helps the lecturer to understand the subject better and the notes more readable.

\tableofcontents

\chapter{Model Categories (Semester 1)}

The following contents are given in the lectures in the autumn semester of 2018. 21 lectures with a total time of 70 hours were given 
to finish the book \textit{Model Categories}.

\section{Basic Concepts}

\subsection{Model Categories and Homotopy Categories}

\defn{
    Suppose $f:A\to B,g:C\to D$ are morphisms in $\catC$. We say $f$ is a \term{retract} of $g$ if there exists a commutative diagram
    $$\xymatrix{A\ar[d]_{f}\ar[r]^{p}&C\ar[d]^{g}\ar[r]^{q}&A\ar[d]^{f}\\B\ar[r]^{r}&D\ar[r]^{s}&B}$$ such that $sr=\1,qp=\1$.
}

\defn{
    A \term{functorial factorization} of a category $\catC$ is an ordered pair $(\alpha,\beta)$ of functors $\catC^{[1]}\to\catC^{[1]}$ such that
    $d\alpha=d,c\alpha=d\beta,c\beta=c$, where $d,c:\catC^{[1]}\to\catC$ are the domain functor and the codomain functor, respectively, and for any
     $f\in\catC^{[1]}$, $f=\beta f\circ\alpha f$. Here $[1]$ denotes the category with 2 objects and 3 morphisms.
}

\defn{
    Suppose $i:A\to B$ and $p:X\to Y$ are morphisms in $\catC$. We say $i$ has \term{left lifting property (LLP)} with respect to $p$, and $p$ has
    \term{right lifting property (RLP)} with respect to $i$, written $i\nearrow p$, if for any commutative diagram
    $$\xymatrix{A\ar[r]^{f}\ar[d]_{i}&X\ar[d]^{p}\\B\ar[r]^{g}&Y,}$$ there exists some $h:B\to X$ such that $ph=g,hi=f$.
}

\defn{
    Suppose $\catC$ is a category. A \term{model structure} on $\catC$ consists of three classes of morphisms $\Cof,\Fib,\W$ together with
     two functorial factorizations $(\alpha,\beta),(\gamma,\delta)$, such that:
    \begin{enumerate}[i)]
    \item $\Cof,\Fib,\W$ are closed under compositions and retracts.
    \item For any morphisms $f,g$ such that $gf$ is defined, if two of the three morphisms $f,g,gf$ are in $\W$, so is the third.
    (This is called the \term{2-out-of-3 property})
    \item $\Cof\cap\W\nearrow\Fib,\Cof\nearrow\Fib\cap\W$.
    \item For any morphism $f$, $\alpha f\in\Cof, \beta f\in\Fib\cap\W, \gamma f\in\Cof\cap\W,\delta f\in\Fib$.
    \end{enumerate} 
    Morphisms in $\Cof,\Fib,\W,\Cof\cap\W,\Fib\cap\W$ are called \term{cofibrations, fibrations, weak equivalences, trivial cofibrations, trivial fibrations},
    respectively.
}

\defn{
    A \term{model category} is a category $\catC$ with all small limits and colimits together with a model structure on $\catC$.
}

\eg{
    Suppose $\catC$ is an arbitrary category, then the following structures are all model structures on $\catC$:
    \begin{enumerate}[i)]
    \item $\Cof=\mathrm{Iso}\,\catC,\Fib=\W=\mathrm{Mor}\,\catC,\alpha f=\gamma f=\1_{df},\beta f=\delta f=f$;
    \item $\Fib=\mathrm{Iso}\,\catC,\Cof=\W=\mathrm{Mor}\,\catC,\alpha f=\gamma f=f,\beta f=\delta f=\1_{cf}$;
    \item $\W=\mathrm{Iso}\,\catC,\Cof=\Fib=\mathrm{Mor}\,\catC,\gamma f=\1_{df},\alpha f=\delta f=f,\beta f=\1_{cf}$,
    \end{enumerate}
    where $\mathrm{Mor}\,\catC$ is the class of all morphisms in $\catC$, $\mathrm{Iso}\,\catC$ is the class of all isomorphisms in $\catC$.
    These model structures are called the \term{trivial model structures}.
}

\eg{
    If $\catC_i(i\in I)$ are model categories and $I$ is a set, then $\prod\limits_{i\in I}\catC_i$ is a model category,
    where $\Cof=\prod\limits_{i\in I}\Cof_{\catC_i},\Fib=\prod\limits_{i\in I}\Fib_{\catC_i},\W=\prod\limits_{i\in I}\W_{\catC_i}$.
    The model structure on $\prod\limits_{i\in I}\catC_i$ is called the \term{product model structure}.
}

\eg{
    If $\catC$ is a model category, then $\catC^\op$ is a model category, where $\Cof_{\catC^\op}=\Fib_{\catC}^\op,\Fib_{\catC^\op}=\Cof_{\catC}^\op,
    \W_{\catC^\op}=\W_{\catC}^\op$. This model structure called the \term{dual model structure}.
}

\defn{
    Suppose $\catC$ is a model category. An object $A$ is called \term{cofibrant} if the morphism $\phi\to A$ is a cofibration, where $\phi$ is the
    initial object; it is called \term{fibrant} if the morphism $A\to*$ is a fibration, where $*$ is the terminal object.
}

\defn{
    A category $\catC$ is called \term{pointed} if the initial object and the terminal object are isomorphic.
}

\defn{
    Suppose $\catC$ is a category with terminal object $*$. Define the \term{pointed category} of $\catC$, written $\catC_*$,
    to be the category with objects $\{(X,v)|X\in\catC,v:*\to X\}$, and $\Hom_{\catC_*}((X,v),(Y,w))=\{f:X\to Y|fv=w\}$.
    If $(X,v)\in\catC_*$, we denote $v$ the \term{basepoint} of $X$. Define the functor $-_+:\catC\to\catC_*$ to be the functor
    that takes $X$ to $X\amalg*$ with basepoint $*$.
}

\lem{
    For any category $\catC$ with terminal object $*$, $\catC_*$ is pointed. Moreover, $\catC$ is pointed if and only if $-_+$ is an isomorphism.
    Furthermore, $-_+$ is left adjoint to the forgetful functor $U:\catC_*\to\catC,(X,v)\mapsto X$.
}

\lem{
    For any category $\catC,\catD$ with terminal objects,
    there is a functor $$-_*:\Adj(\catC,\catD)\to\Adj(\catC_*,\catD_*),(F,G,\vp)\mapsto(F_*,G_*,\vp_*),$$
    where $G_*(Y,w)=(GY,Gw)$ and $F_*(X,v)$ is defined by the pushout diagram:
    $$\xymatrix{
    F*\ar[d]\ar[r]^{Fv}&FX\ar[d]\\
    {*}\ar[r]&F_*(X,v)
    }$$
    Furthermore, $F_*(-_+)$ and $(F-)_+$ are naturally isomorphic.
}

\prop{
    If $\catC$ is a model category, then $\catC_*$ is a model category, where $$\Cof_{\catC_*}=U^{-1}(\Cof_{\catC}),\Fib_{\catC_*}=U^{-1}(\Fib_{\catC}),
    \W_{\catC_*}=U^{-1}(\W_{\catC}).$$
}

\defn{
    Suppose $\catC$ is a model category. We define the functor $Q:\catC\to\catC, X\mapsto d\beta(\phi\to X)$ to be
     the \term{cofibrant replacement functor}, and the functor $R:\catC\to\catC, X\mapsto c\gamma(X\to*)$ to be the \term{fibrant replacement functor}.
}

\lem{
    Suppose $\catC$ is a model category. Then there is a natural transformation $q:Q\to\1$ with $q_X=\beta(\phi\to X)$ for any $X$,
    and a natural transformation $r:\1\to R$ with $r_X=\gamma(X\to*)$ for any $X$. Moreover $q_X$ is a trivial fibration and $r_X$
     is a trivial cofibration. Furthermore, $Q,R$ preserves and reflects weak equivalences.
}

\lem{[The Retract Argument]
    Suppose $\catC$ is a category, and we have a factorization $f=pi$. If $f\nearrow p$ then $f$ is a retract of $i$.
     Dually if $i\nearrow f$ then $f$ is a retract of $p$.
}

\lem{
    Suppose $\catC$ is a model category. Then $f\nearrow\Fib\cap\W$ if and only if $f\in\Cof$, $f\nearrow\Fib$ if and only if $f\in\Cof\cap\W$,
    $\Cof\cap\W\nearrow f$ if and only if $f\in\Fib$, $\Cof\nearrow f$ if and only if $f\in\Fib\cap\W$.
}

\cor{
    Suppose $\catC$ is a model category. Then $\Cof$, $\Cof\cap\W$ are closed under pushouts, $\Fib$, $\Fib\cap\W$ are closed under pullbacks.
}

\lem{[Ken Brown's Lemma]
    Suppose $\catC$ is a model category, $\catD$ is a \term{category with weak equivalences}, i.e. a category with a class of morphisms $\W$
    called weak equivalences, such that $\W$ is closed under composition and satisfies 2-out-of-3 property.
    If $F:\catC\to\catD$ is a functor that takes trivial cofibrations between cofibrant objects to weak equivalences,
    then $F$ takes weak equivalences between cofibrant objects to weak equivalences.
    Dually if $F:\catC\to\catD$ is a functor that takes trivial fibrations between fibrant objects to weak equivalences,
    then $F$ takes weak equivalences between fibrant objects to weak equivalences.
}

\defn{
    Suppose $\catC$ is a model category. Define $\catC_c,\catC_f$ and $\catC_{cf}$ to be the full subcategory of cofibrant, fibrant,
    cofibrant and fibrant objects of $\catC$, respectively.
}

\defn{
    Suppose $\catC$ is a model category, $A\in\catC$. $A'$ is called a \term{cylinder object of $A$} if the map $A\amalg A\to A$ factors through $A'$:
    $A\amalg A\xrightarrow{i_0+i_1}A'\xrightarrow{\sigma}A$, such that $i_0+i_1\in\Cof$ and $\sigma\in\W$. $A''$ is called a \term{path object of $A$}
    if the map $A\to A\times A$ factors through $A''$: $A\xrightarrow{s}A''\xrightarrow{(p_0,p_1)}A\times A$, such that $s\in\W$ and $(p_0,p_1)\in\Fib$.
}

\lem{
    Suppose $\catC$ is a model category, $A\in\catC$, if $A'$ is a cylinder object or a path object of $A$ in $\catC$,
    then $A'$ is a path object or a cylinder object of $A$ in $\catC^\op$, respectively.
}

\lem{
    Suppose $\catC$ is a model category. There is a functor $\Cyl:\catC\to\catC,A\mapsto d\circ\beta(A\amalg A\to A)$, such that $\Cyl(A)$ 
    is a cylinder object, and the structure map $\sigma$ is a trivial fibration. Dually there is a functor $\Path:\catC\to\catC,
    A\mapsto c\circ\gamma(A\to A\times A)$, such that $\Path(A)$ is a path object, and the structure map $s$ is a trivial cofibration.
}

\lem{
    Suppose $\catC$ is a model category, $A\in\catC$. Then for any cylinder object $A'$ of $A$, there exists a map $A'\to\Cyl(A)$
    which is compatible with $i_0,i_1,\sigma$. Dually for any path object $A''$ of $A$, there exists a map $\Path(A)\to A''$ 
    which is compatible with $p_0,p_1,s$.
}

\defn{
    Suppose $\catC$ is a model category, $f,g:A\to X$ are morphisms in $\catC$. A \term{left homotopy} from $f$ to $g$ is a map $H:A'\to X$,
    where $A'$ is some cylinder object of $A$ and $H(i_0+i_1)=f+g$. If such $H$ exists for some cylinder object $A'$ we say $f$ 
    is \term{left homotopic} to $g$, denoted $f\siml g$. Dually a \term{right homotopy} from $f$ to $g$ is a map $K:A\to X'$, 
    where $X'$ is some path object of $X$ and $(p_0,p_1)K=(f,g)$. If such $K$ exists for some path object $K'$ we say $f$ 
    is \term{right homotopic} to $g$, denoted $f\simr g$. Finally we say $f$ is \term{homotopic} to $g$, denoted $f\simeq g$,
    if $f\siml g$ and $f\simr g$. Moreover we call $f$ a \term{homotopy equivalence} if there exists $h:X\to A$ such that $fh\simeq\1_X,hf\simeq\1_B$.
}

\prop{
    Suppose $\catC$ is a model category, $f,g:B\to X,h:A\to B,k:X\to Y$ are morphisms in $\catC$.
    If $f\siml g$ then $kf\siml kg$, and if in addition $X$ is fibrant then $fh\siml gh$.
    Dually if $f\simr g$ then $fh\simr gh$, and if in addition $B$ is cofibrant then $kf\simr kg$.
}

\prop{
    Suppose $\catC$ is a model category, $A,X\in\catC$. If $A$ is cofibrant then $\siml$ is a equivalence relation on $\Hom_\catC(A,X)$.
    Dually if $X$ is fibrant then $\simr$ is a equivalence relation on $\Hom_\catC(A,X)$.
}

\prop{
    Suppose $\catC$ is a model category, $h:A\to B,k:X\to Y$ are morphisms in $\catC$.
    If $B$ is cofibrant and $k$ is a trivial fibration or a weak equivalence between fibrant objects, then 
    $k_*:\Hom_\catC(B,X)/\siml\to\Hom_\catC(B,Y)/\siml$ is a bijection. Dually if $X$ is fibrant and $h$ is a trivial cofibration
    or a weak equivalence between cofibrant objects, then $h^*:\Hom_\catC(B,$ $X)/\simr\to\Hom_\catC(A,X)/\simr$ is a bijection.
}

\prop{
    Suppose $\catC$ is a model category, $f,g:A\to X$ are morphisms in $\catC$.
    If $A$ is cofibrant and $f\siml g$ then $f\simr g$, and the right homotopy may pass any path object of $X$.
    Dually if $X$ is fibrant and $f\simr g$ then $f\siml g$, and the left homotopy may pass any cylinder object of $A$.
}

\cor{
    If $\catC$ is a model category, then $\simeq$ is an equivalence relation on morphisms of $\catC_{cf}$ and is compatible with composition.
}

\prop{[Whitehead]
    If $\catC$ is a model category, then a map in $\catC_{cf}$ is a weak equivalence if and only if it is a homotopy equivalence.
}

\defn{
    Suppose $\catC$ is a category with a class of morphisms $\W$ such that $\W$ is closed under composition. For any $X,Y\in\catC$,
    define $[X,Y]$ to be the class $F(X,Y)/\sim$, where $F(X,Y)$ is the class of all strings $(f_1,\cdots,f_n)$ of morphisms in $\catC$
    or reversals of morphisms in $\W$ of finite length, such that $df_1=X,cf_n=Y,df_i=cf_{i-1}$ for any $1<i\le n$; $\sim$
    is the equivalence relation generated from $()\sim(\1)$, $(f,g)\sim(g\circ f)$, $(w^{-1},v^{-1})\sim((w\circ v)^{-1})$, 
    $()\sim(w,w^{-1})$, $()\sim(w^{-1},w)$ for any $f,g$ morphisms in $\catC$ and $w\in\W$, and $(W_1,W_2,W_4)\sim(W_1,W_3,W_4)$ 
    whenever $W_1,W_2,W_3,W_4$ are strings and $W_2\sim W_3$. If for any $X,Y\in\catC$, $[X,Y]$ is a set,
    define the \term{homotopy category} $\Ho\catC$ to be the category with objects being objects in $\catC$ and $\Hom_{\Ho\catC}(X,Y)=[X,Y]$.
    We denote the obvious functor $\catC\to\Ho\catC$ by $\gamma$.
}

\lem{
    Suppose $\catC$ is a category with a class of morphisms $\W$ such that $\W$ is closed under composition. If $\Ho\catC$ exists 
    and $F:\catC\to\catD$ is a functor which sends morphisms in $\W$ to isomorphisms, then there exists a unique functor 
    $\Ho F:\Ho\catC\to\catD$ such that $(\Ho F)\gamma=F$. Conversely if $\alpha:\catC\to\catE$ is a functor which sends morphisms 
    in $\W$ to isomorphisms and satisfies the above universal property, then $\Ho\catC$ exists and there exists a unique isomorphism 
    $\delta:\Ho\catC\to\catE$ such that $\delta\gamma=\alpha$.
}

\lem{
    Suppose $\catC$ is a category with a class of morphisms $\W$ such that $\W$ is closed under composition, and $\Ho\catC$ exists.
    If $\tau:F\to G$ is a natural transformation between functors which sends morphisms in $\W$ to isomorphisms,
    then $\Ho\tau:\Ho F\to\Ho G$ with $\Ho\tau_X=\tau_X$ is a natural transformation.
}

\prop{
    Suppose $\catC$ is a model category, then $\Ho\catC_{cf}$ exists and there exists a unique isomorphism $\Ho\catC_{cf}\to\catC_{cf}/\!\simeq$,
    such that the isomorphism is identity on objects.
}

\prop{
    Suppose $\catC$ is a model category. Then the inclusions induces isomorphisms $[X,Y]_{\catC_{cf}}\cong[X,Y]_{\catC_{c}}\cong[X,Y]_{\catC}$
    and $[X,Y]_{\catC_{cf}}\cong[X,Y]_{\catC_{f}}\cong[X,Y]_{\catC}$ for any $X,Y\in\catC_{cf}$.
}

\thm{
    Suppose $\catC$ is a model category. Then:
    \begin{enumerate}[i)]
    \item $\Ho\catC$ exists, and we have equivalences between categories $\Ho\catC_{cf}\to\Ho\catC_{c}\to\Ho\catC$
    and $\Ho\catC_{cf}\to\Ho\catC_{f}\to\Ho\catC$ induced by inclusions;
    \item We have for any $X,Y\in\catC$, $$\begin{aligned}
    {[X,Y]} &\cong(\Hom_\catC(QRX,QRY)/\simeq)\\&\cong(\Hom_{\catC}(RQX,RQY)/\simeq)\\&\cong(\Hom_\catC(QX,RY)/\simeq),
    \end{aligned}$$
    and if moreover $X$ is cofibrant and $Y$ is fibrant, then $[X,Y]=\Hom_\catC(X,Y)/\simeq$;
    \item $\gamma:\catC\to\Ho\catC$ identifies maps that are left homotopic or right homotopic;
    \item If $f$ is a morphism in $\catC$ such that $\gamma f$ is an isomorphism, then $f$ is a weak equivalence.
    \end{enumerate}
}

\subsection{Quillen Adjunctions and Derived Functors}

\defn{
    Suppose $\catC$ and $\catD$ are categories, $(F,G,\vp):\catC\to\catD$ is an adjunction. Define its \term{unit} to be the natural transformation
    $\eta:\1\to GF,\eta_X=\vp(FX\to FX)$, and its \term{counit} to be the natural transformation $\ve:FG\to\1,\ve_Y=\vp^{-1}(GY\to GY)$.
}

\defn{
    Suppose $\catC$ and $\catD$ are model categories.
    A functor $F:\catC\to\catD$ is called a \term{left Quillen functor} if it is a left adjoint and preserves cofibrations and trivial cofibrations.
    Dually a functor $G:\catD\to\catC$ is called a \term{right Quillen functor} if it is a right adjoint and preserves fibrations and trivial fibrations.
    An adjunction $(F,G,\vp):\catC\to\catD$ is called a \term{Quillen adjunction} if $F$ is left Quillen.
}

\eg{
    Suppose $\catC$ is a model category and $I$ is a set, then the coproduct functor and the diagonal functor form a Quillen adjunction $\catC^I\to\catC$,
    and the diagonal functor and the product functor form a Quillen adjunction $\catC\to\catC^I$.
}

\eg{
    Suppose $\catC$ is a model category, then $(-_+,U,\vp):\catC\to\catC_*$ is a Quillen adjunction.
}

\lem{
    If $\catC$ and $\catD$ are model categories, then an adjunction $(F,G,\vp):\catC\to\catD$ is a Quillen adjunction if and only if $G$ is right Quillen.
}

\lem{
    Quillen adjunctions are closed under compositions.
}

\lem{
    If $\catC$ and $\catD$ are model categories and $(F,G,\vp):\catC\to\catD$ is a Quillen adjunction, then $(G,F,\vp^{-1}):\catD^\op\to\catC^\op$ 
    is a Quillen adjunction.
}

\prop{
    If $\catC$ and $\catD$ are model categories and $(F,G,\vp):\catC\to\catD$ is a Quillen adjunction, then $(F_*,G_*,\vp_*):\catC_*\to\catD_*$ 
    is a Quillen adjunction.
}

\defn{
    Suppose $\catC$ and $\catD$ are model categories. The \term{left derived functor} of a left Quillen functor $F:\catC\to\catD$, denoted $LF$,
    is the composite $\Ho\catC\xrightarrow{\Ho Q}\Ho\catC_{c}\xrightarrow{\Ho F}\Ho\catD$. The \term{derived natural transformation}
    of a natural transformation $\tau:F\to F'$ between left Quillen functors, denoted $L\tau$, is the composite $\Ho\tau\circ\Ho Q$.
    Dually, the \term{right derived functor} of a right Quillen functor $G:\catD\to\catC$, denoted $RG$, is the composite
    $\Ho\catD\xrightarrow{\Ho R}\Ho\catD_{f}\xrightarrow{\Ho G}\Ho\catC$. The \term{derived natural transformation} of a natural transformation
     $\tau:G\to G'$ between right Quillen functors, denoted $R\tau$, is the composite $\Ho\tau\circ\Ho R$.
}

\lem{
    If $\tau$ is a natural transformation between left Quillen functors, then $L\tau_X=\tau_{QX}$.
    Dually if $\tau$ is a natural transformation between right Quillen functors, then $R\tau_X=\tau_{RX}$.
}

\lem{
    If $\tau:F\to F'$, $\tau':F'\to F''$ are natural transformations between left Quillen functors, then $L(\tau'\circ\tau)=(L\tau')\circ(L\tau)$,
     and $L(\1_F)=\1_{LF}$. Dually if $\tau:G\to G'$, $\tau':G'\to G''$ are natural transformations between right Quillen functors, 
     then $R(\tau'\circ\tau)=(R\tau')\circ(R\tau)$, and $R(\1_G)=\1_{RG}$.
}

\thm{
    For any model categories $\catC,\catD,\catE,\catF$ and left Quillen functors $F:\catC\to\catD,F':\catD\to\catE,F'':\catE\to\catF$,
    there exists natural isomorphisms $\alpha_\catC:L(\1_\catC)\to\1_{\Ho\catC}$ and $\mu_{F'F}:LF'\circ LF\to L(F'\circ F)$,
    given by $(\alpha_\catC)_X:QX\xrightarrow{q}X$ and $(\mu_{F'F})_X:F'QFQX\xrightarrow{F'q_{FQX}}F'FQX$,
    such that the following three diagrams commute:
    $$\xymatrix @C=50pt{
    (LF''\circ LF')\circ LF\ar@{=}[d]\ar[r]^{\mu_{F''F'}\circ LF}&L(F''\circ F')\circ LF\ar[r]^{\mu_{(F''\circ F')F}}&L((F''\circ F')\circ F)\ar@{=}[d]\\
    LF''\circ (LF'\circ LF)\ar[r]^{LF''\circ \mu_{F'F}}&LF''\circ L(F'\circ F)\ar[r]^{\mu_{F''(F'\circ F)}}&L(F''\circ(F'\circ F))
    }$$
    $$\xymatrix @C=40pt{
    L(\1_\catD)\circ LF\ar[d]_{\alpha\circ LF}\ar[r]^{\mu_{\1_\catD F}}&L(\1_\catD\circ F)\ar@{=}[d]\\
    \1_{\Ho\catD}\circ LF\ar@{=}[r]&LF
    }\xymatrix @C=40pt{
    LF\circ L(\1_\catC)\ar[d]_{LF\circ\alpha}\ar[r]^{\mu_{F\1_\catC}}&L(F\circ\1_\catC)\ar@{=}[d]\\
    LF\circ\1_{\Ho\catC}\ar@{=}[r]&LF
    }$$
    Dually there is a statement for right derived functors.
}

\prop{
    Suppose $\sigma:F\to G$ is a natural transformation between two left Quillen functors $\catC\to\catD$,
    $\tau:F'\to G'$ is a natural transformation between two left Quillen functors $\catD\to\catE$,
    then the following diagram is commtative:
    $$\xymatrix @C=40pt{
    LF'\circ LF\ar[d]_{L\tau*L\sigma}\ar[r]^{\mu_{F'F}}&L(F'\circ F)\ar[d]^{L(\tau*\sigma)}\\
    LG'\circ LG\ar[r]^{\mu_{G'G}}&L(G'\circ G)
    }$$
    Dually there is a statement for right derived functors.
}

\prop{
    If $\catC$ and $\catD$ are model categories and $(F,G,\vp):\catC\to\catD$ is a Quillen adjunction, then $(LF,RG,\Ho\vp):\Ho\catC\to\Ho\catD$
    is an adjunction. We denote $\Ho(F,G,\vp)=(LF,RG,\Ho\vp)$ and call it the \term{derived adjunction}.
}

\eg{
    If $\catC$ is a model category, then the right derived functor of the product functor on $\catC$ is a product functor on $\Ho\catC$.
    Dually the left derived functor of the coproduct functor on $\catC$ is a coproduct functor on $\Ho\catC$.
}

\defn{
    Suppose $\catC$ and $\catD$ are model categories. A Quillen adjunction is called a \term{Quillen equivalence},
    if for any $X$ cofibrant in $\catC$ and $Y$ fibrant in $\catD$, a map $f:FX\to Y$ is a weak equivalence if and only if $\vp(f):X\to GY$ 
    is a weak equivalence.
}

\prop{
    Suppose $\catC$ and $\catD$ are model categories and $(F,G,\vp):\catC\to\catD$ is a Quillen adjunction. Then the following statements are equivalent:
    \begin{enumerate}[i)]
    \item $(F,G,\vp)$ is a Quillen equivalence.
    \item For any $X$ cofibrant in $\catC$ the map $X\xrightarrow{\eta}GFX\xrightarrow{Gr_{FX}}GRFX$ is a weak equivalence,
    for any $Y$ fibrant in $\catD$ the map $FQGY\xrightarrow{Fq_{GY}}FGY\xrightarrow{\ve}Y$ is a weak equivalence.
    \item $\Ho(F,G,\vp)$ is an adjoint equivalence of categories.
    \end{enumerate}
}

\cor{
    Suppose $(F,G),(F',G),(F,G')$ are Quillen adjunctions between model categories $\catC$ and $\catD$. If one of the three adjunctions 
    is a Quillen equivalence, so are the other two.
}

\cor{
    Suppose $F:\catC\to\catD,G:\catD\to\catE$ are left (resp. right) Quillen functors. If two of the functors $F,G,GF$ are Quillen equivalences, 
    so is the third.
}

\cor{
    Suppose $\catC$ and $\catD$ are model categories and $(F,G,\vp):\catC\to\catD$ is a Quillen adjunction. Then the following statements are equivalent:
    \begin{enumerate}[i)]
    \item $(F,G,\vp)$ is a Quillen equivalence.
    \item For any morphism $f$ between cofibrant objects in $\catC$ with $Ff$ a weak equivalence, $f$ is a weak equivalence;
    for any $Y$ fibrant in $\catD$ the map $FQGY\xrightarrow{Fq_{GY}}FGY\xrightarrow{\ve}Y$ is a weak equivalence. 
    \item For any morphism $g$ between fibrant objects in $\catD$ with $Gg$ a weak equivalence, $g$ is a weak equivalence;
    for any $X$ cofibrant in $\catC$ the map $X\xrightarrow{\eta}GFX\xrightarrow{Gr_{FX}}GRFX$ is a weak equivalence.
    \end{enumerate}
}

\prop{
    Suppose $\catC$ and $\catD$ are model categories and $F:\catC\to\catD$ is a Quillen equivalence. If the terminal object $*$ of $\catC$ is cofibrant, 
    and $F$ preserves terminal object, then $F_*:\catC_*\to\catD_*$ is a Quillen equivalence.
}

\prop{
    Suppose $\tau:F\to G$ is a natural transformation between left (resp. right) Quillen functors. Then $L\tau$ (resp. $R\tau$) is a natural equivalence
    if and only if for any cofibrant (resp. fibrant) $X$, $\tau_X$ is a weak equivalence.
}

\subsection{2-Categories}

\defn{
    A \term{2-category} $\cattC$ consists of three superclasses $\cattC_0,\cattC_1,$ $\cattC_2$, each called the objects, the 1-morphisms (or morphisms)
    and the 2-morphisms, two maps $d:\cattC_1\to\cattC_0,d:\cattC_2\to\cattC_1$ called the domain maps, two maps 
    $c:\cattC_1\to\cattC_0,c:\cattC_2\to\cattC_1$ called the codomain maps, two maps $\1:\cattC_0\to\cattC_1,\cattC_1\to\cattC_2$ called the identity maps,
    and three maps $\circ:\cattC_1\times_{\cattC_0}\cattC_1\to\cattC_1,\circ:\cattC_2\times_{\cattC_1}\cattC_2\to\cattC_2,
    *:\cattC_2\times_{\cattC_0}\cattC_2\to\cattC_2$, each called the composite, the vertical composite and the horizontal composite,
    where the pullbacks are taken along the maps $c,d$, the maps $c,d$ and the maps $c^2,d^2$, respectively,
    satisfying the following conditions:
    \begin{enumerate}[i)]
    \item As maps $\cattC_2\to\cattC_0$ we have $dc=dd$, $cd=cc$;
    \item Either as a map $\cattC_0\to\cattC_0$ or as a map $\cattC_1\to\cattC_1$ we have $d\,\1$ and $c\,\1$ are the identity maps;
    \item If $f,g$ are 1-morphisms such that $dg=cf$ then $d(g\circ f)=df$ and $c(g\circ f)=cg$;
    if $\sigma,\tau$ are 2-morphisms such that $d\tau=c\sigma$ then $d(\tau\circ\sigma)=d\sigma$ and $c(\tau\circ\sigma)=c\tau$;
    if $\sigma,\tau$ are 2-morphisms such that $d^2\tau=c^2\sigma$ then $d(\tau*\sigma)=d\tau\circ d\sigma$ and $c(\tau\circ\sigma)=c\tau\circ c\sigma$;
    \item For any 1-morphism $f$ we have $\1_{cf}\circ f=f=f\circ\1_{df}$; for any 2-morphism $\sigma$ we have 
    $\1_{c\sigma}\circ\sigma=\sigma=\sigma\circ\1_{d\sigma}$, and $\1_{\1_{cc\sigma}}*\sigma=\sigma=\sigma*\1_{\1_{dd\sigma}}$;
    \item If $f,g,h$ are 1-morphisms such that $dg=cf,dh=cg$ then $(h\circ g)\circ f=h\circ(g\circ f)$;
    if $\rho,\sigma,\tau$ are 2-morphisms such that $d\sigma=c\rho,d\tau=c\sigma$ then $(\tau\circ\sigma)\circ\rho=\tau\circ(\sigma\circ\rho)$;
    if $\rho,\sigma,\tau$ are 2-morphisms such that $d^2\sigma=c^2\rho,d^2\tau=c^2\sigma$ then $(\tau*\sigma)*\rho=\tau*(\sigma*\rho)$;
    \item If $\sigma,\tau,\sigma',\tau'$ are 2-morphisms such that $d\tau=c\sigma,d\tau'=c\sigma',d^2\tau=c^2\sigma'$,
    then $(\tau\circ\sigma)*(\tau'\circ\sigma')=(\tau*\tau')\circ(\sigma*\sigma')$;
    \item For any objects $A,B$, $\Hom_\cattC(A,B):=\{f\in\cattC_1|df=A,cf=B\}$ is a class; for any 1-morphisms $f,g$, 
    $\Hom_\cattC(f,g):=\{\tau\in\cattC_2|d\tau=f,c\tau=g\}$ is a class.
    \end{enumerate}
    If $f\in\Hom_\cattC(A,B)$ where $A,B$ are objects, we denote $f:A\to B$, and if $\tau\in\Hom_\cattC(f,g)$ where $f,g$ are 1-morphisms, 
    we denote $\tau:f\to g$.
}

\eg{
    All categories, functors, natural transformations form a 2-category which we denote $\catt{Cat}$.
    All categories, adjunctions, natural transformations between left adjoints form a 2-category which we denote $\catt{Cat}_{\mathit{ad}}$.
    All model categories, Quillen adjunctions, natural transformations between left Quillen functors form a 2-category which we denote $\catt{Model}$.
}

\defn{
    Suppose $\cattC$ and $\cattD$ are 2-categories. A \term{(covariant) 2-functor} $F:\cattC\to\cattD$ consists of three maps
    $\cattC_0\to\cattD_0,\cattC_1\to\cattD_1,\cattC_2\to\cattD_2$, satisfying the following conditions:
    \begin{enumerate}[i)]
    \item Either as maps $\cattC_1\to\cattD_0$ or as maps $\cattC_2\to\cattD_1$ we have $Fd=dF,Fc=cF$;
    \item For any object $A$ we have $F(\1_A)=\1_{FA}$, for any 1-morphism $f$ we have $F(\1_f)=\1_{Ff}$;
    \item If $f,g$ are 1-morphisms such that $dg=cf$ then $F(g\circ f)=Fg\circ Ff$;
    if $\sigma,\tau$ are 2-morphisms such that $d\tau=c\sigma$ then $F(\tau\circ\sigma)=F\tau\circ F\sigma$;
    if $\sigma,\tau$ are 2-morphisms such that $d^2\tau=c^2\sigma$ then $F(\tau*\sigma)=F\tau*F\sigma$.
    \end{enumerate}
    A \term{contravariant 2-functor} $F:\cattC\to\cattD$ consists of three maps $\cattC_0\to\cattD_0,\cattC_1\to\cattD_1,\cattC_2\to\cattD_2$,
    satisfying the following conditions:
    \begin{enumerate}[i)]
    \item Either as maps $\cattC_1\to\cattD_0$ or as maps $\cattC_2\to\cattD_1$ we have $Fd=cF,Fc=dF$;
    \item For any object $A$ we have $F(\1_A)=\1_{FA}$, For any 1-morphism $f$ we have $F(\1_f)=\1_{Ff}$;
    \item If $f,g$ are 1-morphisms such that $dg=cf$ then $F(g\circ f)=Ff\circ Fg$;
    if $\sigma,\tau$ are 2-morphisms such that $d\tau=c\sigma$ then $F(\tau\circ\sigma)=F\sigma\circ F\tau$;
    if $\sigma,\tau$ are 2-morphisms such that $d^2\tau=c^2\sigma$ then $F(\tau*\sigma)=F\sigma*F\tau$.
    \end{enumerate}
}

\eg{
    $\catt{Cat}_{\mathit{ad}}\to\catt{Cat},\catC\mapsto\catC,(F,G,\vp)\mapsto F,\tau\mapsto\tau$ is a covariant 2-functor. 
    $\catt{Model}\to\catt{Cat}_{\mathit{ad}},\catC\mapsto\catC,(F,G,\vp)\mapsto(F,G,\vp),\tau\mapsto\tau$ is a covariant 2-functor.
}

\defn{
    Given a natural transformation $\tau:F\to F'$ between adjunctions $(F,G)$ and $(F',G')$, we define its \term{dual natural transformation}
    $\tau^\op:G'\to G$ by the composite $G'X\xrightarrow{\eta_{G'X}}GFG'X\xrightarrow{G\tau_{G'X}}GF'G'X\xrightarrow{G\ve'_{X}}GX.$
}

\lem{
    If $\tau:F\to F'$ is a natural transformation between adjunctions $(F,G,\vp)$ and $(F',G',\vp')$, then the following diagram is commutative 
    for any objects $X\in\catC,Y\in\catD$:
    $$\xymatrix @C=40pt{
    \Hom_\catD(F'X,Y)\ar[d]_{\vp'}\ar[r]^{\tau_X^*}& \Hom_\catD(FX,Y)\ar[d]^\vp\\
    \Hom_\catC(X,G'Y)\ar[r]^{(\tau^\op_Y)_*}& \Hom_\catC(X,GY)
    }$$ 
    In particular, $\tau$ is a natural equivalence if and only if $\tau^\op$ is.
}

\lem{
    $$-^\op:\left\{\begin{aligned}\catt{Cat}_{\mathit{ad}}&\to\catt{Cat}_{\mathit{ad}},\\\catC&\mapsto\catC^\op,\\(F,G,\vp)&\mapsto(G,F,\vp^{-1}),
    \\(\tau:(F,G)\to(F',G'))&\mapsto\tau^\op\\\end{aligned}\right.$$ is a contravariant 2-functor,
    and $(-^\op)^2$ is the identity 2-functor. Similarly $$-^\op:\left\{\begin{aligned}\catt{Model}&\to\catt{Model},\\\catC&\mapsto\catC^\op,
    \\(F,G,\vp)&\mapsto(G,F,\vp^{-1}),\\(\tau:(F,G)\to(F',G'))&\mapsto\tau^\op\\\end{aligned}\right.$$
    is a contravariant 2-functor, and $(-^\op)^2$ is the identity 2-functor. They are called the \term{duality 2-functors}.
}

\defn{
    Suppose $\cattC$ and $\cattD$ are 2-categories. A \term{pseudo-2-functor} $F:\cattC\to\cattD$ consists of three maps
    $\cattC_0\to\cattD_0,\cattC_1\to\cattD_1,\cattC_2\to\cattD_2$, and 2-isomorphisms $\alpha_A:F(\1_A)\to\1_{FA}$ for any object $A$, 2-isomorphisms
    $\mu_{gf}:Fg\circ Ff\to F(g\circ f)$ for any 1-morphisms $f,g$ with $dg=cf$, satisfying the following conditions:
    \begin{enumerate}[i)]
    \item Either as maps $\cattC_1\to\cattD_0$ or as maps $\cattC_2\to\cattD_1$ we have $Fd=dF,Fc=cF$;
    \item For any 1-morphism $f$ we have $F(\1_f)=\1_{Ff}$; if $\sigma,\tau$ are 2-morphisms such that $d\tau=c\sigma$ then 
    $F(\tau\circ\sigma)=F\tau\circ F\sigma$;
    \item If $f,g,h$ are 1-morphisms such that $dg=cf,dh=cg$ then the following diagrams are commutative:
    $$\xymatrix @C=50pt{
    (Fh\circ Fg)\circ Ff\ar@{=}[d]\ar[r]^{\mu_{hg}\circ Ff}&F(h\circ g)\circ Ff\ar[r]^{\mu_{(h\circ g)f}}&F((h\circ g)\circ f)\ar@{=}[d]\\
    Fh\circ (Fg\circ Ff)\ar[r]^{Fh\circ \mu_{gf}}&Fh\circ F(g\circ f)\ar[r]^{\mu_{h(g\circ f)}}&F(h\circ(g\circ f))
    }$$
    $$\xymatrix @C=40pt{
    F(\1_{cf})\circ Ff\ar[d]_{\alpha*\1_{Ff}}\ar[r]^{\mu_{\1_{cf}f}}&F(\1_{cf}\circ f)\ar@{=}[d]\\
    \1_{F(cf)}\circ Ff\ar@{=}[r]&Ff
    }\xymatrix @C=40pt{
    Ff\circ F(\1_{df})\ar[d]_{\1_{Ff}*\alpha}\ar[r]^{\mu_{f\1_{df}}}&F(f \circ\1_{df})\ar@{=}[d]\\
    Ff\circ\1_{F(df)}\ar@{=}[r]&Ff
    }$$
    \item if $\sigma:f\to f',\tau:g\to g'$ are 2-morphisms such that $d^2\tau=c^2\sigma$ then the following diagram is commutative:
    $$\xymatrix @C=40pt{
    Fg\circ Ff\ar[d]_{F\tau*F\sigma}\ar[r]^{\mu_{gf}}&F(g\circ f)\ar[d]^{F(\tau*\sigma)}\\
    Fg'\circ Ff'\ar[r]^{\mu_{g'f'}}&F(g'\circ f')
    }$$
    \end{enumerate}
}

\eg{
    We have $$\Ho:\catt{Model}\to\catt{Cat}_{\mathit{ad}},\catC\mapsto\Ho\catC,(F,G,\vp)\mapsto\Ho(F,G,\vp),\tau\mapsto L\tau$$ is a pseudo-2-functor. 
    Moreover $\Ho\circ(-^\op)=(-^\op)\circ\Ho$.
}

\defn{
    Suppose $\cattC,\cattD$ are 2-categories, $F,G:\cattC\to\cattD$ are pseudo-2-functors. A \term{2-natural transformation} $\tau:F\to G$
    consists of a map $\tau_X:FX\to GX$ for any $X\in\cattC_0$, such that for any $f\in\cattC_1$, $Gf\circ\tau_{df}=\tau_{cf}\circ Ff$, 
    and for any $\alpha\in\cattC_2$, $G\alpha*\1_{\tau_{d^2\alpha}}=\1_{\tau_{c^2\alpha}}*F\alpha$. A \term{2-natural isomorphism} 
    is a 2-natural transformation $\tau$ such that $\tau_X$ is an isomorphism for any $X\in\cattC_0$.
}

\section{Examples of Model Categories}

\subsection{Cofibrantly Generated Model Categories}

\defn{
    Suppose $\catC$ is a category with all small colimits and $\lambda$ is a ordinal. A \term{$\lambda$-sequence} $X$ is a colimit-preserving functor
    $\lambda\to\catC$. We denote $X(\alpha)$ as $X_\alpha$. A \term{transfinite composition} of a $\lambda$-sequence is the induced map
    $X_0\to\colim_{\beta<\lambda}X_\beta$. If $I$ is a collection of maps of $\catC$ and for any $\beta+1<\lambda$ the map $X_\beta\to X_{\beta+1}$
    is in $I$, we call the $\lambda$-sequence to be in $I$ and transfinite composition a transfinite composition of maps of $I$.
}

\defn{
    Suppose $\gamma$ is a cardinal. A \term{$\gamma$-filtered ordinal} is a limit ordinal $\alpha$ such that for any $A\subseteq\alpha,|A|\le\gamma$
    we have $\sup A<\alpha$.
}

\defn{
    Suppose $\catC$ is a category with all small colimits, $I$ is a collection of maps of $\catC$, $A$ is an object in $\catC$ and $\kappa$ is a cardinal.
    We say $A$ is \term{$\kappa$-small relative to $I$} if for any $\kappa$-filtered ordinal $\lambda$ and all $\lambda$-sequence $X$ in $I$,
    the map of sets $$\colim_{\beta<\lambda}\Hom_\catC(A,X_\beta)\to\Hom_\catC(A,\colim_{\beta<\lambda}X_\beta)$$ is an isomorphism.
    We say $A$ is \term{small relative to $I$} if there exists some $\kappa$ such that $A$ is $\kappa$-small relative to $I$.
    We say $A$ is \term{small} if A is small relative to $\catC$. We say $A$ is \term{finite relative to $I$} if there exists some finite cardinal
    $\kappa$ such that $A$ is $\kappa$-small relative to $I$. We say $A$ is \term{finite} if A is finite relative to $\catC$.
}

\eg{
    In the category $\cat{Set}$ every object $A$ is $|A|$-small.
}

\eg{
    If $R$ is a ring, in the category $\cat{Mod}_R$ of left $R$-modules every object is small, every finitely presented $R$-module is finite.
}

\rmk{
    In the category $\cat{Top}$ not all objects are small; For example, the indiscrete space with 2 points.
}

\defn{
    Suppose $\catC$ is a category, $I$ is a collection of maps of $\catC$. 
    Denote $\RLP(I)$ to be the class of all maps that has RLP with respect to every may in $I$, and $\LLP(I)$
    to be the class of maps that has LLP with respect to every may in $I$. Denote 
    $\Cof(I)=\LLP(\RLP(I))$ and $\Fib(I)=\RLP(\LLP(I))$. We call the maps in $\Cof(I)$ to be
    \term{$I$-cofibrations} and $\Fib(I)$ to be \term{$I$-fibrations}.
}

\lem{
    Suppose $\catC$ is a category, $I$ is a collection of maps of $\catC$. Then $I\subseteq\Cof(I),I\subseteq\Fib(I)$. Moreover $\RLP(\Cof(I))=\RLP(I)$ 
    and $\LLP(\Fib(I))=\LLP(I)$.
}

\lem{
    Suppose $(F,G)$ is an adjunction between $\catC$ and $\catD$, $I$ is a collection of maps of $\catC$, $J$ is a collection of maps of $\catD$.
    Then we have $G(\RLP(FI))\subseteq\RLP(I),F(\Cof(I))\subseteq\Cof(FI),F(\LLP(GJ))\subseteq\LLP(J),G(\Fib(J))\subseteq\Fib(GJ)$.
}

\defn{
    Suppose $\catC$ is a category with all small colimits and $I$ is a collection of maps of $\catC$. A \term{relative $I$-cell complex}
    is a transfinite composition of pushouts of elements of $I$. Denote $\Cell(I)$ to be the class of all relative $I$-cell complexes. An object $A$ 
    is called an \term{$I$-cell complex} if the map $\phi\to A$ is a relative $I$-cell complex.
}

\lem{
    Suppose $\catC$ is a category with all small colimits and $I$ is a collection of maps of $\catC$. Then $\Cell(I)\subseteq\Cof(I)$.
}

\lem{
    Suppose $\catC$ is a category with all small colimits and $I$ is a collection of maps of $\catC$. Then transfinite composition of pushouts
    of elements of $I$ or isomorphisms is in $\Cell(I)$.
}

\lem{
    Suppose $\catC$ is a category with all small colimits and $I$ is a collection of maps of $\catC$. Then $\Cell(I)$ is closed
     under transfinite composition.
}

\lem{
    Suppose $\catC$ is a category with all small colimits and $I$ is a collection of maps of $\catC$. Then any pushout of coproducts of maps of $I$
    is in $\Cell(I)$.
}

\thm{[The Small Object Argument]
    Suppose $\catC$ is a category with all small colimits and $I$ is a set of maps of $\catC$. Suppose the domains of maps of $I$ are small
    relative to $\Cell(I)$. Then there exists a functorial factorization $(\gamma,\delta)$ on $\catC$ such that for any $f$ a morphism in $\catC$,
    $\gamma(f)\in\Cell(I)$ and $\delta(f)\in\RLP(I)$.
}

\cor{
    Suppose $\catC$ is a category with all small colimits and $I$ is a set of maps of $\catC$. Suppose the domains of maps of $I$ are small 
    relative to $\Cell(I)$. Then for any $f:A\to B\in\Cof(I)$ there exists $g:A\to C\in\Cell(I)$ such that $f$ is a retract of $g$, and the retract fixes $A$.
}

\prop{
    Suppose $\catC$ is a category with all small colimits and $I$ is a set of maps of $\catC$. Suppose the domains of maps of $I$ are small 
    relative to $\Cell(I)$. Then any object that is small relative to $\Cell(I)$ is small relative to $\Cof(I)$.
}

\defn{
    Suppose $\catC$ is a model category. $\catC$ is called \term{cofibrantly generated} if there exist sets of maps $I$ and $J$ of $\catC$,
    such that:
    \begin{enumerate}[i)]
    \item The domains of maps of $I$ are small relative to $\Cell(I)$;
    \item The domains of maps of $J$ are small relative to $\Cell(J)$;
    \item $\Fib=\RLP(J)$;
    \item $\W\cap\Fib=\RLP(I)$.
    \end{enumerate} 
    The sets $I$ and $J$ are called the \term{generating cofibrations} and the \term{generating trivial cofibrations},
    respectively. A cofibrantly generated model category is called \term{finitely generated} if the domains and codomains of maps of $I$ and $J$
    are finite relative to $\Cell(I)$.
}

\prop{
    Suppose $\catC$ is a cofibrantly generated model category with generating cofibrations $I$ and generating trivial cofibrations $J$. Then:
    \begin{enumerate}[i)]
    \item $\Cof=\Cof(I)$, $\W\cap\Cof=\Cof(J)$;
    \item Any cofibration is a retract of a relative $I$-cell complex; any trivial cofibration is a retract of a relative $J$-cell complex;
    \item The domains of maps of $I$ are small relative to $\Cof$; the domains of maps of $J$ are small relative to $\W\cap\Cof$.
    \end{enumerate}
    If $\catC$ is finitely generated then the domains and codomains of maps of $I$ and $J$ are finite relative to $\Cof$.
}

\thm{
    Suppose $\catC$ is a category with all small limits and colimits, $\W$ is a class of maps of $\catC$, $I$ and $J$ are sets of maps in $\catC$. 
    Then $\catC$ is a cofibrantly generated model category with weak equivalence $\W$, generating cofibrations $I$ and generating
    trivial cofibrations $J$ if and only if the following are satisfied:
    \begin{enumerate}[i)]
    \item $\W$ has the 2-out-of-3 property and is closed under retracts;
    \item The domains of maps of $I$ are small relative to $\Cell(I)$;
    \item The domains of maps of $J$ are small relative to $\Cell(J)$;
    \item $\Cell(J)\subseteq\W\cap\Cof(I)$;
    \item $\RLP(I)\subseteq\W\cap\RLP(J)$;
    \item Either $\W\cap \Cof(I)\subseteq \Cof(J)$ or $\W\cap\RLP(J)\subseteq\RLP(I)$.
    \end{enumerate}
}

\prop{
    Suppose $(F,G):\catC\to\catD$ is an adjunction, where $\catC$ and $\catD$ are model categories. Suppose more that $\catC$ is a
    cofibrantly generated model category with generating cofibrations $I$ and generating trivial cofibrations $J$. Then $(F,G)$
    is an Quillen adjunction if and only if $FI\subseteq\Cof_\catD$ and $FJ\subseteq\W_\catD\cap\Cof_\catD$. 
}

\prop{
    If $\catC$ is a cofibrantly generated model category then so is $\catC_*$. If $\catC$ is a finitely generated model category then so is $\catC_*$.
}

\subsection{The Category of Modules over a Frobenius Ring}

\defn{
    A ring $R$ is called a \term{Frobenius ring} if the projective and injective $R$-modules coincide.
}

\defn{
    Suppose $R$ is a ring. Two maps $f,g:M\to N$ of $R$-modules are called \term{stably equivalent}, denote $f\simeq g$,
    if $f-g$ factors through a projective module.
}

\lem{
    Stable equivalence between maps is an equivalence relation that is compatible with composition.
}

\defn{
    Two $R$-modules $M,N$ are called \term{stably equivalent} if there exists maps $f:M\to N$, $g:N\to M$ such that $fg\simeq\1_N$, $gf\simeq\1_M$.
    Such $f,g$ are called \term{stable equivalences}.
}

\defn{
    Suppose $R$ is a ring. Define the set $I=\{A\to R|A$ is an ideal of $R\}$, and the set $J=\{0\to R\}$. Define $\Cof=\Cof(I)$, $\Fib=\RLP(J)$, and $\W$
    to be all stable equivalences.
}

\lem{
    A map $p\in\cat{Mod}_R$ is a fibration if and only if it is surjective.
}

\lem{
    Suppose $R$ is a Frobenius ring. Then a map $p\in\cat{Mod}_R$ is a trivial fibration if and only it is surjective with injective kernel.
}

\prop{
    A map $p\in\cat{Mod}_R$ is in $\RLP(I)$ if and only it is surjective with injective kernel. In particular, if $R$ is a Frobenius ring, 
    then $\W\cap\Fib=\RLP(I)$.
}

\lem{
    A map $i\in\cat{Mod}_R$ is a cofibration if and only if it is injective.
}

\prop{
    A map $i\in\cat{Mod}_R$ is in $\Cof(J)$ if and only it is injective with projective cokernel. In particular, $\Cof(J)\subseteq\W\cap\Cof$.
}

\thm{
    If $R$ is a Frobenius ring, then $\cat{Mod}_R$ is a cofibrantly generated model category with $I$ being its generating cofibrations, 
    $J$ being its generating trivial cofibrations and weak equivalences being stable equivalences. If $R$ is Noetherian 
    then $\cat{Mod}_R$ is finitely generated.
}

\rmk{
    If $\cat{Mod}_R$ is a cofibrantly generated model category with $I$ being its generating cofibrations and $J$ being its 
    generating trivial cofibrations, then $R$ must be a Frobenius ring and weak equivalences must be stable equivalences.
}

\cor{
    If $R$ is a Frobenius ring, every object in $\cat{Mod}_R$ is cofibrant and fibrant, and two maps are left or right homotopic 
    if and only if they are stably equivalent.
}

\prop{
    For any ring homomorphism $R\to S$ between Frobenius rings, the adjunction $(\operatorname{Ind},\operatorname{Res}):\cat{Mod}_R\to\cat{Mod}_S$ 
    is a Quillen adjunction if and only if $S$ is a flat $R$-module, it is a Quillen equivalence if and only if the homomorphism is an isomorphism.
}

\subsection{The Category of Chain Complexes of Modules over a Ring}

\lem{
    If $R$ is a ring, all objects in the category $\cat{Ch}_R$ of chain complexes of $R$-modules are small. All bounded complexes of
    finitely presented $R$-modules are finite.
}

\defn{
    Suppose $R$ is a ring, define the functor $S^n:\cat{Mod}_R\to\cat{Ch}_R$ with $S^n(M)_n=M$ and $S^n(M)_m=0$ if $n\ne m$; 
    define the functor $D^n:\cat{Mod}_R\to\cat{Ch}_R$ with $D^n(M)_n=D^n(M)_{n-1}=M$ and $D^n(M)_m=0$ if $n\ne m,m-1$, and $d_n=\1_M$. 
    Denote $S^n(R)$ by $S^n$ and $D^n(R)$ by $D^n$.
}

\lem{
    $S^n$ is left adjoint to $Z_n:X\to\ker d_n$, and $D^n$ is left adjoint to $-_n$.
}

\defn{
    Suppose $R$ is a ring. Define the set $I=\{S^{n-1}\to D^n|n\in\mathbb{Z}\}$, and the set $J=\{0\to D^n|n\in\mathbb{Z}\}$. 
    Define $\Cof=\Cof(I)$, $\Fib=\RLP(J)$, and $\W$ to be all quasi-isomorphisms.
}

\lem{
    A map $p\in\cat{Ch}_R$ is a fibration if and only if it is surjective.
}

\prop{
    A map $p\in\cat{Ch}_R$ is in $\RLP(I)$ if and only if it is a surjective quasi-isomorphism. In particular, $\RLP(I)=\W\cap\Fib$.
}

\lem{
    Any cofibrant chain complex is degreewise projective. Any bounded below degreewise projective chain complex is cofibrant.
}

\rmk{
    Not every degreewise projective chain complex is cofibrant. For example take $R=\Bbbk[x]/(x^2)$ where $\Bbbk$ is a field, and take $A$
    to be the complex with $A_n=R$, and $d_n$ to be the multiplication by $x$ for any $n$, then $A$ is degreewise projective but not cofibrant.
}

\lem{
    If $C$ is a cofibrant chain complex and $K$ is acyclic, then any map from $C$ to $K$ is nullhomotopic.
}

\prop{
    A map $i\in\cat{Ch}_R$ is a cofibration if and only if it is injective with cofibrant cokernel.
}

\prop{
    A map $i\in\cat{Ch}_R$ is in $\Cof(J)$ if and only if it is injective with projective cokernel. In particular, $\Cof(J)\subseteq\W\cap\Cof$.
}

\thm{
    $\cat{Ch}_R$ is a cofibrantly generated model category with $I$ being its generating cofibrations, $J$ being its generating trivial cofibrations
    and weak equivalences being quasi-isomorphisms. This model structure is called the \term{projective model structure}.
}

\cor{
    Any object in $\cat{Ch}_R$ is projectively fibrant. An object in $\cat{Ch}_R$ is projective if and only if it is acyclic and cofibrant. 
    If two maps in $\cat{Ch}_R$ are chain homotopic then they are right homotopic.
}

\defn{
    Define $\Cof'$ to be all injections in $\cat{Ch}_R$. Define $\Fib'$ to be $\RLP(\W\cap\Cof')$, and call elements in $\Fib'$ \term{injective fibrations}.
    For any $X\in\cat{Ch}_R$ define $|X|=\abs{\bigcup_nX_n}$. Define $\gamma=\sup\{|R|,|\mathbb{Z}|\}$. Define $I'$ to be the set containing a map
    from each isomorphism class of injections whose codomain has cardinality no larger than $\gamma$. Define $J'=I'\cap\W$.
}

\prop{
    $\Cof(I')=\Cof'$. Moreover a map $p\in\cat{Ch}_R$ is in $\RLP(I')$ if and only if it is surjective with injective kernel.
}

\cor{
    Any injective object in $\cat{Ch}_R$ is injectively fibrant and acyclic. Moreover $\RLP(I')\subseteq\W\cap\Fib'$.
}

\lem{
    Any injectively fibrant chain complex is degreewise injective. Any bounded above degreewise injective chain complex is injectively fibrant.
}

\rmk{
    Not every degreewise projective chain complex is injectively fibrant. For example take $R=\Bbbk[x]/(x^2)$ where $\Bbbk$ is a field, and take $A$
    to be the complex with $A_n=R$, and $d_n$ to be the multiplication by $x$ for any $n$, then $A$ is degreewise injective but not injectively fibrant.
}

\lem{
    If $C$ is an acyclic chain complex and $K$ is injectively fibrant, then any map from $C$ to $K$ is nullhomotopic.
}

\prop{
    A map $p\in\cat{Ch}_R$ is in $\Fib'$ if and only if it is surjective with injectively fibrant kernel.
}

\lem{
    Suppose $i:A\to B$ is a map in $\cat{Ch}_R$ such that $i\in\Cof'\cap\W$. If $C$ is a subcomplex of $B$ with $|C|\le\gamma$, 
    then there exists a subcomplex $D$ of $B$ containing $C$ such that $i:D\cap A\to D$ is in $\W$.
}

\prop{
    $\Cof(J')=\Cof'\cap\W$. $\RLP(J')=\Fib'$.
}

\thm{
    $\cat{Ch}_R$ is a cofibrantly generated model category with $I'$ being its generating cofibrations, $J'$ being its generating trivial cofibrations
    and weak equivalences being quasi-isomorphisms. This model structure is called the \term{injective model structure}.
}

\cor{
    Any object in $\cat{Ch}_R$ is injectively cofibrant. An object in $\cat{Ch}_R$ is injective if and only if it is acyclic and injectively fibrant.
}

\prop{
    The identity functor from the projective model structure to the injective model structure is a Quillen equivalence.
}

\prop{
    For any ring homomorphism $R\to S$, the adjunction $(\mathrm{Ind},\mathrm{Res}):\cat{Ch}_R\to\cat{Ch}_S$ is a Quillen adjunction between
    the projective model structures, it is a Quillen equivalence if and only if the homomorphism is an isomorphism.
}

\prop{
    For any ring homomorphism $R\to S$, the adjunction $(\mathrm{Ind},\mathrm{Res}):\cat{Ch}_R\to\cat{Ch}_S$ is a Quillen adjunction between 
    the injective model structures if and only if $S$ is a flat $R$-module, it is a Quillen equivalence if and only if the homomorphism is an isomorphism.
}

\prop{
    $[S^nM,S^0N]\cong\operatorname{Ext}_R^n(M,N)$ for any $R$-module $M,N$.
}

\subsection{The Category of Topological Spaces}

\lem{
    Every topological space is small relative to the inclusions.
}

\defn{
    Define an inclusion to be a \term{closed $T_1$ inclusion} if it is closed and every point not in the image is closed.
}

\prop{
    Every compact space is finite relative to closed $T_1$ inclusions.
}

\defn{
    Define the set $I=\{S^{n-1}\to D^n\mid n\ge 0\}$ ($S^{-1}=\varnothing,D^0=*$), and the set $J=\{D^n\to D^n\times[0,1]\mid n\ge 0\}$. 
    Define $\Cof=\Cof(I)$, $\Fib=\RLP(J)$, and $\W$ to be all weak homotopy equivalences. Define a map to be a \term{Serre fibration} if and only it is in $\Fib$.
    Define a map to be a \term{relative cell complex} if and only it is in $\Cell(I)$.
}

\lem{
    Every relative cell complex is a closed $T_1$ inclusion.
}

\cor{
    Every map in $\Cof$ or $\Cof(J)$ is a closed $T_1$ inclusion.
}

\lem{
    Suppose $X$ is a relative cell complex. Then every compact set of $X$ intersects the interiors of only finitely many cells.
}

\lem{
    Closed $T_1$ inclusions that are also weak equivalence are closed under transfinite composition.
}

\prop{
    $\Cof(J)\subseteq\W\cap\Cof$.
}

\prop{
    $\RLP(I)\subseteq\W\cap\Fib$.
}

\lem{
    A map $p:X\to Y$ is in $\RLP(I)$ if and only for any map $i:A\to B\in I$, $\sHom_{\squ}(i,p):\sHom(B,X)\to\sHom(A,X)\times_{\sHom(A,Y)}\sHom(B,Y)$
    is surjective. In particular, if for any map $i\in I$, $\sHom_{\squ}(i,p)$ is a trivial fibration, then $p\in\RLP(I)$.
}

\lem{
    If $p$ is a fibration and $i\in I$, then $\sHom_{\squ}(i,p)$ is a fibration.
}

\cor{
    Every topological space is fibrant, and for any $n\ge 0$ and $Y$ a space the map $\sHom(D^n,Y)\to\sHom(S^{n-1},Y)$ is a fibration.
}

\lem{
    If $p$ is a weak equivalence, then for any $n\ge 0$, $\sHom(D^n,p)$ is a weak equivalence.
}

\lem{
    If $p$ is a weak equivalence, then for any $n\ge -1$, $\sHom(S^n,p)$ is a weak equivalence.
}

\prop{
    The pullback of any weak equivalence along a fibration is a weak equivalence.
}

\thm{
    $\W\cap\Fib\subseteq\RLP(I)$.
}

\thm{
    $\cat{Top}$ is a finitelyly generated model category with $I$ being its generating cofibrations, $J$ being its generating trivial cofibrations 
    and weak equivalences being weak homotopy equivalences. As a corollary, $\cat{Top}_*$ is a finitely generated model category with $I_+$ 
    being its generating cofibrations, $J_+$ being its generating trivial cofibrations and weak equivalences being weak homotopy equivalences.
}

\defn{
    Suppose $X$ is a space. A subset $U$ of $X$ is called \term{compactly open} if for any map $f:K\to X$ with $K$ compact Hausdorff, $f^{-1}(U)$ is open.
    $X$ is called a \term{$k$-space} if all its compactly open sets are open. Define $\cat{K}$ to be the full subcategory of $\cat{Top}$
    containing all $k$-spaces. Define the \term{$k$-space topology} on $X$, denoted $kX$, to be defining $U$ open in $kX$ if and only if
    it is compactly open in $X$.
}

\prop{
    $k$ is a functor from $\cat{Top}$ to $\cat{K}$, and it has a left adjoint and right inverse $i$, the inclusion.
}

\prop{
    $\cat{K}$ has all small colimits and limits, where colimits are directly taken in $\cat{Top}$ and limits are taken by applying $k$ 
    to the limit in $\cat{Top}$.
}

\prop{
    For any $X,Y\in\cat{K}$, define $C(X,Y)$ to be the space with underlying set $\Hom_{\cat{K}}(X,Y)$, with the topological subbasis
    $$\begin{array}{l}\{\{g:X\to Y\mid g(f(K))\subseteq U\}\mid f:K\to X,\\K\text{ compact, Hausdorff, }U\text{ an open set of }Y\}.\end{array}$$ 
    Define $\sHom(X,Y)=kC(X,Y)$, then we have a natural isomorphism
    $$\Hom_{\cat{K}}(k(X\times Y),Z)\cong\Hom_{\cat{K}}(X,\sHom(Y,Z))$$ for any $X,Y,Z\in\cat{K}$.
    \footnote{The usage of $\Hom,\sHom,\Map,-^-$ are as follows. Suppose we want to say that the Hom-object of $A$ and $B$ is $C$. 
    If $C$ is a set, or $C$ is a simplicial set, or we are talking about Hom-object in enriched categories, we use $\Hom$. If in addition
    $C$ is a simplicial set, we may add $\bullet$ to distinguish it with the case that $C$ is a set. If $C$ is a homotopy type, we use
    $\Map$. If $B,C$ are of the same type, and a) $A$ is of different type, or b) $B,C$ are sets, simplicial sets or homotopy types,
    we use the exponential $-^-$. In other cases, we use $\sHom$.}
}

\thm{
    $\cat{K}$ is a finitely generated model category with $I$ being its generating cofibrations, $J$ being its generating trivial cofibrations 
    and weak equivalences being weak homotopy equivalences. Moreover we have $\Cof_{\cat{K}}=\Cof_{\cat{Top}}\cap\cat{K},
    \Fib_{\cat{K}}=\Fib_{\cat{Top}}\cap\cat{K}$. As a corollary, $\cat{K}_*$ is a finitelyly generated model category with $I_+$ being
    its generating cofibrations, $J_+$ being its generating trivial cofibrations and weak equivalences being weak homotopy equivalences.
    Moreover we have $\Cof_{\cat{K}_*}=\Cof_{\cat{Top}_*}\cap\cat{K}_*,\Fib_{\cat{K}_*}=\Fib_{\cat{Top}_*}\cap\cat{K}_*$. Furthermore 
    the inclusions $\cat{K}\to\cat{Top},\cat{K}_*\to\cat{Top}_*$ are Quillen equivalences.
}

\section{Simplicial Sets}

\subsection{The Definition of Simplicial Sets}

\defn{
    Define the \term{simplicial category} $\d$ to be the category with objects $\{[n]:=\{0,1,\dots,n\}\mid n\in\mathbb{N}\}$
    and $$\Hom_\d([k],[n]):=\{f:[k]\to[n]|\forall 0\le x\le y\le k,f(x)\le f(y)\}.$$
}

\defn{
    Define $\d_+$ to be the subcategory of $\d$ with $$\Hom_{\d_+}([k],[n]):=\{f\in\Hom_\d([k],[n])\mid f\text{ injective}\}.$$
    Define $\d_-$ to be the subcategory of $\d$ with $$\Hom_{\d_-}([k],[n]):=\{f\in\Hom_\d([k],[n])\mid f\text{ surjective}\}.$$
}

\defn{
    For any $n\ge 1,0\le i\le n$, define $$d^i:[n-1]\to[n],d^i(j)=\left\{\begin{array}{cc}j&(j<i)\\j+1&(j\ge i)\end{array}\right.;$$
    For any $n\ge 1,0\le i\le n-1$, define $$s^i:[n]\to[n-1],s^i(j)=\left\{\begin{array}{cc}j&(j\le i)\\j-1&(j>i)\end{array}\right..$$
}

\lem{
    We have the \term{cosimplicial identities}:
    $$
    \begin{aligned}
    d^jd^i&=d^id^{j-1}&(i<j),\\
    s^jd^i&=d^is^{j-1}&(i<j),\\
          &=        \1&(i=j,j+1),\\
          &=d^{i-1}s^j&(i>j+1),\\
    s^js^i&=s^{i-1}s^j&(i>j).
    \end{aligned}$$
}

\defn{
    If $\catC$ is a category, define the category of \term{cosimplicial objects of $\catC$} to be $\catC^\d$, and the category of
    \term{simplicial objects of $\catC$} to be $\catC^{\d^\op}$.
}

\defn{
    Define the category of \term{simplicial sets} to be $\cat{SSet}:=\cat{Set}^{\d^\op}$.
}

\defn{
    If $K$ is a simplicial set, denote the set of \term{$n$-simplices} to be $K_n:=K[n]$. For any $n\ge 1,0\le i\le n$, define the \term{face maps}
    to be $d_i:=K(d^i):K_n\to K_{n-1}$. For any $n\ge 1,0\le i\le n-1$, define the \term{degeneracy maps} to be $s_i:=K(s^i):K_{n-1}\to K_n$.
}

\lem{
    For any simplicial set we have the \term{simplicial identities}:
    $$
    \begin{aligned}
    d_id_j&=d_{j-1}d_i&(i<j),\\
    d_is_j&=s_{j-1}d_i&(i<j),\\
          &=\1        &(i=j,j+1),\\
          &=s_jd_{i-1}&(i>j+1),\\
    s_is_j&=s_js_{i-1}&(i>j);
    \end{aligned}$$
    Conversely, any maps satisfying the simplicial identities determine a simplicial set.
}

\defn{
    Suppose $x,y$ are simplices in a simplicial set $K$. $y$ is called a \term{face} of $x$ if $y$ is the image of $x$ under a composition
    of some face maps, $y$ is called a \term{degeneracy} of $x$ if $y$ is the image of $x$ under a composition of some degeneracy maps.
    $x$ is called \term{nondegenerate} if $x$ is a degeneracy only of $x$.
    $K$ is called \term{finite} if the number of nondegenerate elements is finite.
}

\lem{
    If $K\in\cat{SSet}$, then $\forall x\in K$, $\exists! y\in K$, such that $x$ is a degeneracy of $y$.
}

\lem{
    Every simplicial set is small. Every finite simplicial set is finite.
}

\defn{
    We may define the functor $\d[-]:\d\to\cat{SSet}$, $[n]\mapsto\Hom_\d(-,[n])$, and we call $\d[n]$
    the \term{standard $n$-dimensional simplicial set}. We denote $i_n=\1_{[n]}\in\d[n]_n$.
}

\rmk{
    The above definition generalizes to the functor $\d[-]:\cat{POSet}\to\cat{SSet}$, $P\mapsto\Hom_{\cat{POSet}}(-,P)$,
    via the inclusion $\d\to\cat{POSet}$.
}

\lem{
    All nondegenerate $k$-simplicies of $\d[n]$ are elements in $\Hom_{\d_+}([k],[n])$. In particular, $i_n$ is the only nondegenerate
    $n$-simplex of $\d[n]$.
}

\lem{
    We have a natural isomorphism between the functors $$\Hom_{\cat{SSet}}(\d[n],-)\cong -_n,f\mapsto f(i_n):\cat{SSet}\to\cat{Set}.$$
}

\defn{
    We define the \term{boundary} of $\d[n]$, $\p\d[n]$ to be the simplicial set obtained from $\d[n]$
    by removing all degeneracies of $i_n$.
}

\defn{
    We define the \term{$r$-horn} of $\d[n]$, $\l^r[n]$ to be the simplicial set obtained from $\p\d[n]$
    by removing all degeneracies of $d^r\in\p\d[n]_{n-1}$.
}

\defn{
    There is a functor $\d:\cat{SSet}\to\cat{Cat}$ with
    $$
    \begin{aligned}
    K\mapsto&\9\begin{array}{c}\textrm{Objects: }\{\d[n]\to K\}\\\Hom_{\d K}({(f:\d[k]\to K)},{(g:\d[n]\to K)})\hspace{2em}\\
    \hspace{2em}=\{h\in\Hom_\d([k],[n])|g\circ \d[-](h)=f\}\end{array}\0,\\
    (a:K\to L)\mapsto&\9(f:\d[n]\to K)\mapsto a\circ f\0.
    \end{aligned} 
    $$ 
    For any simplicial set $K$, denote $\d K$ the \term{category of simplices} in $K$.
}

\lem{
    We have $$\colim\9\begin{array}{c}\d K\to\cat{SSet}\\(f:\d[n]\to K)\mapsto\d[n]\end{array}\0=K.$$
}

\defn{
    There is a functor $\d':\cat{SSet}\to\cat{Cat}$ with
    $$
    \begin{aligned}
    K\mapsto&\9\begin{array}{c}\textrm{Objects: }\{f:\d[n]\to K|f(i_n)\textrm{ nondegenerate}\}\\
    \Hom_{\d' K}({(f:\d[k]\to K)},{(g:\d[n]\to K)})\hspace{2em}\\\hspace{2em}=\{h\in\Hom_{\d_+}([k],[n])|g\circ \d[-](h)=f\}\end{array}\0,\\
    (a:K\to L)\mapsto&\9(f:\d[n]\to K)\mapsto a\circ f\0.
    \end{aligned} 
    $$ 
    For any simplicial set $K$, denote $\d'K$ the \term{category of nondegenerate simplices} in $K$.
}

\lem{\label{taga}
    If $K$ is \term{regular}, which means for any nondegenerate $n$-simplex the map $\d[n]\to K$ induced by the $n$-simplex is injective,
    then we have $$\colim\9\begin{array}{c}\d'K\to\cat{SSet}\\(f:\d[n]\to K)\mapsto\d[n]\end{array}\0=K.$$
}

\rmk{
    If $K$ is not regular then Lemma \ref{taga} may not be true. A counterexample is $\d[2]/\p\d[2]$.
}

\prop{\label{tagb}
    If $\catC$ is a category with all small colimits,
    then the functor $\Adj(\cat{SSet},\catC)\to\catC^\d$ with $(F,G,\vp)\mapsto F\circ\d[-]$ is an equivalence between categories
    with inverse being $A^\bullet\mapsto\9A^\bullet\otimes-,\Hom_\catC(A^\bullet,-),\vp\0$,
    where $$A^\bullet\otimes K=\colim\9\begin{array}{c}\d K\to\catC\\(f:\d[n]\to K)\mapsto A^n\end{array}\0$$
    and $$\Hom_\catC(A^\bullet,Y)_n=\Hom_\catC(A^n,Y).$$
}

\prop{\label{tage}
    If $\catC$ is a category with all small limits, then the category $\Adj(\cat{SSet},\catC^\op)^\op$ is naturally equivalent to $\catC^{\d^\op}$.
    We denote the image of $A_\bullet\in\catC^{\d^\op}$ under the equivalence to be $\9A_\bullet^-,\Hom_\catC(-,A_\bullet),\vp\0$.
}

\cor{\label{tagg}
    If $\catC$ is a pointed category with all small colimits, then the category $\Adj(\cat{SSet}_*,\catC)$ is naturally equivalent to $\catC^\d$.
    We denote the image of $A^\bullet\in\catC^\d$ under the equivalence to be $\9A^\bullet\land-,\Hom_{\catC*}(A^\bullet,-),\vp\0$.
    Furthermore we have a natural isomorphism $A^\bullet\land(-_+)\cong A^\bullet\otimes-$. Dually the category $\Adj(\cat{SSet}_*,\catC^\op)^\op$
    is naturally equivalent to $\catC^{\d^\op}$. We denote the image of $A_\bullet\in\catC^{\d^\op}$ under the equivalence to be 
    $\9A_{\bullet*}^-,\Hom_{\catC*}(-,A_\bullet),\vp\0$. Furthermore we have a natural isomorphism $A_{\bullet*}^{-_+}\cong A_\bullet^-$.
}

\eg{
    By Proposition \ref{tagb} the functor $\cat{SSet}\to\cat{SSet}^\d$, $K\mapsto K\times\d[-]$ corresponds to a functor 
    $\cat{SSet}\to\Adj(\cat{SSet},\cat{SSet})$ which we denote by $K\mapsto\9K\times-,-^K,\vp\0$, where $(L^K)_n=\Hom_{\cat{SSet}}(K\times\d[n],L)$.
    The left adjoint is exactly the product in $\cat{SSet}$. We call $-^-$ the \term{function complex functor}.
}

\eg{
    By Proposition \ref{tagb} the element $$\9[n]\mapsto\{(t_0,\dots,t_n)|t_i\ge 0,t_0+\cdots+t_n=1\}\0\in\cat{Top}^\d(\cat{K}^\d)$$
    corresponds to an element in $\Adj(\cat{SSet},\cat{Top}(\cat{K}))$. We denote the adjunction by $\9\abs-,\Sing,\vp\0$ 
    and call $\abs-$ the \term{geometric realization} and $\Sing$ the \term{singular functor}.
}

\lem{
    $\abs{\d[n]}=D^n,\abs{\p\d[n]}=S^{n-1},\abs{\l^r[n]}=D^{n-1}$.
}

\prop{
    As a functor $\cat{SSet}\to\cat{K}$, $\abs-$ preserves finite products.
}

\rmk{
    As a functor $\cat{SSet}\to\cat{Top}$ $\abs-$ need not preserve finite products. This is because the product need not preserve colimits.
}

\subsection{The Standard Model Structure on Simplicial Sets}

\defn{
    Define the set $I=\{i:\p\d[n]\to\d[n]\mid n\ge 0\}$, and the set $J=\{j:\l^r[n]\to\d[n]\mid n>0,0\le r\le n\}$. Define $\Cof=\Cof(I)$, $\Fib=\RLP(J)$, 
    and $\W=\abs-^{-1}(\W_{\cat{Top}})$. Define a map to be a \term{Kan fibration} if and only it is in $\Fib$. Define a simplicial set 
    to be a \term{Kan complex} if and only if it is fibrant. Define a map to be an \term{anodyne extension} if and only it is in $\Cof(J)$.
}

\defn{
    If $p:X\to Y$ is a fibration, $v:\d[0]\to Y$ is a vertex in $Y$, define $\d[0]\times_YX$ to be the \term{fiber of $p$ over $v$}, denoted $X_v$.
}

\prop{
    A map is a cofibration if and only if it is injective. In particular, any simplicial set is cofibrant, and $\Cof(J)\subseteq \Cof(I)$.
    Moreover, any cofibration is a relative $I$-cell complex.
}

\prop{
    We have $\abs{I}$ is the set of generating cofibrations of $\cat{K}$ and $\abs{J}$ is the set of generating trivial cofibrations of $\cat{K}$.
    Thus all maps in $\abs{\Cof(I)}$ are cofibrations, and all maps in $\abs{\Cof(J)}$ are trivial cofibrations. In particular, all anodyne extensions
    are trivial cofibrations. Moreover the singular functor takes fibrations between $k$-spaces to Kan fibrations and takes trivial fibrations
    between $k$-spaces to $\RLP(I)$.
}

\lem{
    As a functor $\cat{SSet}\to\cat{K}$, $\abs-$ preserves all finite limits, and in particular, pullbacks.
}

\lem{
    For any map $f\in\RLP(I)$, $\abs f$ is a fibration.
}

\lem{
    For any map $f\in\RLP(I)$, $f$ is a trivial fibration.
}

\defn{
    Suppose $i:K\to L$ and $p:X\to Y$ are maps between simplicial sets. We define $P(i,p):=(K\times Y)\amalg_{K\times X}(L\times X)$, and $i\squ p$
    being the induced map $P(i,p)\to L\times B$. Moreover we define $p_{\squ}^i$ to be the induced map $X^L\to X^K\times_{Y^K}Y^L$.
}

\lem{
    We have that $f\squ g\nearrow h$ if and only if $f\nearrow h_{\squ}^g$ for any maps $f,g,h$.
}

\lem{
    Suppose $i:\p\d[n]\to\d[n]$, $f:\l^r[1]\to\d[1]$ are the inclusions, where $r=0,1$. Then the map $i\squ f$ is an anodyne extension.
}

\prop{
    Suppose $f:\l^r[1]\to\d[1]$ is the inclusion and $i:K\to L$ is an arbitrary injective map, where $r=0,1$. Then the map $i\squ f$ 
    is an anodyne extension.
}

\prop{
    A map is an anodyne extension if and only if it is in $\Cof(J')$, where $J'=\{i\squ f|i\in I,f:\l^r[1]\to\d[1],r=0,1\}$.
}

\thm{
    If $f$ is an anodyne extenstion and $i$ is an arbitrary injective map, then $i\squ f$ is an anodyne extension.
}

\thm{
    If $p$ is a fibration and $i$ is an arbitrary injective map, then $p_{\squ}^i$ is a fibration.
}

\defn{
    Suppose $X$ is a fibrant simplicial set and $x,y$ are 0-simplices, we define $x$ to be \term{homotopic} to $y$, denoted $x\sim y$,
    if there exists some 1-simplex $z$ with $d_0z=x,d_1z=y$.
}

\lem{
    If $X$ is a fibrant simplicial set, then $\sim$ is an equivalence relation on $X_0$.
}

\defn{
    If $X$ is a fibrant simplicial set, define $\pi_0X=X/\sim$ and denote elements in $\pi_0X$ the \term{path components} of $X$.
    Moreover if $v$ is a 0-simplex of $X$, define $\pi_0(X,v)=(\pi_0X,[v])$.
}

\lem{
    $\pi_0:\cat{SSet}_f\to\cat{Set}$ and $\pi_0:(\cat{SSet}_f)_*\to\cat{Set}_*$ are functorial. If $X$ is a fibrant simplicial set, 
    then $\pi_0X$ and $\pi_0\abs X$ are naturally isomorphic.
}

\defn{
    Suppose $v$ is a vertex of a simplicial set $X$, and $Y$ is another simplicial set.
    We define the \term{constant map at $v$} from $Y$ to $X$ to be the composite $Y\to\d[0]\xrightarrow{v}X$.
}

\defn{
    Suppose $X$ is a fibrant simplicial set and $v$ is a vertex. Let $F$ be the fiber of the fibration $X^{\d[n]}\to X^{\p\d[n]}$ over $v$.
    Define the \term{$n$-th homotopy group} $\pi_n(X,v)$ of $X$ at $v$ to be $\pi_0(F,v)$.
}

\lem{
    Given a map $f:(X,v)\to(Y,w)$, $f$ induces a map $f_*:\pi_n(X,v)\to\pi_n(Y,w)$, thus $\pi_n:(\cat{SSet}_f)_*\to\cat{Set}_*$ is functorial.
}

\defn{
    Suppose $f,g:K\to X$ are maps of simplicial sets, we denote a map $H:K\times\d[1]\to X$ to be a \term{homotopy} from $f$ to $g$
    if $H|_{K\times 0}=f$ and $H|_{K\times 1}=g$. If such $H$ exists we denote $f\simeq g$.
}

\lem{
    If $X$ is fibrant, then for any simplicial set $K$, $\simeq$ is an equivalence relation on $\Hom_{\cat{SSet}}(K,X)$,
    and $f\simeq g$ if and only if $f\sim g$ as vertices in $X^K$. Moreover $$\pi_n(X,v)=(\Hom_{\cat{SSet}}(\d[n],X)/\simeq,[v]).$$
}

\lem{
    Suppose $X$ is fibrant, $v$ is a vertex of $f$, $a$ is an $n$-simplex such that $d_ia=v$ for all $i$.
    Then $[a]=[v]\in\pi_n(X,v)$ if and only if there is an $(n+1)$-simplex $x$ such that $d_{n+1}x=a$ and $d_ix=v$ if $i\le n$.
}

\defn{
    Suppose $K$ is a subsimplicial set of $L$, $i:K\to L$ is the inclusion. A map $r:L\to K$ is called a \term{retraction} from $L$ to $K$ if $ri=\1$;
    A map $H:L\times\d[1]\to L$ is called a \term{deformation retraction} from $L$ to $K$ if $H|_{L\times 0}=\1$, $H|_{L\times 1}$ is a retraction,
    and $H|_{K\times\d[1]}$ is the projection onto $K$.
}

\eg{
    The vertex $n$ is a deformation retraction of $\d[n]$, and the deformation retraction restricts to a deformation retraction from $\l^n[n]$ to $n$.
}

\rmk{
    If $n>0$ then $n\not\simeq\1_{\d[n]}$, thus $\d[n]$ is not fibrant.
}

\prop{
    If $X$ is a nonempty fibrant simplicial set with no nontrivial homotopy groups, then the map $X\to\d[0]$ is in $\RLP(I)$.
}

\defn{
    Suppose $p:X\to Y$ is a fibration between fibrant simplicial sets, $v$ is a vertex of $X$, $F$ is the fiber of $p$ over $p(v)$.
    Define the map $\p:\pi_n(Y,p(v))\mapsto\pi_{n-1}(F,v)$ as follows: for any $[\alpha]\in\pi_n(Y,p(v))$, define $\gamma$ to be a lift of 
    $$\xymatrix{
    \l^n[n]\ar[d]\ar[r]^{v}&X\ar[d]^{p}\\
    \d[n]\ar[r]^{\alpha}\ar@{.>}[ur]^{\gamma}&Y
    }$$
    and define $\p[\alpha]=[d_n\gamma]$.
}

\lem{
    $\p:\pi_n(Y,p(v))\mapsto\pi_{n-1}(F,v)$ is well-defined and is natural for maps between fibrations.
}

\prop{
    Suppose $p:X\to Y$ is a fibration between fibrant simplicial sets, $v$ is a vertex of $X$, $F$ is the fiber of $p$ over $p(v)$.
    Then we have an exact sequence of pointed sets:
    $$
    \begin{aligned}
    \cdots\xrightarrow{\p}\pi_n(F,v)\xrightarrow{i_*}\pi_n(X,v)&\xrightarrow{p_*}\pi_n(Y,p(v))\\
    &\xrightarrow{\p}\pi_{n-1}(F,v)\xrightarrow{i_*}\cdots\xrightarrow{i_*}\pi_0(X,v)\xrightarrow{p_*}\pi_0(Y,p(v)),
    \end{aligned} 
    $$
    which is called the \term{long exact sequence}.
}

\defn{
    We define a fibration $p:X\to Y$ to be \term{locally trivial} if for any simplex $y:\d[n]\to Y$ there exists some simplicial set $F$,
    such that the pullback fibration $y^*p:y^*X:=\d[n]\times_YX\to\d[n]$ is isomorphic over $\d[n]$ to the projection $\d[n]\times F\to\d[n]$.
}

\prop{
    If $p$ is a locally trivial fibration such that every fiber of $p$ is non-empty and has no nontrivial homotopy groups, then $p\in\RLP(I)$.
}

\prop{
    Suppose $p:X\to Y$ is a fibration and $f\simeq g:K\to Y$. Then the pullback fibrations $f^*p$ and $g^*p$ are \term{fiber homotopy equivalent},
    i.e. there exists maps $r:f^*X\to g^*X$ and $s:g^*X\to f^*X$ such that $f^*p\circ s=g^*p,g^*p\circ r=f^*p$,
    and there exists homotopies $H:rs\simeq\1_{g^*X}$ and $H':sr\simeq\1_{f^*X}$ with
    $g^*p\circ H=\pr_1\circ(g^*p\times\1),f^*p\circ H'=\pr_1\circ(f^*p\times\1)$.
}

\cor{
    Suppose $p:X\to Y$ is a fibration, and $y$ is an $n$-simplex of $Y$, then the pullback fibration $y^*p$
    is fiber homotopy equivalent to the projection fibration $\d[n]\times F\to\d[n]$, where $F$ is the fiber of $p$ over $y(n)$.
}

\defn{
    Suppose $p:X\to Y$ is a fibration, and $x,y$ are two $n$-simplicies of $X$. We say $x$ and $y$ are \term{$p$-related},
    written $x\sim_py$, if they represent vertices in the same path component in the same fiber of the map $p_{\squ}^{\p\d[n]\to\d[n]}$.
    We say $p$ is a \term{minimal fibration} if $x\sim_py$ implies $x=y$.
}

\lem{
    Suppose $p:X\to Y$ is a fibration, and $x,y$ are two $n$-simplicies of $X$. Then $\sim_p$ is an equivalence relation, and $x\sim_py$ 
    if and only if $px=py$, $d_ix=d_iy$ for any $0\le i\le n$, and there exists a homotopy $H:x\simeq y$ such that $H|_{\p\d[n]\times\d[1]}$ 
    and $pH$ are constant homotopies.
}

\lem{
    Suppose $p:X\to Y$, $q:Z\to Y$ are fibrations and $q$ is minimal, $f,g:X\to Z$, $H:f\simeq g$ such that $qH=\pr_1(p\times\1)$. 
    If $g$ is an isomorphism, so is $f$.
}

\cor{
    Any minimal fibration is locally trivial.
}

\lem{
    Suppose $p:X\to Y$ is a fibration and $x,y$ are degenerate $n$-simplices in $X$ that are $p$-related. Then $x=y$.
}

\thm{
    Any fibration $p$ can be factored into $p'r$, where $r$ is a retraction onto a subsimplicial set of the domain such that $r\in\RLP(I)$, 
    and $p'$ is a minimal fibration.
}

\cor{
    If $p$ is a fibration such that every fiber of $p$ is non-empty and has no nontrivial homotopy groups, then $p\in\RLP(I)$.
}

\prop{
    If $p$ is a locally trivial fibration then $|p|$ is a fibration.
}

\cor{
    If $p$ is a fibration then $|p|$ is a fibration.
}

\prop{
    There exists a natural isomorphism $\pi_n(X,v)\to\pi_n(|X|,|v|)$ for any $(X,v)\in\cat{SSet}_*$.
}

\rmk{
    Suppose $X$ is a fibrant simplicial set, $v$ is a vertex of $X$. For any $[\alpha],[\beta]\in\pi_1(X,v)$, take $\gamma$ to be a lift 
    of the following diagram:
    $$\xymatrix{
    \l^1[2]\ar[d]\ar[rr]^{\alpha\text{ on }d_2i_2}_{\beta\text{ on }d_0i_2}&&X\ar[d]\\
    \d[2]\ar[rr]\ar@{.>}[urr]_{\gamma}&&{*}
    }$$
    and define $[\alpha]\cdot[\beta]=[d_1\gamma]$; take $\delta$ to be the lift of a following diagram:
    $$\xymatrix{
    \l^0[2]\ar[d]\ar[rr]^{\alpha\text{ on }d_2i_2}_{v\text{ on }d_1i_2}&&X\ar[d]\\
    \d[2]\ar[rr]\ar@{.>}[urr]_{\delta}&&{*}
    }$$
    and define $[\alpha]^{-1}=[d_0\delta]$. Then this gives a natural group structure on $\pi_1(X,v)$ making the map $\pi_1(X,v)\to\pi_1(|X|,|v|)$ 
    a natural group isomorphism.
}

\thm{
    If $p$ is a locally trivial fibration, then $p\in\RLP(I)$.
}

\thm{
    $\cat{SSet}$ is a finitely generated model category with $I$ being its generating cofibrations and $J$ being its generating trivial cofibrations.
    As a corollary, $\cat{SSet}_*$ is a finitely generated model category with $I_+$ being its generating cofibrations and $J_+$ 
    being its generating trivial cofibrations. This model structure is called the \term{standard model structure} on $\cat{SSet}$.
}

\thm{
    $\9\abs-,\Sing,\vp\0$ is a Quillen equivalence between $\cat{SSet}$ and $\cat{K}$. $\9\abs-_*,\Sing_*,\vp_*\0$ is a Quillen equivalence 
    between $\cat{SSet}_*$ and $\cat{K}_*$.
}

\prop{
    Suppose $\catC$ is a model category and $F:\cat{SSet}\to\catC$ is a functor that preserves colimits and cofibrations.
    Then $F$ preserves trivial cofibrations if and only if $F(\d[n])\to F(\d[0])$ is a weak equivalence for any $n$.
}

\cor{
    Suppose $\catC$ is a model category and $F:\cat{SSet}_*\to\catC$ is a functor that preserves colimits and cofibrations.
    Then $F$ preserves trivial cofibrations if and only if $F(\d[n]_+)\to F(\d[0]_+)$ is a weak equivalence for any $n$.
}

\section{Monoidal Model Categories}

\subsection{Monoidal Categories}

\defn{
    Suppose $\catC$ is a category. A \term{monoidal structure} on $\catC$ consists of a quintuple $(\ox,S,a,l,r)$, where $\ox:\catC\times\catC\to\catC$ 
    is a functor, $S$ is an object in $\catC$, $a_{XYZ}:(X\ox Y)\ox Z\to X\ox(Y\ox Z)$, $l_X:S\ox X\to X$, $r_X:X\ox S\to X$ are natural isomorphisms,
    such that the following diagrams are commutative for any objects $X,Y,Z,W$:
    $$\xymatrix @C=50pt{
    ((X\ox Y)\ox Z)\ox W\ar[d]_{a_{(X\ox Y)ZW}}\ar[r]^{a_{XYZ}\ox\1_W}&(X\ox(Y\ox Z))\ox W\ar[dd]^{a_{X(Y\ox Z)W}}\\
    (X\ox Y)\ox(Z\ox W)\ar[d]_{a_{XY(Z\ox W)}}\\
    X\ox(Y\ox(Z\ox W))&X\ox((Y\ox Z)\ox W)\ar[l]_{\1_X\ox a_{YZW}}
    }$$ 
    $$\xymatrix @C=0pt{
    (X\ox S)\ox Y\ar[dr]_{r_X\ox\1_Y}\ar[rr]^{a_{XSY}}&&X\ox(S\ox Y)\ar[dl]^{\1_X\ox l_Y}\\
    &X\ox Y
    }\qquad\xymatrix{
    S\ox S\ar@/^/[d]^{l_S}\ar@/_/[d]_{r_S}\\S
    }$$ 
    A category together with a monoidal structure is called a \term{monoidal category}.
}

\defn{
    Suppose $\catC,\catD$ are monoidal categories. A \term{monoidal functor} between $\catC$ and $\catD$ is a triple $(F,\mu,\alpha)$, 
    where $F:\catC\to\catD$ is a functor, $\mu_{XY}:FX\ox FY\to F(X\ox Y)$ is a natural isomorphism, $\alpha:FS\to S$ is a isomorphism, 
    such that the following diagrams are commutative for any objects $X,Y,Z$:
    $$\xymatrix @C=60pt{
    (FX\ox FY)\ox FZ\ar[d]_{\mu_{XY}\ox\1_{FZ}}\ar[r]^{a_{(FX)(FY)(FZ)}}&FX\ox(FY\ox FZ)\ar[d]^{\1_{FX}\ox\mu_{YZ}}\\
    F(X\ox Y)\ox FZ\ar[d]_{\mu_{(X\ox Y)Z}}&FX\ox F(Y\ox Z)\ar[d]^{\mu_{X(Y\ox Z)}}\\
    F((X\ox Y)\ox Z)\ar[r]^{F(a_{XYZ})}&F(X\ox(Y\ox Z))
    }$$ 
    $$\xymatrix @C=40pt{
    FS\ox FX\ar[d]_{\mu_{SX}}\ar[r]^{\alpha\ox\1_{FX}}&S\ox FX\ar[d]^{l_{FX}}\\
    F(S\ox X)\ar[r]^{F(l_X)}&FX
    }\xymatrix @C=40pt{
    FX\ox FS\ar[d]_{\mu_{XS}}\ar[r]^{\1_{FX}\ox\alpha}&FX\ox S\ar[d]^{r_{FX}}\\
    F(X\ox S)\ar[r]^{F(r_X)}&FX
    }$$ 
}

\defn{
    Suppose $\catC,\catD$ are monoidal categories and $F,F'$ are monoidal functors between $\catC$ and $\catD$. A \term{monoidal natural transformation}
    between $F$ and $F'$ is a natural transformation $\tau:F\to F'$ such that the following diagrams are commutative for any objects $X,Y$:
    $$\xymatrix @C=30pt{
    FX\ox FY\ar[r]^{\mu_{XY}}\ar[d]_{\tau_X\ox\tau_Y}&F(X\ox Y)\ar[d]^{\tau_{X\ox Y}}\\
    F'X\ox F'Y\ar[r]^{\mu'_{XY}}&F'(X\ox Y)
    }\qquad\xymatrix @C=0pt{
    FS\ar[dr]_{\tau_S}\ar[rr]^{\alpha_S}&&S\\
    &F'S\ar[ur]_{\alpha'_S}
    }$$ 
}

\lem{
    Monoidal categories, monoidal functors and monoidal natural transformations form a 2-category, which we denote $\catt{Mon}$.
}

\defn{
    Suppose $\catC$ is a category. A \term{symmetric monoidal structure} on $\catC$ is a monoidal structure $\ox$ together with a natural isomorphism 
    $T_{XY}:X\ox Y\to Y\ox X$, such that the following diagrams are commutative for any objects $X,Y,Z$:
    $$\xymatrix @C=40pt{
    (X\ox Y)\ox Z\ar[r]^{T_{XY}\ox\1_Z}\ar[d]_{a_{XYZ}}&(Y\ox X)\ox Z\ar[r]^{a_{YXZ}}&Y\ox(X\ox Z)\ar[d]^{\1_Y\ox T_{XZ}}\\
    X\ox(Y\ox Z)\ar[r]^{T_{X(Y\ox Z)}}&(Y\ox Z)\ox X\ar[r]^{a_{YZX}}&Y\ox(Z\ox X)
    }$$ 
    $$\xymatrix{
    X\ox Y\ar@/^/[d]^{T_{XY}}\\
    Y\ox X\ar@/^/[u]^{T_{YX}}
    }\qquad\xymatrix{
    S\ox S\ar@/^/[d]^{T_{SS}}\ar@/_/_{\1}[d]\\S\ox S
    }\qquad\xymatrix @C=5pt{
    X\ox S\ar[rr]^{r_X}\ar[dr]_{T_{XS}}&&X\\
    &S\ox X\ar[ur]_{T_{SX}}
    }$$ 
    A category together with a symmetric monoidal structure is called a \term{symmetric monoidal category}.
}

\defn{
    Suppose $\catC,\catD$ are symmetric monoidal categories. A \term{symmetric monoidal functor} between $\catC$ and $\catD$ is a monoidal functor
    $F:\catC\to\catD$, such that the following diagram is commutative for any objects $X,Y$:
    $$\xymatrix @C=30pt{
    FX\ox FY\ar[d]_{T_{(FX)(FY)}}\ar[r]^{\mu_{XY}}&F(X\ox Y)\ar[d]^{FT_{XY}}\\
    FY\ox FX\ar[r]^{\mu_{YX}}&F(Y\ox X)
    }$$ 
}

\lem{
    Symmetric monoidal categories, symmetric monoidal functors and monoidal natural transformations form a 2-category, which we denote $\catt{SymMon}$.
    There is a forgetful 2-functor from $\catt{SymMon}$ to $\catt{Mon}$.
}

\defn{
    Suppose $\catC$ is a monoidal category. A \term{(right) $\catC$-module structure} on a category $\catD$ is a triple $(\ox,a,r)$, 
    where $\ox:\catD\times\catC\to\catD$ is a functor, $a_{XKL}:(X\ox K)\ox L\to X\ox(K\ox L)$, $r_X:X\ox S\to X$ are natural isomorphisms, 
    such that the following diagrams are commutative for any $X\in\catD,K,L,M\in\catC$:
    $$\xymatrix @C=60pt{
    ((X\ox K)\ox L)\ox M\ar[d]_{a_{(X\ox K)LM}}\ar[r]^{a_{XKL}\ox\1_M}&(X\ox(K\ox L))\ox M\ar[dd]^{a_{X(K\ox L)M}}\\
    (X\ox K)\ox(L\ox M)\ar[d]_{a_{XK(L\ox M)}}\\
    X\ox(K\ox(L\ox M))&X\ox((K\ox L)\ox M)\ar[l]_{\1_X\ox a_{KLM}}
    }$$
    $$\xymatrix @C=5pt{
    (X\ox S)\ox K\ar[dr]_{r_X\ox\1_K}\ar[rr]^{a_{XSK}}&&X\ox(S\ox K)\ar[dl]^{\1_X\ox l_K}\\
    &X\ox K
    }$$
    $$\xymatrix @C=5pt{
    (X\ox K)\ox S\ar[dr]_{r_{X\ox K}}\ar[rr]^{a_{XKS}}&&X\ox(K\ox S)\ar[dl]^{\1_X\ox r_K}\\
    &X\ox K
    }$$
    A category together with a $\catC$-module structure is called a \term{(right) $\catC$-module}.
}

\defn{
    Suppose $\catD,\catE$ are $\catC$-modules. A \term{(right) $\catC$-module functor} between $\catD$ and $\catE$ is a pair $(F,\mu)$, 
    where $F:\catD\to\catE$ is a functor, $\mu_{XK}:FX\ox K\to F(X\ox K)$ is a natural isomorphism, such that the following diagrams 
    are commutative for any objects $X\in\catD,K,L\in\catC$:
    $$\xymatrix @C=50pt{
    (FX\ox K)\ox L\ar[d]_{a_{(FX)KL}}\ar[r]^{\mu_{XK}\ox\1_L}&F(X\ox K)\ox L\ar[dd]^{\mu_{(X\ox K)L}}\\
    FX\ox(K\ox L)\ar[d]_{\mu_{X(K\ox L)}}\\
    F(X\ox(K\ox L))&F((X\ox K)\ox L)\ar[l]_{F(a_{XKL})}
    }$$
    $$\xymatrix @C=5pt{
    FX\ox S\ar[dr]_{r_{FX}}\ar[rr]^{\mu_{XS}}&&F(X\ox S)\ar[dl]^{Fr_X}\\
    &X\ox S
    }$$
}

\defn{
    Suppose $\catD,\catE$ are $\catC$-modules and $F,F'$ are $\catC$-module functors between $\catD$ and $\catE$. A \term{(right) $\catC$-module 
    natural transformation} between $F$ and $F'$ is a natural transformation $\tau:F\to F'$ such that the following diagram is commutative 
    for any objects $X\in\catD,K\in\catC$:
    $$\xymatrix @C=30pt{
    FX\ox K\ar[d]_{\tau_X\ox\1_K}\ar[r]^{\mu_{XK}}&F(X\ox K)\ar[d]^{\tau_{X\ox K}}\\
    F'X\ox K\ar[r]^{\mu'_{XK}}&F'(X\ox K)
    }$$ 
}

\lem{
    $\catC$-modules, $\catC$-module functors and $\catC$-module natural transformations form a 2-category, which we denote $\catt{Mod}_\catC$. 
    A monoidal functor between two monoidal categories $\catC$ and $\catD$ induces a forgetful 2-functor from $\catt{Mod}_\catD$ to $\catt{Mod}_\catC$.
}

\defn{
    Suppose $\catC$ is a monoidal category. A \term{$\catC$-algebra structure} on a category $\catD$ is a monoidal structure on $\catD$ 
    together with a monoidal functor $i:\catC\to\catD$. A category together with a $\catC$-algebra structure is called a \term{$\catC$-algebra}.
}

\defn{
    Suppose $\catD,\catE$ are $\catC$-algebras. A \term{$\catC$-algebra functor} between $\catD$ and $\catE$ is a monoidal functor $F:\catD\to\catE$
    together with a monoidal natural transformation $\rho:F\circ i_\catD\to i_\catE$.
}

\defn{
    Suppose $\catD,\catE$ are $\catC$-algebras and $F,F'$ are $\catC$-algebra functors between $\catD$ and $\catE$. A \term{$\catC$-algebra 
    natural transformation} between $F$ and $F'$ is a monoidal natural transformation $\tau:F\to F'$ such that the following diagram is commutative
    for any object $K\in\catC$:
    $$\xymatrix @C=5pt{
    F(i(K))\ar[dr]_{\rho_K}\ar[rr]^{\tau_{i(K)}}&&F'(i(K))\ar[dl]^{\rho'_K}\\
    &i(K)
    }$$ 
}

\lem{
    $\catC$-algebras, $\catC$-algebra functors and $\catC$-algebra natural transformations form a 2-category, which we denote $\catt{Alg}_\catC$. 
    A monoidal functor between two monoidal categories $\catC$ and $\catD$ induces a forgetful 2-functor from $\catt{Alg}_\catD$ to $\catt{Alg}_\catC$.
}

\prop{
    For any monoidal category $\catC$, we have a forgetful 2-functor from $\catt{Alg}_\catC$ to $\catt{Module}_\catC$, where the $\ox$-structure 
    is given by $X\ox K=X\ox iK$ for any object $X$ in some $\catC$-algebra.
}

\defn{
    Suppose $\catC$ is a symmetric monoidal category. A \term{summetric $\catC$-algebra structure} on a category $\catD$ 
    is a symmetric monoidal structure on $\catD$ together with a symmetric monoidal functor $i:\catC\to\catD$. A category 
    together with a symmetric $\catC$-algebra structure is called a \term{symmetric $\catC$-algebra}.
}

\defn{
    Suppose $\catD,\catE$ are symmetric $\catC$-algebras. A \term{symmetric $\catC$-algebra functor} between $\catD$ and $\catE$ 
    is a $\catC$-algebra functor $F:\catD\to\catE$ that is also a symmetric monoidal functor.
}

\lem{
    Symmetric $\catC$-algebras, symmetric $\catC$-algebra functors and $\catC$-algebra natural transformations form a 2-category, 
    which we denote $\catt{SymAlg}_\catC$. There is a forgetful 2-functor from $\catt{SymAlg}_\catC$ to $\catt{Alg}_\catC$ 
    for any monoidal category $\catC$. A monoidal functor between two monoidal categories $\catC$ and $\catD$
    induces a forgetful 2-functor from $\catt{SymAlg}_\catD$ to $\catt{SymAlg}_\catC$.
}

\defn{
    Suppose $\catC,\catD,\catE$ are categories. An \term{adjunction of 2 variables} from $\catC\times\catD$ to $\catE$ 
    is a quintuple $(\ox,\sHom_l,\sHom_r,\vp_l,\vp_r)$, where $$\ox:\catC\times\catD\to\catE,\sHom_l:\catC^\op\times\catE\to\catD,
    \sHom_r:\catD^\op\times\catE\to\catC$$ are functors, $$\begin{aligned}(\vp_l)_{CDE}&:\Hom_\catE(C\ox D,E)\to\Hom_\catD(D,\sHom_l(C,E))
    \\(\vp_r)_{CDE}&:\Hom_\catE(C\ox D,E)\to\Hom_\catC(C,\sHom_r(D,E))\end{aligned}$$ are natural isomorphisms.
}

\defn{
    Suppose $\star$ is one of the structures given above. $\star$ is called \term{closed} if all the structure bifunctors are adjunctions of 2 variables,
    and all the structure functors are left adjoints.
}

\defn{
    If $\catC$ is a closed monoidal category and $\catD$ is a closed $\catC$-module, then we denote $X^K$ to be $\sHom_r(K,X)$ for $X\in\catD,K\in\catC$,
    and $\sHom(X,Y)$ to be $\sHom_l(X,Y)$ for $X,Y\in\catD$.
}

\lem{
    For any 2-category given above, all such categories with closed such structures, all such functors with closed such structures 
    and all such natural transformations with closed such structures form a 2-category. We denote such 2-category by adding a $\catt{Clo}$ 
    before the original one.
}

\prop{
    If $\catC$ is a closed symmetric monoidal category, then there is a duality 2-functor $-^\op$ on $\catt{CloMod}_\catC$, 
    which maps a closed $\catC$-module $$(\catD,\ox,\sHom_l,\sHom_r,a,r)$$ to $$(\catD^\op,\ox^\op,\sHom_l^\op,\sHom_r^\op,a^\op,r^\op), $$
    where $X\ox^\op K=\sHom_r(K,X)$, $\sHom_l^\op(X,Y)=\sHom_l(Y,X)$, $\sHom_r^\op(K,X)=X\ox K$, $a^\op$ is defined by the natural isomorphism
    $$
    \begin{aligned}
         &\Hom_\catD(Y,\sHom_r(L,\sHom_r(K,X)))\\
    \cong&\Hom_\catD(Y\ox L,\sHom_r(K,X))\\
    \cong&\Hom_\catD((Y\ox L)\ox K,X)\\
    \cong&\Hom_\catD(Y\ox(L\ox K),X)\\
    \cong&\Hom_\catD(Y\ox(K\ox L),X)\\
    \cong&\Hom_\catD(Y,\sHom_r(K\ox L,X)),
    \end{aligned} 
    $$
    $r^\op$ is defined by the natural isomorphism $$\Hom_\catD(Y,\sHom_r(S,X))\cong\Hom_\catD(Y\ox S,X)\cong\Hom_\catD(Y,X);$$ 
    and maps a closed $\catC$-module functor $(F,G,\mu)$ to $(G,F,\mu^\op)$, where $\mu^\op$ is defined by the natural isomorphism
    $$
    \begin{aligned}
         &\Hom_\catD(X,\sHom_r(K,GY))\\
    \cong&\Hom_\catD(X\ox K,GY)\\
    \cong&\Hom_\catE(F(X\ox K),Y)\\
    \cong&\Hom_\catE(FX\ox K,Y)\\
    \cong&\Hom_\catE(FX,\sHom_r(K,Y))\\
    \cong&\Hom_\catD(X,G\sHom_r(K,Y));
    \end{aligned} 
    $$
    and maps a closed $\catC$-module natural transformation $\tau$ to $\tau^\op$. Moreover $(-^\op)^2=\1$.
}

\subsection{Monoidal Model Categories}

\defn{
    Suppose that $\catC,\catD,\catE$ are model categories. An adjunction of 2 variables $(\ox,\sHom_l,\sHom_r,\vp_l,\vp_r):\catC\times\catD\to\catE$ 
    is called a \term{Quillen adjunction of 2 variables} if for any $f:A\to B\in\Cof_\catC,g:C\to D\in\Cof_\catD$, the \term{pushout product}
    $$f\squ g:A\ox D\amalg_{A\ox C}B\ox C\to B\ox D$$ is a cofibration, and is trivial if $f$ or $g$ is. If so $\ox$ is called a \term{Quillen bifunctor}.
}

\lem{
    Suppose $(\ox,\sHom_l,\sHom_r):\catC\times\catD\to\catE$ is an adjunction of 2 variables between model categories $\catC,\catD,\catE$. 
    Then the following statements are equivalent:
    \begin{enumerate}[i)]
    \item $\ox$ is a Quillen bifunctor;
    \item For any $f:A\to B\in\Cof_\catC,p:X\to Y\in\Fib_\catE$, the map $$\sHom_{\squ,l}(f,p):
    \sHom_l(A,X)\times_{\sHom_l(B,X)}\sHom_l(B,Y)\to\sHom_l(A,Y)$$ is a fibration, and is trivial if $f$ or $p$ is;
    \item For any $g:C\to D\in\Cof_\catD,p:X\to Y\in\Fib_\catE$, the map $$\sHom_{\squ,r}(g,p):
    \sHom_r(C,X)\times_{\sHom_r(D,X)}\sHom_r(D,Y)\to\sHom_r(C,Y)$$ is a fibration, and is trivial if $g$ or $p$ is.
    \end{enumerate}
}

\lem{
    Suppose $(\ox,\sHom_l,\sHom_r):\catC\times\catD\to\catE$ is a Quillen adjunction of 2 variables between model categories $\catC,\catD,\catE$. 
    If $C$ is cofibrant in $\catC$, then $(C\ox-,\sHom_l(C,-)):\catD\to\catE$ is a Quillen adjunction. If $D$ is cofibrant in $\catD$, 
    then $(-\ox D,\sHom_r(D,-)):\catC\to\catE$ is a Quillen adjunction. If $E$ is fibrant in $\catE$, then $(\sHom_l(-,E),\sHom_r(-,E)):\catC\to\catD^\op$
    is a Quillen adjunction.
}

\lem{
    Suppose $(\ox,\sHom_l,\sHom_r):\catC\times\catD\to\catE$ is an adjunction of 2 variables between categories $\catC,\catD,\catE$. 
    Suppose $I,I'$ are collections of maps in $\catC,\catD$, respectively. Then $\Cof(I)\squ\Cof(I')\subseteq\Cof(I\squ I')$.
}

\cor{
    Suppose $(\ox,\sHom_l,\sHom_r):\catC\times\catD\to\catE$ is an adjunction of 2 variables between model categories $\catC,\catD,\catE$. 
    Suppose more that $\catC,\catD$ are cofibrantly generated model categories, with generating cofibrations $I,I'$, generating trivial cofibrations 
    $J,J'$, respectively. Then $\ox$ is a Quillen bifunctor, if and only if $I\squ I'\subseteq\Cof_\catE,I\squ J'\subseteq\W_\catE\cap\Cof_\catE,
    J\squ I'\subseteq\W_\catE\cap\Cof_\catE$.
}

\rmk{
    If $\catC$ is a closed monoidal category, then $(\squ,\sHom_{\squ,l},\sHom_{\squ,r})$ forms a closed monoidal structure on $\catC^{[1]}$.
}

\defn{
    A \term{monoidal model category} is a model category together with a closed monoidal structure, such that:
    \begin{enumerate}[i)]
    \item The monoidal structure bifunctor $\ox$ is a Quillen bifunctor;
    \item For any $X$ cofibrant, the natural transformation $QS\ox X\to S\ox X$ is a weak equivalence;
    \item For any $X$ cofibrant, the natural transformation $X\ox QS\to X\ox S$ is a weak equivalence.
    \end{enumerate}
    A \term{symmetric monoidal model category} is a monoidal model category that is also a symmetric monoidal category.
}

\lem{
    Suppose $\catC$ is a model category that is also a closed monoidal category. Then for any $X$ cofibrant, the natural transformation 
    $QS\ox X\to X$ is a weak equivalence, if and only if for any fibrant $X$, the natural transformation $X\to\sHom_l(QS,X)$ is a weak equivalence; 
    For any $X$ cofibrant, the natural transformation $X\ox QS\to X$ is a weak equivalence, if and only if for any fibrant $X$, 
    the natural transformation $X\to\sHom_r(QS,X)$ is a weak equivalence.
}

\prop{\label{tagd}
    Suppose $\catC$ is a monoidal model category with the unit being the terminal object $*$. Suppose more that $*$ is cofibrant. 
    Then $\catC_*$ is a monoidal model category, with $X\wedge Y$ defined to be the pushout
    $$\xymatrix @C=80pt{
    X\amalg Y\ar[r]^{(\1_X\ox w)\amalg(v\ox\1_Y)}\ar[d]&X\ox Y\ar[d]\\
    {*}\ar[r]&X\wedge Y
    }$$
    for any $(X,v),(Y,w)\in\catC_*$, with the unit of $\wedge$ being $*_+$; $\sHom_{l,*}(X,Y)$ defined to be the pullback 
    $$\xymatrix @C=80pt{
    \sHom_{l,*}(X,Y)\ar[r]\ar[d]& \sHom_l(X,Y)\ar[d]^{\sHom_l(v,Y)}\\
    {*}\ar[r]^{\sHom_l(X,w)}& \sHom_l(*,Y)
    }$$ 
    for any $(X,v),(Y,w)\in\catC_*$, with the basepoint of $\sHom_{l,*}(X,Y)$ being induced by the maps $*\to *$ and 
    $$\vp_l(X\ox*\cong X\to*\xrightarrow{w}Y):*\to\sHom_l(X,Y);$$ $\sHom_{r,*}(X,Y)$ defined similarly. If $\catC$ is a 
    symmetric monoidal model category, so is $\catC_*$.
}

\prop{
    $\cat{SSet}$ is a symmetric monoidal model category, with $\ox=\times,\sHom_l=\sHom_r=-^-$. $\cat{SSet}_*$ is a symmetric monoidal model category.
}

\prop{
    $\cat{K}$ is a symmetric monoidal model category, with $\ox=\times,\sHom_l=\sHom_r=\sHom$. $\cat{K}_*$ is a symmetric monoidal model category.
}

\rmk{
    $\cat{Top}$ is not a monoidal model category.
}

\prop{\label{tagc}
    If $R$ is a commutative ring, then $\cat{Ch}_R$ with the projective model structure is a symmetric monoidal model category, where $X\ox Y$ 
    is defined to be $$\begin{aligned}(X\ox Y)_n&=\bigoplus_{k\in\mathbb{Z}}X_k\ox_RY_{n-k},\\
    d_n(x\ox y)&=(dx)\ox y+(-1)^{\abs{x}}x\ox(dy);\end{aligned}$$ $\sHom(X,Y):=\sHom_l(X,Y)=\sHom_r(X,Y)$ is defined to be
    $$\begin{aligned}\sHom(X,Y)_n&=\prod_{k\in\mathbb{Z}}\Hom(X_k,Y_{n+k}),\\(d_n(f))(x)&=d(f(x))+(-1)^{n+1}f(dx);\end{aligned}$$ 
    The associativity isomorphism $T$ is defined to be $T(x\ox y)=(-1)^{\abs{x}\abs{y}}y\ox x$.
}

\prop{
    If $R$ is a ring, then the functor $\ox:\cat{Ch}_{R^\op}\times\cat{Ch}_R\to\cat{Ch}_{\mathbb{Z}}$ defined in Proposition \ref{tagc} 
    is a Quillen bifunctor.
}

\rmk{
    $\cat{Ch}_R$ with the injective model structure is not a monoidal model category. For example, in $\cat{Ch}_\mathbb{Z}$,
    $$(S^0(\mathbb{Z})\to S^0(\mathbb{Q}))\squ(0\to S^0(\mathbb{Z}/2\mathbb{Z}))=(S^0(\mathbb{Z}/2\mathbb{Z})\to 0),$$ which is certainly not injective.
}

\defn{
    Suppose $\catC,\catD$ are monoidal model categories. A \term{monoidal Quillen adjunction} from $\catC$ to $\catD$ is a Quillen adjunction $(F,G)$ 
    such that $F$ is monoidal, and the map $FQS\to QS$ is a weak equivalence. If so $F$ is called a \term{monoidal Quillen functor}. 
    A \term{symmetric monoidal Quillen functor} is a monoidal Quillen functor that is a symmetric monoidal functor.
}

\lem{
    The composition of two monoidal Quillen functors is a monoidal Quillen functor.
}

\lem{
    Monoidal model categories, monoidal Quillen functors and monoidal natural transformations form a 2-category, which we denote $\catt{MonModel}$.
    Symmetric monoidal model categories, symmetric monoidal Quillen functors and monoidal natural transformations form a 2-category, 
    which we denote $\catt{SymMonModel}$.
}

\eg{
    $-_+:\cat{SSet}\to\cat{SSet}_*,-_+:\cat{K}\to\cat{K}_*$ are monoidal Quillen adjunctions.
}

\eg{
    If $R\to S$ is a ring homomorphism, then the induced Quillen adjunction $\cat{Ch}_R\to\cat{Ch}_S$ is a monoidal Quillen adjunction.
}

\prop{
    $\abs{-}:\cat{SSet}\to\cat{K},\abs{-}:\cat{SSet}_*\to\cat{K}_*$ are monoidal Quillen adjunctions.
}

\defn{
    Suppose $\catC$ is a monoidal model category. A \term{$\catC$-model category} is a model category $\catD$ together with a 
    closed right $\catC$-module structure, such that $\ox:\catD\times\catC\to\catD$ is a Quillen bifunctor, and for any $X$ cofibrant, 
    the natural transformation $X\ox QS\to X\ox S$ is a weak equivalence. A $\cat{SSet}$-model category is called a \term{simplicial model category}.
}

\defn{
    Suppose $\catC$ is a monoidal model category. A \term{$\catC$-Quillen functor} between $\catC$-model categories is a Quillen functor 
    that is also a closed $\catC$-module functor.
}

\eg{
    $\cat{SSet},\cat{SSet}_*,\cat{K},\cat{K}_*$ are simplicial model categories.
}

\eg{
    For any ring $R$, $\cat{Ch}_R$ is a $\cat{Ch}_\mathbb{Z}$-model category.
}

\lem{
    If $\catC$ is a monoidal model category, all $\catC$-model categories, $\catC$-Quillen functors and $\catC$-module natural transformations 
    form a 2-category, which we denote $\catt{Model}_\catC$. Any monoidal Quillen functor between monoidal model categories $\catC$ and $\catD$ 
    induces a forgetful 2-functor from $\catt{Model}_\catD$ to $\catt{Model}_\catC$.
}

\lem{
    If $\catC$ is a pointed monoidal model category, every $\catC$-model category is pointed.
}

\prop{
    Suppose $\catC$ is a monoidal model category with the unit being the terminal object $*$. Suppose more that $*$ is cofibrant. 
    If $\catD$ is a $\catC$-model category, then $\catD_*$ is a $\catC_*$-model category, where the $\catC_*$-module structure 
    is given similar to Proposition \ref{tagd}.
}

\lem{
    If $\catC$ is a symmetric monoidal model category, the duality 2-functor on $\catt{Model}$ and the duality 2-functor on $\catt{Mod}_\catC$ 
    induces a duality 2-functor on $\catt{Model}_\catC$.
}

\defn{
    Suppose $\catC$ is a monoidal model category. A \term{monoidal $\catC$-model category} is a monoidal model category $\catD$ together with 
    a monoidal Quillen functor $\catC\to\catD$. A \term{monoidal $\catC$-Quillen functor} between monoidal $\catC$-model categories 
    is a Quillen functor that is also a closed $\catC$-algebra functor. We have similar definitions
     for a \term{symmetric monoidal $\catC$-model category} and a \term{symmetric monoidal $\catC$-Quillen functor}.
}

\lem{
    If $\catC$ is a monoidal model category, all monoidal $\catC$-model categories, monoidal $\catC$-Quillen functors and $\catC$-algebra 
    natural transformations form a 2-category, which we denote $\catt{MonModel}_\catC$. Any monoidal Quillen functor between 
    monoidal model categories $\catC$ and $\catD$ induces a forgetful 2-functor from $\catt{MonModel}_\catD$ to $\catt{MonModel}_\catC$. 
    We have a similar atatement for the symmetric case.
}

\eg{
    $\cat{K}$ is a symmetric monoidal $\cat{SSet}$-model category.
}

\prop{
    Suppose $(\ox,\sHom_l,\sHom_r,\vp_l,\vp_r):\catC\times\catD\to\catE$ is a Quillen adjunction of 2 variables between model categories 
    $\catC,\catD,\catE$. Then the derived functors defines an adjunction of 2 variables $$(\ox^L,R\sHom_l,R\sHom_r,\Ho\vp_l,\Ho\vp_r):
    \Ho\catC\times\Ho\catD\to\Ho\catE,$$ where $\Ho\vp_l$ is given by
    $$
    \begin{aligned}
    {[C\ox^LD,E]}&=[QC\ox QD,E]\\
    &\cong[QC\ox QD,RE]\\
    &\cong[QD,\sHom_l(QC,RE)]\\
    &\cong[D,\sHom_l(QC,RE)]=[D,(R\sHom_l)(C,E)];
    \end{aligned} 
    $$
    and $\Ho\vp_r$ is given similarly.
}

\thm{\label{tagj}
    The pseudo-2-functor $\Ho:\catt{Model}\to\catt{Cat}_{ad}$ lifts to a pseudo-2-functor $$\Ho:\catt{MonModel}\to\catt{CloMon},$$ 
    where a monoidal model category $$(\catC,\ox,\sHom_l,\sHom_r,\vp_l,\vp_r,a,l,r,S)$$ is mapped to 
    $$(\Ho\catC,\ox^L,R\sHom_l,R\sHom_r,\Ho\vp_l,\Ho\vp_r,a,l,r,QS),$$ where $a$ is given by
    $$
    \begin{aligned}
    (X\ox^LY)\ox^LZ&=Q(QX\ox QY)\ox QZ\\
    &\cong(QX\ox QY)\ox QZ\\
    &\cong QX\ox(QY\ox QZ)\\
    &\cong QX\ox Q(QY\ox QZ)=X\ox^L(Y\ox^LZ),
    \end{aligned} 
    $$
    $l$ is given by $$S\ox^LX=QS\ox QX\cong S\ox QX\cong QX\cong X,$$ $r$ is given similarly; a monoidal Quillen adjunction 
    $$(F,G,\vp,\mu,\alpha)$$ is mapped to $$(LF,RG,\Ho\vp,\mu,\alpha),$$ where $\mu$ is given by 
    $$
    \begin{aligned}
    (LF)X\ox^L(LF)Y&=QFQX\ox QFQY\\
    &\cong FQX\ox FQY\\
    &\cong F(QX\ox QY)\\
    &\cong FQ(QX\ox QY)=(LF)(X\ox^LY),
    \end{aligned} 
    $$
    $\alpha$ is given by $$(LF)S=FQS\cong FS\cong S;$$ a monoidal natural transformation $\tau$ is mapped to $L\tau$.
}

\thm{
    The pseudo-2-functor $\Ho:\catt{MonModel}\to\catt{CloMon}$ lifts to a pseudo-2-functor $$\Ho:\catt{SymMonModel}\to\catt{CloSymMon},$$ 
    where the natural isomorphism $T$ is given by $$X\ox^LY=QX\ox QY\cong QY\ox QX=Y\ox^LX.$$
}

\thm{
    Suppose $\catC$ is a monoidal model category, then the pseudo-2-functor $\Ho:\catt{Model}\to\catt{Cat}_{ad}$ lifts to a pseudo-2-functor 
    $$\Ho:\catt{Model}_\catC\to\catt{CloMod}_{\Ho\catC},$$ where the natural isomorphisms $a,r,\mu$ are defined similarly to that of 
    Theorem \ref{tagj}; lifts to a pseudo-2-functor $$\Ho:\catt{MonModel}_\catC\to\catt{CloAlg}_{\Ho\catC},$$ where $(\catD,i)$ is mapped to 
    $(\Ho\catD,Li)$, $(F,\rho)$ is mapped to $(LF,L\rho)$; and lifts to a pseudo-2-functor $$\Ho:\catt{SymMonModel}_\catC\to\catt{CloSymAlg}_{\Ho\catC}$$ 
    if $\catC$ is symmetric. In the latter case $$\Ho\circ\9-^\op\0=\9-^\op\0\circ\Ho:\catt{SymMonModel}_\catC\to\catt{CloSymAlg}_{\Ho\catC}.$$
}

\prop{
    $L\abs{-}:\Ho\cat{SSet}\to\Ho\cat{K}$ is a equivalence between closed symmetric monoidal categories. $L\abs{-}_*:\Ho\cat{SSet}_*\to\Ho\cat{K}_*$ 
    is a equivalence between closed symmetric monoidal categories.
}

\section{Framings}

\subsection{Diagrams over Reedy Categories}

\defn{
    Suppose $\catB$ is a small category, and $\lambda$ is an ordinal. A functor $f:\catB\to\lambda$ is called a \term{linear extension} if the image
    of a nonidentity map is nonidentity. If so define $f(i)$ to be the \term{degree} of $i$ for any object $i\in\catB$. If such linear extension exists
    we call $\catB$ a \term{direct category}. If $\catB^\op$ is a direct category we call $\catB$ an \term{inverse category}.
}

\defn{
    Suppose $\catC$ is a category with all small colimits, $\catB$ is a direct category, $i\in\catB$. Define the \term{latching space functor} 
    $L_i:\catC^{\catB}\to\catC$ as the composition $$\catC^\catB\to\catC^{\catB_i}\xrightarrow{\colim}\catC,$$ where $\catB_i$ is defined 
    to be the category of all nonidentity maps in $\catB$ with codomain $i$. Dually, suppose $\catC$ is a category with all small limits, 
    $\catB$ is an inverse category, $i\in\catB$. Define the \term{matching space functor} $M_i:\catC^{\catB}\to\catC$ as the composition
    $$\catC^\catB\to\catC^{\catB^i}\xrightarrow{\lim}\catC,$$ where $\catB^i$ is defined to be the category of all nonidentity maps 
    in $\catB$ with domain $i$.
}

\lem{
    If $\catC$ is a category with all small colimits, $\catB$ is a direct category, $i\in\catB$, then there exists a natural transformation 
    $L_i\to-_i:=-(i)$; Dually if $\catC$ is a category with all small limits, $\catB$ is an inverse category, $i\in\catB$, then there exists 
    a natural transformation $-_i\to M_i$.
}

\prop{
    Suppose $\catC$ is a model category, $\catB$ is a direct category, and we have a commutative diagram in $\catC^\catB$:
    $$\xymatrix{
    A\ar[r]\ar[d]_f&X\ar[d]^p\\
    B\ar[r]&Y
    }$$
    such that $p_i:X_i\to Y_i$ is a fibration for any $i\in\catB$, $g_i:A_i\amalg_{L_iA}L_iB\to B_i$ is a cofibration for any $i\in\catB$. 
    If $f_i$ is a trivial fibration for any $i$ or $g_i$ is a trivial cofibration for any $i$, then the above diagram has a lift.
}

\cor{
    Suppose $\catC$ is a model category, $\catB$ is a direct category, $f:A\to B\in\catC^\catB$. If the map $g_i:A_i\amalg_{L_iA}L_iB\to B_i$ 
    is a (trivial) cofibration for any $i\in\catB$, then $\colim f:\colim A\to\colim B$ is a (trivial) cofibration.
}

\thm{
    Suppose $\catC$ is a model category, $\catB$ is a direct category. Then there exists a model structure on $\catC^\catB$ where a map 
    is a weak equivalence if and only if it is a degreewise weak equivalence; a map is a fibration if and only if it is a degreewise fibration; 
    a map $f:A\to B$ is a (trivial) cofibration if and only if $g_i:A_i\amalg_{L_iA}L_iB\to B_i$ is a (trivial) cofibration for any $i\in\catB$.
    Dually suppose $\catC$ is a model category, $\catB$ is an inverse category. Then there exists a model structure on $\catC^\catB$ where a map
    is a weak equivalence if and only if it is a degreewise weak equivalence; a map is a cofibration if and only if it is a degreewise cofibration;
    a map $p:X\to Y$ is a (trivial) fibration if and only if $q_i:X_i\to Y_i\times_{M_iY}M_iX$ is a (trivial) fibration for any $i\in\catB$.
}

\cor{
    Suppose $\catC$ is a model category, $\catB$ is a direct category. Then the colimit functor $\colim:\catC^\catB\to\catC$ is a left Quillen functor,
    left adjoint to the constant diagram functor. Dually suppose $\catC$ is a model category, $\catB$ is an inverse category. Then the limit functor
    $\lim:\catC^\catB\to\catC$ is a right Quillen functor, right adjoint to the constant diagram functor.
}

\cor{
    Suppose $\catC$ is a model category, $\catB$ is a direct category. If $f:A\to B$ is a (trivial) cofibration in $\catC^\catB$, 
    then for any $i\in\catB$, $f_i$ is a (trivial) cofibration. Dually suppose $\catC$ is a model category, $\catB$ is an inverse category. 
    If $p:X\to Y$ is a (trivial) fibration in $\catC^\catB$, then for any $i\in\catB$, $p_i$ is a (trivial) fibration.
}

\rmk{
    Suppose $\catC$ is a cofibrantly generated model category with generating cofibrations $I$ and generating trivial cofibrations $J$, 
    and $\catB$ is a direct category. Then $\catC^\catB$ is a cofibrantly generated model category with generating cofibrations 
    $\bigcup_{i\in\catB}F_iI$ and generating trivial cofibrations $\bigcup_{i\in\catB}F_iJ$, where $F_i$ is constructed to be a left adjoint of $-_i$.
}

\defn{
    Suppose $\catB$ is a small category. $\catB$ is called a \term{Reedy category} if there exists subcategories $\catB_+,\catB_-$ 
    and a \term{degree function} $d:\mathrm{Obj}(\catB)\to\lambda$ where $\lambda$ is some ordinal, such that all nonidentity morphisms 
    in $\catB_+$ raise degrees, all nonidentity morphisms in $\catB_-$ lower degrees, and for every map $f\in\catB$, there exists unique maps
    $g\in\catB_-,h\in\catB_+$, such that $f=hg$. (In particular, $\catB_+$ is a direct category and $\catB_-$ is an inverse category.)
}

\eg{
    $\d,\d^\op$ are Reedy categories. For any simplicial set $K$, $\d K$ is a Reedy category.
}

\lem{
    If $\catB$ is a Reedy category, then $\catB^\op$ is a Reedy category, where $(\catB^\op)_+=(\catB_-)^\op$ and $(\catB^\op)_-=(\catB_+)^\op$.
}

\defn{
    Suppose $\catC$ is a category with all small colimits and limits, $\catB$ is a Reedy category, $i\in\catB$. Define the \term{latching space functor}
    $L_i:\catC^{\catB}\to\catC$ as the composition $$\catC^\catB\to\catC^{\catB_+}\xrightarrow{L_i}\catC,$$ and the \term{matching space functor} 
    $M_i:\catC^{\catB}\to\catC$ as the composition $$\catC^\catB\to\catC^{\catB_-}\xrightarrow{M_i}\catC.$$
}

\lem{
    If $\catC$ is a category with all small colimits and limits, $\catB$ is a Reedy category, $i\in\catB$, then there exists natural transformations 
    $L_i\to-_i,-_i\to M_i$.
}

\eg{
    Suppose $\catC$ is a category with all small colimits and limits. Then for any $A\in\catC^\d$, $L_1A=A[0]\amalg A[0]$, $M_1A=A[0]$; 
    for any $A\in\catC^{\d^\op}$, $L_1A=A[0]$, $M_1A=A[0]\times A[0]$.
}

\lem{
    Suppose $\catC$ is a category with all small colimits and limits, $\catB$ is a Reedy category, define $\catB_{\beta}$ to be the full subcategory
    of $\catB$ of all objects with degree less than $\beta$ for an ordinal $\beta$. Then given a functor $X:\catB_\beta\to\catC$, an extension of $X$ 
    to $X':\catB_{\beta+1}\to\catC$ is equivalent to a choice of factorizations of maps $L_iX\to M_iX$ for all $i$'s with degree $\beta$; 
    given a natural transformation $\tau$ between two functors $X,Y:\catB_\beta\to\catC$ and extensions of $X,Y$ to $X',Y':\catB_{\beta+1}\to\catC$, 
    an extension of $\tau$ to $\tau':X'\to Y'$ is equivalent to a choice of maps $X'_i\to Y'_i$ making the following diagram commutative:
    $$\xymatrix{
    L_iX\ar[r]\ar[d]_{L_i\tau}&X'_i\ar[d]\ar[r]&M_iX\ar[d]^{M_i\tau}\\
    L_iY\ar[r]&Y'_i\ar[r]&M_iY    
    }$$
    for all $i$'s with degree $\beta$.
}

\prop{
    Suppose $\catC$ is a category with all small colimits and limits. Then there exists functors $l^\bullet,r^\bullet:\catC\to\catC^\d$ 
    such that $l^\bullet(-)[0]=r^\bullet(-)[0]=\1$, $L_{[n]}l^\bullet=l^\bullet(-)[n]$, $M_{[n]}r^\bullet=r^\bullet(-)[n]$. Moreover we have 
    $(l^\bullet,-[0])$, $(-[0],r^\bullet)$ are adjoint pairs, and for any $A\in\catC$, $l^\bullet A[n]=\amalg_{n+1}A$, $r^\bullet A[n]=A$. 
    Furthermore, $(l^\bullet,-[0])$, $(-[0],r^\bullet)$ are Quillen adjunctions. Dually there exists functors $l_\bullet,r_\bullet:\catC\to\catC^{\d^\op}$
    such that $l_\bullet(-)[0]=r_\bullet(-)[0]=\1$, $L_{[n]}l_\bullet=l_\bullet(-)[n]$, $M_{[n]}r_\bullet=r_\bullet(-)[n]$. Moreover we have 
    $(l_\bullet,-[0])$, $(-[0],r_\bullet)$ are adjoint pairs, and for any $A\in\catC$, $l^\bullet A[n]=A$, $r^\bullet A[n]=\Pi_{n+1}A$. Furthermore,
    $(l_\bullet,-[0])$, $(-[0],r_\bullet)$ are Quillen adjunctions.
}

\thm{
    Suppose $\catC$ is a model category, $\catB$ is a Reedy category. Then there exists a model structure on $\catC^\catB$ where a map 
    is a weak equivalence if and only if it is a degreewise weak equivalence; a map $f:A\to B$ is a (trivial) cofibration if and only if 
    $g_i:A_i\amalg_{L_iA}L_iB\to B_i$ is a (trivial) cofibration for any $i\in\catB$; a map $p:X\to Y$ is a (trivial) fibration if and only if 
    $q_i:X_i\to Y_i\times_{M_iY}M_iX$ is a (trivial) fibration for any $i\in\catB$. Such model structure is called the \term{Reedy model structure}.
}

\prop{
    If $\catC$ is a model category, $\catB$ is a Reedy category, then the Reedy model structure on $\catC^{\catB^\op}$ is the same 
    as the Reedy model structure on $(\catC^\op)^\catB$. Moreover if $\catB_1,\catB_2$ are Reedy categories, then the Reedy model structure 
    on $(\catC^{\catB_2})^{\catB_1}$ is the same as the Reedy model structure on $(\catC^{\catB_1})^{\catB_2}$.
}

\rmk{
    If $\catC$ is a model category, $\catB$ is a Reedy category, the colimit functor $\catC^\catB\to\catC$ need not be left Quillen, 
    and the limit functor $\catC^\catB\to\catC$ need not be right Quillen.
}

\lem{[The Cube Lemma]
    Suppose $\catC$ is a model category, $P_i,Q_i,R_i$ $(i=0,1)$ are cofibrant objects in $\catC$, $f_i:P_i\to Q_i,g_i:P_i\to R_i$ $(i=0,1)$ 
    are morphisms in $\catC$, such that $f_0,f_1$ are cofibrations. Define $S_i=Q_i\amalg_{P_i}R_i$ $(i=0,1)$. Suppose more there are morphisms 
    $p:P_0\to P_1,q:Q_0\to Q_1,r:R_0\to R_1$, which induce a morphism $s:S_0\to S_1$. If $p,q,r$ are weak equivalences, then $s$ is a weak equivalence; 
    if $p,(f_1,q):P_1\amalg_{P_0}Q_0\to Q_1,r$ are cofibrations, then $s$ is a cofibration.
} 

\defn{
    Suppose $\catC$ is a model category, $A,B$ are objects in $\catC$.
    \begin{enumerate}[i)]
    \item A \term{cosimplicial frame} on $A$ is a factorization $l^\bullet A\to A^*\to r^\bullet A$ in the category $\catC^\d$, such that 
    the map $l^\bullet A\to A^*$ is a cofibration, the map $A^*\to r^\bullet A$ is a weak equivalence, and the factorization restricts to identity 
    on degree 0. Dually, a \term{simplicial frame} on $A$ is a factorization $l_\bullet A\to A_*\to r_\bullet A$ in the category $\catC^{\d^\op}$, 
    such that the map $l_\bullet A\to A_*$ is a weak equivalence, the map $A_*\to r_\bullet A$ is a fibration, and the factorization 
    restricts to identity on degree 0.
    \item Suppose $A^*,B^*$ are cosimplicial frames on $A,B$, respectively. A \term{map of cosimplicial frames} between $A^*,B^*$ 
    is simply a morphism $A^*\to B^*$ in the category $\catC^\d$. Given a map $f:A\to B$, a \term{map of cosimplicial frames over $f$} 
    (or $A$, if $f$ is the identity map on $A$) is a morphism $A^*\to B^*$ in the category $\catC^\d$ making the following diagram commutative:
    $$\xymatrix{
    l^\bullet A\ar[r]\ar[d]_{l^\bullet f}&A^*\ar[r]\ar[d]&r^\bullet A\ar[d]^{r^\bullet f}\\
    l^\bullet B\ar[r]&B^*\ar[r]&r^\bullet B
    }$$
    Dually, suppose $A_*,B_*$ are simplicial frames on $A,B$, respectively. A \term{map of simplicial frames} between $A_*,B_*$ is simply 
    a morphism $A_*\to B_*$ in the category $\catC^{\d^\op}$. Given a map $f:A\to B$, a \term{map of simplicial frames over $f$} 
    (or $A$, if $f$ is the identity map on $A$) is a morphism $A_*\to B_*$ in the category $\catC^{\d^\op}$ making the following diagram commutative:
    $$\xymatrix{
    l_\bullet A\ar[r]\ar[d]_{l_\bullet f}&A_*\ar[r]\ar[d]&r_\bullet A\ar[d]^{r_\bullet f}\\
    l_\bullet B\ar[r]&B_*\ar[r]&r_\bullet B
    }$$
    \item A \term{left framing} on $\catC$ is a functor $-^*:\catC\to\catC^\d$, such that $-^*[0]$ is naturally isomorphic to the identity functor 
    on $\catC$, and $A^*$ is a cosimplicial frame on $A$ whenever $A$ is cofibrant. Dually, a \term{right framing} on $\catC$ is a functor 
    $-_*:\catC\to\catC^{\d^\op}$, such that $-_*[0]$ is naturally isomorphic to the identity functor on $\catC$, and $A_*$ is a simplicial frame 
    on $A$ whenever $A$ is fibrant.
    \item A \term{framing} on $\catC$ is a left framing together with a right framing.
    \end{enumerate} 
}

\lem{
    Suppose $\catC$ is a model category, $A$ is an object in $\catC$. A cosimplicial frame on $A$ is precisely a cosimplicial object $A^*$ 
    such that $A^*[0]$ is isomorphic to $A$, the induced map $L_nA^*\to A^*[n]$ is a cofibration for any positive $n$, and the induced map 
    $A^*[n]\to A^*[0]$ is a weak equivalence for any $n$. Dually a simplicial frame on $A$ is precisely a simplicial object $A_*$ 
    such that $A_*[0]$ is isomorphic to $A$, the induced map $A_*[0]\to A_*[n]$ is a weak equivalence for any $n$, and the induced map
    $A_*[n]\to M_nA_*$ is a fibration for any positive $n$.
}

\lem{
    Any cosimplicial frame of a cofibrant object is cofibrant, and any simplicial frame of a fibrant object is fibrant.
}

\lem{
    Any left framing on $\catC$ is a right framing on $\catC^\op$, and any right framing on $\catC$ is a left framing on $\catC^\op$.
}

\thm{
    Suppose $\catC$ is a model category. Define $-^\circ:\catC\to\catC^\d$ by induction, with $-^\circ[0]$ being the identity functor,
    and $-^\circ[n]=c\circ\alpha(L_n-^\circ\to M_n-^\circ)$. Then for any $A\in\catC$, $A^\circ$ is a cosimplicial frame on $A$, 
    thus $-^\circ$ is a left framing on $\catC$. Dually, define $-_\circ:\catC\to\catC^{\d^\op}$ by induction, with $-_\circ[0]$ 
    being the identity functor, and $-_\circ[n]=d\circ\delta(L_n-_\circ\to M_n-_\circ)$. Then for any $A\in\catC$, 
    $A_\circ$ is a simplicial frame on $A$, thus $-_\circ$ is a right framing on $\catC$.
}

\prop{
    Suppose $\catC$ is a simplicial model category. Then $-\ox\d[-]:\catC\to\catC^\d$ is a left framing on $\catC$, 
    and $-^{\d[-]}:\catC\to\catC^{\d^\op}$ is a right framing on $\catC$.
}

\rmk{
    Suppose $\catC$ is a simplicial model category. If $A$ is not cofibrant, then $A\ox\d[-]$ may not be a cosimplicial frame on $A$. 
    Dually if $A$ is not fibrant, then $A^{\d[-]}$ may not be a simplicial frame on $A$. Also, the functors $-\ox\d[-],-^\circ$ are usually not the same,
    and the functors $-^{\d[-]},-_\circ$ are usually not the same.
}

\defn{
    Suppose $\catC$ is a model category. By Proposition \ref{tagb}, $-^\circ$ induces functors $\catC\times\cat{SSet}\to\catC,(A,K)\mapsto A^\circ\ox K$,
    and $\catC^\op\times\catC\to\cat{SSet},(A,Y)\mapsto\Hom_\catC(A^\circ,Y)$. We denote $A^\circ\ox K$ by $A\ox K$ (if there is no confusion), 
    $\Hom_\catC(A^\circ,Y)$ by $\Map(A,Y)_l$, and call $(A,Y)\mapsto\Map(A,Y)_l$ the \term{left function complex functor}. 
    Dually by Proposition \ref{tage}, $-_\circ$ induces functors $\catC\times\cat{SSet}^\op\to\catC,(A,K)\mapsto A_\circ^K$,
    and $\catC^\op\times\catC\to\cat{SSet},(A,Y)\mapsto\Hom_\catC(A,Y_\circ)$. We denote $A_\circ^K$ by $A^K$, $\Hom_\catC(A,Y_\circ)$ by $\Map(A,Y)_r$,
    and call $(A,Y)\mapsto\Map(A,Y)_r$ the \term{right function complex functor}.
}

\rmk{
    $A\ox-$ is a left adjoint, but $-\ox K$ need not be.
}

\subsection{Homotopy Categories as \texorpdfstring{$\operatorname{Ho}\protect\mathsf{\protect\underline{SSet}}$}{Ho SSet}-Modules}

\prop{
    Suppose $\catC$ is a model category, and $g:K\to L$ is a cofibration in $\cat{SSet}$. If $f:A^\bullet\to B^\bullet$ is a cofibration in $\catC^\d$,
    then $f\squ g:A^\bullet\ox L\amalg_{A^\bullet\ox K}B^\bullet\ox K\to B^\bullet\ox L$ is a cofibration that is trivial if $f$ is. 
    Dually if $p:X_\bullet\to Y_\bullet$ is a fibration in $\catC^{\d^\op}$, then $p_{\squ}^g:X_\bullet^L\to Y_\bullet^L\times_{Y_\bullet^K}X_\bullet^K$ 
    is a fibration that is trivial if $p$ is.
}

\cor{
    Suppose $\catC$ is a model category, and $K$ is a simplicial set. Then the functor $-\ox K:\catC^\d\to\catC$ preserves cofibrations 
    and trivial cofibrations, and the functor $-^K:\catC^{\d^\op}\to\catC$ preserves fibrations and trivial fibrations.
}

\prop{
    Suppose $\catC$ is a model category, and $g:K\to L$ is a trivial cofibration in $\cat{SSet}$. If $f:A^*\to B^*$ is a cofibration 
    of cosimplicial frames on cofibrant objects, then $f\squ g:A^*\ox L\amalg_{A^*\ox K}B^*\ox K\to B^*\ox L$ is a trivial cofibration. 
    Dually if $p:X_*\to Y_*$ is a fibration of simplicial frames on fibrant objects, then $p_{\squ}^g:X_*^L\to Y_*^L\times_{Y_*^K}X_*^K$ 
    is a trivial fibration.
}

\cor{
    Suppose $\catC$ is a model category, $A$ is cofibrant, $Y$ is fibrant. Then for any cosimplicial frame $A^*$ on $A$, the adjunction 
    $(A^*\ox-,\Hom_\catC(A^*,-)):\cat{SSet}\to\catC$ is a Quillen adjunction. In particular, the adjunction $(A\ox-,\Map(A,-)_l):\cat{SSet}\to\catC$ 
    is a Quillen adjunction. Dually for any simplicial frame $Y_*$ on $Y$, the adjunction $(Y_*^-,\Hom_\catC(-,Y_*)):\cat{SSet}\to\catC^\op$ 
    is a Quillen adjunction. In particular, the adjunction $(Y^-,\Map(-,Y)_r):\cat{SSet}\to\catC^\op$ is a Quillen adjunction.
}

\cor{
    Suppose $\catC$ is a model category, and $K$ is a simplicial set. If $A^*\to B^*$ is a map of cosimplicial frames on cofibrant objects 
    that is a weak equivalence in degree 0, then the map $A^*\ox K\to B^*\ox K$ is a weak equivalence. Dually if $X_*\to Y_*$ is a map of 
    simplicial frames on fibrant objects that is a weak equivalence in degree 0, then the map $X_*^K\to Y_*^K$ is a weak equivalence.
}

\cor{
    Suppose $\catC$ is a model category, and $K$ is a simplicial set. Then the functor $-\ox K:\catC\to\catC$ preserves weak equivalences 
    between cofibrant objects, and the functor $-^K:\catC\to\catC$ preserves weak equivalences between fibrant objects.
}

\defn{
    We call elements in the category $\cat{SSet}^{\d^\op}$ \term{bisimplicial sets}. There is a functor $\cat{SSet}^{\d^\op}\to\cat{SSet}$ 
    induced by the diagonal functor $\d\to\d\times\d$. We call this functor $\diag$.
}

\lem{
    $\diag$ preserves colimits. Furthermore every bisimplicial set is cofibrant with respect to the Reedy model structure.
}

\prop{
    Suppose $\catC$ is a category with all small colimits and limits. Define the functor $F_n:\catC\to\catC^{\d^\op}$ with $(F_nK)_m=\amalg_{\d[n]_m}K$ 
    and the simplicial structure given by the simplicial structure of $\d[n]$ for any $K\in\catC$. Then $F_n$ is left adjoint to $-[n]$. 
    Furthermore for any $X\in\catC^{\d^\op}$, $X$ is the coequalizer of the following diagram:
    $$\xymatrix @C=200pt{\displaystyle\coprod_{[k]\to[n]}F_kX_n\ar@/^/[r]^{F_kX_n\to F_nX_n\text{, induced by the map }\d[k]\to\d[n]}
    \ar@/_/[r]_{F_kX_n\to F_kX_k\text{, induced by the map }X_n\to X_k}&\displaystyle\coprod_{k}F_kX_k}.$$
}

\prop{
    For any simplicial set $K$, we have $\diag F_nK=K\ox\d[n]$. Therefore $\diag$ is left adjoint to the functor 
    $\cat{SSet}\to\cat{SSet}^{\d^\op},K\mapsto K^{\d[-]}$.
}

\lem{
    The functor $K\mapsto K^{\d[-]}$ preserves fibrations and trivial fibrations, hence the adjunction $\9\diag,K\mapsto K^{\d[-]}\0$ 
    is a Quillen adjunction.
}

\prop{
    Suppose $X$ is a bisimplicial set such that for any map $[k]\to[n]$ in $\d$ the induced map $X[n]\to X[k]$ is a weak equivalence. 
    Then the map $X[0]\to\diag X$ is a weak equivalence.
}

\prop{
    Suppose $\catC$ is a model category, $A^*$ is a cosimplicial frame on a cofibrant object, $Y_*$ is a simplicial frame on a fibrant object. 
    Then there exists weak equivalences $$\Hom_\catC(A^*,Y)\to\diag\Hom_\catC(A^*,Y_*)\leftarrow\Hom_\catC(A,Y_*).$$
}

\cor{
    Suppose $\catC$ is a model category, $A$ is cofibrant, $Y$ is fibrant. Then $\Map(-,Y)_l$ preserves weak equivalences between cofibrant objects, 
    and $\Map(A,-)_r$ preserves weak equivalences between fibrant objects.
}

\thm{
    Suppose $\catC$ is a model category. Then the left derived functors of $-\ox-:\catC\times\cat{SSet}\to\catC$ and 
    $-^-:\catC^\op\times\cat{SSet}\to\catC^\op$ exist, which will be denoted $-\ox^L-$ and $-_R^-$. The right derived functors of 
    $\Map(-,-)_l,\Map(-,-)_r:\catC^\op\times\catC\to\cat{SSet}$ exist, which will be denoted $R\Map(-,-)_l,R\Map(-,-)_r$,
    and are naturally isomorphic. There exist natural isomorphisms $$[X\ox^LK,Y]\cong[K,R\Map(X,Y)_l]\cong[K,R\Map(X,Y)_r]\cong[Y^K_R,X],$$ 
    constructing an adjunction of 2 variables $\Ho\catC\times\Ho\cat{SSet}\to\Ho\catC$. Finally $-\ox^L\d[0]$ is naturally isomorphic to the identity.
}

\lem{
    Suppose $\catC$ is a model category, $f:A\to B$ is a morphism in $\catC$. Then for any cosimplicial frame $A^*$ on $A$, there exists a map 
    $A^*\to B^\circ$ of cosimplicial frames over $f$. Dually for any simplicial frame $B_*$ on $B$, there exists a map $A_\circ\to B_*$ 
    of simplicial frames over $f$.
}

\lem{
    Suppose $\catC$ is a model category, $A^*,B^*$ are cosimplicial frames on cofibrant objects $A,B$, respectively. Suppose $f,g:A^*\to B^*$ 
    are two maps of cosimplicial frames that agree on degree 0. Then the natural transformation $A^*\ox^L-\to B^*\ox^L-$ induced by $f$ and $g$ are equal.
}

\thm{
    Suppose $\catC$ is a model category. Then for any $A\in\catC$ and $K\in\cat{SSet}$, $A\ox(K\times\d[-])$ is a cosimplicial frame on $A\ox K$ 
    whenever $A$ is cofibrant, which induces a weak equivalence $$A\ox(K\times L)\cong (A\ox(K\times\d[-]))\ox L\to(A\ox K)\ox L$$ that is natural 
    on $L$. Define $a_{AKL}$ to be the isomorphism in $\Ho\catC$:
    $$\begin{aligned}&(A\ox^LK)\ox^LL=Q(QA\ox QK)\ox QL\xrightarrow{q_{QA\ox QK}\ox QL}(QA\ox QK)\ox QL\\
    &\hspace{0.8em}\to QA\ox(QK\times QL)\xrightarrow{QA\ox q_{QK\times QL}^{-1}}QA\ox Q(QK\times QL)=A\ox^L(K\times^LL),\end{aligned}$$
    then $a$ is natural in $A,K,L$, and $$\9\Ho\catC,-\ox^L-,R\Map(-,-),-_R^-,a,r:-\ox^L\d[0]\cong\1\0$$ gives $\Ho\catC$ 
    a closed $\Ho\cat{SSet}$-module structure.
}

\lem{
    Suppose $\catC,\catD$ are model categories, $(F,G):\catC\to\catD$ is a Quillen adjunction, $A^*$ is a cosimplicial frame 
    on a cofibrant object $A$ in $\catC$, $Y_*$ is a simplicial frame on a fibrant object $Y$ in $\catD$. Then $FA^*$ is a cosimplicial frame 
    on $FA$ and $GY_*$ is a simplicial frame on $GY$.
}

\thm{
    Suppose $\catC,\catD$ are model categories, $(F,G,\vp):\catC\to\catD$ is a Quillen adjunction. Then for any $A\in\catC$, there is a weak equivalence
    $$F(A\ox K)\cong F(A^\circ)\ox K\to FA\ox K$$ whenever $A$ is cofibrant and is natural on $K$. Define $\mu_{AK}$ to be the the isomorphism 
    in $\Ho\catC$: $$\begin{aligned}&(LFA)\ox^LK=QFQA\ox QK\xrightarrow{q_{FQA}^{-1}\ox QK}FQA\ox QK\\
    &\hspace{0.8em}\to F(QA\ox QK)\xrightarrow{Fq_{QA\ox QK}}FQ(QA\ox QK)=(LF)(A\ox^LK),\end{aligned}$$
    then $\mu$ is natural in $A,K$, and $$\9LF,RG,\Ho\vp,\mu\0$$ gives $LF$ a closed $\Ho\cat{SSet}$-module functor structure.
}

\thm{
    The pseudo-2-functor $\Ho:\catt{Model}\to\catt{Cat}_{ad}$ lifts to a pseudo-2-functor $$\Ho:\catt{Model}\to\catt{CloMod}_{\Ho\cat{SSet}}$$ 
    which is given by $$\begin{aligned}\catC&\mapsto\9\Ho\catC,-\ox^L-,R\Map(-,-),-_R^-,a,r\0,\\
    (F,G,\vp)&\mapsto\9LF,RG,\Ho\vp,\mu\0,\\
    \tau&\mapsto L\tau.\end{aligned}$$
    Moreover $\Ho\circ\9-^\op\0=\9-^\op\0\circ\Ho:\catt{Model}\to\catt{CloMod}_{\Ho\cat{SSet}}$, and the identity functor gives a 
    2-natural isomorphism between the pseudo-2-functors $$\catt{Model}_{\cat{SSet}}\xrightarrow{\Ho}\catt{CloMod}_{\Ho\cat{SSet}}$$ 
    and $$\catt{Model}_{\cat{SSet}}\to\catt{Model}\xrightarrow{\Ho}\catt{CloMod}_{\Ho\cat{SSet}}.$$
}

\prop{
    Suppose $\catC$ is a model category. Then for any $X\in\catC$, $X\ox^L-:\Ho\cat{SSet}\to\Ho\catC$ is a closed $\Ho\cat{SSet}$-module functor,
    where the structure map $\mu_{KL}$ is given by $a_{XKL}$. On the other hand, for any $K\in\cat{SSet}$, define $\mu_{XL}$ to be the natural isomorphism 
    $$(X\ox^LK)\ox^LL\to X\ox^L(K\times^LL)\to X\ox^L(L\times^LK)\to (X\ox^LL)\ox^LK,$$ then $-\ox^LK$ together with the structure map given above 
    is a closed $\Ho\cat{SSet}$-module functor.
}

\prop{
    Suppose $\catC$ is a simplicial model category. Then for any $K\in\cat{SSet}$, the closed $\Ho\cat{SSet}$-module functors $-\ox^LK$
    and $-^\circ\ox^LK$ share the same structure map $\mu$.
}

\prop{
    Suppose $\catC$ is a model category. Then for any $X\in\catC$, $K,L\in\cat{SSet}$, the associativity isomorphism is a $\Ho\cat{SSet}$-module 
    natural transformation in the following cases: $$\begin{aligned}
    (X\ox^LK)\ox^L-&\to X\ox^L(K\times^L-),\\
    (X\ox^L-)\ox^LL&\to X\ox^L(-\times^LL),\\
    (-\ox^LK)\ox^LL&\to -\ox^L(K\times^LL).
    \end{aligned} $$
} 

\prop{
    Suppose $\catC$ is a monoidal model category. Then for any $X,Y,Z\in\catC$, $X\ox^L-,-\ox^LY:\Ho\catC\to\Ho\catC$ are 
    closed $\Ho\cat{SSet}$-module functors, constructing following natural isomorphisms for $K\in\cat{SSet}$: 
    $$m^l_{XYK}:(X\ox^LY)\ox^LK\to X\ox^L(Y\ox^LK)$$ and $$m^r_{XYK}:(X\ox^LY)\ox^LK\to (X\ox^LK)\ox^LY.$$ Furthermore
    $$\begin{aligned}
    l:S\ox^L-&\to\1,\\
    r:-\ox^LS&\to\1,\\
    a_{XY-}:(X\ox^LY)\ox^L-&\to X\ox^L(Y\ox^L-),\\
    a_{X-Z}:(X\ox^L-)\ox^LZ&\to X\ox^L(-\ox^LZ),\\
    a_{-YZ}:(-\ox^LY)\ox^LZ&\to -\ox^L(Y\ox^LZ)
    \end{aligned} $$
    are $\Ho\cat{SSet}$-module natural transformations.
}

\thm{\label{tagf}
    The pseudo-2-functor $\Ho:\catt{MonModel}\to\catt{CloMon}$ lifts to a pseudo-2-functor $$\Ho:\catt{MonModel}\to\catt{CloAlg}_{\Ho\cat{SSet}},$$ 
    where a monoidal model category $(\catC,\ox)$ is mapped to $\9\Ho\catC,\ox^L,(i,\mu,\alpha)\0$, where $i:\Ho\cat{SSet}\to\Ho\catC$ is given by 
    $K\mapsto S\ox^LK$, $\mu$ is given by 
    $$\begin{aligned}&iK\ox^LiL=(S\ox^LK)\ox^L(S\ox^LL)\xrightarrow{\9m^l_{(S\ox^LK)SL}\0^{-1}}((S\ox^LK)\ox^LS)\ox^LL\\
    &\hspace{0.8em}\xrightarrow{r_{S\ox^LK}\ox^LL}(S\ox^LK)\ox^LL\xrightarrow{a_{SKL}}S\ox^L(K\times^LL)=i(K\times^LL),\end{aligned}$$
    and $\alpha$ is given by $$i(\d[0])=S\ox^L\d[0]\xrightarrow{r_{S}}S;$$ a monoidal Quillen adjunction $(F,G):\catC\to\catD$ is mapped to
    $(LF,RG,\rho)$, where $\rho$ is given by
    $$(LF)i_\catC K=(LF)(S\ox^LK)\xrightarrow{\mu_{SK}^{-1}}(LFS)\ox^LK\xrightarrow{\alpha_S\ox^LK}S\ox^LK=i_\catD K;$$ 
    a monoidal natural transformation $\tau$ is mapped to $L\tau$. Furthermore, the identity functor gives a 2-natural isomorphism 
    between the pseudo-2-functors $$\catt{MonModel}_{\cat{SSet}}\xrightarrow{\Ho}\catt{CloAlg}_{\Ho\cat{SSet}}$$ and
    $$\catt{MonModel}_{\cat{SSet}}\to\catt{MonModel}\xrightarrow{\Ho}\catt{CloAlg}_{\Ho\cat{SSet}}.$$
}

\thm{
    Suppose $\catC$ is a monoidal model category. Then for any $K\in\cat{SSet}$, the natural isomorphism
    $$X\ox^L(S\ox^LK)\xrightarrow{\9m^l_{XSK}\0^{-1}}(X\ox^LS)\ox^LK\xrightarrow{r_X\ox^LK}X\ox^LK$$ is a $\Ho\cat{SSet}$-module natural transformation 
    between $\Ho\cat{SSet}$-module functors $-\ox^L(S\ox^LK)$ and $-\ox^LK$.
    \footnote{This theorem remains unproved in class. Its proof is in Corollaire 6.7 of \textit{Propri\'et\'es universelles et extensions 
    de Kan d\'eriv\'ees''.}}
}

\thm{
    The pseudo-2-functor $$\Ho:\catt{SymMonModel}\to\catt{CloSymMon}$$ lifts to a pseudo-2-functor 
    $$\Ho:\catt{SymMonModel}\to\catt{CloSymAlg}_{\Ho\cat{SSet}},$$ where the structures are given in Theorem \ref{tagf}. 
    The identity functor gives a 2-natural isomorphism between the pseudo-2-functors 
    $$\catt{SymMonModel}_{\cat{SSet}}\xrightarrow{\Ho}\catt{CloSymAlg}_{\Ho\cat{SSet}}$$ and
    $$\catt{SymMonModel}_{\cat{SSet}}\to\catt{SymMonModel}\xrightarrow{\Ho}\catt{CloSymAlg}_{\Ho\cat{SSet}}.$$
}

\defn{
    Suppose $\catC$ is a pointed model category. By Proposition \ref{tagg}, $-^\circ$ induces functors 
    $\catC\times\cat{SSet}_*\to\catC,(A,K)\mapsto A^\circ\ov K$, and $\catC^\op\times\catC\to\cat{SSet}_*,(A,Y)\mapsto\Hom_{\catC*}(A^\circ,Y)$. 
    We denote $A^\circ\ov K$ by $A\ov K$ (if there is no confusion), $\Hom_{\catC*}\catC(A^\circ,Y)$ by $\Map_*(A,Y)_l$. 
    Dually $-_\circ$ induces functors $\catC\times\cat{SSet}_*^\op\to\catC,(A,K)\mapsto A_{\circ*}^K$,
    and $\catC^\op\times\catC\to\cat{SSet}_*,(A,Y)\mapsto\Hom_{\catC*}(A,Y_\circ)$. We denote $A_{\circ*}^K$ by $A_*^K$, 
    $\Hom_{\catC*}(A,Y_\circ)$ by $\Map_*(A,Y)_r$.
}

\prop{
    Suppose $\catC$ is a pointed model category, and $g:K\to L$ is a cofibration in $\cat{SSet}_*$. If $f:A^\bullet\to B^\bullet$ is a cofibration 
    in $\catC^\d$, then $f\squ_*g:A^\bullet\ov L\amalg_{A^\bullet\ov K}B^\bullet\ov K\to B^\bullet\ov L$ is a cofibration that is trivial if $f$ is. 
    Dually if $p:X_\bullet\to Y_\bullet$ is a fibration in $\catC^{\d^\op}$, then $p_{\squ*}^g:X_{\bullet*}^L\to 
    Y_{\bullet*}^L\times_{Y_{\bullet*}^K}X_{\bullet*}^K$ is a fibration that is trivial if $p$ is.
}

\prop{
    Suppose $\catC$ is a pointed model category, and $g:K\to L$ is a trivial cofibration in $\cat{SSet}_*$. If $f:A^*\to B^*$ is a cofibration 
    of cosimplicial frames on cofibrant objects, then $f\squ_* g:A^*\ov L\amalg_{A^*\ov K}B^*\ov K\to B^*\ov L$ is a trivial cofibration. 
    Dually if $p:X_*\to Y_*$ is a fibration of simplicial frames on fibrant objects, then $p_{\squ*}^g:X_{**}^L\to Y_{**}^L\times_{Y_{**}^K}X_{**}^K$ 
    is a trivial fibration.
}

\prop{
    Suppose $\catC$ is a pointed model category, $A^*$ is a cosimplicial frame on a cofibrant object, $Y_*$ is a simplicial frame on a fibrant object. 
    Then there exists weak equivalences $$\Hom_{\catC*}(A^*,Y)\to\diag\Hom_{\catC*}(A^*,Y_*)\leftarrow\Hom_{\catC*}(A,Y_*).$$
}

\thm{
    The pseudo-2-functor $\Ho:\catt{Model}_*\to\catt{Cat}_{ad*}$ lifts to a pseudo-2-functor $$\Ho:\catt{Model}_*\to\catt{CloMod}_{\Ho\cat{SSet}_*},$$ 
    where the structures are similar to that given above. Moreover $\Ho\circ\9-^\op\0=\9-^\op\0\circ\Ho:\catt{Model}_*\to\catt{CloMod}_{\Ho\cat{SSet}_*}$, 
    and the identity functor gives a 2-natural isomorphism between the pseudo-2-functors
    $$\catt{Model}_{\cat{SSet}_*}\xrightarrow{\Ho}\catt{CloMod}_{\Ho\cat{SSet}_*}$$ and 
    $$\catt{Model}_{\cat{SSet}_*}\to\catt{Model}_*\xrightarrow{\Ho}\catt{CloMod}_{\Ho\cat{SSet}_*}.$$
}

\thm{
    The pseudo-2-functor $\Ho:\catt{MonModel}_*\to\catt{CloMon}_*$ lifts to a pseudo-2-functor $$\Ho:\catt{MonModel}_*\to\catt{CloAlg}_{\Ho\cat{SSet}_*},$$
    where the structures are similar to that given above. Furthermore, the identity functor gives a 2-natural isomorphism between the pseudo-2-functors
    $$\catt{MonModel}_{\cat{SSet}_*}\xrightarrow{\Ho}\catt{CloAlg}_{\Ho\cat{SSet}_*}$$ and
    $$\catt{MonModel}_{\cat{SSet}_*}\to\catt{MonModel}_*\xrightarrow{\Ho}\catt{CloAlg}_{\Ho\cat{SSet}_*}.$$
}

\thm{
    Suppose $\catC$ is a pointed monoidal model category. Then for any $K\in\cat{SSet}_*$, the natural isomorphism
    $$X\ov^L(S\ov^LK)\xrightarrow{\9m^l_{XSK}\0^{-1}}(X\ov^LS)\ov^LK\xrightarrow{r_X\ov^LK}X\ov^LK$$ is a $\Ho\cat{SSet}_*$-module 
    natural transformation between $\Ho\cat{SSet}_*$-module functors $-\ov^L(S\ov^LK)$ and $-\ov^LK$.\footnote{This theorem remains unproved 
    in class as well.}
}

\thm{
    The pseudo-2-functor $$\Ho:\catt{SymMonModel}_*\to\catt{CloSymMon}_*$$ lifts to a pseudo-2-functor 
    $$\Ho:\catt{SymMonModel}_*\to\catt{CloSymAlg}_{\Ho\cat{SSet}_*},$$ where the structures are similar to that given above. 
    The identity functor gives a 2-natural isomorphism between the pseudo-2-functors 
    $$\catt{SymMonModel}_{\cat{SSet}_*}\xrightarrow{\Ho}\catt{CloSymAlg}_{\Ho\cat{SSet}_*}$$ and
    $$\catt{SymMonModel}_{\cat{SSet}_*}\to\catt{SymMonModel}_*\xrightarrow{\Ho}\catt{CloSymAlg}_{\Ho\cat{SSet}_*}.$$
}

\section{Pointed Model Categories}

\subsection{The Suspension and Loop Functors}

\defn{
    Suppose $\catC$ is a pointed model category. Define the \term{suspension functor} $\Sigma$ to be the functor $-\ov^LS^1:\Ho\catC\to\Ho\catC$, 
    and the \term{loop functor} $\Omega$ to be the functor $-_{*R}^{S^1}:\Ho\catC\to\Ho\catC$.
}

\lem{
    $(\Sigma,\Omega)$ is an adjoint pair.
}

\lem{
    Suppose $\catC$ is a pointed model category. For any $X\in\catC$, $\Sigma X$ is the cofiber of the map $QX\amalg QX\to\Cyl(QX)$, 
    and is naturally isomorphic in $\Ho\catC$ to the cofiber of the map $X\amalg X\to\Cyl X$ if $X$ is cofibrant. Dually for any $X\in\catC$,
    $\Omega X$ is the fiber of the map $\Path(RX)\to RX\times RX$, and is naturally isomorphic in $\Ho\catC$ to the fiber of the map 
    $\Path X\to X\times X$ if $X$ is fibrant.
}

\prop{
    Suppose $\catC$ is a pointed model category, $A$ is cofibrant in $\catC$, $X$ is fibrant in $\catC$. Then for any nonnegative integer $t$ 
    there exists natural isomorphisms $$[A,\Omega^tX]\cong[\Sigma^tA,X]\cong\pi_t\sHom_*(A,X)_l\cong\pi_t\sHom_*(A,X)_r.$$
}

\lem{
    Suppose $\catC$ is a pointed model category, $A$ is cofibrant in $\catC$, $X$ is fibrant in $\catC$. Then any element in $[\Sigma A,X]$ 
    can be represented by a map $k:A\to X^{\d[1]}=\Path X$ such that $p_0k=p_1k=0$, and two such maps $k,k'$ represent the same element in $[\Sigma A,X]$
    if and only if there exists a map $K:A\to X^{\d[1]\times\d[1]}$ such that $p_0^0K=k,p_0^1K=k',p_1^0K=p_1^1K=0$, 
    where $$p_0^0,p_0^1,p_1^0,p_1^1:X^{\d[1]\times\d[1]}\to X^{\d[1]}$$ are induced by the inclusions
    $$\begin{aligned}
    i_0^0:\d[1]\times\{0\}&\to\d[1]\times\d[1],\\
    i_0^1:\d[1]\times\{1\}&\to\d[1]\times\d[1],\\
    i_1^0:\{0\}\times\d[1]&\to\d[1]\times\d[1],\\
    i_1^1:\{1\}\times\d[1]&\to\d[1]\times\d[1],
    \end{aligned} $$ respectively. Dually any element in $[\Sigma A,X]$ can be represented by a map $h:\Cyl A=A\ox\d[1]\to X$ such that $hi_0=hi_1=0$, 
    and two such maps $h,h'$ represent the same element in $[\Sigma A,X]$ if and only if there exists a map $H:A\ox(\d[1]\times\d[1])\to X$ such that 
    $Hi_0^0=h,Hi_0^1=h',Hi_1^0=Hi_1^1=0$, where $$i_0^0,i_0^1,i_1^0,i_1^1:A\ox\d[1]\to A\ox(\d[1]\times\d[1])$$ are induced by the inclusions given above.
}

\lem{
    Suppose $\catC$ is a pointed model category, $A$ is cofibrant in $\catC$, $X$ is fibrant in $\catC$. Then two maps $k,k':A\to\Path X$ with 
    $p_0k=p_1k=p_0k'=p_1k'=0$ represent the same element in $[\Sigma A,X]$ if and only if there exists a map $K:\Cyl A\to\Path X$ such that 
    $p_0K=p_1K=0,Ki_0=k,Ki_1=k'$. Dually two maps $h,h':\Cyl A\to X$ with $hi_0=hi_1=h'i_0=h'i_1=0$ represent the same element in $[\Sigma A,X]$ 
    if and only if there exists a map $H:\Cyl A\to\Path X$ such that $p_0H=h,p_1H=h',Hi_0=Hi_1=0$.
}

\lem{
    Suppose $\catC$ is a pointed model category, $A$ is cofibrant in $\catC$, $X$ is fibrant in $\catC$. Then two maps $k:A\to\Path X$ with $p_0k=p_1k=0$
    and $h:\Cyl A\to X$ with $hi_0=hi_1=0$ represent the same element in $[\Sigma A,X]$ if and only if there exists a map $H:\Cyl A\to\Path X$ such that 
    $p_0H=h,p_1H=0,Hi_0=k,Hi_1=0$. Such $H$ is called a \term{correspondence} between $h$ and $k$.
}

\cor{
    Suppose $\catC$ is a pointed model category. Then for any $X$ and positive integer $t$, $\Sigma^tX$ is naturally a cogroup object in $\Ho\catC$ 
    that is naturally abelian if $t\ge 2$. Dually $\Omega^tX$ is naturally a group object in $\Ho\catC$ that is naturally abelian if $t\ge 2$.
}

\rmk{
    Suppose $\catC$ is a pointed model category, $A$ is cofibrant in $\catC$, $X$ is fibrant in $\catC$. Then the group structure on $[\Sigma A,X]$ 
    may be expressed as follows: for any $k,k':A\to\Path X$ with $p_0k=p_1k=p_0k'=p_1k'=0$ representing elements in $[\Sigma A,X]$, define $K$ 
    to be a lift of the following diagram:
    $$\xymatrix @C=70pt{
	{*}\ar[d]\ar[r]&X^{\d[2]}\ar[d]\\
	A\ar[r]^{k\text{ on }X^{\d\{0,1\}}}_{k'\text{ on }X^{\d\{1,2\}}}\ar@{.>}[ur]^{K}&X^{\l^1[2]}
    }$$
    then $[k]\cdot[k']=[X^{d_1}K]$; define $H$ to be a lift of the following diagram:
    $$\xymatrix @C=70pt{
	{*}\ar[d]\ar[r]&X^{\d[2]}\ar[d]\\
	A\ar[r]^{k\text{ on }X^{\d\{0,1\}}}_{0\text{ on }X^{\d\{0,2\}}}\ar@{.>}[ur]^{H}&X^{\l^0[2]}
    }$$
    then $[k]^{-1}=[X^{d_0}H]$.
}

\defn{
    Suppose $\catC$ is a pointed model category, $f:A\to B$ is a cofibration of cofibrant objects with cofiber $g:B\to C$, $X$ is fibrant in $\catC$. 
    Then for any map $h:A\to\Path X$ with $p_0h=p_1h=0$ representing an element in $[\Sigma A,X]$ and map $u:C\to X$ representing an element in $[C,X]$, 
    take $\alpha$ to be a lift of the following diagram:
    $$\xymatrix @C=30pt{
	A\ar[rr]^h\ar[d]_f&& \Path X\ar[d]^{p_0}\\
    B\ar[r]^u\ar@{.>}[urr]^{\alpha}&C\ar[r]^g&X
    }$$
    and take $w$ to be the map induced as follows:
    $$\xymatrix{
	A\ar[rr]^h\ar[rd]^f\ar[dd]&& \Path X\ar[dd]^{p_1}\\
    &B\ar[d]^g\ar[ur]^{\alpha}&\\
    {*}\ar[r]&C\ar@{.>}[r]^w&X
    }$$
    and define $[u]\odot[h]=[w]$. Dually suppose $p:E\to B$ is a fibration of fibrant objects with fiber $i:F\to E$, and $A$ is cofibrant in $\catC$. 
    Then for any map $h:\Cyl A\to B$ with $hi_0=hi_1=0$ representing an element in $[A,\Omega B]$ and map $u:A\to F$ representing an element in $[A,F]$,
    take $\alpha$ to be a lift of the following diagram:
    $$\xymatrix @C=30pt{
	A\ar[r]^u\ar[d]_{i_0}&F\ar[r]^i&E\ar[d]^p\\
    \Cyl A\ar@{.>}[urr]^{\alpha}\ar[rr]^h&&B
    }$$
    and take $w$ to be the map induced as follows:
    $$\xymatrix{
	A\ar@{.>}[r]^{w}\ar[dd]_{i_1}&F\ar[r]\ar[d]_{i}&{*}\ar[dd]\\
    &E\ar[dr]^{p}&\\
    \Cyl A\ar[ur]^{\alpha}\ar[rr]^{h}&&B
    }$$
    and define $[u]\odot[h]=[w]$.
}

\lem{
    Suppose $\catC$ is a pointed model category, $f:A\to B$ is a cofibration of cofibrant objects with cofiber $g:B\to C$. Then there exists 
    cylinder objects $B',C'$ for $B,C$, and maps $\Cyl A\xrightarrow{f'}B'\xrightarrow{g'}C'$, such that $f'$ is a cofibration with cofiber $g'$, 
    the following diagram is commutative:
    $$\xymatrix @C=40pt{
    A\amalg A\ar[r]^{(i_0,i_1)}\ar[d]_{f\amalg f}& \Cyl A\ar[r]^{\sigma}\ar[d]^{f'}&A\ar[d]^{f}\\
    B\amalg B\ar[r]^{(i_0,i_1)}\ar[d]_{g\amalg g}&B'\ar[r]^{\sigma}\ar[d]^{g'}&B\ar[d]^{g}\\
    C\amalg C\ar[r]^{(i_0,i_1)}&C'\ar[r]^{\sigma}&C
    }$$
    and the map $\9B\amalg B\0\amalg_{A\amalg A}\Cyl A\to B'$ is a cofibration. Dually there is a statement for fibration of fibrant objects.
}

\lem{
    Suppose $\catC$ is a pointed model category, $f:A\to B$ is a cofibration of cofibrant objects with cofiber $g:B\to C$, $X$ is fibrant in $\catC$.
    Then the map $\odot:[C,X]\times[\Sigma A,X]\to[C,X]$ is well-defined. Dually suppose $p:E\to B$ is a fibration of fibrant objects
    with fiber $i:F\to E$, and $A$ is cofibrant in $\catC$. Then the map $\odot:[A,F]\times[A,\Omega B]\to[A,F]$ is well-defined.
}

\lem{
    Suppose $\catC$ is a pointed model category, $f:A\to B$ is a cofibration of cofibrant objects with cofiber $g:B\to C$, $X$ is fibrant in $\catC$.
    Then the map $\odot:[C,X]\times[\Sigma A,X]\to[C,X]$ is natural in $X$. Dually suppose $p:E\to B$ is a fibration of fibrant objects with fiber 
    $i:F\to E$, and $A$ is cofibrant in $\catC$. Then the map $\odot:[A,F]\times[A,\Omega B]\to[A,F]$ is natural in $A$.
}

\thm{\label{tagh}
    Suppose $\catC$ is a pointed model category, $f:A\to B$ is a cofibration of cofibrant objects with cofiber $g:B\to C$, $X$ is fibrant in $\catC$.
    Then the map $\odot:[C,X]\times[\Sigma A,X]\to[C,X]$ defines a right $[\Sigma A,X]$-action on $[C,X]$, constructing a right $\Sigma A$-coaction 
    on $C$ in $\Ho\catC$. Dually suppose $p:E\to B$ is a fibration of fibrant objects with fiber $i:F\to E$, and $A$ is cofibrant in $\catC$. 
    Then the map $\odot:[A,F]\times[A,\Omega B]\to[A,F]$ defines a right $[A,\Omega B]$-action on $[A,F]$,
    constructing a right $\Omega B$-action on $F$ in $\Ho\catC$.
}

\prop{
    Suppose $\catC$ is a pointed model category, $f_i:A_i\to B_i$ are cofibrations of cofibrant objects with cofibers $g_i:B_i\to C_i$ for $i=1,2$, 
    $p:A_1\to A_2,q:B_1\to B_2$ are maps in $\catC$ such that $qf_1=f_2p$. Then the induced map $r:C_1\to C_2$ is \term{$\Sigma p$-equivariant} 
    in $\Ho\catC$, namely the following diagram is commutative in $\Ho\catC$:
    $$\xymatrix{
	C_1\ar[r]\ar[d]_{r}&C_1\amalg\Sigma A_1\ar[d]^{r\amalg\Sigma p}\\
    C_2\ar[r]&C_2\amalg\Sigma A_2
    }$$
    Dually there is a statement for fibrations of fibrant objects.
}

\defn{
    Suppose $\catC$ is a pointed model category. A \term{cofiber sequence} in $\Ho\catC$ is a diagram $X\to Y\to Z$ in $\Ho\catC$ 
    together with a right $\Sigma X$-coaction on $Z$ in $\Ho\catC$, such that there exists a commutative diagram in $\Ho\catC$:
    $$\xymatrix{
	X\ar[r]\ar[d]_{p}&Y\ar[r]\ar[d]^{q}&Z\ar[d]^{r}\\
    A\ar[r]^f&B\ar[r]^g&C
    }$$
    such that $f:A\to B$ is a cofibration of cofibrant objects with cofiber $g:B\to C$, $p,q,r$ are isomorphisms and $r$ is $\Sigma p$-equivariant. 
    Dually a \term{fiber sequence} in $\Ho\catC$ is a diagram $X\to Y\to Z$ in $\Ho\catC$ together with a right $\Omega Z$-coaction on $X$ in $\Ho\catC$,
    such that there exists a commutative diagram in $\Ho\catC$:
    $$\xymatrix{
	X\ar[r]\ar[d]_{s}&Y\ar[r]\ar[d]^{t}&Z\ar[d]^{u}\\
    F\ar[r]^i&E\ar[r]^p&B
    }$$
    such that $p:E\to B$ is a fibration of fibrant objects with fiber $i:F\to E$, $s,t,u$ are isomorphisms and $s$ is $\Omega u$-equivariant.
}

\lem{
    Suppose $\catC$ is a pointed model category. For any cofiber sequence or fiber sequence $X\to Y\to Z$, the composite of the two maps is $0$.
}

\defn{
    Suppose $\catC$ is a pointed model category. For any cofiber sequence $X\to Y\to Z$, define the \term{boundary map} $\p:Z\to\Sigma X$ to be 
    the composite $$Z\xrightarrow{\text{coaction}}Z\amalg\Sigma X\xrightarrow{0\amalg\1}\Sigma X.$$ Dually for any fiber sequence $X\to Y\to Z$, 
    define the \term{boundary map} $\p:\Omega Z\to X$ to be the composite $$\Omega Z\xrightarrow{(0,\1)}X\times\Omega Z\xrightarrow{\text{action}}X.$$
}

\lem{
    Suppose $\catC$ is a pointed model category. For any cofiber sequence $X\to Y\to Z$ and $\theta\in[\Sigma X,A]$, $\theta\p=[0]\odot\theta$. 
    Dually for any fiber sequence $X\to Y\to Z$ and $\theta\in[A,\Omega Z]$, $\p\theta=[0]\odot\theta$. 
}

\subsection{Cofiber and Fiber Sequences}

\prop{
    Suppose $\catC$ is a pointed model category. Then any diagram isomorphic to a cofiber sequence is a cofiber sequence, 
    and any diagram isomorphic to a fiber sequence is a fiber sequence.
}

\prop{
    Suppose $\catC$ is a pointed model category. Then for any $X\in\catC$, the diagram $*\to X\xrightarrow{\1}X$ together with the trivial coaction 
    of $\Sigma *=*$ on $X$ is a cofiber sequence, and the diagram $X\xrightarrow{\1}X\to*$ together with the trivial action of $\Omega *=*$ on $X$ 
    is a fiber sequence.
}

\prop{
    Suppose $\catC$ is a pointed model category. Then for any $f:X\to Y\in\Ho\catC$, there exists a cofiber sequnce $X\xrightarrow{f}Y\to Z$, 
    and there exists a fiber sequnce $W\to X\xrightarrow{f}Y$.
}

\prop{
    Suppose $\catC$ is a pointed model category. For any cofiber sequence $X\xrightarrow{f}Y\xrightarrow{g}Z$, the sequence 
    $Y\xrightarrow{g}Z\xrightarrow{\p}\Sigma X$ is a cofiber sequence, where the coaction is given by 
    $$\Sigma X\xrightarrow{m}\Sigma X\amalg\Sigma X\xrightarrow{\1\amalg\Sigma f}\Sigma X\amalg\Sigma Y\xrightarrow{\1\amalg i}\Sigma X\amalg\Sigma Y,$$
    where $m$ is the comultiplication and $i$ is the coinverse. Dually for any fiber sequence $X\xrightarrow{f}Y\xrightarrow{g}Z$, the sequence 
    $\Omega Z\xrightarrow{\p}X\xrightarrow{f}Y$ is a fiber sequence, where the action is given by 
    $$\Omega Z\times\Omega Y\xrightarrow{\1\times i}\Omega Z\times\Omega Y\xrightarrow{\1\times\Omega g}\Omega Z\times\Omega Z\xrightarrow{m}\Omega Z,$$ 
    where $m$ is the multiplication and $i$ is the inverse.
}

\prop{
    Suppose $\catC$ is a pointed model category. If we have a commutative diagram in $\Ho\catC$ as follows:
    $$\xymatrix{
    X\ar[r]^f\ar[d]_\alpha&Y\ar[r]^g\ar[d]_\beta&Z\\
    X'\ar[r]^{f'}&Y'\ar[r]^{g'}&Z'
    }$$
    where the two rows are cofiber sequences, then there exists a map $\gamma:Z\to Z'$ making the following diagram commutative:
    $$\xymatrix{
    X\ar[r]^f\ar[d]_\alpha&Y\ar[r]^g\ar[d]_\beta&Z\ar[r]\ar[d]_\gamma&Z\amalg\Sigma X\ar[d]^{\gamma\amalg\Sigma\alpha}\\
    X'\ar[r]^{f'}&Y'\ar[r]^{g'}&Z'\ar[r]&Z'\amalg\Sigma X'
    }$$
    Dually if we have a commutative diagram in $\Ho\catC$ as follows:
    $$\xymatrix{
    X\ar[r]^f&Y\ar[r]^g\ar[d]^\beta&Z\ar[d]^\alpha\\
    X'\ar[r]^{f'}&Y'\ar[r]^{g'}&Z'
    }$$
    where the two rows are fiber sequences, then there exists a map $\gamma:X\to X'$ making the following diagram commutative:
    $$\xymatrix{
    X\times\Omega Z\ar[r]\ar[d]_{\gamma\times\Omega\alpha}&X\ar[r]^f\ar[d]^\alpha&Y\ar[r]^g\ar[d]^\beta&Z\ar[d]^\alpha\\
    X'\times\Omega Z'\ar[r]&X'\ar[r]^{f'}&Y'\ar[r]^{g'}&Z'
    }$$
}

\prop{
    Suppose $\catC$ is a pointed model category. If $v:X\to Y,u:Y\to Z$ are maps in $\Ho\catC$, then there exists cofiber sequences
    $$X\xrightarrow{v}Y\xrightarrow{d}U,\quad X\xrightarrow{uv}Z\xrightarrow{a}V,\quad Y\xrightarrow{u}Z\xrightarrow{f}W,\quad
     U\xrightarrow{r}V\xrightarrow{s}W,$$
    such that the following diagram is commutative:
    $$\xymatrix{
    X\ar[r]^v\ar[d]_{\1}&Y\ar[r]^d\ar[d]_u&U\ar[r]\ar[d]_r&U\amalg\Sigma X\ar[d]^{r\amalg\1}\\
    X\ar[r]^{uv}\ar[d]_v&Z\ar[r]^a\ar[d]_{\1}&V\ar[r]\ar[d]_s&V\amalg\Sigma X\ar[d]^{s\amalg\Sigma v}\\
    Y\ar[r]^u&Z\ar[r]^f&W\ar[r]\ar[d]&W\amalg\Sigma Y\ar[ld]^{\1\amalg\Sigma d}\\
    &&W\amalg\Sigma U&
    }$$
    Dually there is a statement for fiber sequences.
}

\prop{
    Suppose $\catC$ is a pointed model category, $X\xrightarrow{f}Y\xrightarrow{g}Z$ is a cofiber sequence, $X'\xrightarrow{i}Y'\xrightarrow{p}Z'$ 
    is a fiber sequence. If we have a commutative diagram as follows:
    $$\xymatrix{
    X\ar[r]^f\ar[d]_\alpha&Y\ar[r]^g\ar[d]_\beta&Z\ar[r]^\p&\Sigma X\ar[d]^{-\vp^{-1}\alpha}\\
    \Omega Z'\ar[r]^\p&X'\ar[r]^i&Y'\ar[r]^p&Z'
    }$$
    where $\vp$ is the natural isomorphism $[\Sigma X,Z]\cong[X,\Omega Z]$ and the negative sign is taken with respect to the group structure of
    $[\Sigma X,Z]$, then there exists a fill-in map $\gamma:Z\to Y'$ making the diagram commutative. Dually if we have a commutative diagram as follows:
    $$\xymatrix{
    X\ar[r]^f\ar[d]_{-\vp\delta}&Y\ar[r]^g&Z\ar[r]^\p\ar[d]^\gamma&\Sigma X\ar[d]^\delta\\
    \Omega Z'\ar[r]^\p&X'\ar[r]^i&Y'\ar[r]^p&Z'
    }$$
    then there exists a fill-in map $\beta:Y\to X'$ making the diagram commutative.
}

\prop{
    Suppose $(F,G):\catC\to\catD$ is a Quillen adjunction between pointed model categories. Then $LF:\Ho\catC\to\Ho\catD$ maps any cofiber sequence
    $X\xrightarrow{f}Y\xrightarrow{g}Z$ to a cofiber sequence $LFX\xrightarrow{LFf}LFY\xrightarrow{LFg}LFZ$, where the coaction is given by
    $$LFZ\to LF(Z\amalg\Sigma X)\cong LFZ\amalg LF\Sigma X\xrightarrow{\1\amalg\mu^{-1}}LFZ\amalg\Sigma LFX,$$ where $\mu$ is the structure isomorphism 
    $(LFX)\ov^LS^1\cong LF(X\ov^LS^1)$ of $LF$ as a $\Ho\cat{SSet}_*$-module functor. Dually $RG:\Ho\catD\to\Ho\catC$ maps any fiber sequence 
    $X\xrightarrow{f}Y\xrightarrow{g}Z$ to a fiber sequence $RGX\xrightarrow{RGf}RGY\xrightarrow{RGg}RGZ$, where the action is given by 
    $$RGX\times\Omega RGZ\xrightarrow{\1\times\9\mu^\op\0^{-1}}RGX\times RG\Omega Z\cong RG(X\times\Omega Z)\to RGX,$$
    where $\mu^\op$ is the dual natural transformation of $\mu$.
}

\prop{
    Suppose $\catC$ is a pointed model category. Then for any $A\in\Ho\catC$, $A\ov^L-$ maps cofiber sequences in $\Ho\cat{SSet}_*$ 
    into cofiber sequences in $\Ho\catC$, $A_{*R}^-$ maps cofiber sequences in $\Ho\cat{SSet}_*$ into fiber sequences in $\Ho\catC$, 
    $R\Map_*(-,A)_l\cong R\Map_*(-,A)_r$ maps fiber sequences in $\Ho\catC$ into fiber sequences in $\Ho\cat{SSet}_*$, 
    $R\Map_*(A,-)_l\cong R\Map_*(A,-)_r$ maps cofiber sequences in $\Ho\catC$ into fiber sequences in $\Ho\cat{SSet}_*$.
}

\lem{\label{tagi}
    Suppose $\catC$ is a pointed model category, $f:A\to B$ is a cofibration of cofibrant objects with cofiber $g:B\to C$. Take a cylinder object
    $A'$ of $A$, and define $r:A'\to A''$ to be the cofiber of $A\amalg A\xrightarrow{(i_0,i_1)}A'$. Define $\tilde{B}$ to be the pushout of the diagram:
    $$\xymatrix{A\amalg A\ar[r]^{(i_0,i_1)}\ar[d]_{f\amalg f}&A'\ar[d]\\B\amalg B\ar[r]&\tilde{B}}$$ and factor the map 
    $(f\sigma,\1\amalg\1):\tilde{B}\to B$ into a cofibration $\tilde{B}\to B'$ and a trivial fibration $\sigma:B'\to B$. 
    Then $B'$ is a cylinder object of $B$, and it induces a map $f':A'\to B'$ commuting with $i_0,i_1,\sigma$.
    Let $B_f$ to be the pushout of the diagram: $$\xymatrix{A\ar[r]^{i_0}\ar[d]_f&A'\ar[d]\\B\ar[r]^f&B_f}$$ 
    Then the following diagram is a pushout square: $$\xymatrix{A\ar[r]^{i_1}\ar[d]_{jf}&A'\ar[d]\\B_f\ar[r]&\tilde{B}}$$ Let $\tilde{C}$ to be 
    the pushout of the diagram: $$\xymatrix{B_f\ar[r]^{(i_0,f')}\ar[d]_{(g,r)}&B'\ar[d]^k\\Cof\amalg A''\ar[r]^\pi&\tilde{C}}$$ then $\pi$ 
    is a weak equivalence, and $ki_1f=0:A\to\tilde{C}$, which induces a map $i_1:=(ki_1,0):C\to\tilde{C}$. Take a map $t:A'\to\Cyl A$ 
    commuting with $i_0,i_1,\sigma$, and call the induced map $A''\to A\ov S^1$ also by $t$. Now if we define $\psi:C\to C\amalg\Sigma A$ 
    to be the composite in $\Ho\catC$:
    $$C\xrightarrow{i_1}\tilde{C}\xrightarrow{\pi^{-1}}C\amalg A''\xrightarrow{\1\amalg t}C\amalg(A\ov S^1)\cong C\amalg(QA\ov QS^1)=C\amalg\Sigma A,$$ 
    then $\psi$ coincides with the right coaction given in Theorem \ref{tagh}. Dually there is a statement for fibration of fibrant objects.
}

\lem{
    Suppose $\catC$ is a pointed model category. If $f:A\to B$ is a cofibration of cofibrant objects with cofiber $g:B\to C$, 
    then there exists cosimplicial frames $B^*,C^*$ for $B,C$ and a cofibration $f^*:A^\circ\to B^*$ in $\catC^\d$ covering $f$ with cofiber 
    $g^*:B^*\to C^*$ covering $g$. Dually if $p:E\to B$ is a fibration of fibrant objects with fiber $i:F\to E$, then there exists simplicial frames 
    $F_*,E_*$ for $F,E$ and a fibration $p_*:E_*\to B_\circ$ in $\catC^{\d^\op}$ covering $p$ with fiber $i_*:F_*\to E_*$
    covering $i$.
}

\prop{
    Suppose $\catC,\catD$ are pointed model categories. Then any functor between $\catC$ and $\catD$ preserving cofibrations, weak equivalence 
    between cofibrant objects and colimits preserves cofiber sequences, where the coaction is simply given by acting the functor on the map 
    $\psi$ given in Lemma \ref{tagi}, and any functor between $\catC$ and $\catD$ preserving fibrations, weak equivalence between fibrant objects 
    and limits preserves fiber sequences, where the action is simply given by acting the functor on the map
    given in the dual statement in Lemma \ref{tagi}.
}

\prop{
    Suppose $\catC$ is a pointed model category. Then for any $K\in\cat{SSet}_*$, $-\ov^LK$ maps a cofiber sequence $X\xrightarrow{f}Y\xrightarrow{g}Z$
    in $\Ho\catC$ into a cofiber sequence $X\ov^LK\xrightarrow{f\ov^L\1}Y\ov^LK\xrightarrow{g\ov^L\1}Z\ov^LK$ in $\Ho\catC$, where the coaction
    is given by $$\begin{aligned}&Z\ov^LK\to(Z\amalg\Sigma X)\ov^LK\\&\hspace{0.8em}\cong Z\ov^LK\amalg\Sigma X\ov^LK
    \xrightarrow{\1\amalg\mu^{-1}}Z\ov^LK\amalg\Sigma(X\ov^LK),\end{aligned}$$ where $\mu$ is the structure isomorphism 
    $(X\ov^LK)\ov^LS^1\cong(X\ov^LS^1)\ov^LK$ of $-\ov^LK$ as a $\Ho\cat{SSet}_*$-module functor. Dually $-_{*R}^K$ maps a fiber sequence
    $X\xrightarrow{f}Y\xrightarrow{g}Z$ in $\Ho\catC$ into a fiber sequence $X_{*R}^K\xrightarrow{f_{*R}^\1}Y_{*R}^K\xrightarrow{g_{*R}^\1}Z_{*R}^K$ 
    in $\Ho\catC$, where the action is given by $$Z_{*R}^K\times\Omega X_{*R}^K\xrightarrow{\1\times\9\mu^\op\0^{-1}}Z_{*R}^K\times(\Omega X)_{*R}^K
    \cong(Z\times\Omega X)_{*R}^K\to Z_{*R}^K,$$ where $\mu^\op$ is the dual natural transformation of $\mu$.
}

\subsection{Pre-Triangulated Categories}

\defn{
    Suppose $(\catS,-\ov-,\Map_*(-,-),-_*^-)$ is a closed $\Ho\cat{SSet}_*$-module. Define the \term{suspension functor} $\Sigma$ to be the functor 
    $-\ov S^1:\catS\to\catS$, and the \term{loop functor} $\Omega$ to be the functor $-_*^{S^1}:\catS\to\catS$.
}

\lem{
    Suppose $\catS$ is a closed $\Ho\cat{SSet}_*$-module. Then for any $X$ and positive integer $t$, $\Sigma^tX$ is naturally a cogroup object 
    in $\catS$ that is naturally abelian if $t\ge 2$. Dually $\Omega^tX$ is naturally a group object in $\catS$ that is naturally abelian if $t\ge 2$.
}

\defn{
    Suppose $\catS$ is a closed $\Ho\cat{SSet}_*$-module. A \term{pre-triangulation} on $\catS$ is a collection of \term{cofiber sequences} 
    or \term{left triangles}, and a collection of \term{fiber sequences} or \term{right triangles}, satisfying the following conditions:
    \begin{enumerate}[i)]
    \item A cofiber sequence is a diagram of the form $X\xrightarrow{f}Y\xrightarrow{g}Z$ with a right $\Sigma X$-coaction on $Z$.
    \item The dual statement of i) holds.
    \item Any diagram isomorphic to a cofiber sequence is a cofiber sequence.
    \item The dual statement of iii) holds.
    \item For any $X$, the diagram $*\to X\xrightarrow{\1}X$ together with the trivial coaction of $\Sigma *=*$ on $X$ is a cofiber sequence.
    \item The dual statement of v) holds.
    \item For any $f:X\to Y$, there exists a cofiber sequnce $X\xrightarrow{f}Y\to Z$.
    \item The dual statement of vii) holds.
    \item For any cofiber sequence $X\xrightarrow{f}Y\xrightarrow{g}Z$, the sequence $Y\xrightarrow{g}Z\xrightarrow{\p}\Sigma X$ is a cofiber sequence, 
    where $\p$ is the map $Z\to Z\amalg\Sigma X\xrightarrow{0\amalg\1}\Sigma X$ and the coaction is given by
    $$\Sigma X\xrightarrow{m}\Sigma X\amalg\Sigma X\xrightarrow{\1\amalg\Sigma f}\Sigma X\amalg\Sigma Y\xrightarrow{\1\amalg i}\Sigma X\amalg\Sigma Y,$$ 
    where $m$ is the comultiplication and $i$ is the coinverse.
    \item The dual statement of ix) holds.
    \item For any commutative diagram as follows:
    $$\xymatrix{
    X\ar[r]^f\ar[d]_\alpha&Y\ar[r]^g\ar[d]_\beta&Z\\
    X'\ar[r]^{f'}&Y'\ar[r]^{g'}&Z'
    }$$
    where the two rows are cofiber sequences, there exists a map $\gamma:Z\to Z'$ making the following diagram commutative:
    $$\xymatrix{
    X\ar[r]^f\ar[d]_\alpha&Y\ar[r]^g\ar[d]_\beta&Z\ar[r]\ar[d]_\gamma&Z\amalg\Sigma X\ar[d]^{\gamma\amalg\Sigma\alpha}\\
    X'\ar[r]^{f'}&Y'\ar[r]^{g'}&Z'\ar[r]&Z'\amalg\Sigma X'
    }$$
    \item The dual statement of xi) holds.
    \item For any maps $v:X\to Y,u:Y\to Z$, there exists cofiber sequences
    $$X\xrightarrow{v}Y\xrightarrow{d}U,\quad X\xrightarrow{uv}Z\xrightarrow{a}V,\quad Y\xrightarrow{u}Z\xrightarrow{f}W,\quad 
    U\xrightarrow{r}V\xrightarrow{s}W,$$
    such that the following diagram is commutative:
    $$\xymatrix{
    X\ar[r]^v\ar[d]_{\1}&Y\ar[r]^d\ar[d]_u&U\ar[r]\ar[d]_r&U\amalg\Sigma X\ar[d]^{r\amalg\1}\\
    X\ar[r]^{uv}\ar[d]_v&Z\ar[r]^a\ar[d]_{\1}&V\ar[r]\ar[d]_s&V\amalg\Sigma X\ar[d]^{s\amalg\Sigma v}\\
    Y\ar[r]^u&Z\ar[r]^f&W\ar[r]\ar[d]&W\amalg\Sigma Y\ar[ld]^{\1\amalg\Sigma d}\\
    &&W\amalg\Sigma U&
    }$$
    \item The dual statement of xiii) holds.
    \item For any cofiber sequence $X\xrightarrow{f}Y\xrightarrow{g}Z$ and fiber sequence $X'\xrightarrow{i}Y'\xrightarrow{p}Z'$ 
    and a commutative diagram as follows:
    $$\xymatrix{
    X\ar[r]^f\ar[d]_\alpha&Y\ar[r]^g\ar[d]_\beta&Z\ar[r]^\p&\Sigma X\ar[d]^{-\vp^{-1}\alpha}\\
    \Omega Z'\ar[r]^\p&X'\ar[r]^i&Y'\ar[r]^p&Z'
    }$$
    where $\vp$ is the natural isomorphism $[\Sigma X,Z]\cong[X,\Omega Z]$ and the negative sign is taken with respect to the group structure 
    of $[\Sigma X,Z]$, there exists a fill-in map $\gamma:Z\to Y'$ making the diagram commutative.
    \item The dual statement of xv) holds.
    \item For any cofiber sequence $X\xrightarrow{f}Y\xrightarrow{g}Z$ in $\catS$ and $K\in\Ho\cat{SSet}_*$, the sequence 
    $X\ov K\xrightarrow{f\ov\1}Y\ov K\xrightarrow{g\ov\1}Z\ov K$ is a cofiber sequence, where the coaction is given by
    $$\begin{aligned}&Z\ov K\to(Z\amalg\Sigma X)\ov K\\&\hspace{0.8em}\cong Z\ov K\amalg\Sigma X\ov K\xrightarrow{\1\amalg\mu^{-1}}
    Z\ov K\amalg\Sigma(X\ov K),\end{aligned}$$
    where $\mu$ is the structure isomorphism $(X\ov K)\ov S^1\cong(X\ov S^1)\ov K$ of $-\ov K$ as a $\Ho\cat{SSet}_*$-module functor.
    \item The dual statement of xvii) holds.
    \item For any cofiber sequence $K\xrightarrow{f}L\xrightarrow{g}M$ in $\Ho\cat{SSet}_*$ and $X\in\catS$, the sequence 
    $X\ov K\xrightarrow{\1\ov f}X\ov L\xrightarrow{\1\ov g}X\ov M$ is a cofiber sequence, where the coaction is given by
    $$\begin{aligned}&X\ov M\to X\ov(M\amalg\Sigma K)\\&\hspace{0.8em}\cong X\ov M\amalg X\ov\Sigma K\xrightarrow{\1\amalg\mu^{-1}}
    X\ov M\amalg\Sigma(X\ov K),\end{aligned}$$ where $\mu$ is the structure isomorphism $(X\ov S^1)\ov K\cong(X\ov K)\ov S^1$ of $-\ov S^1$ 
    as a $\Ho\cat{SSet}_*$-module functor.
    \item The dual statement of xix) holds.
    \item For any fiber sequence $X\xrightarrow{f}Y\xrightarrow{g}Z$ in $\catS$ and $A\in\catS$, the sequence
    $\Map_*(A,X)\xrightarrow{\Map_*(A,f)}\Map_*(A,Y)\xrightarrow{\Map_*(A,g)}\Map_*(A,Z)$ is a fiber sequence, where the action is given by
    $$\begin{aligned}&\Map_*(A,X)\times\Omega\Map_*(A,Z)\xrightarrow{\1\times\mu^\op}\Map_*(A,X)\times\Map_*(A,\Omega Z)\\
    &\hspace{0.8em}\cong\Map_*(A,X\times\Omega Z)\to\Map_*(A,X),\end{aligned}$$
    where $\mu^\op$ is the dual of the structure isomorphism $(A\ov K)\ov S^1\cong A\ov(K\ov^LS^1)$ of $A\ov-$ as a $\Ho\cat{SSet}_*$-module functor.
    \item The dual statement of xxi) holds.
    \end{enumerate}
    A \term{pre-triangulated category} is a closed $\Ho\cat{SSet}_*$-module with all small products and coproducts together with a pre-triangulation.
}

\prop{
    Suppose $\catS$ is a pre-triangulated category.
    \begin{enumerate}[i)]
    \item Suppose $X\xrightarrow{f}Y\xrightarrow{g}Z$ is a cofiber sequence, $W$ is an object. Then there exists a long exact sequences of pointed sets:
    $$\begin{aligned}\cdots\xrightarrow{(\Sigma\p)^*}[\Sigma Z,W]\xrightarrow{(\Sigma g)^*}[\Sigma Y,W]&\xrightarrow{(\Sigma f)^*}[\Sigma X,W]\\
    &\xrightarrow{\p^*}[Z,W]\xrightarrow{g^*}[Y,W]\xrightarrow{f^*}[X,W],\end{aligned}$$
    with the following addition properties:
    \begin{itemize}
    \item For any $a,b\in[Z,W]$, $g^*a=g^*b$ if and only if there is an $x\in[\Sigma X,W]$ such that $a\odot x=b$;
    \item For any $a,d\in[\Sigma X,W]$, $\p^*c=\p^*d$ if and only if there is an $y\in[\Sigma Y,W]$ such that $c\cdot(\Sigma f)^*y=d$.
    \end{itemize} 
    \item Suppose $X\xrightarrow{f}Y\xrightarrow{g}Z$ and $X'\xrightarrow{f'}Y'\xrightarrow{g'}Z'$ are cofiber sequences and we have 
    a commutative diagram as follows:
    $$\xymatrix{X\ar[r]^f\ar[d]_a&Y\ar[r]^g\ar[d]_b&Z\ar[r]\ar[d]_c&Z\amalg\Sigma X\ar[d]^{c\amalg\Sigma a}\\
    X'\ar[r]^{f'}&Y'\ar[r]^{g'}&Z'\ar[r]&Z'\amalg\Sigma X'}$$
    Then if $a,b$ are isomorphisms, so is $c$.
    \end{enumerate} 
    Dually there is a statement for fiber sequences.
}

\defn{
    Suppose $\catS$ and $\catT$ are pre-triangulated categories. A functor $F:\catS\to\catT$ is called a \term{right exact functor} 
    if it is a left adjoint, it is a $\Ho\cat{SSet}_*$-module functor, and it maps any cofiber sequence $X\xrightarrow{f}Y\xrightarrow{g}Z$ 
    to a cofiber sequence $FX\xrightarrow{Ff}FY\xrightarrow{Fg}FZ$, where the coaction is given by
    $$FZ\to F(Z\amalg\Sigma X)\cong FZ\amalg F\Sigma X\xrightarrow{\1\amalg\mu^{-1}}FZ\amalg\Sigma FX,$$ where $\mu$ is the structure isomorphism 
    $(FX)\ov S^1\cong F(X\ov^LS^1)$ of $F$ as a $\Ho\cat{SSet}_*$-module functor. Dually a functor $G:\catT\to\catS$ is called 
    a \term{left exact functor} if it is a right adjoint, its left adjoint is a $\Ho\cat{SSet}_*$-module functor, and it maps any fiber sequence
    $X\xrightarrow{f}Y\xrightarrow{g}Z$ to a cofiber sequence $GX\xrightarrow{Gf}GY\xrightarrow{Gg}GZ$, where the action is given by
    $$GX\times\Omega GZ\xrightarrow{\1\times\9\mu^\op\0^{-1}}GX\times G\Omega Z\cong G(X\times\Omega Z)\to GX,$$ where $\mu^\op$ is 
    the dual natural transformation of the structure isomorphism of its left adjoint as a $\Ho\cat{SSet}_*$-module functor. An adjunction 
    $(F,G):\catS\to\catT$ is called an \term{exact adjunction} if $F$ is right exact and $G$ is left exact.\footnote{The notations here 
    are OPPOSITE from the notations in the original book, since the notations given here coincide with our usual thoughts.}
}

\lem{
    Pre-triangulated categories, exact adjunctions and $\Ho\cat{SSet}_*$-module natural transformations form a 2-category, 
    which we denote $\catt{PreTr}$. There exists a duality 2-functor $-^\op$ on $\catt{PreTr}$.
}

\thm{
    The pseudo-2-functor $$\Ho:\catt{Model}_*\to\catt{CloModule}_{\Ho\cat{SSet}_*}$$ lifts to a pseudo-2-functor $$\Ho:\catt{Model}_*\to\catt{PreTr}.$$
    Moreover $\Ho\circ\9-^\op\0=\9-^\op\0\circ\Ho:\catt{Model}_*\to\catt{PreTr}$.
}

\defn{
    A \term{closed monoidal pre-triangulated category} is a closed $\Ho\cat{SSet}_*$-algebra $(\catS,\ov,i)$ with all small products and coproducts,
    together with a pre-triangulation, such that for any $X\in\catS$, the functors $X\ov-,-\ov X:\catS\to\catS$ are right exact, and the functors
    $\sHom_l(X,-),\sHom_r(X,-):\catS\to\catS,\sHom_l(-,X),\sHom_r(-,X):\catS^\op\to\catS$ are left exact. We define a \term{closed symmetric monoidal 
    pre-triangulated category} to be a closed monoidal pre-triangulated category that is also a closed symmetric $\Ho\cat{SSet}_*$-algebra.
}

\defn{
    For any closed monoidal pre-triangulated category $\catS$, define $S^n=\Sigma^nS=i(S^n)$ for any nonnegative integer $n$.
}

\lem{
    Suppose $\catS$ is a closed symmetric monoidal pre-triangulated category, then the following diagram is commutative for any nonnegative integers $m,n$:
    $$\xymatrix{S^m\ov S^n\ar[r]^\mu\ar[d]_T&S^{m+n}\ar[d]^{(-1)^{mn}\1}\\S^n\ov S^m\ar[r]^\mu&S^{m+n}}$$ where $\mu$ is the structure isomorphism 
    of the monoidal functor $i$ and the negative sign is taken with respect to the cogroup structure of $S^{m+n}$.
}

\defn{
    We define a \term{closed monoidal exact adjunction} between closed monoidal pre-triangulated categories to be an exact adjunction that is also 
    a closed $\Ho\cat{SSet}_*$-algebra functor. A \term{closed symmetric monoidal exact adjunction} between closed monoidal pre-triangulated categories 
    is an exact adjunction that is also a closed symmetric $\Ho\cat{SSet}_*$-algebra functor.
}

\lem{
    Closed monoidal pre-triangulated categories, closed monoidal exact adjunctions and $\Ho\cat{SSet}_*$-algebra natural transformations 
    form a 2-category, which we will denote $\catt{CloMonPreTr}$. Closed symmetric monoidal pre-triangulated categories, closed symmetric monoidal 
    exact adjunctions and $\Ho\cat{SSet}_*$-algebra natural transformations form a 2-category, which we denote $\catt{CloSymMonPreTr}$.
}

\thm{
    The pseudo-2-functor $$\Ho:\catt{MonModel}_*\to\catt{CloAlg}_{\Ho\cat{SSet}_*}$$ lifts to a pseudo-2-functor 
    $$\Ho:\catt{MonModel}_*\to\catt{CloMonPreTr}.$$ The pseudo-2-functor $$\Ho:\catt{SymMonModel}_*\to\catt{CloSymAlg}_{\Ho\cat{SSet}_*}$$ 
    lifts to a pseudo-2-functor $$\Ho:\catt{SymMonModel}_*\to\catt{CloSymMonPreTr}.$$
}

\section{Triangulated Categories and Stable Homotopy Categories}

\subsection{Triangulated Categories}

\defn{
    A \term{triangulated category} is a pre-triangulated category such that $\Sigma$ is an equivalence of categories. A pointed model category is
    \term{stable} if its homotopy category is triangulated. A \term{closed (symmetric) monoidal triangulated category} is a closed (symmetric) monoidal 
    pre-triangulated category that is also a triangulated category. The full sub-2-category of $\catt{PreTr}$ ($\catt{CloMonPreTr}$, 
    $\catt{CloSymMonPreTr}$) consisting of all triangulated categories will be denoted $\catt{Tr}$ ($\catt{CloMonTr}$, $\catt{CloSymMonTr}$).
}

\lem{
    Triangulated categories are additive.
}

\rmk{
    Any cofiber sequence $X\xrightarrow{f}Y\xrightarrow{g}Z$ is completely determined by the maps $f,g,\p$, and for any cofiber sequences 
    $X\xrightarrow{f}Y\xrightarrow{g}Z$ and $X'\xrightarrow{f'}Y'\xrightarrow{g'}Z'$ and maps $a:X\to X',b:Y\to Y',c:Z\to Z'$, 
    $c$ is $\Sigma a$-equivariant if and only if $\Sigma a\circ\p=\p\circ c$. Dually any fiber sequence $X\xrightarrow{f}Y\xrightarrow{g}Z$ 
    is completely determined by the maps $f,g,\p$, and for any fiber sequences $X\xrightarrow{f}Y\xrightarrow{g}Z$ and
    $X'\xrightarrow{f'}Y'\xrightarrow{g'}Z'$ and maps $a:X\to X',b:Y\to Y',c:Z\to Z'$, $a$ is $\Omega c$-equivariant if and only if 
    $a\circ\p=\p\circ\Omega c$.
}

\prop{
    Any triangulated category is a classical triangulated category.
}

\lem{
    Suppose $\catS$ is a triangulated category, and $\Sigma X\xrightarrow{\Sigma f}\Sigma Y\xrightarrow{\Sigma g}\Sigma Z\xrightarrow{\Sigma h}\Sigma^2X$
    is a cofiber sequence. Then so is $X\xrightarrow{-f}Y\xrightarrow{-g}Z\xrightarrow{-h}\Sigma X$.
}

\prop{
    Suppose $\catS$ is a triangulated category. Then $X\xrightarrow{f}Y\xrightarrow{g}Z\xrightarrow{h}\Sigma X$ is a cofiber sequence, if and only if
    $\Omega Z\xrightarrow{-\eta_X^{-1}\circ(\Omega h)}X\xrightarrow{f}Y\xrightarrow{\ve_Z^{-1}\circ g}\Omega\Sigma Z$ is a cofiber sequence.
}

\lem{
    Suppose $\catS$ is a triangulated category. Then for any cofiber sequence $X\xrightarrow{f}Y\xrightarrow{g}Z\xrightarrow{h}\Sigma X$ and object $W$,
    the sequence $$[W,X]\xrightarrow{f_*}[W,Y]\xrightarrow{g_*}[W,Z]\xrightarrow{h_*}[W,\Sigma X]$$ is exact.
}

\thm{
    Suppose $\catS$ is a triangulated category. Then $\Omega Z\xrightarrow{f}X\xrightarrow{g}Y\xrightarrow{h}Z$ is a fiber sequence, if and only if
    $\Omega Z\xrightarrow{f}X\xrightarrow{g}Y\xrightarrow{-\ve_Z^{-1}\circ h}\Omega\Sigma Z$ is a cofiber sequence.
}

\defn{
    Suppose $\catC$ is a model category, and we have a commutative square in $\catC$: $$\xymatrix{W\ar[r]^u\ar[d]_v&X\ar[d]^g\\Z\ar[r]^f&Y}$$ 
    Take the following pullback square: $$\xymatrix @C=40pt{W'\ar[r]\ar[d]&X'\ar[d]^{\delta(r_Y\circ g)}\\Z'\ar[r]^{\delta(r_Y\circ f)}&RY}$$ 
    The original square is called a \term{homotopy pullback square} if the induced map $W\to W'$ is a weak equivalence. 
    Dually take the following pushout square: $$\xymatrix@C=40pt{QW\ar[r]^{\alpha(u\circ q_W)}\ar[d]_{\alpha(v\circ q_W)}&X''\ar[d]\\Z''\ar[r]&Y''}$$
    The original square is called a \term{homotopy pushout square} if the induced map $Y''\to Y$ is a weak equivalence.
} 

\prop{
    Suppose $\catC$ is a pointed model category. Then homotopy pullback squares and homotopy pushout square coincide if and only if $\catC$ is stable.
}

\prop{
    Suppose $\catS,\catT$ are triangulated categories. An adjunction $(F,G):\catS\to\catT$ is an exact adjunction if and only if $F$ is right exact, 
    if and only if $G$ is left exact.
}

\lem{
    Suppose $\catS$ is a closed symmetric monoidal triangulated category. Then the following diagram is commutative for any integers $m,n$:
    $$\xymatrix{S^m\ov S^n\ar[r]^\mu\ar[d]_T&S^{m+n}\ar[d]^{(-1)^{mn}\1}\\S^n\ov S^m\ar[r]^\mu&S^{m+n}}$$ where $S^{-k}=\Omega^kS$ for any $k>0$, 
    $\mu$ is the structure isomorphism of the monoidal functor $i$, combined with the unit or counit isomorphisms, if necessary.
}

\subsection{Stable Homotopy Categories}

\defn{
    Suppose $\catS$ is a (pre-)triangulated category, $\mathcal{G}$ is a set of objects. $\mathcal{G}$ is called \term{a set of weak generators} 
    if $[\Sigma^nG,X]=0$ for any $G\in\mathcal{G}$ and (nonnegative) integers $n$ implies $X\cong *$. (When $n<0$, take $\Sigma^nG=\Omega^{-n}G$)
}

\defn{
    Suppose $\catS$ is a pre-triangulated category. An object $X$ is called \term{algebraically finite} if for any set $\{Y_\alpha\}_{\alpha\in K}$ 
    of objects, the induced map $$\colim\limits_{S\subseteq K,S\text{ finite}}\left[X,\coprod_{\alpha\in S}Y_\alpha\right]\to
    \left[X,\coprod_{\alpha\in K}Y_\alpha\right]$$ is an isomorphism.
}

\defn{
    An \term{algebraic stable homotopy category} is a closed symmetric monoidal triangulated category with a set of 
    algebraically finite weak generators.
}

\prop{
    Suppose $\catC$ is a pointed model category, $\lambda$ is an ordinal, $X$ is a $\lambda$-sequence of cofibrations of cofibrant objects 
    with colimit $X$, and $Y$ is fibrant. If $[X_\beta,Y]=0$ for any $\beta<\lambda$, then $[X,Y]=0$.
    \footnote{This proposition remains unproved in class.}
}

\thm{
    Suppose $\catC$ is a cofibrantly generated pointed model category with generating cofibrations $I$. Then the set of cofibers of maps of $I$ 
    is a set of weak generators for $\Ho\catC$.
}

\lem{
    Suppose $\catC$ is a finitely generated model category, $\lambda$ is an ordinal, $X,Y$ are $\lambda$-sequences of cofibrations, 
    $p:X\to Y$ is a natural transformation such that $p_\alpha$ is a (trivial) fibration for any $\alpha<\lambda$. 
    Then $\colim p_\alpha$ is a (trivial) fibration.
}

\cor{
    Suppose $\catC$ is a finitely generated model category with generating cofibrations $I$, $\catD$ is a subcategory such that domains and codomains 
    of $I$ are small relative to $\catD$. Then transfinite compositions of weak equivalences in $\catD$ are weak equivalences.
}

\thm{
    Suppose $\catC$ is a finitely generated pointed model category. Then any cofibrant object that is finite relative to cofibrations 
    is algebraically finite in $\Ho\catC$.
}

\thm{
    The pseudo-2-functor $\Ho$ lifts to a pseudo-2-functor $\Ho$ from the 2-category of finitely generated stable symmetric monoidal model categories 
    to the 2-category of algebraic stable homotopy categories.
}

\chapter{Infinity Categories (Semesters 2 \& 3)}

The following contents are given in the lectures in the spring and autumn semester of 2019. 24 lectures with a total time of 121 hours 
%The following contents are given in the lectures in the spring and autumn semester of 2019. 25 lectures with a total time of 128 hours 
have been given so far in order to finish the book \textit{Higher Topos Theory}. Infinity categories are really abstract things, 
so motivations, together with sketches of hard proofs are given in order to understand them better. 
However, details of proofs to the statements are still omitted. A few statements 
remain unproofed in class due to lack of corresponding knowledge; they are marked with $\dagger$. These statements do not include 
those that are too boring to present the proofs, which do not contain any technical difficulties, however. 

\setcounter{section}{-1}

\section{Preliminaries}

\subsection{Preliminaries for Classical Category Theory}

\label{secc}

We now give some review to some ideas in classical category theory. 

First we give some definitions on cardinal theory:

\defn{
    Suppose $\kappa$ is an infinte cardinal. $\kappa$ is called \term{regular}, if for any collection of sets $\{A_i\}_{i\in I}$,
    such that $\abs I<\kappa$ and $\abs{A_i}<\kappa$ for all $I$, we have $\abs{\coprod_{i\in I}A_i}<\kappa$.
    $\kappa$ is called \term{strongly inaccessible}, if $\kappa$ is regular, uncountable, and for any cardinal $\tau<\kappa$
    we have $2^\tau<\kappa$.
}

By definition, if $\kappa$ is strongly inaccessible, all sets with cardinality less than $\kappa$ satisfies the ZF axioms.
Thus the existence of strongly inaccessible cardinals cannot be proven in the ZF axioms.

We note more that for any cardinal $\kappa$, there exists a cardinal $\kappa'>\kappa$ such that $\kappa$ is regular.

\defn{
    Suppose $\kappa$ and $\tau$ are regular cardinals. We write $\tau\ll\kappa$, if for any $\tau_0<\tau$ and $\kappa_0<\kappa$
    we have $\kappa_0^{\tau_0}<\kappa$.
}

We note that for any regular cardinal $\tau$, there exists a regular cardinal $\kappa>\tau$ such that $\kappa\gg\tau$.
Also note that a regular cardinal $\kappa$ satisfies that $\kappa\ll\kappa$ if and only if $\kappa=\omega$
or $\kappa$ is strongly inaccessible.

We now deduce with the set-theoretic technicalities. In the theory of 1-categories, mostly we may assume that 
the hom-set between two objects to be indeed a set but allowing the object to form a proper class.
(A category with all hom's between two objects to be (small) sets is called \term{locally small}.) However,
we will see that in our theory of $\infty$-categories, objects and morphisms are treated equally,
meaning that we are facing set-theoretic problems. To fix this problem, we shall use the following terminology:
for any cardinal $\tau$, we assume that there exists a strongly inaccessible cardinal
$\kappa$ that is greater than $\tau$. We will mostly fix a strongly inaccessible cardinal $\kappa$, and we will mainly deal with
mathematical objects that are $\kappa$-small. When we need to deal with mathematical objects that are too large to be
$\kappa$-small, we shall take another strongly inaccessible cardinal $\kappa'$ that is greater than $\kappa$, and to work
with mathematical objects that are $\kappa'$-small, in which case it is okay to talk about objects that are not $\kappa$-small.
This terminology will work throughout this chapter.

Next, we discuss the theory of enriched categories.

\defn{
    Suppose $(\catC,\ox,\1_{\catC})$ is a monoidal category (where $\1_{\catC}$ is the tensor unit). A \term{$\catC$-enriched category},
    $\catD$, consists of the following data:
    \begin{enumerate}[i)]
        \item A class of objects;
        \item A object $\Hom_{\catD}(X,Y)\in\catC$ for every pair of object $X,Y\in\catD$;
        \item An identity $\1_{\catC}\to\Hom_{\catD}(X,X)$ for every object $X\in\catD$;
        \item A composition map $\Hom_{\catD}(Y,Z)\ox\Hom_{\catD}(X,Y)\to\Hom_{catD}(X,Z)$ for every triple of objects $X,Y,Z\in\catD$;
        \item Such that the composition is associative and unital, i.e. the following diagrams are commutative:
        $$\xymatrix{(\Hom_{\catD}(Z,W)\ox\Hom_{\catD}(Y,Z))\ox\Hom_{\catD}(X,Y)\ar[dd]\ar[r]&\Hom_{\catD}(Y,W)\ox\Hom_{\catD}(X,Y)\ar[d]\\
        &\Hom_{\catD}(X,W)\\\Hom_{\catD}(Z,W)\ox(\Hom_{\catD}(Y,Z)\ox\Hom_{\catD}(X,Y))\ar[r]&\Hom_{\catD}(Z,W)\ox\Hom_{\catD}(X,Z),\ar[u]}$$
        $$\xymatrix{\1_{\catC}\ox\Hom_{\catD}(X,Y)\ar[rr]\ar[rd]&&\Hom_{\catD}(Y,Y)\ox\Hom_{\catD}(X,Y)\ar[ld]\\&\Hom_{\catD}(X,Y),&}$$
        $$\xymatrix{\Hom_{\catD}(X,Y)\ox\1_{\catC}\ar[rr]\ar[rd]&&\Hom_{\catD}(X,Y)\ox\Hom_{\catD}(X,X)\ar[ld]\\&\Hom_{\catD}(X,Y).&}$$
    \end{enumerate}
    For objects $X,Y$ in $\catD$, a \term{morphism} $f$ from $X$ to $Y$ is a map $\1_\catC\to\Hom_{\catD}(X,Y)$ in $\catC$.
    When $\catC=\cat{SSet}$, a $\catC$-enriched category is also called a \term{simplicial category}.
}

For example, for any closed left $\catC$-module $\catD$, $\catD$ can be regarded as a $\catC$-enriched category, by making
$\Hom_{\catD}(X,Y)=\sHom(X,Y)$.

\defn{
    Suppose $\catC$ is a monoidal category, and $\catD,\catD'$ are $\catC$-enriched categories. A \term{$\catC$-enriched functor}
    $F$ from $\catD$ to $\catD'$ consists of the following data:
    \begin{enumerate}[i)]
        \item A map from the objects of $\catD$ to the objects of $\catD'$;
        \item A morphism $\Hom_{\catD}(X,Y)\to\Hom_{\catD'}(FX,FY)$ for any pair of objects $X,Y\in\catD$;
        \item Such that the following diagrams are commutative:
        $$\xymatrix{\Hom_{\catD}(Y,Z)\ox\Hom_{\catD}(X,Y)\ar[r]\ar[d]&\Hom_{\catD}(X,Z)\ar[d]\\
        \Hom_{\catD'}(FY,FZ)\ox\Hom_{\catD'}(FX,FY)\ar[r]&\Hom_{\catD'}(FX,FZ),}$$
        $$\xymatrix{&\1\ar[ld]\ar[rd]\\\Hom_{\catD}(X,X)\ar[rr]&&\Hom_{\catD'}(FX,FX).}$$
    \end{enumerate}
    When $\catC=\cat{SSet}$, a $\catC$-enriched functor is also called a \term{simplicial functor}.
}

\defn{
    Suppose $\catC$ is a monoidal category, $F,F':\catD\to \catD'$ are two $\catC$-enriched functors between $\catC$-enriched categories.
    A \term{$\catC$-enriched natural transformation} $\tau$ from $F$ to $F'$ consists of a map $\1\to\Hom_{\catD'}(FX,F'X)$ for any object
    $X\in\catD$, such that the following diagram commutes:
    $$\xymatrix{\Hom_{\catD'}(FX,FY)\ar[r]&\1\ox\Hom_{\catD'}(FX,FY)\ar[r]&\Hom_{\catD'}(FY,F'Y)\ox\Hom_{\catD'}(FX,FY)\ar[d]\\
    \Hom_{\catD}(X,Y)\ar[d]\ar[u]&&\Hom_{\catD'}(FX,F'Y)\\
    \Hom_{\catD'}(F'X,F'Y)\ar[r]&\Hom_{\catD'}(F'X,F'Y)\ox\1\ar[r]&\Hom_{\catD'}(F'X,F'Y)\ox\Hom_{\catD'}(FX,F'X)\ar[u]}$$
    When $\catC=\cat{SSet}$, a $\catC$-enriched natural transformation is also called a \term{simplicial natural transformation}.
}

Note that these definitions restricts to ordinary category theory when $\catC=(\cat{Set},\times)$.

By the above definitions, we may form the category $\cat{Cat}_{\catC}$ of all (small) $\catC$-enriched categories,
and the 2-category $\catt{Cat}_{\catC}$ of all (small) $\catC$-enriched categories. Also, for any two $\catC$-enriched categories
$\catD,\catD'$, we have the category $\Fun(\catD,\catD')$ of all $\catC$-enriched functors.

We also make the following definition:

\defn{
    Suppose $\catC,\catD$ are monoidal categories. A \term{lax monoidal functor} between $\catC$ and $\catD$ is a triple $(F,\mu,\beta)$, 
    where $F:\catC\to\catD$ is a functor, $\mu_{XY}:FX\ox FY\to F(X\ox Y)$ is a natural transformation, 
    $\beta:\1_{\catD}\to F(\1_{\catC})$ is a map, such that the following diagrams are commutative for any objects $X,Y,Z$:
    $$\xymatrix @C=60pt{
    (FX\ox FY)\ox FZ\ar[d]_{\mu_{XY}\ox\1_{FZ}}\ar[r]^{a_{(FX)(FY)(FZ)}}&FX\ox(FY\ox FZ)\ar[d]^{\1_{FX}\ox\mu_{YZ}}\\
    F(X\ox Y)\ox FZ\ar[d]_{\mu_{(X\ox Y)Z}}&FX\ox F(Y\ox Z)\ar[d]^{\mu_{X(Y\ox Z)}}\\
    F((X\ox Y)\ox Z)\ar[r]^{F(a_{XYZ})}&F(X\ox(Y\ox Z))
    }$$ 
    $$\xymatrix @C=40pt{
    F\1\ox FX\ar[d]_{\mu_{\1X}}&\1\ox FX\ar[d]^{l_{FX}}\ar[l]^{\beta\ox\1_{FX}}\\
    F(\1\ox X)\ar[r]^{F(l_X)}&FX
    }\xymatrix @C=40pt{
    FX\ox F\1\ar[d]_{\mu_{X\1}}&FX\ox \1\ar[d]^{r_{FX}}\ar[l]^{\1_{FX}\ox\beta}\\
    F(X\ox \1)\ar[r]^{F(r_X)}&FX
    }$$ 
}

By definition, a monoidal functor is a lax monoidal functor. For example, if $\catC$ is a monoidal category,
then $\Hom_{\catC}(\1,-):\catC\to\cat{Set}$ is a lax monoidal functor.

\rmk{
    Now, suppose we have a lax monoidal functor $F:\catC\to\catC'$ between monoidal categories. Then for any $\catC$-enriched category
    $\catD$, we may form a $\catC'$-enriched category $F\catD$, with:
    \begin{enumerate}[i)]
        \item Objects being the objects of $\catD$;
        \item For any $X,Y\in\catD$, $\Hom_{F\catD}(X,Y)=F\Hom_{\catD}(X,Y)\in\catC'$;
        \item The unit is defined to be $$\1_{\catC'}\to F\1_{\catC}\to F\Hom_{\catD}(X,X)=\Hom_{F\catD}(X,X);$$
        \item The composition is defined to be $$\begin{aligned}\Hom_{F\catD}(Y,Z)\ox\Hom_{F\catD}(X,Y)
        &=(F\Hom_{\catD}(Y,Z))\ox(F\Hom_{\catD}(X,Y))\\&\to F(\Hom_{\catD}(Y,Z)\ox\Hom_{\catD}(X,Y))\\
        &\to F\Hom_{\catD}(X,Z)=\Hom_{F\catD}(X,Z).\end{aligned}$$
    \end{enumerate}
    This construction extends to a functor $F:\cat{Cat}_{\catC}\to\cat{Cat}_{\catC'}$ and a $2$-functor
    $F:\catt{Cat}_{\catC}\to\catt{Cat}_{\catC'}$, and also a functor $F:\Fun(\catD,\catD')\to\Fun(F\catD,F\catD')$ 
    for any two $\catC$-enriched categories $\catD,\catD'$.

    In particular, if the functor $F$ is chosen to be $U=\Hom_{\catC}(\1,-):\catC\to\cat{Set}$, then for any $\catC$-enriched category
    $\catD$, the ordinary category $U\catD$ is called the \term{underlying category} of $\catD$. In this case, 
    we notice that for objects $X,Y$ in $\catD$, the set of all morphisms from $X$ to $Y$ is simply $\Hom_{U\catD}(X,Y)$. 
    Therefore it is reasonable to call a morphism $f:X\to Y$ in $\catD$ an \term{isomorphism},
    if it is an isomorphism in $U\catD$.
}

We next make the following definition.

\defn{
    Suppose $\catD$ is a $\catC$-enriched category, where $\catC$ is a closed monoidal category. $\catD$ is called \term{tensored over $\catC$},
    if for all $D\in\catD$ and $C\in\catC$, the functor $$\catD\to\catC,E\mapsto\sHom_r(C,\Hom_{\catD}(D,E))$$ is representable. 
    We denote the representation of the functor by $C\ox D$. $\catD$ is called \term{cotensored over $\catC$},
    if for all $D\in\catD$ and $C\in\catC$, the functor $$\catD^\op\to\catC,E\mapsto\sHom_l(C,\Hom_{\catD}(E,D))$$ is representable. 
    We denote the representation of the functor by $D^C$.
}

We note that for any $\catC$-enriched category $\catD$ that is tensored over $\catC$, $\catD$ is naturally a left $\catC$-module.
If moreover $\catD$ is also tensored over $\catC$, the module structure is closed. 

We next give the following proposition, whose prove may be found in [nLab]:

\prop{[$\dagger$]
    For any two $\catC$-enriched category $\catD,\catD'$ where $\catC$ is closed symmetric monoidal, the functor category
    $\Fun(\catD,\catD')$ has a natural $\catC$-enriched structure, which we will denote by $(\catD')^\catD$ (or again $\Fun(\catD,\catD')$),
    whose underlying category is the ordinary category $\Fun(\catD,\catD')$, such that we have the following reciprocity formula
    between $\catC$-enriched categories:
    $$\Fun(\catD,\Fun(\catD',\catD''))\cong\Fun(\catD\ox\catD',\catD''),$$
    where $\catD\ox\catD'$ is the $\catC$-enriched category with objects being pairs $(D,D')$ where $D\in\catD,D'\in\catD'$,
    and $\Hom_{\catD\ox\catD'}((X,X'),(Y,Y'))=\Hom_{\catD}(X,Y)\ox\Hom_{\catD'}(X',Y')$ for $X,Y\in\catD,X',Y'\in\catD'$.
}

The next concept is the slice categories.

\defn{
    Suppose $\catC$ is a category, $p:\catI\to\catC$ is a diagram in $\catC$. The \term{category of objects of $\catC$ over $p$}, $\catC_{/p}$
    is the following category:
    \begin{enumerate}[i)]
        \item The objects in $\catC_{/p}$ are pairs $(X,\tau)$, where $X$ is an object in $\catC$ and $\tau:\const_X\to p$
        is a natural transformation;
        \item A morphism from $(X,\tau)$ to $(Y,\sigma)$ is a map $f:X\to Y$ in $\catC$ such that $\sigma\circ\const_f=\tau$.
    \end{enumerate}
    Dually, the \term{category of objects of $\catC$ under $p$}, $\catC_{p/}$ is the following category:
    \begin{enumerate}[i)]
        \item The objects in $\catC_{/p}$ are pairs $(X,\tau)$, where $X$ is an object in $\catC$ and $\tau:p\to\const_X$
        is a natural transformation;
        \item A morphism from $(X,\tau)$ to $(Y,\sigma)$ is a map $f:X\to Y$ in $\catC$ such that $\const_f\circ\tau=\sigma$.
    \end{enumerate}
}

We also use the following notation:

\defn{
    Suppose $\catC$ is a category and $Z\in\catC$. For any $X,Y\in\catC$, we define $\Hom_Z(X,Y):=\Hom_{\catC_{/Z}}(X,Y)$.
}

The slice category enjoy the following property:

\prop{
    Suppose $\catC$ is a category, $p:\catI\to\catC$ is a diagram in $\catC$. The initial object of $\catC_{p/}$ is exactly
    colimit of $p$ in $\catC$. Dually the terminal object of $\catC_{/p}$ is exactly the limit of $p$ in $\catC$.
}

The slice category can also be defined by a universal property. We first make the following property:

\defn{
    Suppose $\catC,\catC'$ are two categories. The \term{join} of the two categories, $\catC\star\catC'$, is defined as follows:
    \begin{enumerate}[i)]
        \item The objects of $\catC\star\catC'$ is the disjoint union of the objects $\catC$ and $\catC'$;
        \item $$\Hom_{\catC\star\catC'}(C,C')=\begin{cases}\Hom_{\catC}(C,C'),&(C,C'\in\catC)\\\{*\},&(C\in\catC,C'\in\catC')\\
        \varnothing,&(C\in\catC',C'\in\catC)\\\Hom_{\catC'}(C,C').&(C,C'\in\catC')\end{cases}$$
    \end{enumerate}
}

We note that there exists natural inclusions $\catC\to\catC\star\catC'$ and $\catC'\to\catC\star\catC'$.

Now we may show that the slice category enjoys the following universal property:

\prop{
    Suppose $\catC$ is a category, $p:\catI\to\catC$ is a diagram in $\catC$. Then for any category $\catJ$, there exists a
    natural isomorphism $$\Fun(\catJ,\catC_{/p})\cong\Fun(\catJ\star\catI,\catC)\times_{\Fun(\catI,\catC)}\{p\}.$$
    There is a dual statement for undercategories.
}

Now we can generalize this idea to joins, overcategories, undercategories, limits and colimits to $\catC$-enriched categories,
where $\catC$ is a closed symmetric monoidal category whose tensor unit is the terminal object. Details are left to the reader.

We next introduce the language of Kan extensions:

\defn{
    Suppose $\catC$ is a category, $i:\catI\to\catJ$ is a functor between categories, which induces a functor
    $i^*:\catC^\catJ\to\catC^\catI.$ The \term{left Kan extension} of a functor $F:\catI\to\catC$, is a functor 
    $i_!F:\catJ\to\catC$, together with a natural transformation $F\to i^*i_!F$, such that for any functor $G:\catJ\to\catC$,
    the natural transformation $$\Hom_{\catC^\catJ}(i_!F,G)\to\Hom_{\catC^\catI}(i^*i_!F,i^*G)\to\Hom_{\catC^\catI}(F,i^*G)$$
    is a natural isomorphism. Dually the \term{right Kan extension} of a functor $F:\catI\to\catC$, is a functor 
    $i_*F:\catJ\to\catC$, together with a natural transformation $i^*i_*F\to F$, such that for any functor $G:\catJ\to\catC$,
    the natural transformation $$\Hom_{\catC^\catJ}(G,i_*F)\to\Hom_{\catC^\catI}(i^*G,i^*i_*F)\to\Hom_{\catC^\catI}(i^*G,F)$$
    is a natural isomorphism. A functor $i_!:\catC^\catI\to\catC^\catJ$ is called a \term{left Kan extension functor}
    if it is a left adjoint to $i^*$. Dually a functor $i_*:\catC^\catI\to\catC^\catJ$ is called a \term{right Kan extension functor}
    if it is a right adjoint to $i^*$.
}

Kan extensions are generalizations of limits and colimits, as one can see from the following proposition:

\prop{
    Suppose $\catC$ is a category, $i:\catI\to\catJ$ is a functor between categories. If $\catC$ has all colimits,
    then the left Kan extension functor exists, and for any functor $F:\catI\to\catC$ and object $Y\in\catJ$,
    $$i_!F=\colim_{X\in\catI,iX\to Y}F(X).$$ Dually, if $\catC$ has all limits,
    then the right Kan extension functor exists, and for any functor $F:\catI\to\catC$ and object $Y\in\catJ$,
    $$i_*F=\lim_{X\in\catI,Y\to iX}F(X).$$
}

The following definition and proposition also hold in enriched category theory. We will discuss the theory of homotopy Kan extensions
in the next section, and its $\infty$-categorical generalization in Section \ref{secj}.

We conclude this section with a brief discussion of presentable categories and accessible categories:

\defn{
    Suppose $\kappa$ is a regular cardinal. A category $\catC$ is called \term{$\kappa$-filtered},
    if for any category $\catD$ that is $\kappa$-small (that is, the arrows in $\catD$ hava cardinality less than $\kappa$),
    and any diagram $p:\catD\to\catC$, the category $\catC_{p/}$ is nonempty. A diagram is called \term{$\kappa$-filtered}
    if it is indexed by a $\kappa$-filtered category. A colimit is called \term{$\kappa$-filtered} if 
    it is the colimit of a $\kappa$-filtered diagram. We say a structure is \term{filtered} if it is $\omega$-filtered.
}

\defn{
    Suppose $\kappa$ is a regular cardinal. We call a category \term{$\kappa$-accessible} if the category is locally small,
    every object of the category is small, the category admits small $\kappa$-filtered colimits, and there exists a small set of objects
    that generates the category under small $\kappa$-filtered colimits. We call a category \term{accessible} if it is $\kappa$-accessible
    for some regular cardinal $\kappa$.
}

\defn{
    We call a category \term{presentable} if it is accessible and has all small colimits.
}

\lem{
    A category is presentable if the category is locally small, every object of the category is small,
    the category admits small colimits, and there exists a small set of objects that generates the category under small colimits.
}

\defn{
    Suppose $\catC$ is a category, $S$ is a class of morphisms. $S$ is called \term{weakly saturated}, if $S$ is closed under
    pushouts, retracts and transfinite compositions.
}

By the small object argument, we deduce the following:

\prop{
    Suppose $\catC$ is a presentable category. Then for any set of morphisms $S$ in $\catC$, the smallest weakly saturated class
    of morphisms in $\catC$ containing $S$ is $\Cof(S)$.
}

For further use, we state the following proposition:

\prop{[$\dagger$]
    Suppose $\catC$ is a presentable category, $\catC'$ is a full subcategory of $\catC$ that is closed under small $\kappa$-filtered colimits
    for some regular cardinal $\kappa$. The following conditions are equivalent:
    \begin{enumerate}[i)]
        \item There exists a small set of objects in $\catC'$ that generates $\catC'$ under small colimits;
        \item For any sufficiently large cardinals $\tau$ with $\tau\gg\kappa$, any $\tau$-filtered partially ordered set $A$,
        and any diagram $\{X_\alpha\}_{\alpha\in A}$ of $\kappa$-small objects in $\catC$ indexed by $A$, if $X_A\in\catC'$,
        where $X_B$ is the colimit of $\{X_\alpha\}_{\alpha\in B}$ for all $B\subseteq A$, then for every $\tau$-small subset $C\subseteq A$,
        there exists a $\tau$-small $\kappa$-filtered subset $B\subseteq A$ containing $C$ such that $X_B\in\catC_0$.
    \end{enumerate}
}

\cor{
    Suppose $F:\catC\to\catD$ is a functor between presentable categories which preserves $\kappa$-filtered colimits and 
    $\catD'$ is a $\kappa$-filtered full subcategory of $\catD$. Then $F^{-1}(\catD')$ is a $\kappa$-filtered full subcategory of $\catC$.
}

We will discuss more about presentable categories and accessible categories in Section \ref{seck} and Section \ref{secl}.
\footnote{We note that even though we are discussing on presentable and accessible $\infty$-categories in Section \ref{seck}
and Section \ref{secl}, all arguments are also available for ordinary categories, and not circularity will result from
using the results given in the two stated sections now.}

\subsection{Preliminaries for Model Category Theory}

% To be added: Enriched homotopy theory

\section{Introduction to Infinity Categories}

\subsection{Foundations of Infinity Categories}

We use some simple examples to give the motivation of $\infty$-categories.

\eg{\label{tagk}
    Suppose $X$ is a topological space. The \term{fundamental groupoid} of $X$, denoted $\Pi_{\le 1}(X)$, is the category with:
    \begin{itemize}
        \item objects being all points in $X$; and
        \item $\Hom_{\Pi_{\le 1}(X)}(a,b)$ being the homotopy classes of all paths from $a$ to $b$.
    \end{itemize}
    This is some definition that we shall already have seen in algebraic topology. This gives rise to a functor $\Pi_{\le 1}:\cat{Top}\to\cat{Gpd}$.
    Moreover, for any map $f:X\to Y$ of topological spaces, $\Pi_{\le 1}(f)$ is an equivalence of categories, if and only if 
    it is fully faithful and essentially surjective, where the first assertion indicates that $\pi_1(f)$ is a bijection, and the second assertion
    indicates that $\pi_0(f)$ is a bijection. 

    However, a topological space, contains far more information than its 0th and 1st homotopy groups. There also not only exist
    homotopies between paths, there also exist homotopies between homotopies, and homotopies between the homotopies between homotopies, etc.
    How shall we encode such higher structures? We may think of a ``category'', with not only morphisms between objects, but also morphisms
    between morphisms, and morphisms between the morphisms between morphisms, etc. Specifically, we would like to define $\Pi(X)$ as follows:
    \begin{itemize}
        \item Objects being all points $a,b,\cdots$ in $X$; 
        \item $\Hom_{\Pi(X)}(a,b)$ being all paths $\gamma,\delta,\cdots$ from $a$ to $b$;
        \item $\Hom_{\Pi(X)}(\gamma,\delta)$ being all homotopies $H,K,\cdots$ from $\gamma$ to $\delta$;
        \item $\cdots$
    \end{itemize}
    And we would like to obtain the property that for any map $f:X\to Y$ of topological spaces, $\Pi(f)$ is an ``equivalence'' of
    ``categories'', if and only if $f$ induces isomorphisms on all homotopy groups; that is, $f$ is a weak homotopy equivalence.
    The theory of $\infty$-categories are introduced to make this thought precise and to study their properties.
}

\eg{
    We recall homotopies between maps in model categories: Given a model category $\catC$ and cofibrant-fibrant objects $A,B$,
    maps $f,g:A\to B$, a homotopy from $f$ to $g$ is equivalent to a map $h:A\ox\d[1]\to B$, such that $h|_{A\ox0}=f,h|_{A\ox1}=g$.
    However, cosimplicial frames not only allow us to construct the cylinder object of $A$, but also higher structures.
    For example, a map $A\ox\d[2]\to B$ may be viewed as a homotopy between the three homotopies given by its restriction on
    $A\ox\p\d[2]$, a map $A\ox\d[3]\to B$ may be viewed as a homotopy between the four homotopies between homotopies
    given by its restriction on $A\ox\p\d[3]$. The theory of $\infty$-categories gives us a tool on how to 
    formalize this kind of higher structures, and to study their properties.
}

Notice, that in these two examples, higher morphisms are ``invertible'' (homotopies are invertible, at least up to higher homotopies).
This also coincides with most usages of $\infty$-categories, as in most applications we may see. Therefore, we only talk about
$\infty$-categories, whose ``$k$-morphisms'' are invertible whenever $k\ge 2$. \footnote{In higher category theory, we use 
the term $(\infty,n)$-category to indicate an $\infty$-category whose $k$-morphisms are invertible whenever $k\ge n+1$. In other words,
what we are focusing about is $(\infty,1)$-categories.}

So how shall we define them? The most na\"ive approach is to define them as what we did in Section \ref{seca}. However, notice one thing: 
in both examples, associativity and unit axioms fails to be strict, but only hold up to homotopy, i.e. higher morphisms. To fix this issue,
one must add certain extra diagrams to give the correct notion of higher categories, rather than the ones given in Section \ref{seca}.
(The reason we introduce them in the first chapter is to make the language we use concise. However, pseudo-2-functors, DO give an approach to
functors between $\infty$-categories, as they keep the homotopy-coherent ideas. From now on, we will call that kind of 2-category
introduced in Section \ref{seca} \term{strict 2-categories}.) On the other hand, even the explicit definition of a 3-category 
is EXTREMELY complicated, and things are just getting worse when we pass to 4-categories and beyond.

Fortunately, we have an alternative approach to the theory of $\infty$-categories.

From our na\"ive thought, if we take the ``Hom-thing'' between two object in an $\infty$-category, it will result in an $(\infty,0)$-category.
In other words, $\infty$-categories are categories enriched over $(\infty,0)$-categories. Consequencely, we only have to make a model for
$(\infty,0)$-categories.

Notice that in Example \ref{tagk}, all morphisms, including 1-morphisms, are invertible. This means that $\Pi(X)$ ``should be
an $(\infty,0)$-category''. On the other hand, there is a general axiom, known as the \term{homotopy hypothesis},
accepted by higher categorists, which is roughly stated as follows:

\begin{itemize}
    \item Every $(\infty,0)$-category has the form $\Pi(X)$ for some topological space $X$. Moreover two $(\infty,0)$-categories 
    are equivalent, if and only if their corresponding topological spaces are weakly equivalent.
\end{itemize}

This axiom ensures that we have a one-to-one correspondence between $(\infty,0)$-categories and equivalences classes of topological spaces,
where two spaces are equivalent if and only if they are weakly equivalent. That is, we use objects 
in $\Ho\cat{Top}\simeq\Ho\cat{K}\simeq\Ho\cat{SSet}$ to describe $(\infty,0)$-categories. Because of such important role they play in
higher category theory, we give them a special name: \term{homotopy types}. We write $\cat{H}$ for the category of homotopy types.

Now we have a schematic view of $(\infty,1)$-categories: loosely speaking, they are just $\cat{H}$-enriched categories.
However, doing computations in $\cat{H}$-enriched categories are really difficult. Moreover, not all $\cat{H}$-enriched categories
are $(\infty,1)$-categories. This is because of the difference between homotopy-commutative diagrams and homotopy-coherent diagrams,
as we will discuss in the next section. Therefore, we need to give correct models for $(\infty,1)$-categories. 
Moreover, we require these models to be easy to compute on. We shall now give three different models for us to work with.

\rmk{
    Although we are giving different models, they are actually representing the same thing: $(\infty,1)$-categories. Therefore it is important 
    to show that these models are equivalent, at least up to equivalence between categories. We will devolope how equivalences between categories
    are defined in these different models, and how one model can be translated into another, and show that under equivalences between categories,
    all these models are equivalent.
}

The first model is topological categories. 

\defn{
    A \term{topological category} is a category enriched over $\cat K$, the category of all $k$-spaces.
    \footnote{The reason we use $k$-spaces here, is that $\cat K$ is a closed monoidal category, and the geometric realization plays well 
    with $k$-spaces. From now on, unless otherwise stated, when we say ``topological spaces'', what we actually mean is ``$k$-spaces.''}
}

By the definition of a topological category, if we apply the localization functor $\cat K\to\cat H$, a topological category becomes an
$\cat{H}$-enriched category. We will now specify this localization functor with the symbol $\h$. Therefore, topological categories are indeed
models for $\infty$-categories. 

\defn{
    The \term{homotopy category} of a topological category $\catC$ is $\h\catC$. Two topological categories are said to be 
    \term{(weakly) equivalent} if their homotopy categories are equivalent (viewed as $\cat H$-enriched categories).
}

Note that an equivalence between topological categories, at least in higher category theory, does not mean it is an equivalence
between $\cat K$-enriched categories. This is because essentially the morphisms spaces we are dealing with are actually homotopy types,
not real spaces.

The second model is simplicial categories.

\defn{
    A \term{simplicial category} is a category enriched over $\cat{SSet}$, the category of all simplicial sets.
    \footnote{Note that a simplicial category is not equivalent to a simplicial object in the category of the categories.}
}

By the definition of a simplicial category, if we apply the localization functor $\cat{SSet}\to\cat H$, a simplicial category becomes an
$\cat{H}$-enriched category. We will, again, specify this localization functor with the symbol $\h$. Therefore, simplicial categories
are also indeed models for $\infty$-categories. 

\defn{
    The \term{homotopy category} of a simplicial category $\catC$ is $\h\catC$. Two simplicial categories are said to be 
    \term{(weakly) equivalent} if their homotopy categories are equivalent (viewed as $\cat H$-enriched categories).
}

We now consider the relations between these two definitions. We already know that there exists an adjunction $(\abs{-},\Sing):\cat{SSet}\to\cat K$,
that is a closed symmetric monoidal Quillen equivalence. Consequencely, we get an adjunction 
$(\abs{-},\Sing):\cat{Cat}_{\cat{SSet}}\to\cat{Cat}_{\cat K}$. By Section \ref{secb}, this adjunction is a Quillen equivalence between the
canonical model structures on both sides. Thus both the left and right adjoints preserve equivalences, and both the unit morphisms 
and counit morphisms are equivalences. Thus the two models can be regarded as equivalent models for $(\infty,1)$-categories.

However, these two models have disadvantages. As we have said before, in most $(\infty,1)$-categories (that we actually work with), compositions
of morphisms is associative only up to homotopy. But in a topological category, composition of morphisms are associative up to nose. 
Therefore we must figure out how to transform a homotopy coherent diagram into a strictly commutative diagram. Although this is always
possible (See Section \texttt{\color{ff0000}[ERROR! SECTION NOT FOUND]}), it is convenient for us to get a more flexible model 
of $(\infty,1)$-categories.

We note that in some cases, simplicial sets can regarded as categories. We may regard 0-simplices of the simplicial set as objects of the category,
and 1-simplices of the simplicial set as morphisms of the category. For example, the simplicial set $\d[n]$ can be regarded as the category, 
that comes from the partially ordered set with $n+1$ objects $[n]$. This gives rise to the following definition:

\defn{
    We consider the functor $\d\to\cat{Cat},[n]\mapsto[n]$. This gives a cosimplicial object in the category of all categories. By
    Proposition \ref{tagb}, this gives rise to an adjunction $\cat{SSet}\to\cat{Cat}$. We call the right adjoint of this adjunction
    the \term{nerve functor}, denoted $\N$.
}

Now, for any category $\catC$, all $n$-simplices in $\N\catC$ are simply all composable sequences of maps 
$$X_0\xrightarrow{f_0}X_1\xrightarrow{f_1}\cdots\xrightarrow{f_{n-1}}X_n.$$

The nerve functor enjoys the following property:

\prop{
    Suppose $K$ is a simplicial set. The following conditions are equivalent:
    \begin{enumerate}[i)]
        \item There exists a category $\catC$ and an isomorphism of simplicial sets $K\cong \N(\catC)$;
        \item Any map $\l^i[n]\to K$, where $0<i<n$, extends \textnormal{uniquely} to a map $\d[i]\to K$.
    \end{enumerate}
}

Let us take a pause here to see what does this proposition means. What is a map $\d[2]\to \N(\catC)$? it means taking three morphisms
$f:X\to Y,g:Y\to Z$ and $h:X\to Z$ in $\catC$, such that $h=gf$. Similarly, a map $\l^1[2]\to \N(\catC)$ means taking two morphisms 
$f:X\to Y,g:Y\to Z$. Now the meaning for the statement ``any map $\l^1[2]\to \N(\catC)$, extends uniquely to a map $\d[2]\to \N(\catC)$'' means: 
it simply means that every two morphisms that are tip-to-tail are uniquely composable. Similarly, the statement ``any map $\l^i[3]\to \N(\catC)$,
where $i=1,2$, extends uniquely to a map $\d[3]\to \N(\catC)$'' means that composition is strictly associative. Note that we do not
include the cases $i=0$ and $i=n$. This is because ``any map $\l^0[2]\to \N(\catC)$, extends uniquely to a map $\d[2]\to \N(\catC)$''
will indicate that every map has a left inverse, which is only possible whenever $\catC$ is a groupoid.

We have seen that certain simplicial sets can be used to model ordinary categories. We may want to show that certain simplicial sets can 
be used to model $(\infty,1)$-categories. 

Now, suppose $K$ is a simplicial set, that is considered to be an $(\infty,1)$-category. 
What kind of properties would we like $K$ to have? We would like to consider 0-simplices as objects, 1-simplices as morphisms, as usual.
We would also like every two morphisms that are tip-to-tail are composable. This means that any map $\l^1[2]\to K$, extends to a map $\d[2]\to K$.
However, if we still require the composition to be unique, we are on our way back to ordinary categories. In fact, it is unreasonable,
and unnatural, to require composition of two morphisms to be unique. For example, there exists multiple ways to join two paths 
to obtain a longer path in any topological space $X$, meaning that there exists multiple choices of compositions in $\Pi(X)$. Nevertheless,
these choices are homotopic. The condition that different compositions are homotopic, can be recovered by the statement
``any map $\l^i[3]\to K$, where $i=1,2$, extends to a map $\d[3]\to K$''. Similarly, higher homotopy relations can be recovered by the statement
``any map $\l^i[n]\to K$, where $0<i<n$, extends to a map $\d[i]\to K$''. Again, we do not include the cases $i=0$ and $i=n$,
since they are only satisfied when every morphism in $K$ is ``invertible''. 

We now make our discussion precise:

\defn{
    A \term{quasi-category}, is a simplicial set $K$, such that $$(K\to *)\in\RLP(\{\l^i[n]\to\d[n]\mid 0<i<n\}).$$
}

Since this is the mainly used model for $(\infty,1)$-categories, from now on, when we say an $\infty$-category, we refer to a quasi-category,
unless stated otherwise.

\eg{
    The nerve of an ordinary category is an $\infty$-category.
}

\eg{
    Kan complexes are $\infty$-categories. Moreover, they enjoy right lifting property with respect to $\l^i[n]\to\d[n]$ even if $i=0$ or $i=n$,
    meaning that every morphism in a Kan complex is ``invertible''. For this reason, we simply define \term{$\infty$-groupoids} or
    \term{$(\infty,0)$-categories} to be Kan complexes, and say that two $\infty$-groupoids are \term{equivalent} if they are weak equivalent.
    This give a nice realization of the homotopy hypothesis, and the construction $\Pi$ taking a topological space to an $\infty$-groupoid
    can be simply realized by $\Sing$.
}

We now discuss the relations between $\infty$-categories and simplicial categories. In order to show that these two models are equivalent
(in an appropriate sense), we must from a pair of functors $\cat{SSet}\to\cat{Cat}_{\cat{SSet}}$ and $\cat{Cat}_{\cat{SSet}}\to\cat{SSet}$.
As shown in Proposition \ref{tagb}, a cosimplicial simplicial category will give rise to an adjunction $\cat{SSet}\to\cat{Cat}_{\cat{SSet}}$.
In the case of ordinary categories, we gave the cosimplicial category $\d\to\cat{Cat},[n]\mapsto[n]$. This makes the composition of morphisms
in a simplicial set strictly associative, which is what in the case of an ordinary category we want. Consequencely, we would like
to ``thicken the morphism space'' to give rise to the homotopy-coherent idea of composing the morphisms. This gives rise to
the following definition:

\defn{
    We define the functor $\C:\cat{TOSet}\to\cat{Cat}_{\cat{SSet}}$ as follows. For any totally ordered set $J$, define $\C[J]$ to be the
    simplicial category with:
    \begin{itemize}
        \item Objects being all objects of $J$;
        \item For any $i,j\in J$, $\Hom_{\C[J]}(i,j)=\emptyset$ if $i>j$, and $\Hom_{\C[J]}(i,j)=\N(P_{i,j})$ if $i\le j$, where
        $P_{i,j}$ is the partially ordered set $\{I\subseteq J\mid i,j\in I;\forall k\in I,i\le k\le j\}$;
        \item For $i\le j\le k$, the composition law $\Hom_{\C[J]}(i,j)\times\Hom_{\C[J]}(j,k)\to\Hom_{\C[J]}(i,k)$ is induced by the map
        of partially ordered sets $P_{i,j}\times P_{j,k}\to P_{i,k},(I,I')\mapsto I\cup I'$;
    \end{itemize}
    and for any maps $f:J\to J'$ between totally ordered sets, define $\C[f]$ to be the functor with:
    \begin{itemize}
        \item $\C[f](i)=f(i)$ for any object $i\in J$;
        \item For any $i,j\in J$, the map $\Hom_{\C[J]}(i,j)\to \Hom_{\C[J']}(f(i),f(j))$ is induced by the map 
        $P_{i,j}\to P_{f(i),f(j)},I\mapsto f(I)$.
    \end{itemize}
    This functor induces a cosimplicial object in the category of simplicial categories, which further induces an adjunction
    $\cat{SSet}\to\cat{Cat}_{\cat{SSet}}$. We denote this adjunction by $(\C,\N)$, and call the right adjoint the \term{simplicial nerve}
    functor.
}

This is really a complicated definition, so we shall pause again to figure out this definition. Take $\C[\d[3]]$ as example. We have
$$\Hom_{\C[\Delta[3]]}=\9\vcenter{\xymatrix{03\ar[r]\ar[d]\ar[rd]&013\ar[d]\\023\ar[r]&0123}}\0.$$
(Notice that it also contains two nondegenerate $2$-simplices that are not shown.) We notice that each vertex of this 
Hom-space represents a way from $0$ to $3$. In particular, there is a way from $0$ to $3$ that does not pass through any other vertices,
which is the $03$ vertex. We denote the morphism $0\to 3$ represented by the vertex $03$ as $p_{03}$. Now we notice that
$p_{03}\ne p_{13}p_{01}$. However, the $1$-simplex from $03$ to $013$ indicates that these two morphisms are homotopic to each other.
Moreover, the fact that the morphism space is weakly contractible, means that all possible compositions are canonically homotopic.
This corresponds to the point that $\C[n]$ is the ``thicken version'' of $[n]$ that we have dropped the strict associative condition
and replace it by the idea that compositions are commutative only up to coherent homotopy. Take another example, $\C[\p\d[3]]$.
We have $$\Hom_{\C[\p\d[3]]}=\9\vcenter{\xymatrix{03\ar[r]\ar[d]&013\ar[d]\\023\ar[r]&0123}}\0.$$ We may see that the path of maps
$03\to 013\to 0123\leftarrow 023\leftarrow 03$ is not homotopic to the constant path, which corresponds well with the fact that
$\p\d[3]$ is not weakly contractible.

As an exercise, the reader can show that if $0\le i\le j\le n$, then the map $\Hom_{\C[\p\d[n]]}(i,j)\to\Hom_{\C[\d[n]]}(i,j)$
is identity unless $i=0,j=n$, in which case it is a cofibration; then the map $\Hom_{\C[\l[n]]}(i,j)\to\Hom_{\C[\d[n]]}(i,j)$
is identity unless $i=0,j=n$, in which case it is an anodyne extension.

\defn{
    The \term{topological nerve} of a topological category $\catC$ is defined to be $\N\Sing\catC$.
}

\prop{
    Suppose $\catC$ is a fibrant simplicial category (in the canonical model structure of the category of simplicial categories)
    (which is equivalent to all of its Hom-spaces are Kan complexes). Then $\N\catC$ is an $\infty$-category.
}

\cor{
    The topological nerve of any topological category is an $\infty$-category.
}

We now give the definition to equivalences between $\infty$-categories.

\defn{
    Suppose $S$ is a simplicial set. The \term{homotopy category} of $S$ is $\h S:=\h\C[S]$. A map $f:S\to T$ between simplicial sets
    is called a \term{categorical equiavlence} if $\h f$ is an equivalence of $\cat{H}$-enriched categories.
}

We now state the following theorem, which will be proved in Section \texttt{\color{ff0000}[ERROR! SECTION NOT FOUND]}.

\thm{
    There exists a model structure on $\cat{SSet}$, called the \term{Joyal model structure}, where weak equivalences are 
    all the categorical equivalences and fibrant objects are the quasi-categories, such that the adjunction
    $$\xymatrix{(\cat{SSet},\text{Joyal})\ar@(ru,lu)[r]^{\C}\ar@{}[r]|{\bot}&(\cat{Cat}_{\cat{SSet}},\text{std})\ar@(ld,rd)[l]^{\N}}$$
    is a Quillen equivalence.
}

This means that the theory of quasi-categories and the theory of simplicial categories are equivalent models of $\infty$-categories.

Finally we give the following definition:

\defn{
    Suppose $\catC$ is a simplicial category, or a topological category, or a simplicial set. We define $\Ho\catC:=\pi_0\h\catC$.
}

To get a better view of what we have done in this section, it would be better if you have this graph in mind. Here all the arrows
above the categories are left adjoints, where all the arrows below the categories are right adjoints, equivalences
means Quillen equivalences.
$$\xymatrix @C=40pt{
    (\cat{SSet},\text{Joyal})\ar@(ru,lu)[r]^{\C}\ar@{}[r]|{\simeq}
    &(\cat{Cat}_{\cat{SSet}},\text{std})\ar@(ld,rd)[l]^{\N}\ar@(ru,lu)[r]^{\abs-}\ar@{}[r]|{\simeq}
    &(\cat{Cat}_{\cat{K}},\text{std})\ar@(ld,rd)[l]^{\Sing}\ar[r]^{\h}\ar@(ru,lu)[rr]^{\Ho}
    &\cat{Cat}_{\cat H}\ar[r]^{\pi_0}
    &\cat{Cat}\ar@(ld,rd)[ll]
}$$

\subsection{The Language of Infinity Categories}

\label{seci}

We now describe how most aspects of ordinary categories extends to infinity categories. 

\rmk{
    As we have stated above, since the three models are equivalent models for $(\infty,1)$-categories, what wen have defined for 
    $\infty$-categories (that are invariant under categorical equivalences) should extends naturally to topological categories
    and simplicial categories. The readers may feel free to use the results we have stated for $\infty$-categories on topological categories
    and simplicial categories, if they prefer.
}

\defn{
    Suppose $\catC$ is an $\infty$-category. An \term{object} in $\catC$ is a vertex in $\catC$. A \term{morphism} in $\catC$ is a $1$-simplex
    in $\catC$. The \term{domain} and \term{codomain} of a morphism $\phi:\d[1]\to\catC$ are $\phi(0)$ and $\phi(1)$, respectively.
    If a morphism $\phi$ has domain $x$ and codomain $y$, we write $\phi:x\to y$. The \term{identity morphism} of an object $x$ is
    $\1_x:=s_0(x):x\to y$.
}

\defn{
    Suppose $S$ is a simplicial set. We define the \term{opposite} of $S$ to be the simplicial set $S^\op$, with $S^\op_n=S_n$ but
    $$(d_i:S^\op_n\to S^\op_{n-1})=(d_{n-i}:S_n\to S_{n-1}),(s_i:S^\op_n\to S^\op_{n+1})=(s_{n-i}:S_n\to S_{n+1}).$$
    If $\catC$ is an $\infty$-category, we call $\catC^\op$ the \term{opposite category} of $\catC$.
}

We next consider equivalences between morphisms.

\defn{
    Suppose $\catC$ is an $\infty$-category. Two morphisms $f,g:x\to y$ in $\catC$ are said to be \term{homotopic}, 
    written as $f\simeq g$, if one of the following equivalence relations hold:
    \begin{enumerate}[i)]
        \item There exists a $2$-simplex $\vcenter{\xymatrix @R=5pt{x\ar[rd]^f\ar[dd]_{\1_x}&\\&y\\x\ar[ru]_g&}}$ in $\catC$;
        \item There exists a $2$-simplex $\vcenter{\xymatrix @R=5pt{x\ar[rd]^g\ar[dd]_{\1_x}&\\&y\\x\ar[ru]_f&}}$ in $\catC$;
        \item There exists a $2$-simplex $\vcenter{\xymatrix @R=5pt{&y\ar[dd]^{\1_y}\\x\ar[ru]^f\ar[rd]_g&\\&y}}$ in $\catC$;
        \item There exists a $2$-simplex $\vcenter{\xymatrix @R=5pt{&y\ar[dd]^{\1_y}\\x\ar[ru]^g\ar[rd]_f&\\&y}}$ in $\catC$.
    \end{enumerate}
}

\prop{
    The homotopic relation is an equivalence relation.
}

\defn{
    Suppose $\catC$ is an $\infty$-category. We define a category $\pi(\catC)$ as follows:
    \begin{enumerate}[i)]
    \item Its objects are vertices of $\catC$;
    \item For any $x,y\in\catC_0$, define
    $\Hom_{\pi(\catC)}(x,y)=\{f:x\to y\in\catC_1\}/\simeq$;
    \item For any $x\in\catC_0$, define $\1_x=[s_0x]$;
    \item For any morphisms $[f]:x\to y,[g]:y\to z$ in $\pi(\catC)$, take a lift
    of the following diagram:
    $$\xymatrix @C=40pt{
    \l^1[2]\ar[rr]^{f\text{ on }\d\{0,1\}}_{g\text{ on }\d\{1,2\}}\ar[d]&&\catC\\
    \d[2]\ar@{.>}[rru]_{h}&&
    }$$
    and define $[g]\circ[f]=[d_1h]$.
    \end{enumerate}
}

\prop{
    The category $\pi(\catC)$ is isomorphic to the category $\Ho(\catC)$, meaning that two morphisms $f\simeq g$ if and only if thet are equal in 
    $\Ho\catC$.
}

Next we discuss the mapping spaces between objects.

\defn{
    Suppose $S$ is a simplicial set, $x,y$ are vertices in $S$. We define the \term{mapping space} from $x$ to $y$ to be 
    $\Map_S(x,y):=\Hom_{\h S}(x,y)\in\cat H$.
}

So how shall we compute this mapping space? By definition, $\Map_S(x,y)$ is the homotopy type of $\Hom_{\C[S]}(x,y)$.
The advantage of this definition is that it has a natural composition law, and it works for all simplicial sets, not just $\infty$-categories.
However, this construction is quite complicated, and it is usually not a Kan complex, which makes it difficult to verify its properties.

We shall now introduce another simplicial set that represents the homotopy type $\Map_S(x,y)$, at least when $S$ is an $\infty$-category.

\defn{
    Suppose $S$ is a simplicial set, $x,y$ are vertices in $S$. We define $$\Hom^R_S(x,y)=\{\sigma\in S_{n+1}
    \mid\sigma|_{\d\{0,\cdots,n\}}=x,\sigma(n+1)=y\},$$ and $$\Hom^L_S(x,y)=\{\sigma\in S_{n+1}\mid\sigma(0)=x,\sigma|_{\d\{1,\cdots,n+1\}}=y\}.$$
    Moreover we define $$\Hom_S(x,y)=\{\sigma:\d[n]\times\d[1]\to S\mid \sigma|_{\d[n]\times\{0\}}=x,\sigma|_{\d[n]\times\{1\}}=y\}.$$
}

The following proposition will be proved in Section \ref{secd} and Section \ref{secf}.

\prop{
    If $S$ is an $\infty$-category, then for any objects $x,y$, the simplicial sets $\Hom^R_S(x,y),\Hom^L_S(x,y),\Hom_S(x,y)$ are Kan complexes. 
    There exists a canonical isomorphism $$\Hom^R_S(x,y)\xrightarrow{\simeq}\Hom_{\C[S]}(x,y)$$ in the category $\cat H$. The natural inclusions
    $$\Hom^R_S(x,y)\hookrightarrow\Hom_S(x,y)\hookleftarrow\Hom^L_S(x,y)$$ are homotopy equivalences.
}

We next discuss when can we say a morphism is an equivalence.

\defn{
    Suppose $\catC$ is a topological category, of a simplicial category, or an $\infty$-category. A morphism $f$ in $\catC$ is called an
    \term{equivalence}, or \term{invertible}, if it is an isomorphism in $\h\catC$.
}

The following criterion is useful while we want to determine whether a morphism is an equivalence or not.

\prop{
    Suppose $\catC$ is a topological category, $f:X\to Y$ is a morphism in $\catC$. The following conditions are equivalent:
    \begin{enumerate}[i)]
        \item $f$ is an equivalence;
        \item $f$ has a homotopy inverse; that is, there exists a morphism $g:Y\to X$ such that $fg\simeq \1_Y$ and $gf\simeq\1_X$;
        \item For any $Z\in\catC$, composition with $f$ induces a homotopy equivalence $\Hom_\catC(Z,X)\to\Hom_\catC(Z,Y)$;
        \item For any $Z\in\catC$, composition with $f$ induces a weak homotopy equivalence $\Hom_\catC(Z,X)\to\Hom_\catC(Z,Y)$;
        \item For any $Z\in\catC$, composition with $f$ induces a homotopy equivalence $\Hom_\catC(Y,Z)\to\Hom_\catC(X,Z)$;
        \item For any $Z\in\catC$, composition with $f$ induces a weak homotopy equivalence $\Hom_\catC(Y,Z)\to\Hom_\catC(X,Z)$.
    \end{enumerate}
}

The following proposition will be proved in Section \ref{secg}.

\prop{\label{tagm}
    Suppose $\catC$ is an $\infty$-category, $\phi$ is a morphism in $\catC$. $\phi$ is an equivalence, if and only if for every
    map $f:\l^0[n]\to\catC$ such that $f|_{\d\{0,1\}}=\phi$, $f$ extends to a map $\d[n]\to\catC$.
}

Using this proposition we obtain the following result:

\prop{\label{tagn}
    Suppose $\catC$ is a simplicial set. The following conditions are equivalent:
    \begin{enumerate}[i)]
        \item $\catC$ is an $\infty$-groupoid;
        \item $\catC$ is an $\infty$-category and $\Ho\catC$ is a groupoid;
        \item $\catC$ is an $\infty$-category and $\catC$ has extension property with respect to all inclusions $\l^0[n]\to\d[n]$;
        \item $\catC$ is an $\infty$-category and $\catC$ has extension property with respect to all inclusions $\l^n[n]\to\d[n]$.
    \end{enumerate}
}

Next we talk about the ``underlying $\infty$-groupoid of an $\infty$-category''.

\prop{
    Suppose that $\catC$ is an $\infty$-category. Let $\catC'\subseteq\catC$ be the lergest simplicial subset of $\catC$
    such that all edges of $\catC'$ are invertible in $\catC$. Then $\catC$ is a Kan complex. Moreover for any Kan complex $K$,
    the induced map $\Hom_{\cat{SSet}}(K,\catC')\to\Hom_{\cat{SSet}}(K,\catC)$ is an isomorphism. We call $\catC'$ the \term{largest
    Kan complex} contained in $\catC$. If $f:\catC\to\catD$ is a categorical equivalence between $\infty$-categories, then the induced map
    $f':\catC'\to\catD'$ between the largest Kan complexes is a homotopy equivalence.
    \footnote{The lecturer cannot figure out a proof to the last assertion while not using the Joyal model structure of simplicial sets,
    although the book said that it can be proved directly.}
}

We next consider diagrams on $\infty$-categories. This gives rises to the difference between homotopy-commutative diagrams
and homotopy-coherent diagrams, as it has been promised that we will discuss it in this section. 

The issue is as follows. Suppose $\catC$ is an $\infty$-category (or a topological category, simplicial category). To a first approximation,
it is very similar to work on $\catC$ and $\h\catC$, since they have the same objects and morphisms. However, in $\catC$, we should ask
whether or not two morphisms are equivalent or homotopic, but not equal. In this case, the homotopy between the morphisms need to
be taken in consider, since they may as well affect the higher-dimensional homotopies. However, a commutative diagram in $\h\catC$,
does not contain the information on the homotopies between the morphisms. Consequencely, commutative diagrams on $\h\catC$, which corresponds
to the notion of \term{homotopy-commutative diagrams} on $\catC$, is quite unnatural and need to be refined by the notion of
\term{homotopy-coherent diagrams} on $\catC$.

Let us give two examples to show their differences.

\eg{
    Consider a monoid object $G$ in $\cat H$, and consider the $\cat H$-enriched category $\cat BG$. 
    Then a $\cat BG$-valued diagram in a topological category is a homotopy-commutative diagram. On the other hand, the mutiplication on $G$
    is homotopy commutative, i.e. the maps $(xy)z,x(yz):G\times G\times G\to G$ are homotopic, which means that $G$ is an $A_3$-algebra.
    However, in practice we would like to require that $G$ is not just an $A_3$-algebra, but also satisfies higher 
    homotopy-coherent properties. For example, we would like the following diagram of homotopies between morphisms $G\times G\times G\times G\to G$
    to be commutative up to a homotopy:
    $$\xymatrix @C=-5pt @R=30pt{&(x(yz))w\ar@{-}[rr]\ar@{-}[ld]&&x((yz)w)\ar@{-}[rd]&\\
    ((xy)z)w\ar@{-}[rrd]&&&&x(y(zw))\ar@{-}[lld]\\&&(xy)(zw);&&}$$
    and also the following diagram of homotopies between homotopies between maps $G\times G\times G\times G\times G\to G$
    to be commutative up to a homotopy:
    $$\xymatrix @!0 @C=44pt{&&&&&(((ab)c)d)e\ar@{-}[rrrr]\ar@{-}[llld]\ar@{-}'[dd]'[ddddd][dddddd]&&&&((ab)(cd))e\ar@{-}[lldd]\ar@{-}[ddddd]\\
    &&((a(bc))d)e\ar@{-}[ld]\ar@{-}'[d]'[dddd][dddddd]&&&&&&&\\
    &(a((bc)d))e\ar@{-}[rrrrrr]\ar@{-}[dddl]&&&&&&(a(b(cd)))e\ar@{-}[dddl]&&\\
    &&&&&&&&&\\
    &&&&&&&&&\\
    a(((bc)d)e)\ar@{-}[rrrrrr]\ar@{-}[dddd]&&&&&&a((b(cd))e)\ar@{-}[ddd]&&&(ab)((cd)e)\ar@{-}[lld]\ar@{-}[lllddd]\\
    &&&&&((ab)c)(de)\ar@{-}[llld]\ar@{-}'[r][rr]&&(ab)(c(de))\ar@{-}'[ld][lllddd]&&\\
    &&(a(bc))(de)\ar@{-}[lldd]&&&&&&&\\
    &&&&&&a(b((cd)e))\ar@{-}[lld]&&&\\
    a((bc)(de))\ar@{-}[rrrr]&&&&a(b(c(de)))&&&&&}$$
    This requirement is equivalent to assuming that $G$ is an $A_\infty$-algebra. But there exists $A_3$ algebras that are not $A_\infty$-algebras.
    For such $G$, $\cat BG$-valued diagrams are ``bad'' since they cannot recover all the homotopy relations that we want. Also, 
    because of the lack of higher homotopies, the $\cat H$-enriched category $\cat BG$ should not be viewed as an $\infty$-category,
    which corresponds to the fact that not all $\cat{H}$-enriched categories are $(\infty,1)$-categories, as it has been mentioned
    in the previous section.
}

\eg{
    Now suppose that we have an $[n]$-valued diagram in the simplicial category $\cat{SSet}$, that we wish to be homotopy-coherent.
    We denote this diagram by $F$. For simplicity, we suppose that $F(i)=*$ for every $0\le i<n$.
    (The reader should also think of what will happen if $F(i)$ are arbitrary simplicial sets, as we will come back to this 
    in several sections.) We define $e_i$ to be the image of $F(i)$ in $F(n)$. Then, for every $0\le i\le j<n$,
    since the maps $i\to j\to n$ and $i\to n$ are equal, we would like to assume that there exists a $1$-simplex $e_{ij}$
    from $e_i$ to $e_j$ in $F(n)$. Also, for every $0\le i\le j\le k<n$, since the diagram 
    $$\xymatrix{&j\ar[r]\ar[rrd]&k\ar[rd]\\i\ar[ru]\ar[rru]\ar[rrr]&&&n}$$ is commutative, we would like to assume that 
    there exists a $2$-simplex $e_{ijk}$ in $F(n)$ filling the homotopies $e_{ij},e_{ik},e_{jk}$; and for every $0\le i\le j\le k\le l<n$,
    we would like to assume that there exists a $3$-simplex $e_{ijkl}$ in $F(n)$ filling the homotopies $e_{ijk},e_{ijl},e_{ikl},e_{jkl}$;
    and homotopies of higher degrees as well. Such a diagram contains much more information than a homotopy-commutative diagram,
    where the latter only contains the information of $e_{ij}$'s, and losing the informations of higher homotopies.
}

So how shall we define a homotopy-coherent diagram? We could define this by working degree by degree 
in topological categories or simplicial categories. However, this is a great amount of data and is really difficult to actually work with.
But in $\infty$-categories, things are really easy to formulate: Suppose $\catI$ is an ordinary category and $\catC$ is an $\infty$-category,
we may simply define an $\catI$-valued (homotopy-coherent) diagram in $\catC$ to be a map $\N(\catI)\to\catC$ of simplicial sets. (This is,
in fact, one of the reasons that quasi-categories are one of the most widely-used model for $(\infty,1)$-categories.)

The notion of homotopy-coherent diagrams in higher categories is a special case of functors between higher categories. Again, this is 
very easy to define in the settings of $\infty$-categories:

\defn{
    Suppose $\catC,\catD$ are $\infty$-categories. A \term{functor} from $\catC$ to $\catD$ is simply a map $F:\catC\to\catD$ of simplicial sets.
    We define $\Fun(K,\catD):=\catD^K$, for every simplicial set $K$. (In this notation, $K$ will often, but not always, be a simplicial set.)
}

The following proposition will be proved in Section \ref{sece}.

\prop{
    Suppose $K,K'$ are simplicial sets, $\catC,\catD$ are $\infty$-categories. Then:
    \begin{enumerate}[i)]
        \item $\Fun(K,\catC)$ is an $\infty$-category;
        \item If $F:\catC\to\catD$ is a categorical equivalence, then the induced map $\Fun(K,\catC)\to\Fun(K,\catD)$ is a categorical equivalence;
        \item If $f:K\to K'$ is a categorical equivalence, then the induced map $\Fun(K',\catC)\to\Fun(K,\catC)$ is a categorical equivalence.
    \end{enumerate}
}

\defn{
    Suppose $\catC,\catD$ are $\infty$-categories. We define the \term{$\infty$-category of functors} from $\catC$ to $\catD$ to be 
    $\Fun(\catC,\catD)$. We define \term{natural transformations} between functors from $\catC$ to $\catD$ to be morphisms in $\Fun(\catC,\catD)$,
    and call a natural transformation a \term{natural equivalence} if it is invertible in $\Fun(\catC,\catD)$.
}

Next we discuss slice categories. It is shown in Section \ref{secc} that in the case of ordinary categories, if $p:\catC\to\catD$
is a diagram in $\catD$, then for any $\catC'$, we have isomorphism
$$\Fun(\catC',\catD_{/p})\cong\Fun(\catC'\star\catC,\catD)\times_{\Fun(\catC,\catD)}\{p\},$$ that is natural in $\catC'$. 
This gives rise to the following definitions:

\defn{
    Suppose $S,S'$ are simplicial sets. We define the \term{join} of $S$ and $S'$, denoted $S\star S'$, to be the simplicial set
    with $$\Hom_{\cat{SSet}}(\d[n],S\star S')=\left\{(f,\sigma,\sigma')\mathrel{\Bigg|}
    \begin{aligned}&f:[n]\to[1],\\&\sigma:\d(f^{-1}(0))\to S,\\&\sigma':\d(f^{-1}(1))\to S'\end{aligned}\right\}.$$
}

If the reader has trouble understanding the definition, it is suggested to take $S=\d[2],S'=\d[1]$ as an example to get a better understanding.

We also give the following definition:

\defn{
    Suppose $K$ is a simplicial set. We define the \term{left cone} of $K$ to be $K^\tril:=*\star K$, and the \term{right cone} of $K$ to be
    $K^\trir:=K\star *$. The distinguished vertex in either cone will be called the \term{cone point}, and will be written as
    $-\infty$ in the left cone or $\infty$ in the right cone.
}

With this definition, one can verify that $(\cat{SSet},\star)$ is a closed monoidal category, and the following proposition holds:

\prop{
    If $S,S'$ are $\infty$-categories, then $S\star S'$ is an $\infty$-category.
}

Moreover, we have the following proposition:

\prop{
    For any simplicial category $\catC,\catC'$ we have a canonical isomorphism $\N(\catC\star\catC')\cong\N(\catC)\star\N(\catC')$.
}

On the other hand, $\star$ does not commute with the functor $\C$. Nevertheless, we have the following proposition 
which will be proved in Section \ref{secf}:

\prop{
    For any simplicial sets $S,S'$ there exists a canonical map $\C[S\star S']\to\C[S]\star\C[S']$, which is an equivalence
    between simplicial categories.
}

Now we can define what a slice category is:

\prop{
    Suppose $p:K\to S$ is a map of simplicial sets. The functor $$\cat{SSet}\to\cat{SSet},
    X\mapsto\Fun(X\star K,S)\times_{\Fun(K,S)}\{p\}$$ is representable. We define $S_{/p}$ to be the representation of this functor,
    and call it the \term{$\infty$-category of objects of $S$ over $p$} whenever $S$ is an $\infty$-category.
    Dually, we have the definition of the \term{$\infty$-category of objects of $S$ under $p$}, which will be denoted $S_{p/}$.
}

To be more specifically, $$(S_{/p})_n=\{\sigma:\d[n]\star K\to S\mid \sigma|_{K}=p\}.$$ Similarly we may directly characterize
$S_{p/}$.

This definition is indeed a generalization of the slice category of ordinary categories:

\prop{
    If $p:\catC\to\catD$ is a diagram in an ordinary category $\catD$, then $\N(\catD)_{/\N(p)}\cong\N(\catD_{/p})$. The case for undercategories
    is similar.
}

The following proposition will be proved in Section \ref{secg} and Section \ref{sech}.

\prop{\label{tagl}
    Suppose $\catC$ is an $\infty$-category, $p:K\to\catC$ is a \term{diagram} in $\catC$ (which simply means that 
    $p$ is a map of simplicial sets). Then $\catC_{/p}$ is an $\infty$-category. Moreover, if $q:\catC\to\catD$ 
    is a categorical equivalence between $\infty$-categories, then the induced map $\catC_{/p}\to\catD_{/qp}$ is also a categorical equivalence.
    The case for undercategories is similar.
}

We next talk about fully faithful and essentially surjective functors, together with subcategories.

\defn{
    Suppose $f:\catC\to\catD$ is a map of simplicial sets (or other models for $\infty$-categories). $f$ is called \term{essentially surjective}
    if $\h f$ is esseentially surjective. $f$ is called \term{fully faithful} if $\h f$ is fully faithful.
}

By definition, $f$ is a categorical equivalence if and only if it is fully faithful and essentially surjective.

\defn{
    Suppose $\catD$ is a subcategory of $\Ho\catC$, where $\catC$ is an $\infty$-category. We define $\catC\times_{\N\Ho\catC}\N\catD$
    to be the \term{subcategory of $\catC$ spanned by $\catD$}. We say that a simplicial subset $K\subseteq\catC$ is a \term{subcategory}
    of $\catC$ if it is isomorphic to $\catC\times_{\N\Ho\catC}\N\catD$ for some subcategory $\catD$ of $\Ho\catC$. If $\catD$
    is a full subcategory of $\Ho\catC$, we also call $\catC\times_{\N\Ho\catC}\N\catD$ the \term{full subcategory of $\catC$ spanned by $S$},
    where $S$ is the set of objects in $\catD$.
}

By definition, if $\catC'$ is a full subcategory of $\catC$, then the inclusion $\catC'\hookrightarrow\catC$ is fully faithful.

Next we discuss initial and terminal objects together with limits and colimits. Since they are dual concepts, we will only talk about
initial objects in details, leave the arguments for terminal objects to the reader.

Our first approach to initial objects is the following:

\defn{
    Suppose $\catC$ is a simplicial set (or other models for $\infty$-categories). An object $x$ in $\catC$ is called \term{initial},
    if for any object $Y\in\catC$, the mapping space $\Map_\catC(x,y)$ is weakly contractible.
}

This definition is really natural and is useful. However, in the setting of $\infty$-categories, there exists an alternative definition
which is also very useful:

\defn{
    Suppose $\catC$ is a simplicial set. An object $x$ in $\catC$ is called \term{strongly initial} if the projection $\catC_{x/}\to\catC$
    is a trivial fibration.
}

In other words, $x$ is strongly initial, if and only if any map $\p\d[n]\to\catC$ with the vertex $0$ mapped to $x$ extends to a map
$\d[n]\to\catC$. The reader should verify that if $\catC$ is the nerve of an ordinary category, then $x$ is strongly initial
if and only if $x$ is initial.

The following proposition will be proved in Section \ref{secg}.

\prop{
    Suppose $\catC$ is an $\infty$-category. An object $x$ in $\catC$ is strongly initial, if and only if for every object $y\in\catC$, 
    the Kan complex $\Hom^L_\catC(x,y)$ is contractible.
}

\cor{
    If $\catC$ is a simplicial set, then every strongly final vertex in $\catC$ are final. The converse holds if $\catC$ is an $\infty$-category.
}

The initial object, if exists, is unique in the following sense:

\prop{
    Suppose $\catC$ is an $\infty$-category. The full subcategory of $\catC$ spanned by all initial objects, is either empty,
    or a contractible Kan complex.
}

We are now at the position to define colimits and limits.

\defn{
    Suppose $p:K\to\catC$ is a diagram in an $\infty$-category $\catC$. A \term{limit} of $p$ is a terminal object pf $\catC_{/p}$. 
    A \term{colimit} of $p$ is a initial object of $\catC_{p/}$. More generally, 
    a diagram $\overline{p}:K^\tril\to\catC$ is called a \term{limit diagram} if $\overline{p}$ is a limit of 
    $p:=\overline{p}|_K$; a diagram $\overline{p}:K^\trir\to\catC$ is called a \term{colimit diagram} if $\overline{p}$ is a colimit of 
    $p:=\overline{p}|_K$.
}

By the uniqueness of initial objects, the colimit of a diagram, if exists, is unique up to a contractible choice.

We note that there has already existed a theory of homotopy limits and colimits in simplicial or topological categories,
as well as in model categories, that we have defined them in Section \ref{secb}. We will discuss the relations between these theories
of limits and colimits in Section \ref{secf}. By now, we will just give the basic definitions and examples.

\eg{
    Suppose $\catC'$ is a full subcategory in $\catC$, and $p:K\to\catC'$ is a diagram in $\catC'$.
    Then $\catC'_{p/}\cong\catC_{p/}\times_\catC\catC'$ is a full subcategory of $\catC_{p/}$. In particular,
    if $p$ has a colimit in $\catC$ and the colimit lies in $\catC'$, then the same object can be regarded as the colimit of $p$
    in $\catC'$.
}

\defn{
    Suppose $f:\catC\to\catD$ is a functor between $\infty$-categories, $p:K\to\catC$ is a diagram in $\catC$ with colimit
    $x\in\catC_{p/}$. If $f(x)\in\catD_{fp/}$ is the colimit of $fp$, then we say that $f$ \term{preserves} the colimit of $p$.
    The definition for limits is similar.
}

We now introduce a simple example of colimit-preserving functors.

\prop{
    Suppose $\catC$ is an $\infty$-category, $q:T\to\catC$ is a diagram in $\catC$, $f:\catC_{/q}\to\catC$ is the projection. 
    Suppose more that $p:K\to\catC_{/q}$ is a diagram in $\catC_{/q}$, and $p_0=fp$. If $p_0$ has a colimit in $\catC$,
    then $p$ has a colimit in $\catC_{/q}$, and the colimit is preserved by $f$. Moreover an extension $\overline{p}:K^\trir\to\catC$
    is a colimit of $p$ if and only if $f\overline{p}$ is a colimit of $p_0$.
}

To end this section, we shall give names to some specific $\infty$-categories.

\defn{
    We define $\cat{Kan}$ to be the simplicial category of all (small) Kan complexes, viewed as a full subcategory of $\cat{SSet}$.
    We define $\cat S:=\N(\cat{Kan})$, and call it the \term{$\infty$-category of spaces}.
    We define $\cat{Cat}_\infty^\d$ to be the simplicial category with objects being all (small) $\infty$-categories,
    and $\Hom_{\cat{Cat}_\infty^\d}(\catC,\catD)$ is the largest Kan complex in $\Fun(\catC,\catD)$ for $\infty$-categories $\catC,\catD$.
    We define $\cat{Cat}_\infty=\N(\cat{Cat}_\infty^\d)$, and call it the \term{$\infty$-category of $\infty$-categories}.
}

The reader can verify that $\cat S$ and $\cat{Cat}_\infty$ are indeed $\infty$-categories. The $\infty$-category
$\cat S$ plays a central role in the theory of $\infty$-categories, just as the category $\cat{Set}$ plays a central role 
in the theory of ordinary categories. On the other hand, $\cat{Cat}_\infty$ to the theory of $\infty$-categories is very similar to the category 
$\cat{Cat}$ of all categories to the theory of ordinary categories.

We remark that we also need to distinguish small objects and large objects. In this case, we will use $\hatcat S$ and
$\hatcat{Cat}_\infty$ to describe the $\infty$-category of not necessarily small spaces, and the $\infty$-category 
of not necessarily small $\infty$-categories, respectively. We notice that $\cat S$ and $\cat{Cat}_\infty$ are large
$\infty$-categories and $\hatcat S$ and $\hatcat{Cat}_\infty$ are even larger.

\section{Grothendieck Construction}

\setcounter{subsection}{-1}

\subsection{Grothendieck Construction in Classical Category Theory}

In this section, we will discuss the Grothendieck construction in $\infty$-category theory.
We will see how this is useful to link simplicial categories and topological categories.
We will start with the Grothendieck construction in classical category theory to get some get some first sense,
and we will generalize it to the case of $\infty$-categories.

The Grothendieck construction in classical category theory is as follows. Suppose $F:\catC\to\catD$ is a functor of categories.
Then for every object $D\in\catD$, the fiber $\catC_{D}$ is a category. We sometimes wish that the action ``taking the fiber''
yields a functor $\catD\to\cat{Cat}$ (in the covariant fashion), or $\catD^\op\to\cat{Cat}$ (in the contravariant fashion);
or at least, in a homotopy-coherent sense. We will show that this can be done in the case that the functor $F$ is some ``fibration''.
The translation between certain ``fibrations'' over $\catD$, and homotopy-coherent functors $\catD\to\cat{Cat}$,
is called the \term{Grothendieck construction}.

\eg{
    Consider the category of all quasi-coherent sheaves, $\cat{QCoh}$. It has objects $(X,\mathscr{F})$, where $X$ is a scheme
    and $\mathscr{F}$ is a quasi-coherent sheaf on $X$; and a morphism from $(X,\mathscr{F})$ to $(Y,\mathscr{G})$
    consists of a map $f:Y\to X$ of schemes, together with a map $f^\sharp:\mathscr{F}\to f_*\mathscr{G}$ of $\mathscr{O}_X$-modules.
    Now take the functor $\cat{QCoh}\to\cat{Sch}^\op$ that takes a pair $(X,\mathscr{F})$ to its underlying scheme $X$.
    Then for every scheme $X$, the fiber of this functor at $X$ is the category of all quais-coherent sheaves on $X$,
    $\cat{QCoh}(X)$. Therefore the action ``taking the fiber'' actually yields a homotopy-coherent functor
    $\cat{QCoh}(-):\cat{Sch}^\op\to\cat{Cat}$.
}

To give a partial solution to the question, let us restrict to the case where the fibers of $F$ are groupoids, so we would
like to form a functor $\catD\to\cat{Gpd}$. The converse is easy: for any functor $\chi:\catD\to\cat{Gpd}$, we may form a functor
$F:\catC_\chi\to\catD$, where the category $\catC_\chi$ has:
\begin{itemize}
    \item Objects being $(D,\eta)$, where $D$ is an object in $\catD$ and $\eta$ is an object in $\chi(D)$;
    \item A morphism $(D,\eta)\to(D',\eta')$ consists of a morphism $f:D\to D'$ in $\catD$, and a morphism $\alpha:\chi(f)(\eta)\to\eta'$
    in $\chi(D')$;
\end{itemize}
and the functor $F$ takes every object $(D,\eta)$ to $D$, ignoring the $\eta$.
(In fact, this construction is also available in the case that $\chi$ is a homotopy-coherent functor,
or is a (homotopy-coherent) functor $\catD\to\cat{Cat}$.) 

For the converse, we make the following observation: for every map $f:D\to D'$ in $\catD$, we must construct a transport functor
$f_!:\catC_D\to\catC_{D'}$. In other words, we require the functor $F$ satisfies the following properties:
\begin{itemize}
    \item Transport of objects: for any object $C\in\catC$ such that $F(C)=D$, there exists a morphism $g:C\to C'$
    in $\catC$ such that $F(g)=f$;
    \item Transport of morphisms: for any diagram $$\xymatrix{C\ar[r]^{g}\ar[d]_{h}&C'\\\overline{C}\ar[r]^{\overline{g}}&\overline{C}',}$$
    such that $F(g)=F(\overline g)=f,F(h)=\1_D$, there exists UNIQUELY a map $h':C'\to\overline{C'}$ such that $h'g=\overline{g}h$
    and $F(h')=\1_{D'}$.
\end{itemize}
We now make this idea into a definition:

\defn{
    Suppose $F:\catC\to\catD$ is a functor. $F$ is called a \term{left fibration} if the following conditions are satisfied:
    \begin{itemize}
        \item The lifting problem $$\xymatrix{\Ho\l^0[1]\ar[r]\ar[d]&\catC\ar[d]\\\Ho\d[1]\ar[r]\ar@{-->}[ru]&\catD}$$ has a solution;
        \item The lifting problem $$\xymatrix{\Ho\l^0[2]\ar[r]\ar[d]&\catC\ar[d]\\\Ho\d[2]\ar[r]\ar@{-->}[ru]&\catD}$$ has a unique solution.
    \end{itemize}
}

The reader should verify that this definition is equivalent to the transport axioms given above, and that this definition
yields that every fiber of $F$ is a groupoid. Moreover, by the discussion given above, we have the following theorem:

\thm{
    A left fibration between small categories $\catC\to\catD$ gives rise to a 2-functor $\catD\to\cat{Gpd}$ by taking the fibers. 
    Conversely, every 2-functor $\catD\to\cat{Gpd}$ corresponds to a left fibration $\catC\to\catD$ in this manner.
}

We finally make the following definition.

\defn{
    We call a functor $F:\catC\to\catD$ a \term{right fibration} if $F^\op$ is a left fibration.
}

Under this definition, right fibrations $\catC\to\catD$ correspond to 2-functors $\catD^\op\to\cat{Gpd}$.

We will generalize the idea to left and right fibrations between simplicial sets, and discuss the corresponding
Grothendieck construction (on $\infty$-groupoids, which is equivalent to spaces) in the first three subsections.
In the last four subsections, we will be more general and discuss the Grothendieck construction on $\infty$-categories.

\rmk{
    In Lurie's book, he mainly considered the contravariant-fashioned structures, such as right fibrations and cartesian fibrations.
    This is also what the lecturer had done in class. However, the lecturer decided to write the notes mainly using
    the covariant-fashioned structures, such as left fibrations and cocartesian fibrations, in order to make it more intuitive
    and readable.
}

\subsection{Left Fibrations and the Covariant Model Structure}

\label{secg}

We first give the following definitions:

\defn{
    A map $p:X\to S$ is called:
    \begin{enumerate}[i)]
        \item An \term{inner fibration} if it is in $\RLP\{\l^i[n]\to\d[n]\mid 0<i<n\}$;
        \item A \term{left fibration} if it is in $\RLP\{\l^i[n]\to\d[n]\mid 0\le i<n\}$;
        \item A \term{right fibration} if it is in $\RLP\{\l^i[n]\to\d[n]\mid 0<i\le n\}$.
    \end{enumerate}
}

An inner fibration is what we generally want a ``good functor'' between $\infty$-categories look like, since for every functor $F:\catC\to\catD$
between ordinary categories, $\N(F)$ is an inner fibration. On the other hand, left and right fibrations come from the previous section,
and the reader may verify that a functor $F:\catC\to\catD$ between ordinary categories is a left (or right) fibration if and only if
$\N(F)$ is.

We also shall give the following definition:

\defn{
    A map $i:A\to B$ is called:
    \begin{enumerate}[i)]
        \item An \term{inner anodyne extension} if it is in $\Cof\{\l^i[n]\to\d[n]\mid 0<i<n\}$;
        \item A \term{left anodyne extension} if it is in $\Cof\{\l^i[n]\to\d[n]\mid 0\le i<n\}$;
        \item A \term{right anodyne extension} if it is in $\Cof\{\l^i[n]\to\d[n]\mid 0<i\le n\}$.
    \end{enumerate}
}

We will first study the stability properties of such maps under certain constructions.

\lem{
    Suppose $f:A\to A',g:B\to B'$ are cofibrations of simplicial sets. Suppose that $f$ is right anodyne or $g$ is left anodyne.
    Then the pushout product $$(A\star B')\coprod_{A\star B}(A'\star B)\to A'\star B'$$ is inner anodyne.
}

\lem{
    Suppose $f:A\to A',g:B\to B'$ are cofibrations of simplicial sets. Suppose that $f$ is anodyne.
    Then the pushout product $$(A\star B')\coprod_{A\star B}(A'\star B)\to A'\star B'$$ is left anodyne.
    (The dual statement is left for the reader.)
}

We now deduce the following proposition.

\prop{
    Suppose $i:A\to B$ is a cofibration of simplicial sets, $p:B\to X$ is a map of simplicial sets, and $q:X\to S$
    is an inner fibration. Define $r=qp,r_0=ri,p_0=pi$. Then:
    \begin{enumerate}[i)]
        \item The induced map $$X_{p/}\to X_{p_0/}\times_{S_{r_0/}}S_{r/}$$ is a left fibration;
        \item If $q$ is a right fibration then the induced map $$X_{p/}\to X_{p_0/}\times_{S_{r_0/}}S_{r/}$$ is a Kan fibration;
        \item If $i$ is right anodyne then the induced map $$X_{p/}\to X_{p_0/}\times_{S_{r_0/}}S_{r/}$$ is a trivial fibration;
        \item If $i$ is anodyne and $q$ is a left fibration then the induced map $$X_{p/}\to X_{p_0/}\times_{S_{r_0/}}S_{r/}$$ is a trivial fibration.
    \end{enumerate} 
}

We may now deduce the following proposition mentioned in \ref{seci}:

\cor{[Part 1 of Proposition \ref{tagl}]
    Suppose $\catC$ is an $\infty$-category, $p:K\to\catC$ is a diagram in $\catC$. Then the projection $\catC_{p/}\to\catC$ 
    is a left fibration. In particular, $\catC_{p/}$ is an $\infty$-category.
}

We furthermore hane the following propositions:

\prop{
    Suppose 
}

\subsection{The Unmarked Straightening and Unstraightening Functor}

\label{secd}

We now make the na\"ive idea to the Grothendieck construction to left fibrations precise, by introducing the straightening 
and unstraightening functors.

The construction is as follows. \tbc

\subsection{The Joyal Model Structure, Inner Fibrations and \texorpdfstring{$n$}{n}-Categories}

\label{sece}

\subsection{Cartesian and Cocartesian Fibrations}

\label{sech}

\subsection{Marked Simplicial Sets and the Cocartesian Model Structure}



\nyw

\subsection{The Marked Straightening and Unstraightening Functor}



\nyw

\subsection{Applications}



\nyw

\section{Limits and Colimits in Infinity Categories}

\subsection{Cofinality}



\nyw

\subsection{Techniques for Computing Colimits}

\label{secf}

\nyw

\subsection{Kan Extensions}



\nyw

\subsection{Applications}



\nyw

\section{Presentable and Accessible Infinity Categories}

\subsection{The Infinity Category of Presheaves}



\nyw

\subsection{Adjoint Functors}



\nyw

\subsection{The Infinity Category of Inductive Colimits}



\nyw

\subsection{Accessible Infinity Categories}

\label{seck}

\nyw

\subsection{Presentable Infinity Categories}



\nyw

\end{document}